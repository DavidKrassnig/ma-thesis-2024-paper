\chapter{Forschungsdaten im Repositorium der Leibniz Universität Hannover}\label{ch:luh-repo}
\parsum{Thema des Kapitels}
Dieses Kapitel behandelt die Auswertung von eingebetteten, begleitenden sowie referenzierten \glspl{forschungsdaten} von Dissertationen der \gls{luh}, die im institutionellen \gls{luh-repo} veröffentlicht worden sind.
Die Arbeit beschränkt sich hierbei exklusiv auf \glspl{forschungsdaten}, die originelle Primärdaten darstellen, die im Rahmen des Promotionsvorhabens entstanden sind.
Es wird hierbei überprüft, welcher Anteil an Dissertationen originelle \glspl{forschungsdaten} beinhaltet, auf welche Art und Weise die \glspl{forschungsdaten} inkludiert wurden, wie sich diese über die einzelnen Fakultäten verteilen, wie sich diese über die letzten zwölf Jahre entwickelt haben und wie die Existenz von \glspl{forschungsdaten} in den Metadaten kenntlich gemacht wurden (sowohl im \gls{luh-repo} wie auch in etwaigen externen \gls{forschungsdaten}-Repositorien).

\parsum{Aufbau des Kapitels}
Hierfür wird in \cref{sec:luh-repo-material-methods} aufgeführt, wie die zu untersuchenden Dissertationen ausgewählt wurden, wie das entsprechende Material gesammelt wurde und mit welchen Methoden dieses daraufhin ausgewertet wurde.
In \cref{sec:luh-repo-results} werden die entsprechenden Ergebnisse der Materialauswertung dargestellt.
Abschließend werden in \cref{sec:luh-repo-discussion} die dargestellten Ergebnisse evaluiert und diskutiert.

\section{Material \&\ Methoden}\label{sec:luh-repo-material-methods}
\parsum{Aufbau des Abschnitts}
In diesem Abschnitt wird das zu untersuchende Material in \cref{sec:luh-repo-material} und die Methoden der Untersuchung in \cref{sec:luh-repo-methods} dargestellt.

\subsection{Material}\label{sec:luh-repo-material}
\parsum{Datengrundlage}
Als Datengrundlage für dieses Kapitel gilt die Metadaten-Datenbank aller Dissertationen des \gls{luh-repo}, welche via \gls{oai-pmh} der Öffentlichkeit frei zugänglich sind \autocite{luh-repo}.
Da sich das Thema dieses Kapitels explizit auf Dissertationen beschränkt, wurden von der Metadaten-Datenbank des \gls{luh-repo} alle Einträge der Sammlung \textit{Dissertationen} die am 21.03.2024 um 10:41 Uhr (UTC+01:00) existierten ($n=5095$) über die Administrationsübersicht des \gls{luh-repo} heruntergeladen \autocite{my-dataset}.

\parsum{Grundmengenbeschreibung}
Da die zentrale Forschungsfrage dieses Kapitels sich auf den Zeitraum von 2012--2023 beschränkt, wurde diese Liste durch \cref{lst:python-luh-repo-stratification} auf nur jene Metadateneinträge gefiltert, deren Publikationsjahr in diese Zeitspanne fällt und deren Sperrfrist auch, insofern vorhanden, spätestens 2023 endete ($n=1898$).
Die daraus resultierende Dissertationsliste enthält Einträge zu jeder Fakultät der \gls{luh}.
Aus Gründen der Übersichtlichkeit und des Platzes nutzen wir folgende Abkürzungen für die Fakultäten der \gls{luh}:
\begin{itemize}
    \item \gls{fakultät2}
    \item \gls{fakultät3}
    \item \gls{fakultät4}
    \item \gls{fakultät5}
    \item \gls{fakultät6}
    \item \gls{fakultät7}
    \item \gls{fakultät8}
    \item \gls{fakultät9}
    \item \gls{fakultät10}
\end{itemize}
Der Name der Fakultät wird fortan nur noch separat erwähnt, wenn dieser relevant zur Diskussion und Verständlichkeit der Daten erscheint.
Der Zeitraum von 2012--2023 wurde für die weitere Bearbeitung wiederum in drei kontinuierliche Zeitintervalle von jeweils vier Jahren aufgeteilt.
Die relative sowie die absolute Distribution aller Metadateneinträge nach Zeitraum und Fakultät ist in \cref{tab:luh-repo-grundmenge-beschreibung} gegeben.
\begin{table}[!htbp]
	\caption{Die Verteilung der Grundmengen-Metadateneinträge nach $\text{\textit{Fakultät}}\times\text{\textit{Zeitraum}}$ aufgegliedert.
    Absolute Werte in Klammern angegeben.}
    \resizebox{\ifdim\width>\textwidth\textwidth\else\width\fi}{!}{%
	\begin{tabular}{lS[table-format=3.2]@{\,}S[table-text-alignment = left]lS[table-format=3.2]@{\,}S[table-text-alignment = left]lS[table-format=3.2]@{\,}S[table-text-alignment = left]lS[table-format=3.2]@{\,}S[table-text-alignment = left]l}
		\toprule
		& \multicolumn{3}{c}{\textbf{2012--2015}} & \multicolumn{3}{c}{\textbf{2016--2019}} & \multicolumn{3}{c}{\textbf{2020--2023}} & \multicolumn{3}{c}{\textbf{Summe}}    \\
		\midrule
		\textbf{\gls{fakultät2}}  & 0,68  & \si{\percent} & (13)  & 1,21  & \si{\percent} & (23)  & 1,37  & \si{\percent} & (26) & 3,27   & \si{\percent} & (62)         \\
		\textbf{\gls{fakultät3}}  & 1,00  & \si{\percent} & (19)  & 1,79  & \si{\percent} & (34)  & 3,64  & \si{\percent} & (69)  & 6,43    & \si{\percent} & (122)          \\
		\textbf{\gls{fakultät4}}  & 1,90  & \si{\percent} & (36)  & 2,63  & \si{\percent} & (50)  & 4,21  & \si{\percent} & (80)  & 8,75   & \si{\percent} & (166)          \\
		\textbf{\gls{fakultät5}}  & 0,00  & \si{\percent} & (0)   & 0,16  & \si{\percent} & (3)   & 0,05  & \si{\percent} & (1)  & 0,21    & \si{\percent} & (4)           \\
		\textbf{\gls{fakultät6}}  & 1,69  & \si{\percent} & (32)  & 1,95  & \si{\percent} & (37)  & 3,48  & \si{\percent} & (66)  & 7,11    & \si{\percent} & (135)           \\
		\textbf{\gls{fakultät7}}  & 5,32  & \si{\percent} & (101)  & 4,48  & \si{\percent} & (85)  & 6,38  & \si{\percent} & (121)  & 16,17    & \si{\percent} & (307)           \\
		\textbf{\gls{fakultät8}}  & 13,22 & \si{\percent} & (251) & 11,80  & \si{\percent} & (224) & 15,12 & \si{\percent} & (287)  & 40,15    & \si{\percent} & (762)           \\
		\textbf{\gls{fakultät9}}  & 2,00  & \si{\percent} & (38)  & 1,95  & \si{\percent} & (37)  & 3,37  & \si{\percent} & (64)  & 7,32    & \si{\percent} & (139)           \\
		\textbf{\gls{fakultät10}} & 3,16  & \si{\percent} & (60)  & 3,37  & \si{\percent} & (64)  & 4,06  & \si{\percent} & (7)  & 10,59    & \si{\percent} & (201)           \\
		\midrule
		\textbf{Summe}            & 28,98 & \si{\percent} & (550) & 29,35 & \si{\percent} & (557) & 41,68 & \si{\percent} & (791) & 100,00  & \si{\percent} & (1898)         \\
		\bottomrule
	\end{tabular}
}
	\label{tab:luh-repo-grundmenge-beschreibung}
\end{table}

\noindent Für eine Liste aller inkludierter Dissertationsmetadaten, siehe \fxfatal*{Fix bibliographic data}{\autocite{my-dataset}}.

\parsum{Stichprobenziehung}
Diese Liste an Dissertationsmetadaten bildete die Grundmenge für die Ziehung einer mehrschichtigen Zufallsstichprobe.
Die Schichten der Zufallsstichprobe entsprachen dabei $\text{\textit{Fakultät}}\times\text{\textit{Jahresspanne}}$ und ergaben daher insgesamt $\num{9}\times\num{3}=\num{27}$ Stichprobengruppierungen.
Für jede Stichprobengruppierung wurde durch \cref{lst:python-luh-repo-stratification} eine eigene CSV-Tabellendatei erstellt.
Auf die einzelnen Stichprobengruppierungen wurde dann jeweils das Verfahren einer einfachen Stichprobenziehung angewandt.
Bei der Auswahl der Stichproben wurde jeweils ein Konfidenzintervall von \SI{95}{\percent} und eine Fehlerspanne von \SI{5}{\percent} zugrunde gelegt.
Diese Parameter gewährleisten, dass die Ergebnisse der Stichprobe mit hoher Wahrscheinlichkeit repräsentativ für die gesamte Population sowie der einzelnen Stichprobengruppierungen sind und die Unsicherheit der Schätzungen innerhalb akzeptabler Grenzen bleibt.
Um den Prozess der Stichprobenziehung zu automatisieren und eine zufällige Auswahl zu gewährleisten, wurde eine auf Python basierende Software \autocite{Krassnig2024-csv} genutzt, welche im Rahmen dieser Arbeit geschrieben wurde.%
\footnote{%
Die Software von \citeauthor{Krassnig2024-csv} \autocite{Krassnig2024-csv} nutzt standardmäßig die Anzahl an Nanosekunden seit dem Beginn der System-Epoche (1970-01-01T00:00:00Z) als Startwert für die Zufallsfunktion.
Der genutzte Startwert wird als begleitendes Metadatum der Stichprobe abgespeichert.
Die Ziehung ist somit wiederholbar und das Datum der Ziehung verifizierbar.} 

\parsum{Stichprobenbeschreibung}
Die so gezogene Stichprobe ($n=1441$) besteht aus ca.~\SI{76}{\percent} aller Metadateneinträge der Grundmenge.
Die relative sowie die absolute Distribution aller Institutionen in der Stichprobe nach \textit{Fakultät} und \textit{Zeitraum} ist in \cref{tab:luh-repo-stichprobe-beschreibung} gegeben.
\begin{table}[!htbp]
	\caption{Die Verteilung der Stichproben-Metadateneinträge nach $\text{\textit{Fakultät}}\times\text{\textit{Zeitraum}}$ aufgegliedert.
    Absolute Werte in Klammern angegeben.}
    \resizebox{\ifdim\width>\textwidth\textwidth\else\width\fi}{!}{%
	\begin{tabular}{lS[table-format=3.2]@{\,}S[table-text-alignment = left]lS[table-format=3.2]@{\,}S[table-text-alignment = left]lS[table-format=3.2]@{\,}S[table-text-alignment = left]lS[table-format=3.2]@{\,}S[table-text-alignment = left]l}
		\toprule
		& \multicolumn{3}{c}{\textbf{2012--2015}} & \multicolumn{3}{c}{\textbf{2016--2019}} & \multicolumn{3}{c}{\textbf{2020--2023}} & \multicolumn{3}{c}{\textbf{Summe}}    \\
		\midrule
		\textbf{\gls{fakultät2}}  & 0,90  & \si{\percent} & (13)  & 1,53  & \si{\percent} & (22)  & 1,73  & \si{\percent} & (25) & 4,16   & \si{\percent} & (60)         \\
		\textbf{\gls{fakultät3}}  & 1,32  & \si{\percent} & (19)  & 2,22  & \si{\percent} & (32)  & 4,09  & \si{\percent} & (59)  & 7,63    & \si{\percent} & (110)          \\
		\textbf{\gls{fakultät4}}  & 2,29  & \si{\percent} & (33)  & 3,12  & \si{\percent} & (45)  & 4,65  & \si{\percent} & (67)  & 10,06   & \si{\percent} & (145)          \\
		\textbf{\gls{fakultät5}}  & 0,00  & \si{\percent} & (0)   & 0,21  & \si{\percent} & (3)   & 0,07  & \si{\percent} & (1)  & 0,28    & \si{\percent} & (4)           \\
		\textbf{\gls{fakultät6}}  & 2,08  & \si{\percent} & (30)  & 2,36  & \si{\percent} & (34)  & 3,96  & \si{\percent} & (57)  & 8,40    & \si{\percent} & (121)           \\
		\textbf{\gls{fakultät7}}  & 5,62  & \si{\percent} & (81)  & 4,86  & \si{\percent} & (70)  & 6,45  & \si{\percent} & (93)  & 16,93    & \si{\percent} & (244)           \\
		\textbf{\gls{fakultät8}}  & 10,62 & \si{\percent} & (153) & 9,85  & \si{\percent} & (142) & 11,45 & \si{\percent} & (165)  & 31,92    & \si{\percent} & (460)           \\
		\textbf{\gls{fakultät9}}  & 2,43  & \si{\percent} & (35)  & 2,36  & \si{\percent} & (34)  & 3,82  & \si{\percent} & (55)  & 8,61    & \si{\percent} & (124)           \\
		\textbf{\gls{fakultät10}} & 3,68  & \si{\percent} & (53)  & 3,82  & \si{\percent} & (55)  & 4,51  & \si{\percent} & (65)  & 12,01    & \si{\percent} & (173)           \\
		\midrule
		\textbf{Summe}            & 28,94 & \si{\percent} & (417) & 30,33 & \si{\percent} & (437) & 40,74 & \si{\percent} & (587) & 100,00  & \si{\percent} & (1441)         \\
		\bottomrule
	\end{tabular}
}
	\label{tab:luh-repo-stichprobe-beschreibung}
\end{table}
Der jeweils relative Anteil der Stichprobengruppierungen zu dem entsprechenden Datensatz aus der Grundmenge sowie die Differenz zwischen der respektiven Anzahl ist, auch nach \textit{Fakultät} und \textit{Zeitraum} aufgegliedert, in \cref{tab:luh-repo-stichprobe-beschreibung-relativ} gegeben.
\begin{table}[!htbp]
	\caption{Die Stichproben-Metadateneinträge nach $\text{\textit{Fakultät}}\times\text{\textit{Zeitraum}}$ aufgegliedert relativ zu der Anzahl an Metadateneinträgen aus der Grundmenge.
    Absolute Differenzwerte in Klammern angegeben.}
    \resizebox{\ifdim\width>\textwidth\textwidth\else\width\fi}{!}{%
	\begin{tabular}{lS[table-format=3.2]@{\,}S[table-text-alignment = left]lS[table-format=3.2]@{\,}S[table-text-alignment = left]lS[table-format=3.2]@{\,}S[table-text-alignment = left]lS[table-format=3.2]@{\,}S[table-text-alignment = left]l}
		\toprule
		& \multicolumn{3}{c}{\textbf{2012--2015}} & \multicolumn{3}{c}{\textbf{2016--2019}} & \multicolumn{3}{c}{\textbf{2020--2023}} & \multicolumn{3}{c}{\textbf{Alle}}    \\
		\midrule
		\textbf{\gls{fakultät2}}  & 100,00  & \si{\percent} & (0)  & 95,65  & \si{\percent} & (-1)  & 96,15  & \si{\percent} & (-1) & 96,77   & \si{\percent} & (-2)         \\
		\textbf{\gls{fakultät3}}  & 100,00  & \si{\percent} & (0)  & 94,12  & \si{\percent} & (-2)  & 85,51  & \si{\percent} & (-10)  & 90,16    & \si{\percent} & (-12)          \\
		\textbf{\gls{fakultät4}}  & 91,67  & \si{\percent} & (-3)  & 90,00  & \si{\percent} & (-5)  & 83,75  & \si{\percent} & (-13)  & 87,35   & \si{\percent} & (-21)          \\
		\textbf{\gls{fakultät5}}  & \multicolumn{1}{r}{---}  &  & (0)   & 100,00  & \si{\percent} & (0)   & 100,00  & \si{\percent} & (0)  & 100,00    & \si{\percent} & (0)           \\
		\textbf{\gls{fakultät6}}  & 93,75  & \si{\percent} & (-2)  & 91,89  & \si{\percent} & (-3)  & 86,36  & \si{\percent} & (-9)  & 89,63    & \si{\percent} & (-14)           \\
		\textbf{\gls{fakultät7}}  & 80,20  & \si{\percent} & (-20)  & 82,35  & \si{\percent} & (-15)  & 76,86  & \si{\percent} & (-28)  & 79,48    & \si{\percent} & (-63)           \\
		\textbf{\gls{fakultät8}}  & 60,96 & \si{\percent} & (-98) & 63,39  & \si{\percent} & (-82) & 57,49 & \si{\percent} & (-122)  & 60,37    & \si{\percent} & (-302)           \\
		\textbf{\gls{fakultät9}}  & 92,11  & \si{\percent} & (-3)  & 91,89  & \si{\percent} & (3)  & 85,94  & \si{\percent} & (-9)  & 89,21    & \si{\percent} & (-15)           \\
		\textbf{\gls{fakultät10}} & 88,33  & \si{\percent} & (-7)  & 85,94  & \si{\percent} & (-9)  & 84,42  & \si{\percent} & (-12)  & 86,07    & \si{\percent} & (-28)           \\
		\midrule
		\textbf{Alle}            & 75,82 & \si{\percent} & (-133) & 78,46 & \si{\percent} & (-120) & 74,21 & \si{\percent} & (587) & 75,92  & \si{\percent} & (-457)         \\
		\bottomrule
	\end{tabular}
}
	\label{tab:luh-repo-stichprobe-beschreibung-relativ}
\end{table}

\parsum{Dateisammlung}
Für die Evaluation, inwiefern die Dissertationen der Stichprobe \glspl{forschungsdaten} beinhalten oder auf solche verweisen wurden alle Dateien, die mit Metadateneinträgen assoziiert werden, heruntergeladen.
Dieser Prozess wurde dadurch verkompliziert, dass DSpace~5, auf welches das \gls{luh-repo} zum Zeitpunkt dieser Arbeit noch basierte, keine eingebaute Möglichkeit anbietet, alle Dateien einer Sammlung (jenseits einer schnell erreichten Grenze) oder einer bestimmten Metadatenliste herunterzuladen:
weder intern mit administrativen Rechten noch extern durch die Nutzung einer Schnittstelle.
Daher wurde im Rahmen dieser Arbeit \cref{lst:simple-dspace5-downloader} entwickelt, welches alle Links zu den entsprechenden Dateien aus dem öffentlichen Quellcode der Webseiten extrahiert und automatisch herunterlädt und nach dem Metadaten-Handle sortiert \autocite{Krassnig2024-dspace}.

Hierbei wurden insgesamt \num{1480} Dateien zur Weiterverarbeitung gefunden, heruntergeladen und sortiert.

\parsum{Zahlenspiegel der \gls{luh}}
Zusätzlich zu den oben aufgeführten Dissertationen wurden auch die von der \gls{luh} veröffentlichten Zahlenspiegel für den zu untersuchenden Zeitraum gesammelt, da diese die jährliche Gesamtzahl veröffentlichter Dissertationen enthalten.
Hierbei wurden die Zahlenspiegel von 2013--2023 ausgesucht, da diese jeweils das vorherige Jahr betreffen \autocite{Zahlenspiegel2013,Zahlenspiegel2014,Zahlenspiegel2015,Zahlenspiegel2016,Zahlenspiegel2017,Zahlenspiegel2018,Zahlenspiegel2019,Zahlenspiegel2020,Zahlenspiegel2021,Zahlenspiegel2022,Zahlenspiegel2023}.
Die aktuellen Daten für 2023, bzw. der Zahlenspiegel aus 2024 wurde zum Zeitpunkt dieser Arbeit noch nicht veröffentlicht.

\subsection{Methoden}\label{sec:luh-repo-methods}
\parsum{Klassifikation}
Die in \cref{sec:luh-repo-material} gesammelten Dissertationsdateien wurden dann, um die zentrale Forschungsfrage dieses Kapitels zu beantworten, nach ihrem Inhalt klassifiziert, ob und auf welche Arte und Weise sie primäre \glspl{forschungsdaten} beinhalten:
\glspl{forschungsdaten} konnten entweder in die PDF-Datei integriert, als Begleitdaten im \gls{luh-repo} eingereicht worden oder auf ein externes \gls{forschungsdaten}-Repositorium hochgeladen worden sein.
Damit diese Klassifikation stattfinden konnte, musste jedoch zuerst bestimmt werden, welcher Inhalt als \gls{forschungsdaten} gewertet wird (hierbei orientierte sich diese Arbeit an \autocite{dfg-richtlinie,Simukovic2014InterviewFD}) und wo sich dieser typischerweise im Dokument befindet.
Hierfür wurden von jeder Fakultät, gleichmäßig auf die drei Zeiträume aufgeteilt, zwölf zufällige Dissertationen ausgewählt und vorläufig evaluiert.
Bei Stichprobengruppierungen von weniger als vier Dissertationen wurden stattdessen alle Dokumente vorläufig ausgewertet.

\parsum{Klassifikationshierarchie}
Bei dieser Auswertung wurde ein provisorisches Klassifikationssystem aufgebaut, welches für den Rest der Arbeit beibehalten wurde.
Hierbei wurden \glspl{forschungsdaten} in drei hierarchische Stufen eingeteilt.
Diese reichen von \textit{Stufe 1}, welche eindeutige und zweifelsfreie Primärdaten beinhalten, zu \textit{Stufe 3}, welche Daten beinhalten, die entweder kompromittiert worden sind (z.B.~durch starke Kompression), keine besondere Leistung darstellen (z.B.~einfache Fragebögen) oder durch den Autor kaum auf Originalität überprüfbar waren.
Unter \textit{Stufe 2} befinden sich jene \glspl{forschungsdaten}, welche zwar originell sind, jedoch weniger direkte Wiederverwendbarkeit oder Qualität im Vergleich zu \glspl{forschungsdaten} aus \textit{Stufe 1} haben.
Es folgt eine Auflistung der verschiedenen \gls{forschungsdaten}-Klassifikationen der entsprechenden Klassifikationsstufen.\\
\textbf{Stufe 1:} rohe Beobachtungs-~/~Messdaten, unkomprimierte Rohbilder, Videos, Skripte~/~Software, Transkriptionen von Interviews, (anonymisierte) Beantwortungen von Fragebögen\\
\textbf{Stufe 2:} Pseudocode, Algorithmen, komprimierte Bilder von Gelfärbungen\footnote{Für komprimierte Bilder von Gelfärbungen wurde auf Anraten einer Wissenschaftlerin aus dem Bereich \textit{Life Science} eine Ausnahme gemacht und als \gls{forschungsdaten} der zweiten Stufe kategorisiert \autocite{SarahPC}.}\\
\textbf{Stufe 3:} komprimierte Bilder, Spektraldiagramme, Gensequenzen, Fragebögen, Leitfäden, Montagezeichnungen

\parsum{Ort der \glspl{forschungsdaten}}
Bei der vorlläufigen Testklassifikation konnten keine bestimmten Teile eines Dokumentes vollständig ausgeschlossen werden.
Während die meisten hochstufigen \glspl{forschungsdaten} in dem jeweiligen Appendix zu finden waren, waren z.B.~\glspl{forschungsdaten} der zweiten und dritten Stufe zu großen Teilen im gesamten Dokument verteilt.
Auch \glspl{forschungsdaten} aus \textit{Stufe 1} waren teilweise in anderen Bereichen der Dissertationen zu finden.
Dies galt auch für externe Forschungsdaten.
Hier wurde teilweise in der Präambel darauf hingewiesen, dass einige oder alle \glspl{forschungsdaten} auf ein externes Repositorium hochgeladen wurde.
Teilweise wurden diese externen Datensätze aber auch erst an der jeweils relevanten Stelle im Hauptteil des Dokumentes zitiert.
Für den restlichen Verlauf der Klassifikation wurde daher beschlossen, dass sämtliche Seiten der PDF-Dateien zumindest kurzzeitig begutachtet werden müssen.

\parsum{Klassifikationsstrategie}
Nach Abschluss der vorläufigen Testklassifikation und Aufbau des hierarchischen Klassifikationssystems wurde dann wie folgt vorgegangen.
Es wurden alle PDF-Dateien einzeln evaluiert.
Insofern ein Typ an \gls{forschungsdaten} im Dokument gefunden wurde, so wurde die \gls{forschungsdaten}-Art und Publikationsart des \gls{forschungsdaten} in der zur Stichprobengruppierungen zugehörigen CSV-Tabellendatei vermerkt.
Hierbei wurde für interne und beigefügte \glspl{forschungsdaten} nur der Typ und nicht die Seite innerhalb des Dokumentes vermerkt.
Für extern publizierte \glspl{forschungsdaten} wurden zusätzlich noch die jeweilig dazugehörigen Seiten und Art des externen Repositoriums vermerkt (z.B.~Git-Repositorium oder dediziertes \gls{forschungsdaten}-Repositorium).
Bei externen \glspl{forschungsdaten} wurden zusätzlich die \gls{doi}, das relevante Stichwort oder die dazugehörige Domäne notiert und in einer separaten Datei eingetragen.
Diese Liste wurde dann am Ende der Evaluation genutzt, um \cref{lst:luh-repo-document-search} zu erstellen.
Dieses Skript durchsuchte automatisch den Text aller Textdatei mit den notierten Wörtern, um etwaige übersehene externe \glspl{forschungsdaten} im Nachhinein noch erfassen zu können.

\parsum{Auswertung der Ergebnisse}
Nach der Klassifizierung aller Dateien wurden die vollständig evaluierten CSV-Dateien der Stichprobengruppierungen durch \cref{lst:bash-luh-repo-csv-combiner} einerseits in kombinierte Fakultätstabellen und andererseits in eine Gesamttabelle zusammengeführt.
Die vorhandenen Metadaten wurden dann auf plausible Faktoren untersucht, die einen etwaigen Einfluss auf die Präsenz, Art und Publikationsform von \glspl{forschungsdaten}.
Für die Kreuzprodukte aller vermuteter Faktoren sowie für die Ergebnisse der Klassifikationsarbeit aller \gls{forschungsdaten}-Publikationsarten wurden Chi-Quadrat-Tests für Unabhängigkeit durchgeführt, um zu überprüfen, ob statistisch signifikante Relationen zwischen den jeweiligen Faktoren bzw.~Ergebnissen besteht.
Hierbei wurden für alle zu überprüfenden Relationen die Nullhypothese angenommen.
Die Nullhypothese gilt als widerlegt wenn der respektive Chi-Quadrat-Test für Unabhängigkeit einen Signifikanzwert von $p<0,05$ erzeugt.
Bei Signifikanzwerten von $p\geqslant0,05$ gilt die Nullhypothese als bestätigt.
Die zu jeweils zu überprüfende Nullhypothese wird an der jeweiligen Stelle in \cref{sec:luh-repo-results} formuliert.
Da p-Werte nichts über die Stärke einer Abhängigkeit aussagen, wurden für alle Testergebnisse mit $p<0,05$ zusätzlich noch der respektive Cramérs V-Wert ($\phi_C$) berechnet, um zu überprüfen, wie stark die statistisch signifikante Abhängigkeit ist.
Bei einem Cramérs V-Wert von $\phi_C>\num{0,1}$ ist von einem schwachem, bei $\phi_C>\num{0,3}$ von einem moderaten und bei $\phi_C>\num{0,5}$ von einem starken Zusammenhang bzw.~Einfluss auszugehen.

\parsum{Zahlenspiegel der \gls{luh}}
Die in \cref{sec:luh-repo-material} gesammelten Zahlenspiegel der \gls{luh} wurden dann wie folgt ausgewertet.
Es wurden alle Gesamtanzahlen der Dissertationen pro Fakultät für jeden Jahrgang extrahiert und in eine gemeinsame Tabelle eingetragen, wo sie in die Jahresgruppen 2012--2015, 2016--2019 und 2020--2023 geklumpt wurden.
Da die offiziellen Zahlen des Jahres 2023 zum Zeitpunkt dieser Arbeit noch nicht veröffentlicht waren, wurden diese durch das arithmetische Mittel der Fakultätswerte aus den Jahren 2012--2022 simuliert, um Werte vergleichbare Werte zu derivieren.
Die resultierende Verteilung ist in \cref{tab:luh-repo-zahlenspiegel-summary} zu sehen.
\begin{table}[!htbp]
	\caption{Die Verteilung der Dissertationen laut den Zahlenspiegeln der \gls{luh} nach $\text{\textit{Fakultät}}\times\text{\textit{Zeitraum}}$ aufgegliedert.
    Absolute Werte in Klammern angegeben.}
    \resizebox{\ifdim\width>\textwidth\textwidth\else\width\fi}{!}{%
	\begin{tabular}{lS[table-format=3.2]@{\,}S[table-text-alignment = left]lS[table-format=3.2]@{\,}S[table-text-alignment = left]lS[table-format=3.2]@{\,}S[table-text-alignment = left]lS[table-format=3.2]@{\,}S[table-text-alignment = left]l}
		\toprule
		& \multicolumn{3}{c}{\textbf{2012--2015}} & \multicolumn{3}{c}{\textbf{2016--2019}} & \multicolumn{3}{c}{\textbf{2020--2023}} & \multicolumn{3}{c}{\textbf{Summe}}    \\
		\midrule
		\textbf{\gls{fakultät2}}  & 1,04  & \si{\percent} & (40)  & 0,88  & \si{\percent} & (22)  & 1,73  & \si{\percent} & (25) & 4,16   & \si{\percent} & (60)         \\
		\textbf{\gls{fakultät3}}  & 2,18  & \si{\percent} & (84)  & 3,13  & \si{\percent} & (32)  & 4,09  & \si{\percent} & (59)  & 7,63    & \si{\percent} & (110)          \\
		\textbf{\gls{fakultät4}}  & 3,42  & \si{\percent} & (132)  & 3,60  & \si{\percent} & (45)  & 4,65  & \si{\percent} & (67)  & 10,06   & \si{\percent} & (145)          \\
		\textbf{\gls{fakultät5}}  & 2,62  & \si{\percent} & (101)   & 1,66  & \si{\percent} & (3)   & 0,07  & \si{\percent} & (1)  & 0,28    & \si{\percent} & (4)           \\
		\textbf{\gls{fakultät6}}  & 5,70  & \si{\percent} & (220)  & 6,84  & \si{\percent} & (34)  & 3,96  & \si{\percent} & (57)  & 8,40    & \si{\percent} & (121)           \\
		\textbf{\gls{fakultät7}}  & 5,02  & \si{\percent} & (194)  & 4,04  & \si{\percent} & (70)  & 6,45  & \si{\percent} & (93)  & 16,93    & \si{\percent} & (244)           \\
		\textbf{\gls{fakultät8}}  & 9,94 & \si{\percent} & (384) & 9,53  & \si{\percent} & (142) & 11,45 & \si{\percent} & (165)  & 31,92    & \si{\percent} & (460)           \\
		\textbf{\gls{fakultät9}}  & 3,68  & \si{\percent} & (142)  & 4,09  & \si{\percent} & (34)  & 3,82  & \si{\percent} & (55)  & 8,61    & \si{\percent} & (124)           \\
		\textbf{\gls{fakultät10}} & 2,77  & \si{\percent} & (107)  & 3,00  & \si{\percent} & (55)  & 4,51  & \si{\percent} & (65)  & 12,01    & \si{\percent} & (173)           \\
		\midrule
		\textbf{Summe}            & 36,35 & \si{\percent} & (1404) & 36,77 & \si{\percent} & (437) & 40,74 & \si{\percent} & (587) & 100,00  & \si{\percent} & (1441)         \\
		\bottomrule
	\end{tabular}
}
	\label{tab:luh-repo-zahlenspiegel-summary}
\end{table}
Die $\text{\textit{Fakultät}}\times\text{\textit{Jahresgruppe}}$-Zahlen wurden dann mit den Daten aus dem \gls{luh-repo} verglichen.
Aus diesem Vergleich wurde dann jeweils abgeleitet, zu welchem Anteil die Promovierenden der verschiedenen Fakultäten das \gls{luh-repo} nutzen.

\section{Resultate}\label{sec:luh-repo-results}
In diesem Abschnitt werden die Klassifizierungsresultate und die statistische Auswertung jener Resultate dargestellt.
In \cref{sec:luh-repo-results-general} werden die Resultate und die statistische Auswertung der Dissertationen relativ zu der gesamten Stichpobe aufgelistet.
In \cref{sec:luh-repo-results-specific} werden die Resultate und die statistische Auswertung der Dissertationen relativ zu den einzelnen Fakultäten aufgelistet.

\subsection{Allgemeine Resultate}\label{sec:luh-repo-results-general}
\subsection{Fakultätsspezifische Resultate}\label{sec:luh-repo-results-specific}
\subsubsection{\gls{fakultät2}}
\subsubsection{\gls{fakultät3}}
\subsubsection{\gls{fakultät4}}
\subsubsection{\gls{fakultät5}}
\subsubsection{\gls{fakultät6}}
\subsubsection{\gls{fakultät7}}
\subsubsection{\gls{fakultät8}}
\subsubsection{\gls{fakultät9}}
\subsubsection{\gls{fakultät10}}

\section{Diskussion}\label{sec:luh-repo-discussion}
\subsection{Allgemeingültige Dokumente}\label{sec:luh-repo-discussion-general}
\subsection{Promotionsspezifische Dokumente}\label{sec:luh-repo-discussion-specific}

