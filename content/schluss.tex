\chapter{Schlussfolgerungen}\label{ch:schlussfolgerungen}\glsunset{fair}
\parsum{Forschungs\-fragen}
Die einzelnen Forschungsfragen dieser Abschlussarbeit lauteten, wie in \cref{ch:einfuehrung} aufgelistet, wie folgt:
\begin{enumerate}
    \item Inwiefern wird der Umgang mit \gls{forschungsdaten} für Promotionsvorhaben in Deutschland bereits in verbindlichen verwaltungsrechtliche Dokumente geregelt?
    \begin{enumerate}
        \item Inwiefern wird der Umgang mit \gls{forschungsdaten} in den allgemeinen Richtlinien einer Institution geregelt, welche auch Promovierende und ihr Forschungvorhaben betreffen?
        \item Inwiefern wird der Umgang mit \gls{forschungsdaten} in Promotions- und Prüfungsordnungen geregelt?
    \end{enumerate}
    \item Auf welche Art und Weise haben Promovierende, welche ihre Dissertationen 2012--2023 in einem instutionellen Repositorium publiziert haben, ihre \gls{forschungsdaten} in ihre Publikation integriert?
    \begin{enumerate}
        \item Für welchen Anteil an Dissertationen wurden \gls{forschungsdaten} in welcher Form nach dem Klassifkationssystem von \citeauthor{ReillyEtAl2011} \autocite{ReillyEtAl2011} veröffentlicht?
        \item Inwiefern hat sich das Publikationsverhalten zu \gls{forschungsdaten} in Dissertationen in den letzten zwölf Jahren verändert?
        \item Inwiefern unterscheiden sich die verschiedenen wissenschaftlichen Fachrichtungen in ihrem Publikationsverhalten in Bezug auf \gls{forschungsdaten} aus Dissertationen?
        \item Wie wird in den Metadaten von \gls{forschungsdaten} aus Dissertationen sichtbar gemacht, dass es eine dazugehörige Dissertation gibt?
        \item Wie wird in den Metadaten von Dissertationen sichtbar gemacht, dass es dazugehörige \gls{forschungsdaten} gibt?
    \end{enumerate}
\end{enumerate}
Der Kontext der zweiten Forschungsfrage beschränkte sich dabei, aus Gründen der Relevanz und des Umfangs dieser Abschlussarbeit, jeweils auf Dissertationen und auf das \gls{luh-repo}.

\parsum{Erste Forschungsfrage}
Die erste Forschungsfrage, wie auch ihre spezifischeren Unterforschungsfragen, wurde in \cref{ch:richtlinien} behandelt.

\parsum{Allgemeine Richtlinien}
Für die erste Unterfrage, die sich auf allgemeine Richtlinien bezieht, wurde durch eine quantitative Auswertung einer repräsentativen Stichprobe ($n=\num{115}$) aller promotionsberechtigter Institutionen in Deutschland ($n=\num{163}$) gezeigt, dass \SI{44.35}{\percent} ($n=\num{51}$) der Institutionen eine dedizierte \glspl{forschungsdaten}-Richtlinie besitzen.
Weitere \SI{43.48}{\percent} ($n=\num{50}$) besitzen hierbei zumindest eine Verpflichtung zum Einhalten der Regeln der \gls{gwp}.
Hierbei wurden jeweils statistisch signifikante Abhängigkeiten zwischen der Trägerschaft und dem Typ der Forschungsinstitution zu der Richtlinien-Klassifikationsstufe der Institution festgestellt, jedoch nicht mit dem Bundesland, in dem sich die Institution befindet.
Diese Ergebnisse aus \cref{ch:richtlinien} sind konform mit den bisher bekannten Daten zu \gls{forschungsdaten}-Richtlinien in Deutschland, wie sie in \cref{ch:forschungsstand} zusammengefasst wurden.
Zusätzlich zeigen die Ergebnisse auf, dass, zumindest bisher, keine Homogenisierung der \gls{forschungsdaten}-Richtlinien-Bezeichnungen in Deutschland stattgefunden hat.

\parsum{Spezifische Richtlinien}
Für die zweite Unterfrage, welche sich auf promotionsspezifische Dokumente bezieht, wurde gezeigt, dass nur \SI{13.91}{\percent} ($n=\num{16}$) aller Stichproben-Institutionen mindestens eine Promotions- oder Prüfungsordnung besitzen, die verpflichtende Richtlinien zum Umgang mit \glspl{forschungsdaten} enhält.
Weitere \SI{53.04}{\percent} ($n=\num{61}$) enthalten jedoch verpflichtende Richtlinien zum Einhalten der \gls{gwp}.
Es wurden statistisch signifikante Abhängigkeiten zwischen der Richtlinien-Klassifikationsstufe für promotionsspezifische Dokumente der Institution und dem Bundesland, dem Institutionstyp und der höchsten Klassifikationsstufe für allgemeine Richtlinien der Institution gefunden.

\parsum{Antwort}
Hiermit konnte die erste Forschungsfrage quantitativ beantwortet werden.
In Deutschland haben fast die Hälfte aller promotionsberechtigter Institutionen eine allgemeine (und verpflichtende) \gls{forschungsdaten}-Richtlinie, während nur die Wenigsten auch \glspl{forschungsdaten} oder \gls{fdm} in den Promotions- bzw. Prüfungsordnungen erwähnen.
Für Details zu den einzelnen Ergebnissen und Verhältnissen wird auf \cref{sec:policy-results,sec:policy-discussion} verwiesen.

\parsum{Zweite Forschungsfrage}
Die zweite Forschungsfrage, wie auch ihre spezifischeren Unterforschungsfragen, wurde in \cref{ch:luh-repo} behandelt.

\parsum{Publikationsarten}
Für die erste Unterfrage, wurde gezeigt, dass ca.~\SI{61,34}{\percent} ($n=\num{768}$) aller \glspl{pdd} ($n=\num{1252}$) aus der mehrschichtigen Stichprobe ($n=\num{1441}$) zumindest einen Teil ihrer \glspl{forschungsdaten} veröffentlicht haben.
Von den \glspl{pdd} haben hierbei \SI{57,03}{\percent} ($n=\num{714}$) zumindest einen Teil ihrer \glspl{forschungsdaten} in der Dissertation integriert, \SI{1,76}{\percent} ($n=\num{22}$) als begleitende Dateien und \SI{7,43}{\percent} ($n=\num{93}$) auf einer externen Plattform veröffentlicht.

\parsum{Zeitliche Entwicklung}
Bei der zweiten Unterfrage, welche die zeitliche Entwicklung betrifft, konnte festgestellt werden, dass sich das \gls{forschungsdaten}-Publikationsverhalten über einen Zeitraum von zwölf Jahren statistisch signifikant geändert hat und vor allem der Anteil an externen Publikationen in den letzten vier Jahren stark zugenommen hat.

\parsum{Fakultäten}
Für die dritte Unterfrage, welche die Fakultäten betrifft, konnte festgestellt werden, dass es statistisch signifikante Unterschiede zwischen den Fakultäten in ihrem \gls{forschungsdaten}-Publikationsverhalten gibt.
So zeichnen sich insbesondere \gls{fakultät4}, \gls{fakultät3} und \gls{fakultät8} dadurch aus, einen höheren Anteil an externen Publikationen zu besitzen, als die anderen Fakultäten.
Ein weiterer interessanter Unterschied zwischen den Fakultäten ist auch, dass diese zu unterschiedlich großen Anteilen das \gls{luh-repo} nutzen, um Dissertationen zu veröffentlichen.
So lassen sich die Fakultäten hierfür in drei Gruppen einteilen:
Geringnutzer (\gls{fakultät5} und \gls{fakultät6}), Intermediärnutzer (\gls{fakultät3},\gls{fakultät4} und \gls{fakultät9}) und Intensivnutzer (\gls{fakultät2}, \gls{fakultät7}, \gls{fakultät8} und \gls{fakultät10}).
\pagebreak

\parsum{Metadaten}
Für die vierte und fünfte Unterfrage, welche die Metadaten betreffen, wurde festgestellt, dass keine der \glspl{pdd} auf der externen \glspl{forschungsdaten}-Plattform in den dortigen Metadaten festhält, dass die \glspl{forschungsdaten} mit einer Dissertation verknüpft sein sollten.
Dies ist aber, zumindest teilweise, dadurch bedingt, dass die wenigsten externen Publikationen auf dedizierten \gls{forschungsdaten}-Repositorien veröffentlicht wurden, sondern auf Plattformen, die keine Funktion für derlei Metadaten besitzen (z.B. \textit{GitHub} oder \textit{GitLab}).
Von Seiten der Dissertationen aus, wiederum, wurde nur eine allgemeine Relation genutzt, aus der nicht ersichtlich ist, dass es sich dabei um \glspl{forschungsdaten} handelt---stattdessen wird diese Relation hauptsächlich dafür genutzt, Artikel einer kumulativen Dissertation zu referenzieren.

\parsum{Antwort}
Somit konnte auch die zweite Forschungsfrage weitestgehend quantitativ beantwortet werden.
Die Dissertationen aus dem \gls{luh-repo} produzieren und veröffentlichen zum Großteil mindestens einen Anteil ihrer \glspl{forschungsdaten}.
Dabei sind die meisten \glspl{forschungsdaten} aber auf eine Art und Weise veröffentlicht, die sie nur schwer für weitere Forschung wiederverwendbar macht.
Insgesamt, lässt sich zusammenfassen, dass nur die wenigsten \glspl{forschungsdaten} den\gls{fair}-Prinzipien entsprechen.
Dies deutet an, dass Promovierende in diesem Bereich vermehrt Informationen und Beratung benötigen.
Entsprechende Vorschläge, wie dies bewerktstelligt werden könnte, wurden in \cref{sec:luh-repo-discussion} gegeben.
Für Details zu den einzelnen Ergebnissen und Verhältnissen wird auf \cref{sec:luh-repo-results,sec:luh-repo-discussion} verwiesen.