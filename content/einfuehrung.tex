\chapter{Einführung}\label{ch:einfuehrung}\glsunset{fair}
\parsum{Einleitung}
Die Veröffentlichung, langfristige Verfügbarkeit sowie allgemeine Nachnutzbarkeit von\linebreak \glspl{forschungsdaten} nach den \gls{fair}-Prinzipien \autocite{Wilkinson2016} gewinnen in der Wissenschaft zunehmend an Bedeutung \autocite{TenopirEtAl2017}.
Dies wird, zumindest in Teilen, durch ein wachsendes Bewusstsein für die Bedeutung von Transparenz und Nachvollziehbarkeit wissenschaftlicher Ergebnisse und den damit einhergehend zunehmenden Forderungen nach Open Access und Open Data Initiativen getrieben.
So rückt das Thema \gls{fdm} zunehmend in den Fokus der Wissenschaft.

\parsum{Bisherige Forschung}
Entsprechend dieser stetig wachsenden Bedeutung von \gls{fdm} sind in der Bibliotheks- und Informationswissenschaft folgende Themen vermehrt Gegenstand präskriptiver und deskriptiver Forschung:
(i)~wie \glspl{forschungsdaten} langfristig gelagert und zugänglich gemacht werden können und wie diese bisher tatsächlich gelagert und zugänglich gemacht werden,
(ii)~welche Metadatenklassifikationssysteme hierzu genutzt werden sollten und welche in der Praxis bisher genutzt werden,
(iii)~welche Dateitypen für langfristige Lagerung und Zugänglichkeit geeignet sind und welche nicht,
(iv)~welche Möglichkeiten es gibt, den richtigen Umgang mit \glspl{forschungsdaten} durch die Forschenden zu fördern und
(v)~wie eine entsprechende Infrastruktur aufgebaut werden kann und welche Infrastruktur bereits existiert.

\parsum{Klassifikation}
Da die Klassifikation von \glspl{forschungsdaten} für den Aufbau und die Bearbeitung dieser Abschlussarbeit von sehr hoher Relevanz ist, ist hier spezifisch zu erwähnen, dass sich in der Literatur eine hierarchische Klassifikationsmöglichkeit für \glspl{forschungsdaten} nach Publikationsart herauskristallisiert hat.
Diese evaluiert, inwiefern die \glspl{forschungsdaten} mit dem dazugehörigen Text verknüpft bzw. in diesen integriert worden sind.
So lässt sich zwischen drei verschiedenen Publikationsformen für \glspl{forschungsdaten} unterscheiden~\autocites[S.~36ff.]{ReillyEtAl2011}:
\begin{enumerate}
    \item vollständig integrierte Daten (z.B.~Tabellen und Grafiken, die in der PDF-Datei eingebettet worden sind),
    \item beigefügte Daten (z.B.~Dateien, die gemeinsam mit der PDF-Datei der Publikation auf dem Publikationsserver hochgeladen worden sind)
    \item Daten, die separat zu der dazugehörigen Publikation auf einer Plattform für \glspl{forschungsdaten} hochgeladen werden (z.B.~fachspezifische \gls{forschungsdaten}-Repositorien)
\end{enumerate}
Für Details zu dieser Klassifikationshierarchie, siehe \cref{ch:forschungsstand}.

\parsum{Wissenslücke}
Die meiste Literatur zum Thema \gls{fdm} ist eher allgemeiner Natur oder hat ein Hauptaugenmerk auf jene \glspl{forschungsdaten}, welche mit Artikeln in wissenschaftlichen Journals produziert wurden \autocite{Piwowar2013-DataReuse}. Ein zumeist wenig beachteter Aspekt sind jene Daten, die im Rahmen des Verfassens einer Hochschulschrift generiert worden sind und daher nicht primär mit einem Journalartikel assoziiert werden.
Dies bedeutet allerdings nicht, dass zu diesem Thema eine vollständige Forschungsarmut besteht.
Jedoch sind Publikationen zu diesem Thema, zumindest bisher, meist eher präskriptiver statt deskriptiver Natur:
So geben im wissenschaftlichen Kontext z.B. präskriptive Publikationen aus dem DFG-Förderprojekt \enquote{eDissPlus} \autocite{Weisbrod2017eDissPlus, KleinebergKaden2018, Weisbrod2018} sowie die \enquote{Policy für dissertationsbezogene \glspl{forschungsdaten}} der Deutschen Nationalbibliothek \autocite{dnb2017} vermehrt Richtlinien für den Umgang mit \glspl{forschungsdaten} in Hochschulschriften.
Es fehlen bisher allerdings umfassende Studien zur Wirksamkeit bzw.
Durchsetzung dieser Richtlinien bei Studierenden und Nachwuchsforschenden (z.B.~durch entsprechende Prüfungsordnungen und Beratungsangebote zu diesem Thema durch Universitätsbibliotheken).
Hier existieren bisher größtenteils nur spezialisierte bzw. fachspezifische Untersuchungen \autocite{Wünsche2018Forschungsdaten} oder Studien, die nur integrierte und begleitende \glspl{forschungsdaten} aus dem Appendix der jeweiligen Dissertationen untersuchen \autocite{Schöpfel2015}.

\parsum{Motivation}
Dabei sind Hochschulschriften--insbesondere Dissertationen--und deren Erschließung, Publikation und Archivierung ein zentraler Bestandteil jener Dienstleistungen, die an Hochschulbibliotheken anfallen.
So sind Dissertationen, neben Habilitationsschriften, die wichtigsten Hochschulschriften, die an Forschungsinstitutionen produziert werden.
Noch dazu werden diese Schriften zunehmend in institutionellen Repositorien veröffentlicht, welche wiederum zumeist entweder von der wissenschaftlichen Bibliothek betrieben oder zumindest mitbetreut werden.
Entsprechend ist eine fachgerechte, langfristige sowie zugängliche \gls{forschungsdaten}-Infrastruktur, die sich an relevante internationale Standards hält, sowie passende Beratungsgebote für Forschende zum Thema \gls{fdm} in Dissertationen unabdingbar.
Um dies effizient und zufriedenstellend bewerkstelligen zu können, benötigen Bibliothekare jedoch ausreichende empirisch-deskriptive Daten sowie präskriptive Richtlinien, an denen sie sich orientieren und von denen sie ihr lokales Beratungs- und Infrastrukturangebot ableiten können \autocite{Martin2013Wissenschaftliche}.
Daher ist es unabdingbar, diese Lücke in der wissenschaftlichen Literatur sukzessive zu minimieren, um die Effizienz und Relevanz bibliothekarischer Arbeit in diesem Bereich zu optimieren.

\parsum{Forschungs\-fragen}
Um etwas zu diesen Bemühungen beizusteuern, stellt diese Arbeit zwei thematisch zentrale Forschungsfragen, welche sich respektiv in zwei oder fünf untergeordnete Forschungsfragen untergliedern lassen.
Hierbei konzentriert sich diese Arbeit bei den Hochschulschriften exklusiv auf \textit{Dissertationen}, da diese durch ihre wissenschaftlich hohe Relevanz und Frequenz, den wichtigsten Teil der Hochschulschriften ausmachen.
Die Forschungsfragen lauten wie folgt:
\begin{enumerate}
    \item Inwiefern wird der Umgang mit \gls{forschungsdaten} für Dissertationen in Deutschland bereits in verbindlichen verwaltungsrechtlichen Dokumenten geregelt?
    \begin{enumerate}
        \item Inwiefern wird der Umgang mit \gls{forschungsdaten} in den allgemeinen Richtlinien einer Institution geregelt, welche auch Promovierende und ihr Forschungvorhaben betreffen?
        \item Inwiefern wird der Umgang mit \gls{forschungsdaten} in Promotions- und Prüfungsordnungen geregelt?
    \end{enumerate}
    \item Auf welche Art und Weise haben Promovierende, welche ihre Dissertationen 2012--2023 in einem instutionellen Repositorium publiziert haben, ihre \gls{forschungsdaten} in ihre Publikation integriert?
    \begin{enumerate}
        \item Für welchen Anteil an Dissertationen wurden \gls{forschungsdaten} in welcher Form nach dem Klassifkationssystem von \citeauthor{ReillyEtAl2011} \autocite{ReillyEtAl2011} veröffentlicht?
        \item Inwiefern hat sich das Publikationsverhalten zu \gls{forschungsdaten} in Dissertationen in den letzten zwölf Jahren verändert?
        \item Inwiefern unterscheiden sich die verschiedenen wissenschaftlichen Fachrichtungen in ihrem Publikationsverhalten in Bezug auf \gls{forschungsdaten} aus Dissertationen?
        \item Wie wird in den Metadaten von \gls{forschungsdaten} aus Dissertationen sichtbar gemacht, dass es eine dazugehörige Dissertation gibt?
        \item Wie wird in den Metadaten von Dissertationen sichtbar gemacht, dass es dazugehörige \gls{forschungsdaten} gibt?
    \end{enumerate}
\end{enumerate}
Die erste Forschungsfrage behandelt die deutschlandweite Situation zu \gls{forschungsdaten} in Dissertationen aus verwaltungsrechtlicher Sicht, während die zweite zentrale Forschungsfrage dieser Arbeit das tatsächliche Publikationsverhalten von Promovierenden in Bezug auf \gls{forschungsdaten} in den letzten zwölf Jahren untersucht.

\parsum{Gründe der Auswahl}
Der Zeitraum 2012--2023 wurde ausgesucht, da 2012 das \glspl{forschungsdaten}-Repositoriumsverzeich\-nis \textit{re3data} gegründet wurde \autocite{Pampel2013} und die darauffolgenden Jahre von vermehrtem Interesse am korrekten Umgang mit \glspl{forschungsdaten} geprägt waren (siehe \cref{ch:forschungsstand} für Details)
Da eine umfassende und differenzierte Auswertung aller in diesem Zeitraum in Deutschland erschienenen Dissertationen im Rahmen einer Masterarbeit zur Beantwortung der zweiten Forschungsfrage nicht machbar wäre, konzentriert sich diese Arbeit explizit nur auf jene Dissertationen, die an der \gls{luh} im \gls{luh-repo} der \gls{tib} veröffentlicht wurden.
Dieses Repositorium wurde wegen zwei Gründen ausgewählt:
Der erste Grund war, dass sich die \gls{tib} zum Verfassungszeitpunkt dieser Abschlussarbeit im Rahmen eines Projektes--\textit{FoHop!}--damit beschäftigt hat, inwiefern sich \glspl{forschungsdaten} bereits im \gls{luh-repo} befinden und wie diese gekennzeichnet werden können.
Dadurch war es im Rahmen dieser Abschlussarbeit möglich, administrativen Zugriff auf das \gls{luh-repo} zu erhalten und direkt einen praktischen Zweck für die Ergebnisse dieser Arbeit zu finden.
Der zweite Grund war, dass das \gls{luh-repo} aufgrund seiner großen Anzahl an vorhandenen Dissertationen sowie wegen seines breiten fachlichen Spektrums, welches von den großen Fächern hauptsächlich nur die Medizin und darstellende Kunst vermisst, ein fast ideales repräsentatives institutionelles Repositorium darstellt.

\parsum{Ziele und Dokumentstruktur}
Diese Masterarbeit verfolgt einen mehrstufigen methodischen Ansatz zur Beantwortung der einzelnen Forschungsfragen.
Die Arbeit ist hierfür in mehrere Kapitel gegliedert, die jeweils spezifische Aspekte der Forschungsfragen untersuchen und beantworten.
%
In \cref{ch:forschungsstand} wird zunächst der aktuelle Stand der Forschung zu \glspl{forschungsdaten} und \gls{fdm}, insbesondere in Bezug auf Dissertationen, dargestellt.
%
In \cref{ch:richtlinien} werden verwaltungsrechtliche Dokumente einer repräsentativen Stichprobe aller promotionsberechtigter Institutionen aus Deutschland analysiert.
Der Fokus liegt dabei auf Richtlinien und Anforderungen, die den Umgang mit \glspl{forschungsdaten} betreffen.
Dies umfasst allgemeine Dokumente (z.B.~\gls{forschungsdaten}-Richtlinien) und promotionsspezifische Dokumente (z.B.~Promotions- und Prüfungsordnungen).
%
In \cref{ch:luh-repo} wird das \gls{luh-repo} und die sich darin befindenden Dissertationen in Bezug auf primäre \glspl{forschungsdaten}, die während des dazugehörigen Forschungsprozesses generiert worden sind, untersucht.
Es wird geprüft, wie die Fakultäten der \gls{luh} sich im Umgang mit \glspl{forschungsdaten} unterscheiden und wie sich deren Praxis über einen Zeitraum von zwölf Jahren entwickelt hat.
Auch werden hier spezifische Handlungsempfehlungen für die \gls{luh} und die \gls{tib} in Bezug auf \glspl{forschungsdaten} und \gls{fdm} formuliert.
%
Abschließend werden in \cref{ch:schlussfolgerungen} die Erkenntnisse aus den vorherigen Kapiteln zusammengeführt und die Forschungsdaten noch einmal explizit beantwortet.
%
Der Aufbau der einzelnen Kapitel wird jeweils zu Beginn des jeweiligen Kapitels detailliert.