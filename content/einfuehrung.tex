\chapter{Einführung}\label{ch:einfuehrung}\glsunset{fair}

Die Veröffentlichung, langfristige Verfügbarkeit sowie Nachnutzbarkeit von \glspl{forschungsdaten} nach den \gls{fair}-Prinzipien (siehe \cref{sec:forschungsstand-basics-gwp-fair}) gewinnen in der Wissenschaft zunehmend an Bedeutung \fxfatal{Quellen hinzufügen}.
Dies wird, zumindest in Teilen, durch ein wachsendes Bewusstsein für die Bedeutung von Transparenz und Nachvollziehbarkeit wissenschaftlicher Ergebnisse\fxfatal{Quellen hinzufügen} und den damit einhergehend zunehmenden Forderungen nach Open Access und Open Data Initiativen, getrieben.
So rückt das Thema \gls{fdm} zunehmend in den Fokus der Wissenschaft.

\parsum{Bisherige Forschung}
Entsprechend dieser stetig wachsenden Bedeutung von \gls{fdm} sind in den Bibliotheks- und Informationswissenschaften folgende Themen hierzu vermehrt Gegenstände präskriptiver und deskriptiver Forschung:
(i)~wie \glspl{forschungsdaten} langfristig gelagert und zugänglich gemacht werden können und wie diese bisher tatsächlich gelagert und zugänglich gemacht werden,\fxfatal{Quellen hinzufügen}
(ii)~welche Metadatenklassifikationssysteme hierzu genutzt werden sollten und welche in der Praxis bisher genutzt werden,\fxfatal{Quellen hinzufügen}
(iii)~welche Dateitypen für langfristige Lagerung und Zugänglichkeit geeignet sind und welche nicht,\fxfatal{Quellen hinzufügen}
(iv)~welche Möglichkeiten es gibt, den richtigen Umgang mit \glspl{forschungsdaten} durch die Forschenden zu fördern\fxfatal{Quellen hinzufügen} und
(v)~wie eine entsprechende Infrastruktur aufgebaut werden kann und welche Infrastruktur bereits existiert.\fxfatal{Quellen hinzufügen}

\parsum{Klassifikation}
Da sie für den Aufbau und die Bearbeitung von sehr hoher Relevanz ist, ist hier spezifisch zu erwähnen, dass, als Teil der deskriptiven Forschung zu \gls{fdm}, sich für die Klassifikation von \glspl{forschungsdaten} in der Literatur eine hierarchische Klassifikationsmöglichkeiten nach Publikationsart herauskristallisiert hat.
Diese evaluiert, inwiefern die \glspl{forschungsdaten} mit dem dazugehörigen Text verknüpft bzw. in diesen integriert worden sind.
So lässt sich zwischen drei verschiedenen Publikationsformen für \glspl{forschungsdaten} unterschieden~\autocites[S.~36ff.]{ReillyEtAl2011}:
\begin{enumerate}
    \item vollständig integrierte Daten (z.B.~Tabellen und Grafiken, die in der PDF-Datei eingebettet worden sind),
    \item beigefügte Daten (z.B.~Dateien, die gemeinsam mit der PDF-Datei der Publikation auf den Publikationsserver hochgeladen worden sind)
    \item Daten, die separat zu der dazugehörigen Publikation auf einer Plattform für \gls{forschungsdaten} hochgeladen werden (z.B.~fachspezifische \gls{forschungsdaten}-Repositorien)
\end{enumerate}
Details zu dieser Klassifikationshierarchie werden in \cref{ch:forschungsstand} näher erläutert.

\parsum{Wissenslücke}
Die meiste Literatur zum Thema \gls{fdm} ist eher allgemeiner Natur oder hat ein Hauptaugenmerk auf jene \glspl{forschungsdaten}, welche mit Artikeln in wissenschaftlichen Journals produziert wurden.\fxfatal{Quellen hinzufügen} Ein zumeist weniger beachteter Aspekt sind jene Daten, die im Rahmen des Verfassens einer Hochschulschriften generiert worden sind und daher nicht primär mit einem Journalartikel assoziiert werden.
Dies bedeutet allerdings nicht, dass zu diesem Thema eine absolute Untersuchungsarmut besteht.
Jedoch sind Publikationen zu diesem Thema zumindest bisher meist eher präskriptiver statt deskriptiver Natur:
So geben im wissenschaftlichen Kontext z.B. präskriptive Publikationen aus dem DFG-Förderprojekt \enquote{eDissPlus} \autocite{Weisbrod2017eDissPlus, KleinebergKaden2018, Weisbrod2018} sowie die \enquote{Policy für dissertationsbezogene \glspl{forschungsdaten}} der Deutschen Nationalbibliothek \autocite{dnb2017} vermehrt Richtlinien für den Umgang mit \glspl{forschungsdaten} in Hochschulschriften.
Es fehlen bisher allerdings umfassende Studien zur Wirksamkeit bzw.
Durchsetzung dieser Richtlinien bei Studierenden und Nachwuchsforschenden (z.B.~durch entsprechende Prüfungsordnungen und Beratungsangebote zu diesem Thema durch Universitätsbibliotheken).
Hier existieren bisher größtenteils nur eher spezialisierte oder fachspezifische Untersuchungen.\fxfatal{Quellen hinzufügen}

\parsum{Motivation}
Dabei sind Hochschulschriften---insbesondere Promotionsvorhaben---von besonderer Relevanz für das lokale Angebot von wissenschaftlichen Bibliotheken:
Dissertationen sind, neben Habilitationsschriften, die wichtigsten Hochschulschriften, die an Institutionen von Forschenden produziert werden.
Noch dazu werden diese Schriften zunehmend in institutionellen Repositorien veröffentlicht, welche wiederum zumeist entweder von der wissenschaftlichen Bibliothek betrieben oder zumindest mitbetreut werden.
Entsprechend ist eine fachgerechte, langfristige sowie zugängliche \gls{forschungsdaten}-Infrastruktur, die sich an relevante internationale Standards hält, sowie passende Beratungsgebote für Forschende zum Thema \gls{fdm} unabdingbar.
Um dies jedoch effizient bewerkstelligen zu können, benötigen Bibliothekare ausreichende empirisch-deskriptive Daten sowie präskriptive Richtlinien, an denen sie sich orientieren und von denen sie ihr lokales Beratungs- und Infrastrukturangebot ableiten können.
Daher ist es unabdingbar, diese Lücke in der wissenschaftlichen Literatur sukzessive zu minimieren, um die Effizienz und Relevanz bibliothekarischer Arbeit in diesem Bereich zu optimieren.

\parsum{Forschungsfragen}
Um dieser sukzessiven Bemühung beizusteuern, stellt diese Arbeit zwei thematisch zentrale Forschungsfragen, welche sich jeweils in zwei und fünf untergeordnete Forschungsfragen aufgliedern lassen.
Hierbei konzentriert sich diese Arbeit speziell auf das Unterthema Dissertationen und Promotionsvorhaben, da diese durch ihre wissenschaftlich hohe Relevanz und Frequenz, den wichtigsten Teil der Hochschulschriften ausmachen.
Die Forschungsfragen lauten wie folgt:
\begin{enumerate}
    \item Inwiefern wird der Umgang mit \gls{forschungsdaten} für Promotionsvorhaben in Deutschland bereits in verbindlichen verwaltungsrechtliche Dokumente geregelt?
    \begin{enumerate}
        \item Inwiefern wird der Umgang mit \gls{forschungsdaten} in den allgemeinen Richtlinien einer Institution geregelt, welche auch Promovierende und ihr Forschungvorhaben betreffen?
        \item Inwiefern wird der Umgang mit \gls{forschungsdaten} in Promotions- und Prüfungsordnungen geregelt?
    \end{enumerate}
    \item Auf welche Art und Weise haben Promovierende, welche ihre Dissertationen 2012--2023 in einem instutionellen Repositorium publiziert haben, ihre \gls{forschungsdaten} in ihre Publikation integriert?
    \begin{enumerate}
        \item Für welchen Anteil an Dissertationen wurden \gls{forschungsdaten} in welcher Form nach dem Klassifkationssystem von \citeauthor{ReillyEtAl2011} \autocite{ReillyEtAl2011} veröffentlicht?
        \item Inwiefern hat sich das Publikationsverhalten zu \gls{forschungsdaten} in Dissertationen in den letzten zwölf Jahren verändert?
        \item Inwiefern unterscheiden sich die verschiedenen wissenschaftlichen Fachrichtungen in ihrem Publikationsverhalten in Bezug auf \gls{forschungsdaten} aus Dissertationen?
        \item Wie wird in den Metadaten von \gls{forschungsdaten} aus Dissertationen sichtbar gemacht, dass es eine dazugehörige Dissertation gibt?
        \item Wie wird in den Metadaten von Dissertationen sichtbar gemacht, dass es dazugehörige \gls{forschungsdaten} gibt?
    \end{enumerate}
\end{enumerate}
Die erste Forschungsfrage behandelt die deutschlandweite Situation zu \gls{forschungsdaten} in Promotionsvorhaben aus verwaltungsrechtlicher Sicht, während die zweite zentrale Forschungsfrage dieser Arbeit das tatsächliche Publikationsverhalten von Promovierenden in Bezug auf \gls{forschungsdaten} in den letzten zwölf Jahren untersucht.
Da zur zweiten Forschungsfrage eine umfassende und differenzierte Auswertung aller in diesem Zeitraum in Deutschland erschienenen Dissertationen im Rahmen einer Masterarbeit nicht stemmbar wäre, konzentriert sich diese Arbeit explizit nur auf jene Dissertationen, die an der \gls{luh} im \gls{luh-repo} der \gls{tib} veröffentlicht wurden.

\parsum{Ziele und Dokumentstruktur}
Diese Masterarbeit verfolgt einen mehrstufigen methodischen Ansatz zur Beantwortung der einzelnen Forschungsfragen.
Die Arbeit ist hierfür in mehrere Kapitel gegliedert, die jeweils spezifische Aspekte der Forschungsfragen untersuchen und beantworten.
%
In \cref{ch:forschungsstand} wird zunächst der aktuelle Stand der Forschung zu \glspl{forschungsdaten} und \glspl{fdm}, insbesondere in Bezug auf Promotionsvorhaben, dargestellt.
Dieses Kapitel beleuchtet dabei die bibliothekarische Klassifikation von \glspl{forschungsdaten} sowie die Veröffentlichungsraten und -arten in verschiedenen wissenschaftlichen Disziplinen.
%
\cref{ch:richtlinien} analysiert verwaltungsrechtliche Dokumente einer repräsentativen Stichprobe promotionsberechtigter Institutionen in Deutschland.
Der Fokus liegt dabei auf Richtlinien und Anforderungen, die den Umgang mit \glspl{forschungsdaten} betreffen, einschließlich allgemeiner und promotionsspezifischer Dokumente wie Promotions- und Prüfungsordnungen.
%
In \cref{ch:luh-repo} wird das \gls{luh-repo} und die sich darin befindenden Dissertationen in Bezug auf primäre \glspl{forschungsdaten}, die während des dazugehörigen Promotionsvorhabens generiert worden sind, untersucht.
Es wird geprüft, wie sich die Fakultäten im Umgang mit \glspl{forschungsdaten} unterscheiden und wie sich diese Praxis über einen Zeitraum von zwölf Jahren entwickelt hat.
Zusätzlich wird in diesem Kapitel untersucht, inwiefern die darin geleistete Arbeit sich dafür eignet, via maschinelles Lernen einen automatischen binären Klassifikator zu entwickeln.
%
Abschließend werden in \cref{ch:schlussfolgerungen} die Erkenntnisse aus den vorherigen Kapiteln zusammengeführt.
Es werden allgemeine Schlussfolgerungen zu den Forschungsfragen der Masterarbeit gezogen und spezifische Empfehlungen für die \gls{luh} und die \gls{tib} formuliert.
%
Der Aufbau der einzelnen Kapitel wird jeweils zu Beginn des jeweiligen Kapitels detailliert erläutert.