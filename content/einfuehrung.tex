\chapter{Einführung}\label{ch:einfuehrung}

\parsum{Allgemeines}\fxfatal*{GWP hinzufügen}{Die Veröffentlichung, Archivierung und die damit einhergehende langfristige Verfügbarkeit von \glspl{forschungsdaten} gewinnen in der wissenschaftlichen Gemeinschaft zunehmend an Bedeutung \fxfatal{Quellen}. Dies ist nicht nur eine Folge des wachsenden Bewusstseins für die Relevanz von Transparenz und Nachvollziehbarkeit wissenschaftlicher Ergebnisse\fxfatal{Quellen}, sondern auch ein Ergebnis zunehmender Forderungen nach Open Access und Open Data Initiativen.\fxfatal{Quellen}}

In den Bibliotheks- und Informationswissenschaften hierzu häufig untersuchte Forschungsgegenstände sind (i)~

Im wissenschaftlichen Kontext geben präskriptive Artikel aus dem DFG-Förderprojekt \enquote{eDissPlus} \autocite{Weisbrod2017eDissPlus, KleinebergKaden2018, Weisbrod2018} sowie die \enquote{Policy für dissertationsbezogene Forschungsdaten} der Deutschen Nationalbibliothek \autocite{dnb2017} vermehrt Richtlinien für den Umgang mit FD für HSS. Es fehlen bisher allerdings umfassende Studien zur Wirksamkeit bzw. Durchsetzung dieser Richtlinien bei Studierenden (z.B. durch entsprechende Prüfungsordnungen und Beratungen zu diesem Thema durch Universitätsbibliotheken). Hier existieren bisher höchstens hochspezialisierte und fachbezogene Untersuchungen.


Es gibt drei Publikationsformen für \glspl{forschungsdaten} in \glspl{hochschulschriften}: \textbf{(i)}~vollständig in HSS integrierte Daten (z.B.~Tabellen und Grafiken, die in der PDF-Datei der HSS eingebettet worden sind), \textbf{(ii)}~HSS-beigefügte Daten (z.B.~Dateien, die gemeinsam mit der PDF-Datei der Hochschulschrift auf den Publikationsserver der Hochschule hochgeladen worden sind)  und \textbf{(iii)}~auf ein separates Repositorium hochgeladene Daten, auf die innerhalb der HSS verwiesen wird \autocites[S.~5f.]{ReillyEtAl2011}.

Die zentrale Forschungsfrage der Arbeit lautet: „Auf welche Art und Weise wurden im institutionellen Repositorium der Leibniz Universität Hannover \glspl{forschungsdaten} von \glspl{hochschulschriften} bis einschließlich Dezember 2023 publiziert?“ Diese Hauptfrage wird durch mehrere untergeordnete Forschungsfragen ergänzt, die sich mit den spezifischen Methoden der \glspl{forschungsdaten}-Veröffentlichung und der Auszeichnung dieser Daten in den \glspl{hochschulschriften}-Metadaten befassen.

Zur Beantwortung dieser Fragen wird ein mehrstufiger methodischer Ansatz verfolgt. Zunächst werden die Promotionsordnungen und relevanten Richtlinien deutscher Hochschulen analysiert. Anschließend erfolgt eine manuelle Klassifikation der \glspl{hochschulschriften} im LUH-Repositorium nach \glspl{forschungsdaten}-Status. Die Ergebnisse dieser beiden Module werden ausgewertet, um Handlungsempfehlungen zu entwickeln, die zu einem besseren Umgang mit \glspl{forschungsdaten} in \glspl{hochschulschriften} führen können. Abschließend wird ein Modell trainiert, das die automatische Klassifikation von \glspl{hochschulschriften} nach ihrem \glspl{forschungsdaten}-Status ermöglichen soll.

Die vorliegende Masterarbeit zielt darauf ab, diese Lücke zu schließen, indem sie eine umfassende Untersuchung der Publikationspraktiken für \glspl{forschungsdaten} in \glspl{hochschulschriften} durchführt. Der Schwerpunkt liegt hierbei auf dem institutionellen Repositorium der Leibniz Universität Hannover (LUH-Repositorium). Durch die Analyse von \glspl{hochschulschriften}-Publikationen bis einschließlich Dezember 2023 soll ermittelt werden, welche Methoden zur Veröffentlichung von \glspl{forschungsdaten} am häufigsten verwendet werden und wie diese Daten in den Metadaten der \glspl{hochschulschriften} sichtbar gemacht werden.

\fxfatal*{Dokumentstruktur hinzufügen}{In \cref{ch:modul2} ...}