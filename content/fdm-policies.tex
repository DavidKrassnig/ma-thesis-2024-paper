\chapter{Richtlinien zu Forschungsdaten aus deutschen Promotionsvorhaben}\label{ch:richtlinien}

\parsum{Thema des Kapitels}
Dieses Kapitel behandelt die verschiedenen verwaltungsrechtlichen Dokumente wissenschaftlicher Institutionen, die ein Promotionsvorhaben in Bezug auf \gls{fdm} entweder spezifisch oder auch nur allgemein betreffen.
Es wird überprüft, inwiefern promotionsberechtigte Institutionen in Deutschland bereits derlei Richtlinien erlassen haben, in welcher Form diese existieren und welche Anforderungen diese stellen.

\parsum{Aufbau des Kapitels}
Hierfür wird in \cref{sec:policy-material-methods} aufgeführt, wie die zu untersuchenden Institutionen ausgewählt wurden, wie welche Materialien der Institutionen ausgesucht wurden und mit welchen Methoden das gesammelte Material daraufhin ausgewertet wurde.
In \cref{sec:policy-results} werden die entsprechenden Ergebnisse der Materialauswertung dargestellt.
Abschließend werden in \cref{sec:policy-discussion} die dargestellten Ergebnisse evaluiert und diskutiert.

\section{Material \&\ Methoden}\label{sec:policy-material-methods}
\parsum{Aufbau des Abschnitts}
In diesem Abschnitt wird das zu untersuchende Material in \cref{sec:policy-material} und die Methoden der Untersuchung in \cref{sec:policy-methods} dargestellt.
\subsection{Material}\label{sec:policy-material}
\parsum{Datengrundlage}
Da es in Deutschland hierzu keine offizielle und öffentlich zugängliche Liste aller Universitäten mit Promotionsrecht seitens des Bundesministeriums für Bildung gibt, wird als Datengrundlage für dieses Kapitel die von der \citeauthor{Hochschulkompass-Liste} geführte Liste aller wissenschaftlichen Institutionen aus dem tertiären Bildungsbereich in Deutschland \autocite{Hochschulkompass-Liste} genutzt.
Diese Liste wird tagesaktuell geführt, basiert auf der Selbstauskunft aller involvierten Institutionen ($n=428$), kodifiziert unter anderem welche Institutionen das Promotions- und Habilationsrecht führen, umfasst auch Institutionen, die nicht Mitglied der \citeauthor{Hochschulkompass-Liste} sind und besitzt einen \textit{de facto} wenn auch nicht \textit{de jure} Status als Datengrundlage für allgemeine Informationen zu wissenschaftlichen Institutionen aus Deutschland. Für die tagesspezifische Version der Liste, die für diese Arbeit genutzt wurde, siehe \fxfatal*{ADD LINK TO DIGITAL APPENDIX!}{FIXME!}.

\parsum{Grundmengenbeschreibung}
Um die zentrale Forschungsfrage dieses Kapitels zu beantworten, wurde diese Liste auf nur jene Institutionen gefiltert, welche das Promotionsrecht besitzen ($n=163$).
Die resultierende Liste promotionsberechtigter Institutionen besteht aus Forschungsinstutionen verschiedener Hochschultypen sowie unterschiedlicher Trägerschaften.
Von den Hochschultypen her umfasst die Liste Universitäten, \glspl{fh}, \glspl{haw}, \glspl{kh}, eine \gls{vh} sowie eine \gls{hset}.
Von den Trägerschaften her umfasst die Liste öffentlich-rechtliche, private sowie kirchliche Institutionen. Alle Institutionen sind staatlich anerkannt.
Die relative sowie die absolute Distribution aller promotionsberechtigter Institutionen in Deutschland nach Hochschultyp und Trägerschaft ist in \cref{tab:grundmenge-beschreibung-art} gegeben.
\begin{table}[!htbp]
	\caption{Die Verteilung aller promotionsberechtigter Institutionen in Deutschland nach $\text{\textit{Hochschultyp}}\times\text{\textit{Trägerschaft}}$ aufgegliedert. Absolute Werte in Klammern angegeben.}
    \resizebox{\ifdim\width>\textwidth\textwidth\else\width\fi}{!}{%
        \begin{tabular}{lS[table-format=3.2]@{\,}S[table-text-alignment = left]lS[table-format=3.2]@{\,}S[table-text-alignment = left]lS[table-format=3.2]@{\,}S[table-text-alignment = left]lS[table-format=3.2]@{\,}S[table-text-alignment = left]l}
            \toprule
            & \multicolumn{3}{c}{\textbf{Öffentlich-Rechtlich}} & \multicolumn{3}{c}{\textbf{Privat}} & \multicolumn{3}{c}{\textbf{Kirchlich}} & \multicolumn{3}{c}{\textbf{Summe}}    \\
            \midrule
            \textbf{Universität}  & 53,37 & \si{\percent} & (87)  & 7,98 & \si{\percent} & (13) & 6,13 & \si{\percent} & (10) & 67,48  & \si{\percent} & (110) \\
            \textbf{\gls{fh} / \gls{haw}}     & 6,75  & \si{\percent} & (11)  & 0,00 & \si{\percent} & (0)  & 0,00 & \si{\percent} & (0)  & 6,75   & \si{\percent} & (11)  \\
            \textbf{\gls{kh}}          & 23,93 & \si{\percent} & (39)  & 0,61 & \si{\percent} & (1)  & 0,00 & \si{\percent} & (0)  & 24,54  & \si{\percent} & (40)  \\
            \textbf{\gls{hset}}         & 0,61  & \si{\percent} & (1)   & 0,00 & \si{\percent} & (0)  & 0,00 & \si{\percent} & (0)  & 0,61   & \si{\percent} & (1)   \\
            \textbf{\gls{vh}}          & 0,61  & \si{\percent} & (1)   & 0,00 & \si{\percent} & (0)  & 0,00 & \si{\percent} & (0)  & 0,61   & \si{\percent} & (1)   \\\midrule
            \textbf{Summe}        & 85,28 & \si{\percent} & (139) & 8,59 & \si{\percent} & (14) & 6,13 & \si{\percent} & (10) & 100,00 & \si{\percent} & (163) \\
            \bottomrule
        \end{tabular}
    }
	\label{tab:grundmenge-beschreibung-art}
\end{table}

\noindent Geografisch gesehen sind, in der gefilterten Liste, zu unterschiedlich hohen Anteilen, Institutionen aus allen deutschen Bundesländern vertreten.
Die genaue Verteilung ist in \cref{fig:DE-grundmenge-beschreibung} wiedergegeben.
\begin{figure}[!htbp]
    \centering
    \begin{tikzpicture}[y=1cm, x=1cm, yscale=.8,xscale=.8, every node/.append style={scale=1}, inner sep=0pt, outer sep=0pt]
  \footnotesize
  \drawgermany
  \drawbw{colorblindC1!93}{25}
  \drawbav{colorblindC1!78}{21}
  \drawbrandenburg{colorblindC1!15}{4}
  \drawhessen{colorblindC1!59}{16}
  \drawmecklenburg{colorblindC1!11}{3}
  \drawniedersachsen{colorblindC1!48}{13}
  \drawnrw{colorblindC1!100}{27}
  \drawrheinland{colorblindC1!30}{8}
  \drawsaarland{colorblindC1!11}{3}
  \drawsachsen{colorblindC1!33}{9}
  \drawsachsenanhalt{colorblindC1!26}{7}
  \drawschleswig{colorblindC1!19}{5}
  \drawthuringen{colorblindC1!19}{5}
  \drawbremen{colorblindC1!7}{2}
  \drawhamburg{colorblindC1!30}{8}
  \drawberlin{colorblindC1!26}{7}
\end{tikzpicture}
    \caption{Die absolute Anzahl promotionsberechtigter Institutionen nach Bundesland.}
    \label{fig:DE-grundmenge-beschreibung}
\end{figure}

\noindent Für die gesamte Liste promotionsberechtigter Institutionen, siehe \fxfatal*{ADD LINK TO DIGITAL APPENDIX!}{FIXME!}

\parsum{Stichprobenziehung}
Diese Liste promotionsberechtigter Institutionen bildete die Grundmenge für die Ziehung einer einfachen Zufallsstichprobe.
Bei der Auswahl der Stichprobe wurde ein Konfidenzintervall von \SI{95}{\percent} und eine Fehlerspanne von \SI{5}{\percent} zugrunde gelegt.
Diese Parameter gewährleisten, dass die Ergebnisse der Stichprobe mit hoher Wahrscheinlichkeit repräsentativ für die gesamte Population sind und die Unsicherheit der Schätzungen innerhalb akzeptabler Grenzen bleibt.
Um den Prozess der Stichprobenziehung zu automatisieren und eine zufällige Auswahl zu gewährleisten, wurde eine auf Python basierende Software \autocite{Krassnig2024-csv} genutzt, welche im Rahmen dieser Arbeit geschrieben wurde.%
\footnote{%
Die Software von \citeauthor{Krassnig2024-csv} \autocite{Krassnig2024-csv} nutzt standardmäßig die Anzahl an Nanosekunden seit dem Beginn der System-Epoche (1970-01-01T00:00:00Z) als Startwert für die Zufallsfunktion.
Der genutzte Startwert wird als begleitendes Metadatum der Stichprobe abgespeichert.
Die Ziehung ist somit wiederholbar und das Datum der Ziehung verifizierbar.} 

\parsum{Stichprobenbeschreibung}
Die so gezogene Stichprobe ($n=115$) besteht aus ca. \SI{71}{\percent} aller promotionsberechtigter Institutionen.
Die Stichprobe umfasst Institutionen aller Trägerschaften aus der Grundmenge: öffentlich-rechtliche, private sowie kirchliche Institutionen.
Darüber hinaus umfasst die Stichprobe von den Hochschultypen her Universitäten, \glspl{fh}, \glspl{haw} und eine \gls{hset}.
In der Stichprobe befindet sich nicht die \gls{vh}, die sich in der Grundmenge befindet.
Mit dieser Ausnahme sind somit alle anderen Hochschultypen vertreten.
Die relative sowie die absolute Distribution aller Institutionen in der Stichprobe nach Hochschultyp und Trägerschaft ist in \cref{tab:stichprobe-beschreibung-art} gegeben.
\begin{table}[!htbp]
	\caption{Die Verteilung der Institutionen in der Stichprobe nach $\text{\textit{Hochschultyp}}\times\text{\textit{Trägerschaft}}$ aufgegliedert. Absolute Werte in Klammern angegeben.}
    \resizebox{\ifdim\width>\textwidth\textwidth\else\width\fi}{!}{%
        \begin{tabular}{lS[table-format=3.2]@{\,}S[table-text-alignment = left]lS[table-format=3.2]@{\,}S[table-text-alignment = left]lS[table-format=3.2]@{\,}S[table-text-alignment = left]lS[table-format=3.2]@{\,}S[table-text-alignment = left]l}
            \toprule
            & \multicolumn{3}{c}{\textbf{Öffentlich-Rechtlich}} & \multicolumn{3}{c}{\textbf{Privat}} & \multicolumn{3}{c}{\textbf{Kirchlich}} & \multicolumn{3}{c}{\textbf{Summe}}    \\
            \midrule
            \textbf{Universität}  & 55,65 & \si{\percent} & (64)  & 8,70 & \si{\percent} & (10) & 4,35 & \si{\percent} & (5) & 68,70  & \si{\percent} & (79) \\
            \textbf{\gls{fh} / \gls{haw}}     & 6,96  & \si{\percent} & (8)  & 0,00 & \si{\percent} & (0)  & 0,00 & \si{\percent} & (0)  & 6,96   & \si{\percent} & (8)  \\
            \textbf{\gls{kh}}          & 22,61 & \si{\percent} & (26)  & 0,87 & \si{\percent} & (1)  & 0,00 & \si{\percent} & (0)  & 23,48  & \si{\percent} & (27)  \\
            \textbf{\gls{hset}}         & 0,87  & \si{\percent} & (1)   & 0,00 & \si{\percent} & (0)  & 0,00 & \si{\percent} & (0)  & 0,87   & \si{\percent} & (1)   \\
            \textbf{\gls{vh}}          & 0,00  & \si{\percent} & (0)   & 0,00 & \si{\percent} & (0)  & 0,00 & \si{\percent} & (0)  & 0,00   & \si{\percent} & (0)   \\\midrule
            \textbf{Summe}        & 86,09 & \si{\percent} & (99) & 9,57 & \si{\percent} & (11) & 4,35 & \si{\percent} & (5) & 100,00 & \si{\percent} & (115) \\
            \bottomrule
        \end{tabular}
    }
	\label{tab:stichprobe-beschreibung-art}
\end{table}

\noindent Geografisch gesehen sind, in der Stichprobe, zu unterschiedlich hohen Anteilen, Institutionen aus allen deutschen Bundesländern vertreten.
Die genaue Verteilung ist in \cref{fig:DE-stichprobe-beschreibung} wiedergegeben.
\begin{figure}[!htbp]
    \centering
    \begin{tikzpicture}[y=1cm, x=1cm, yscale=.8,xscale=.8, every node/.append style={scale=1}, inner sep=0pt, outer sep=0pt]
  \node[text=black,line width=0.0092cm,anchor=west] (title1) at (0,7){\textbf{(A)}};
  \footnotesize
  \drawgermany
  \drawbw{colorblindC1!77}{17}
  \drawbav{colorblindC1!55}{12}
  \drawbrandenburg{colorblindC1!18}{4}
  \drawhessen{colorblindC1!45}{10}
  \drawmecklenburg{colorblindC1!9}{2}
  \drawniedersachsen{colorblindC1!50}{11}
  \drawnrw{colorblindC1!100}{22}
  \drawrheinland{colorblindC1!23}{5}
  \drawsaarland{colorblindC1!9}{2}
  \drawsachsen{colorblindC1!23}{5}
  \drawsachsenanhalt{colorblindC1!27}{6}
  \drawschleswig{colorblindC1!9}{2}
  \drawthuringen{colorblindC1!9}{2}
  \drawbremen{colorblindC1!9}{2}
  \drawhamburg{colorblindC1!36}{8}
  \drawberlin{colorblindC1!23}{5}
\end{tikzpicture}\hfill%
\begin{tikzpicture}[y=1cm, x=1cm, yscale=.8,xscale=.8, every node/.append style={scale=1}, inner sep=0pt, outer sep=0pt]
  \node[text=black,line width=0.0092cm,anchor=west] (title1) at (-0.4,7){\textbf{(B)}};
  \footnotesize
  \drawgermany
  \drawbw{colorblindC1!68}{\SI{68}{\percent}}
  \drawbav{colorblindC1!57}{\SI{57}{\percent}}
  \drawbrandenburg{colorblindC1!100}{\SI{100}{\percent}}
  \drawhessen{colorblindC1!63}{\SI{63}{\percent}}
  \drawmecklenburg{colorblindC1!67}{\SI{67}{\percent}}
  \drawniedersachsen{colorblindC1!85}{\SI{85}{\percent}}
  \drawnrw{colorblindC1!81}{\SI{81}{\percent}}
  \drawrheinland{colorblindC1!63}{\SI{63}{\percent}}
  \drawsaarland{colorblindC1!67}{\SI{67}{\percent}}
  \drawsachsen{colorblindC1!56}{\SI{56}{\percent}}
  \drawsachsenanhalt{colorblindC1!86}{\SI{86}{\percent}}
  \drawschleswig{colorblindC1!40}{\SI{40}{\percent}}
  \drawthuringen{colorblindC1!40}{\SI{40}{\percent}}
  \drawbremen{colorblindC1!100}{\SI{100}{\percent}}
  \drawhamburg{colorblindC1!100}{\SI{100}{\percent}}
  \drawberlin{colorblindC1!71}{\SI{71}{\percent}}
\end{tikzpicture}

    \caption{Verteilung der Institutionen in der gezogenen Stichprobe nach Bundesland. \textbf{(A)}~Die absolute Anzahl der Institutionen nach Bundesland. \fxfatal*{Deutlicher formulieren!}{\textbf{(B)}~Der relative Anteil der aus der Grundmenge gezogenen Institutionen nach Bundesland.}}
    \label{fig:DE-stichprobe-beschreibung}
\end{figure}

\parsum{Dokumentesammlung}
Für die Evaluation, inwiefern die Institutionen der Stichprobe verwaltungsrechtliche Dokumente besitzen, die entweder allgemeine \gls{fdm}-Richtlinien für alle Forschenden der Institution oder spezifische Regelungen für Promovierende beinhalten, wurde deren gesamte öffentlich zugängliche Online-Präsenz nach relevanten Dateien und Webseiten durchsucht.
Diese Suche fand im Allgemeinen auf zwei Wege statt: über die interne Suchfunktion der Institution und die externe Durchsuchung der Institutions-Domäne via der Suchmaschine \href{https://www.duckduckgo.com/}{DuckDuckGo}.

\parsum{Allgemeine Dokumente}
Für allgemeingültige Richtlinien wurden in erster Linie eigenständige \gls{fdm}-Richtlinien gesucht.
Hierfür wurde, jenseits der allgemeinen Suchmethode auch die dedizierte Forschungsdatenpolicies-Liste der Informationsplattform \citeauthor{Forschungsdaten2024} genutzt \autocite{Forschungsdaten2024}.%
\footnote{%
    Die \gls{fdm}-Richtlinien, welche sich nicht auf diesem Portal haben finden lassen, werden vom Autor nach Beendigung dieser Arbeit dort nach Möglichkeit eingetragen.%
}
Jenseits von spezifischen \gls{fdm}-Richtlinien wurden auch Richtlinien zur Sicherung der \gls{gwp} sowie andere Richtlinien gesucht, die anderweitige allgemeine wissenschaftliche Empfehlungen bzw. Auflagen für Forschende aussprechen.
Wenn es von einem Dokument sowohl eine HTML- wie auch eine PDF-Datei gab, so wurde nur die PDF-Datei zur Evaluation weitergenutzt.
Hierbei wurden insgesamt 142 Dokumente zur Weiterverarbeitung aufgenommen.

\parsum{Promotionsspezifische Dokumente}
Für promotionsspezifische Richtlinien wurden Promotions- und Prüfungsordnungen gesucht.
Dabei wurden sowohl fachspezifische Ordnungen wie auch verbindliche Rahmenbedingungen und anderweitige übergreifende Ordnungen aufgenommen.
Die heterogene Handhabung dieser Dokumente seitens der Institutionen führte zu folgenden Selektionsregeln bei der Auswahl:
Forschungsdaten.org
(i)~Wenn es eine aktuelle Lesefassung der Promotionsordnung gibt, so wird diese bevorzugt.
(ii)~Sollte es keine aktuelle Lesefassung der Promotionsordnung geben, so wird die aktuellste Gesamtversion der Promotionsordnung bevorzugt.
(iii)~Sollte es keine aktuelle Gesamtversion geben, so werden zusätzlich zu der letzten Version der Promotionsordnung auch alle seither erschienen relevanten verwaltungsrechtlichen Addenda aufgenommen.
(iv)~Wenn es von einem Dokument sowohl eine HTML- wie auch eine PDF-Datei gab, so wurde nur die PDF-Datei zur Evaluation weitergenutzt.
Hierbei wurden insgesamt 754 Dokumente zur Weiterverarbeitung aufgenommen.

\subsection{Methoden}\label{sec:policy-methods}
\parsum{Allgemeine Dokumente}
Die in \cref{sec:policy-material} gesammelten allgemeingültigen Dokumente und deren Institutionen wurden dann wie folgt klassifiziert:
(i)~Jedes Dokument wurde durch manuelle Überprüfung einem Typ zugeordnet.
Diese Typen waren, hierarchisch geordnet: 
\begin{enumerate}
    \item Nicht relevant
    \item Richtlinie zu \gls{gwp}
    \item Anderweitige Richtlinie die \gls{fdm} beinhaltet
    \item Richtlinie zu \gls{forschungsdaten}~/~\gls{fdm}
\end{enumerate}    
(ii)~Bei Dokumenten, welche als \enquote{Richtlinie zu \gls{forschungsdaten}~/~\gls{fdm}} klassifiziert wurden, wird zusätzlich notiert, ob es sich dabei um eine Leitlinie, einen Grundsatz, eine Policy, eine Empfehlung oder eine Richtlinie handelt (nach dokumenteigener Angabe).
(iii)~Jede Institution erhielt dann die Klassifikation des Dokumentes, welche die höchste hierarchische Klassifikationsstufe besitzt, es sei denn, die Institution hatte kein öffentlich zugängliches Dokument dieser Art.
In diesem Fall wurde dies stattdessen als Klassifikationsstufe vermerkt.

\parsum{Promotionsspezifische Dokumente}
Die in \cref{sec:policy-material} gesammelten promotionsspezifischen Dokumente und deren Institutionen wurden dann wie folgt klassifiziert:
(i)~Es wurde überprüft, ob alle PDF-Dateien lesbaren eingebetteten Text besitzen.
Hierfür wurde mit \cref{lst:pdfheadtail} der Anfang und das Ende des eingebetteten Textes via \emph{pdftotext} aus der \emph{Poppler}-Softwaresammlung \autocite{Poppler} angezeigt.
\lstinputlisting[language=Bash,label={lst:pdfheadtail},caption={Ein Bash-Skript, welches die ersten und letzten 20 Zeilen aller PDF-Dokumente in allen Unterordnern des jetzigen Ordners anzeigt.}]{content/code/pdfheadtail.sh}
(ii)~Texte, die sich als maschinell nicht lesbar erwiesen haben, wurden notiert und später manuell untersucht und klassifiziert.
(iii)~Die restlichen Texte wurden zuerst mit \cref{lst:fdmchecker} daraufhin überprüft, ob sie Text beinhalten, der \glspl{forschungsdaten}, \gls{fdm} oder \gls{gwp} betrifft.
\lstinputlisting[language=Bash,label={lst:fdmchecker},caption={Ein Bash-Skript, welches Erwähnung von \glspl{forschungsdaten} und \gls{gwp} in PDF-Dateien überprüft und den dazugehörigen Kontext anzeigt.}]{content/code/fdmchecker.sh}
(iv)~Die durch das Skript angezeigten Treffer wurden dann auf Kontext überprüft und entsprechend manuell klassifiziert.
Hierbei gab es fünf Klassifkationsstufen:
\begin{enumerate}
    \item Keinerlei Richtlinien zu \glspl{forschungsdaten}~/~\gls{fdm} enhalten
    \item Richtlinien zu \gls{gwp} als Empfehlung enthalten
    \item Richtlinien zu \gls{gwp} als Verpflichtung enthalten
    \item Richtlinien zu \glspl{forschungsdaten}~/~\gls{fdm} als Empfehlung enthalten
    \item Richtlinien zu \glspl{forschungsdaten}~/~\gls{fdm} als Verpflichtung enthalten
\end{enumerate}
(v)~Jede Institution erhielt dann die Klassifikation des Dokumentes, welche die höchste hierarchische Klassifikationsstufe besitzt, es sei denn, die Institution hatte kein öffentlich zugängliches Dokument dieser Art.
In diesem Fall wurde dies stattdessen als Klassifikationsstufe vermerkt.
Es wurde zusätzlich notiert, ob die promotionsspezifischen Richtlinien für alle Promovierenden gelten oder ob dies nur auf eine Teilmenge zutrifft.

\section{Resultate}\label{sec:policy-results}
\begin{table}[!htbp]
	\caption{Klassifikation der allgemeingültigen verwaltungsrechtlichen Dokumente in relativer Angabe nach Hochschultyp. Absolute Werte in Klammern angegeben.}
    \resizebox{\ifdim\width>\textwidth\textwidth\else\width\fi}{!}{%
        \begin{tabular}{lS[table-format=3.2]@{\,}S[table-text-alignment = left]lS[table-format=3.2]@{\,}S[table-text-alignment = left]lS[table-format=3.2]@{\,}S[table-text-alignment = left]lS[table-format=3.2]@{\,}S[table-text-alignment = left]l}
            \toprule
            & \multicolumn{3}{c}{\textbf{Keine Verfügbar}} & \multicolumn{3}{c}{\textbf{\gls{gwp}-Richtlinien}} & \multicolumn{3}{c}{\textbf{Andere Richtlinien}} & \multicolumn{3}{c}{\textbf{\gls{forschungsdaten}-Richtlinien}}    \\
            \midrule
            \textbf{Universität}  & 5,06 & \si{\percent} & (4)  & 35,44 & \si{\percent} & (13) & 0,00 & \si{\percent} & (0) & 59,49  & \si{\percent} & (47) \\
            \textbf{\gls{fh} / \gls{haw}}     & 0,00  & \si{\percent} & (0)  & 75,00 & \si{\percent} & (6)  & 0,00 & \si{\percent} & (0)  & 25,00   & \si{\percent} & (2)  \\
            \textbf{\gls{kh}}          & 37,04 & \si{\percent} & (10)  & 55,56 & \si{\percent} & (15)  & 0,00 & \si{\percent} & (0)  & 7,41  & \si{\percent} & (2)  \\
            \textbf{\gls{hset}}         & 0,00  & \si{\percent} & (0)   & 100,00 & \si{\percent} & (1)  & 0,00 & \si{\percent} & (0)  & 0,00   & \si{\percent} & (0)   \\\midrule
            \textbf{Alle}        & 12,17 & \si{\percent} & (14) & 43,48 & \si{\percent} & (50) & 0,00 & \si{\percent} & (0) & 44,35 & \si{\percent} & (51) \\
            \bottomrule
        \end{tabular}
    }
	\label{tab:stichprobe-klassifikation}
\end{table}

\begin{figure}
    \begin{tikzpicture}[y=1cm, x=1cm, yscale=\globalscale,xscale=\globalscale, every node/.append style={scale=\globalscale}, inner sep=0pt, outer sep=0pt]
    \path[fill=cebebeb,line cap=round,line join=round,line width=0.04cm,miter 
    limit=10.0] (1.62, 15.49) rectangle (23.94, 2.09);
  
  
  
    \path[draw=white,line cap=butt,line join=round,line width=0.02cm,miter 
    limit=10.0] (1.62, 4.22) -- (23.94, 4.22);
  
  
  
    \path[draw=white,line cap=butt,line join=round,line width=0.02cm,miter 
    limit=10.0] (1.62, 7.27) -- (23.94, 7.27);
  
  
  
    \path[draw=white,line cap=butt,line join=round,line width=0.02cm,miter 
    limit=10.0] (1.62, 10.31) -- (23.94, 10.31);
  
  
  
    \path[draw=white,line cap=butt,line join=round,line width=0.02cm,miter 
    limit=10.0] (1.62, 13.36) -- (23.94, 13.36);
  
  
  
    \path[draw=white,line cap=butt,line join=round,line width=0.04cm,miter 
    limit=10.0] (1.62, 2.7) -- (23.94, 2.7);
  
  
  
    \path[draw=white,line cap=butt,line join=round,line width=0.04cm,miter 
    limit=10.0] (1.62, 5.75) -- (23.94, 5.75);
  
  
  
    \path[draw=white,line cap=butt,line join=round,line width=0.04cm,miter 
    limit=10.0] (1.62, 8.79) -- (23.94, 8.79);
  
  
  
    \path[draw=white,line cap=butt,line join=round,line width=0.04cm,miter 
    limit=10.0] (1.62, 11.84) -- (23.94, 11.84);
  
  
  
    \path[draw=white,line cap=butt,line join=round,line width=0.04cm,miter 
    limit=10.0] (1.62, 14.88) -- (23.94, 14.88);
  
  
  
    \path[draw=white,line cap=butt,line join=round,line width=0.04cm,miter 
    limit=10.0] (2.45, 2.09) -- (2.45, 15.49);
  
  
  
    \path[draw=white,line cap=butt,line join=round,line width=0.04cm,miter 
    limit=10.0] (3.83, 2.09) -- (3.83, 15.49);
  
  
  
    \path[draw=white,line cap=butt,line join=round,line width=0.04cm,miter 
    limit=10.0] (5.21, 2.09) -- (5.21, 15.49);
  
  
  
    \path[draw=white,line cap=butt,line join=round,line width=0.04cm,miter 
    limit=10.0] (6.58, 2.09) -- (6.58, 15.49);
  
  
  
    \path[draw=white,line cap=butt,line join=round,line width=0.04cm,miter 
    limit=10.0] (7.96, 2.09) -- (7.96, 15.49);
  
  
  
    \path[draw=white,line cap=butt,line join=round,line width=0.04cm,miter 
    limit=10.0] (9.34, 2.09) -- (9.34, 15.49);
  
  
  
    \path[draw=white,line cap=butt,line join=round,line width=0.04cm,miter 
    limit=10.0] (10.71, 2.09) -- (10.71, 15.49);
  
  
  
    \path[draw=white,line cap=butt,line join=round,line width=0.04cm,miter 
    limit=10.0] (12.09, 2.09) -- (12.09, 15.49);
  
  
  
    \path[draw=white,line cap=butt,line join=round,line width=0.04cm,miter 
    limit=10.0] (13.47, 2.09) -- (13.47, 15.49);
  
  
  
    \path[draw=white,line cap=butt,line join=round,line width=0.04cm,miter 
    limit=10.0] (14.85, 2.09) -- (14.85, 15.49);
  
  
  
    \path[draw=white,line cap=butt,line join=round,line width=0.04cm,miter 
    limit=10.0] (16.22, 2.09) -- (16.22, 15.49);
  
  
  
    \path[draw=white,line cap=butt,line join=round,line width=0.04cm,miter 
    limit=10.0] (17.6, 2.09) -- (17.6, 15.49);
  
  
  
    \path[draw=white,line cap=butt,line join=round,line width=0.04cm,miter 
    limit=10.0] (18.98, 2.09) -- (18.98, 15.49);
  
  
  
    \path[draw=white,line cap=butt,line join=round,line width=0.04cm,miter 
    limit=10.0] (20.36, 2.09) -- (20.36, 15.49);
  
  
  
    \path[draw=white,line cap=butt,line join=round,line width=0.04cm,miter 
    limit=10.0] (21.73, 2.09) -- (21.73, 15.49);
  
  
  
    \path[draw=white,line cap=butt,line join=round,line width=0.04cm,miter 
    limit=10.0] (23.11, 2.09) -- (23.11, 15.49);
  
  
  
    \path[draw=black,fill=cee8866,line cap=butt,line join=miter,line 
    width=0.04cm,miter limit=10.0] (1.83, 5.57) rectangle (3.07, 2.7);
  
  
  
    \path[draw=black,fill=ceedd88,line cap=butt,line join=miter,line 
    width=0.04cm,miter limit=10.0] (1.83, 10.58) rectangle (3.07, 5.57);
  
  
  
    \path[draw=black,fill=c77aadd,line cap=butt,line join=miter,line 
    width=0.04cm,miter limit=10.0] (1.83, 14.88) rectangle (3.07, 10.58);
  
  
  
    \path[draw=black,fill=cee8866,line cap=butt,line join=miter,line 
    width=0.04cm,miter limit=10.0] (3.21, 4.73) rectangle (4.45, 2.7);
  
  
  
    \path[draw=black,fill=ceedd88,line cap=butt,line join=miter,line 
    width=0.04cm,miter limit=10.0] (3.21, 10.82) rectangle (4.45, 4.73);
  
  
  
    \path[draw=black,fill=c77aadd,line cap=butt,line join=miter,line 
    width=0.04cm,miter limit=10.0] (3.21, 14.88) rectangle (4.45, 10.82);
  
  
  
    \path[draw=black,fill=cee8866,line cap=butt,line join=miter,line 
    width=0.04cm,miter limit=10.0] (4.59, 5.14) rectangle (5.83, 2.7);
  
  
  
    \path[draw=black,fill=ceedd88,line cap=butt,line join=miter,line 
    width=0.04cm,miter limit=10.0] (4.59, 7.57) rectangle (5.83, 5.14);
  
  
  
    \path[draw=black,fill=c77aadd,line cap=butt,line join=miter,line 
    width=0.04cm,miter limit=10.0] (4.59, 14.88) rectangle (5.83, 7.57);
  
  
  
    \path[fill=cee8866,line cap=butt,line join=miter,line width=0.04cm,miter 
    limit=10.0] ;
  
  
  
    \path[draw=black,fill=ceedd88,line cap=butt,line join=miter,line 
    width=0.04cm,miter limit=10.0] (5.96, 5.75) rectangle (7.2, 2.7);
  
  
  
    \path[draw=black,fill=c77aadd,line cap=butt,line join=miter,line 
    width=0.04cm,miter limit=10.0] (5.96, 14.88) rectangle (7.2, 5.75);
  
  
  
    \path[fill=cee8866,line cap=butt,line join=miter,line width=0.04cm,miter 
    limit=10.0] ;
  
  
  
    \path[draw=black,fill=ceedd88,line cap=butt,line join=miter,line 
    width=0.04cm,miter limit=10.0] (7.34, 8.79) rectangle (8.58, 2.7);
  
  
  
    \path[draw=black,fill=c77aadd,line cap=butt,line join=miter,line 
    width=0.04cm,miter limit=10.0] (7.34, 14.88) rectangle (8.58, 8.79);
  
  
  
    \path[draw=black,fill=cee8866,line cap=butt,line join=miter,line 
    width=0.04cm,miter limit=10.0] (8.72, 4.22) rectangle (9.96, 2.7);
  
  
  
    \path[draw=black,fill=ceedd88,line cap=butt,line join=miter,line 
    width=0.04cm,miter limit=10.0] (8.72, 13.36) rectangle (9.96, 4.22);
  
  
  
    \path[draw=black,fill=c77aadd,line cap=butt,line join=miter,line 
    width=0.04cm,miter limit=10.0] (8.72, 14.88) rectangle (9.96, 13.36);
  
  
  
    \path[draw=black,fill=cee8866,line cap=butt,line join=miter,line 
    width=0.04cm,miter limit=10.0] (10.1, 3.92) rectangle (11.33, 2.7);
  
  
  
    \path[draw=black,fill=ceedd88,line cap=butt,line join=miter,line 
    width=0.04cm,miter limit=10.0] (10.1, 8.79) rectangle (11.33, 3.92);
  
  
  
    \path[draw=black,fill=c77aadd,line cap=butt,line join=miter,line 
    width=0.04cm,miter limit=10.0] (10.1, 14.88) rectangle (11.33, 8.79);
  
  
  
    \path[draw=black,fill=cee8866,line cap=butt,line join=miter,line 
    width=0.04cm,miter limit=10.0] (11.47, 8.79) rectangle (12.71, 2.7);
  
  
  
    \path[draw=black,fill=ceedd88,line cap=butt,line join=miter,line 
    width=0.04cm,miter limit=10.0] (11.47, 14.88) rectangle (12.71, 8.79);
  
  
  
    \path[fill=c77aadd,line cap=butt,line join=miter,line width=0.04cm,miter 
    limit=10.0] ;
  
  
  
    \path[draw=black,fill=cee8866,line cap=butt,line join=miter,line 
    width=0.04cm,miter limit=10.0] (12.85, 3.81) rectangle (14.09, 2.7);
  
  
  
    \path[fill=ceedd88,line cap=butt,line join=miter,line width=0.04cm,miter 
    limit=10.0] ;
  
  
  
    \path[draw=black,fill=c77aadd,line cap=butt,line join=miter,line 
    width=0.04cm,miter limit=10.0] (12.85, 14.88) rectangle (14.09, 3.81);
  
  
  
    \path[draw=black,fill=cee8866,line cap=butt,line join=miter,line 
    width=0.04cm,miter limit=10.0] (14.23, 3.81) rectangle (15.47, 2.7);
  
  
  
    \path[draw=black,fill=ceedd88,line cap=butt,line join=miter,line 
    width=0.04cm,miter limit=10.0] (14.23, 8.79) rectangle (15.47, 3.81);
  
  
  
    \path[draw=black,fill=c77aadd,line cap=butt,line join=miter,line 
    width=0.04cm,miter limit=10.0] (14.23, 14.88) rectangle (15.47, 8.79);
  
  
  
    \path[fill=cee8866,line cap=butt,line join=miter,line width=0.04cm,miter 
    limit=10.0] ;
  
  
  
    \path[draw=black,fill=ceedd88,line cap=butt,line join=miter,line 
    width=0.04cm,miter limit=10.0] (15.6, 12.45) rectangle (16.84, 2.7);
  
  
  
    \path[draw=black,fill=c77aadd,line cap=butt,line join=miter,line 
    width=0.04cm,miter limit=10.0] (15.6, 14.88) rectangle (16.84, 12.45);
  
  
  
    \path[fill=cee8866,line cap=butt,line join=miter,line width=0.04cm,miter 
    limit=10.0] ;
  
  
  
    \path[draw=black,fill=ceedd88,line cap=butt,line join=miter,line 
    width=0.04cm,miter limit=10.0] (16.98, 8.79) rectangle (18.22, 2.7);
  
  
  
    \path[draw=black,fill=c77aadd,line cap=butt,line join=miter,line 
    width=0.04cm,miter limit=10.0] (16.98, 14.88) rectangle (18.22, 8.79);
  
  
  
    \path[fill=cee8866,line cap=butt,line join=miter,line width=0.04cm,miter 
    limit=10.0] ;
  
  
  
    \path[draw=black,fill=ceedd88,line cap=butt,line join=miter,line 
    width=0.04cm,miter limit=10.0] (18.36, 10.01) rectangle (19.6, 2.7);
  
  
  
    \path[draw=black,fill=c77aadd,line cap=butt,line join=miter,line 
    width=0.04cm,miter limit=10.0] (18.36, 14.88) rectangle (19.6, 10.01);
  
  
  
    \path[fill=cee8866,line cap=butt,line join=miter,line width=0.04cm,miter 
    limit=10.0] ;
  
  
  
    \path[draw=black,fill=ceedd88,line cap=butt,line join=miter,line 
    width=0.04cm,miter limit=10.0] (19.74, 10.82) rectangle (20.98, 2.7);
  
  
  
    \path[draw=black,fill=c77aadd,line cap=butt,line join=miter,line 
    width=0.04cm,miter limit=10.0] (19.74, 14.88) rectangle (20.98, 10.82);
  
  
  
    \path[draw=black,fill=cee8866,line cap=butt,line join=miter,line 
    width=0.04cm,miter limit=10.0] (21.11, 8.79) rectangle (22.35, 2.7);
  
  
  
    \path[draw=black,fill=ceedd88,line cap=butt,line join=miter,line 
    width=0.04cm,miter limit=10.0] (21.11, 14.88) rectangle (22.35, 8.79);
  
  
  
    \path[fill=c77aadd,line cap=butt,line join=miter,line width=0.04cm,miter 
    limit=10.0] ;
  
  
  
    \path[fill=cee8866,line cap=butt,line join=miter,line width=0.04cm,miter 
    limit=10.0] ;
  
  
  
    \path[draw=black,fill=ceedd88,line cap=butt,line join=miter,line 
    width=0.04cm,miter limit=10.0] (22.49, 8.79) rectangle (23.73, 2.7);
  
  
  
    \path[draw=black,fill=c77aadd,line cap=butt,line join=miter,line 
    width=0.04cm,miter limit=10.0] (22.49, 14.88) rectangle (23.73, 8.79);
  
  
  
    \node[anchor=south] (text76) at (2.45, 4.26){24};
  
  
  
    \node[anchor=south] (text77) at (2.45, 3.75){(4)};
  
  
  
    \node[anchor=south] (text78) at (2.45, 8.2){41};
  
  
  
    \node[anchor=south] (text79) at (2.45, 7.7){(7)};
  
  
  
    \node[anchor=south] (text80) at (2.45, 12.86){35};
  
  
  
    \node[anchor=south] (text81) at (2.45, 12.35){(6)};
  
  
  
    \node[anchor=south] (text82) at (3.83, 3.84){17};
  
  
  
    \node[anchor=south] (text83) at (3.83, 3.34){(2)};
  
  
  
    \node[anchor=south] (text84) at (3.83, 7.9){50};
  
  
  
    \node[anchor=south] (text85) at (3.83, 7.4){(6)};
  
  
  
    \node[anchor=south] (text86) at (3.83, 12.98){33};
  
  
  
    \node[anchor=south] (text87) at (3.83, 12.47){(4)};
  
  
  
    \node[anchor=south] (text88) at (5.21, 4.05){20};
  
  
  
    \node[anchor=south] (text89) at (5.21, 3.54){(1)};
  
  
  
    \node[anchor=south] (text90) at (5.21, 6.48){20};
  
  
  
    \node[anchor=south] (text91) at (5.21, 5.98){(1)};
  
  
  
    \node[anchor=south] (text92) at (5.21, 11.35){60};
  
  
  
    \node[anchor=south] (text93) at (5.21, 10.85){(3)};
  
  
  
    \node[anchor=south] (text94) at (6.58, 4.35){25};
  
  
  
    \node[anchor=south] (text95) at (6.58, 3.84){(1)};
  
  
  
    \node[anchor=south] (text96) at (6.58, 10.44){75};
  
  
  
    \node[anchor=south] (text97) at (6.58, 9.94){(3)};
  
  
  
    \node[anchor=south] (text98) at (7.96, 5.87){50};
  
  
  
    \node[anchor=south] (text99) at (7.96, 5.37){(1)};
  
  
  
    \node[anchor=south] (text100) at (7.96, 11.96){50};
  
  
  
    \node[anchor=south] (text101) at (7.96, 11.46){(1)};
  
  
  
    \node[anchor=south] (text102) at (9.34, 3.59){12};
  
  
  
    \node[anchor=south] (text103) at (9.34, 3.08){(1)};
  
  
  
    \node[anchor=south] (text104) at (9.34, 8.92){75};
  
  
  
    \node[anchor=south] (text105) at (9.34, 8.41){(6)};
  
  
  
    \node[anchor=south] (text106) at (9.34, 14.25){12};
  
  
  
    \node[anchor=south] (text107) at (9.34, 13.74){(1)};
  
  
  
    \node[anchor=south] (text108) at (10.71, 3.44){10};
  
  
  
    \node[anchor=south] (text109) at (10.71, 2.93){(1)};
  
  
  
    \node[anchor=south] (text110) at (10.71, 6.48){40};
  
  
  
    \node[anchor=south] (text111) at (10.71, 5.98){(4)};
  
  
  
    \node[anchor=south] (text112) at (10.71, 11.96){50};
  
  
  
    \node[anchor=south] (text113) at (10.71, 11.46){(5)};
  
  
  
    \node[anchor=south] (text114) at (12.09, 5.87){50};
  
  
  
    \node[anchor=south] (text115) at (12.09, 5.37){(1)};
  
  
  
    \node[anchor=south] (text116) at (12.09, 11.96){50};
  
  
  
    \node[anchor=south] (text117) at (12.09, 11.46){(1)};
  
  
  
    \node[anchor=south] (text118) at (13.47, 3.38){9};
  
  
  
    \node[anchor=south] (text119) at (13.47, 2.88){(1)};
  
  
  
    \node[anchor=south] (text120) at (13.47, 9.47){91};
  
  
  
    \node[anchor=south] (text121) at (13.47, 8.97){(10)};
  
  
  
    \node[anchor=south] (text122) at (14.85, 3.38){9};
  
  
  
    \node[anchor=south] (text123) at (14.85, 2.88){(2)};
  
  
  
    \node[anchor=south] (text124) at (14.85, 6.43){41};
  
  
  
    \node[anchor=south] (text125) at (14.85, 5.92){(9)};
  
  
  
    \node[anchor=south] (text126) at (14.85, 11.96){50};
  
  
  
    \node[anchor=south] (text127) at (14.85, 11.46){(11)};
  
  
  
    \node[anchor=south] (text128) at (16.22, 7.7){80};
  
  
  
    \node[anchor=south] (text129) at (16.22, 7.19){(4)};
  
  
  
    \node[anchor=south] (text130) at (16.22, 13.79){20};
  
  
  
    \node[anchor=south] (text131) at (16.22, 13.29){(1)};
  
  
  
    \node[anchor=south] (text132) at (17.6, 5.87){50};
  
  
  
    \node[anchor=south] (text133) at (17.6, 5.37){(1)};
  
  
  
    \node[anchor=south] (text134) at (17.6, 11.96){50};
  
  
  
    \node[anchor=south] (text135) at (17.6, 11.46){(1)};
  
  
  
    \node[anchor=south] (text136) at (18.98, 6.48){60};
  
  
  
    \node[anchor=south] (text137) at (18.98, 5.98){(3)};
  
  
  
    \node[anchor=south] (text138) at (18.98, 12.57){40};
  
  
  
    \node[anchor=south] (text139) at (18.98, 12.07){(2)};
  
  
  
    \node[anchor=south] (text140) at (20.36, 6.89){67};
  
  
  
    \node[anchor=south] (text141) at (20.36, 6.38){(4)};
  
  
  
    \node[anchor=south] (text142) at (20.36, 12.98){33};
  
  
  
    \node[anchor=south] (text143) at (20.36, 12.47){(2)};
  
  
  
    \node[anchor=south] (text144) at (21.73, 5.87){50};
  
  
  
    \node[anchor=south] (text145) at (21.73, 5.37){(1)};
  
  
  
    \node[anchor=south] (text146) at (21.73, 11.96){50};
  
  
  
    \node[anchor=south] (text147) at (21.73, 11.46){(1)};
  
  
  
    \node[anchor=south] (text148) at (23.11, 5.87){50};
  
  
  
    \node[anchor=south] (text149) at (23.11, 5.37){(1)};
  
  
  
    \node[anchor=south] (text150) at (23.11, 11.96){50};
  
  
  
    \node[anchor=south] (text151) at (23.11, 11.46){(1)};
  
  
  
    \path[draw=c333333,line cap=butt,line join=round,line width=0.04cm,miter 
    limit=10.0] (2.45, 1.99) -- (2.45, 2.09);
  
  
  
    \path[draw=c333333,line cap=butt,line join=round,line width=0.04cm,miter 
    limit=10.0] (3.83, 1.99) -- (3.83, 2.09);
  
  
  
    \path[draw=c333333,line cap=butt,line join=round,line width=0.04cm,miter 
    limit=10.0] (5.21, 1.99) -- (5.21, 2.09);
  
  
  
    \path[draw=c333333,line cap=butt,line join=round,line width=0.04cm,miter 
    limit=10.0] (6.58, 1.99) -- (6.58, 2.09);
  
  
  
    \path[draw=c333333,line cap=butt,line join=round,line width=0.04cm,miter 
    limit=10.0] (7.96, 1.99) -- (7.96, 2.09);
  
  
  
    \path[draw=c333333,line cap=butt,line join=round,line width=0.04cm,miter 
    limit=10.0] (9.34, 1.99) -- (9.34, 2.09);
  
  
  
    \path[draw=c333333,line cap=butt,line join=round,line width=0.04cm,miter 
    limit=10.0] (10.71, 1.99) -- (10.71, 2.09);
  
  
  
    \path[draw=c333333,line cap=butt,line join=round,line width=0.04cm,miter 
    limit=10.0] (12.09, 1.99) -- (12.09, 2.09);
  
  
  
    \path[draw=c333333,line cap=butt,line join=round,line width=0.04cm,miter 
    limit=10.0] (13.47, 1.99) -- (13.47, 2.09);
  
  
  
    \path[draw=c333333,line cap=butt,line join=round,line width=0.04cm,miter 
    limit=10.0] (14.85, 1.99) -- (14.85, 2.09);
  
  
  
    \path[draw=c333333,line cap=butt,line join=round,line width=0.04cm,miter 
    limit=10.0] (16.22, 1.99) -- (16.22, 2.09);
  
  
  
    \path[draw=c333333,line cap=butt,line join=round,line width=0.04cm,miter 
    limit=10.0] (17.6, 1.99) -- (17.6, 2.09);
  
  
  
    \path[draw=c333333,line cap=butt,line join=round,line width=0.04cm,miter 
    limit=10.0] (18.98, 1.99) -- (18.98, 2.09);
  
  
  
    \path[draw=c333333,line cap=butt,line join=round,line width=0.04cm,miter 
    limit=10.0] (20.36, 1.99) -- (20.36, 2.09);
  
  
  
    \path[draw=c333333,line cap=butt,line join=round,line width=0.04cm,miter 
    limit=10.0] (21.73, 1.99) -- (21.73, 2.09);
  
  
  
    \path[draw=c333333,line cap=butt,line join=round,line width=0.04cm,miter 
    limit=10.0] (23.11, 1.99) -- (23.11, 2.09);
  
  
  
    \node[text=c4d4d4d,anchor=south east,cm={ 0.71,0.71,-0.71,0.71,(2.66, 
    -16.08)}] (text168) at (0.0, 17.78){DE-BW};
  
  
  
    \node[text=c4d4d4d,anchor=south east,cm={ 0.71,0.71,-0.71,0.71,(4.04, 
    -16.08)}] (text169) at (0.0, 17.78){DE-BY};
  
  
  
    \node[text=c4d4d4d,anchor=south east,cm={ 0.71,0.71,-0.71,0.71,(5.42, 
    -16.08)}] (text170) at (0.0, 17.78){DE-BE};
  
  
  
    \node[text=c4d4d4d,anchor=south east,cm={ 0.71,0.71,-0.71,0.71,(6.8, -16.08)}]
     (text171) at (0.0, 17.78){DE-BB};
  
  
  
    \node[text=c4d4d4d,anchor=south east,cm={ 0.71,0.71,-0.71,0.71,(8.17, 
    -16.08)}] (text172) at (0.0, 17.78){DE-HB};
  
  
  
    \node[text=c4d4d4d,anchor=south east,cm={ 0.71,0.71,-0.71,0.71,(9.55, 
    -16.08)}] (text173) at (0.0, 17.78){DE-HH};
  
  
  
    \node[text=c4d4d4d,anchor=south east,cm={ 0.71,0.71,-0.71,0.71,(10.93, 
    -16.08)}] (text174) at (0.0, 17.78){DE-HE};
  
  
  
    \node[text=c4d4d4d,anchor=south east,cm={ 0.71,0.71,-0.71,0.71,(12.31, 
    -16.08)}] (text175) at (0.0, 17.78){DE-MV};
  
  
  
    \node[text=c4d4d4d,anchor=south east,cm={ 0.71,0.71,-0.71,0.71,(13.68, 
    -16.08)}] (text176) at (0.0, 17.78){DE-NI};
  
  
  
    \node[text=c4d4d4d,anchor=south east,cm={ 0.71,0.71,-0.71,0.71,(15.06, 
    -16.08)}] (text177) at (0.0, 17.78){DE-NW};
  
  
  
    \node[text=c4d4d4d,anchor=south east,cm={ 0.71,0.71,-0.71,0.71,(16.44, 
    -16.08)}] (text178) at (0.0, 17.78){DE-RP};
  
  
  
    \node[text=c4d4d4d,anchor=south east,cm={ 0.71,0.71,-0.71,0.71,(17.81, 
    -16.08)}] (text179) at (0.0, 17.78){DE-SL};
  
  
  
    \node[text=c4d4d4d,anchor=south east,cm={ 0.71,0.71,-0.71,0.71,(19.19, 
    -16.08)}] (text180) at (0.0, 17.78){DE-SN};
  
  
  
    \node[text=c4d4d4d,anchor=south east,cm={ 0.71,0.71,-0.71,0.71,(20.57, 
    -16.08)}] (text181) at (0.0, 17.78){DE-ST};
  
  
  
    \node[text=c4d4d4d,anchor=south east,cm={ 0.71,0.71,-0.71,0.71,(21.95, 
    -16.08)}] (text182) at (0.0, 17.78){DE-SH};
  
  
  
    \node[text=c4d4d4d,anchor=south east,cm={ 0.71,0.71,-0.71,0.71,(23.32, 
    -16.08)}] (text183) at (0.0, 17.78){DE-TH};
  
  
  
    \node[text=c4d4d4d,anchor=south east] (text184) at (1.45, 2.59){0\%};
  
  
  
    \node[text=c4d4d4d,anchor=south east] (text185) at (1.45, 5.63){25\%};
  
  
  
    \node[text=c4d4d4d,anchor=south east] (text186) at (1.45, 8.68){50\%};
  
  
  
    \node[text=c4d4d4d,anchor=south east] (text187) at (1.45, 11.73){75\%};
  
  
  
    \node[text=c4d4d4d,anchor=south east] (text188) at (1.45, 14.77){100\%};
  
  
  
    \path[draw=c333333,line cap=butt,line join=round,line width=0.04cm,miter 
    limit=10.0] (1.53, 2.7) -- (1.62, 2.7);
  
  
  
    \path[draw=c333333,line cap=butt,line join=round,line width=0.04cm,miter 
    limit=10.0] (1.53, 5.75) -- (1.62, 5.75);
  
  
  
    \path[draw=c333333,line cap=butt,line join=round,line width=0.04cm,miter 
    limit=10.0] (1.53, 8.79) -- (1.62, 8.79);
  
  
  
    \path[draw=c333333,line cap=butt,line join=round,line width=0.04cm,miter 
    limit=10.0] (1.53, 11.84) -- (1.62, 11.84);
  
  
  
    \path[draw=c333333,line cap=butt,line join=round,line width=0.04cm,miter 
    limit=10.0] (1.53, 14.88) -- (1.62, 14.88);
  
  
  
    \node[anchor=south,cm={ 0.0,1.0,-1.0,0.0,(0.47, -8.99)}] (text192) at (0.0, 
    17.78){Anteil in Prozent (\%)};
  
  
  
    \begin{scope}[shift={(-0.47, -0.0)}]
      \path[fill=cebebeb,line cap=round,line join=round,line width=0.04cm,miter 
    limit=10.0] (8.67, 16.71) rectangle (9.28, 16.1);
  
  
  
      \path[fill=c77aadd,line cap=butt,line join=miter,line width=0.04cm,miter 
    limit=10.0] (8.7, 16.68) rectangle (9.26, 16.12);
  
  
  
      \path[fill=cebebeb,line cap=round,line join=round,line width=0.04cm,miter 
    limit=10.0] (12.27, 16.71) rectangle (12.88, 16.1);
  
  
  
      \path[fill=ceedd88,line cap=butt,line join=miter,line width=0.04cm,miter 
    limit=10.0] (12.29, 16.68) rectangle (12.85, 16.12);
  
  
  
      \path[fill=cebebeb,line cap=round,line join=round,line width=0.04cm,miter 
    limit=10.0] (16.22, 16.71) rectangle (16.83, 16.1);
  
  
  
      \path[fill=cee8866,line cap=butt,line join=miter,line width=0.04cm,miter 
    limit=10.0] (16.24, 16.68) rectangle (16.8, 16.12);
  
  
  
      \node[shift={(-0.11, -0.69)},anchor=south west] (text198) at (9.48, 
    16.96){\gls{forschungsdaten}-Richtlinie};
  
  
  
      \node[shift={(0.42, -0.69)},anchor=south west] (text199) at (12.54, 
    16.96){\gls{gwp}-Richtlinie};
  
  
  
      \node[shift={(0.95, -0.69)},anchor=south west] (text200) at (15.96, 
    16.96){Keine};
  
  
  
    \end{scope}
  
  \end{tikzpicture}
    \caption{test}
    \label{fig:policy-klassifikation-allgemein-absolut}
\end{figure}

\subsection{Allgemeingültige Dokumente}

\subsection{Promotionsspezifische Dokumente}

\section{Diskussion}\label{sec:policy-discussion}