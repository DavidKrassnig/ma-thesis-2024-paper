\chapter{Richtlinien zu Forschungsdaten aus deutschen Promotionsvorhaben}\label{ch:richtlinien}

\parsum{Thema des Kapitels}
Dieses Kapitel behandelt die verschiedenen verwaltungsrechtlichen Dokumente wissenschaftlicher Institutionen, die ein Promotionsvorhaben in Bezug auf \gls{fdm} entweder spezifisch oder auch nur allgemein betreffen.
Es wird überprüft, inwiefern promotionsberechtigte Institutionen in Deutschland bereits derlei Richtlinien erlassen haben, in welcher Form diese existieren und welche Anforderungen diese stellen.

\parsum{Aufbau des Kapitels}
Hierfür wird in \cref{sec:policy-material-methods} aufgeführt, wie die zu untersuchenden Institutionen ausgewählt wurden, wie welche Materialien der Institutionen ausgesucht wurden und mit welchen Methoden das gesammelte Material daraufhin ausgewertet wurde.
In \cref{sec:policy-results} werden die entsprechenden Ergebnisse der Materialauswertung dargestellt.
Abschließend werden in \cref{sec:policy-discussion} die dargestellten Ergebnisse evaluiert und diskutiert.

\section{Material \&\ Methoden}\label{sec:policy-material-methods}
\parsum{Aufbau des Abschnitts}
In diesem Abschnitt wird das zu untersuchende Material in \cref{sec:policy-material} und die Methoden der Untersuchung in \cref{sec:policy-methods} dargestellt.
\subsection{Material}\label{sec:policy-material}
\parsum{Datengrundlage}
Da es in Deutschland hierzu keine offizielle und öffentlich zugängliche Liste aller Universitäten mit Promotionsrecht seitens des Bundesministeriums für Bildung gibt, wird als Datengrundlage für dieses Kapitel die von der \citeauthor{Hochschulkompass-Liste} geführte Liste aller wissenschaftlichen Institutionen aus dem tertiären Bildungsbereich in Deutschland \autocite{Hochschulkompass-Liste} genutzt.
Diese Liste wird tagesaktuell geführt, basiert auf der Selbstauskunft aller involvierten Institutionen ($n=428$), kodifiziert unter anderem welche Institutionen das Promotions- und Habilationsrecht führen, umfasst auch Institutionen, die nicht Mitglied der \citeauthor{Hochschulkompass-Liste} sind und besitzt einen \textit{de facto} wenn auch nicht \textit{de jure} Status als Datengrundlage für allgemeine Informationen zu wissenschaftlichen Institutionen aus Deutschland. Für die tagesspezifische Version der Liste, die für diese Arbeit genutzt wurde, siehe \fxfatal*{ADD LINK TO DIGITAL APPENDIX!}{FIXME!}.

\parsum{Grundmengenbeschreibung}
Um die zentrale Forschungsfrage dieses Kapitels zu beantworten, wurde diese Liste auf nur jene Institutionen gefiltert, welche das Promotionsrecht besitzen ($n=163$).
Die resultierende Liste promotionsberechtigter Institutionen besteht aus Forschungsinstutionen verschiedener Hochschultypen sowie unterschiedlicher Trägerschaften.
Von den Hochschultypen her umfasst die Liste Universitäten, \glspl{fh}, \glspl{haw}, \glspl{kh}, eine \gls{vh} sowie eine \gls{hset}.
Von den Trägerschaften her umfasst die Liste öffentlich-rechtliche, private sowie kirchliche Institutionen. Alle Institutionen sind staatlich anerkannt.
Die relative sowie die absolute Distribution aller promotionsberechtigter Institutionen in Deutschland nach Hochschultyp und Trägerschaft ist in \cref{tab:grundmenge-beschreibung-art} gegeben.
\begin{table}[!htbp]
	\caption{Die Verteilung aller promotionsberechtigter Institutionen in Deutschland nach $\text{\textit{Hochschultyp}}\times\text{\textit{Trägerschaft}}$ aufgegliedert. Absolute Werte in Klammern angegeben.}
    \resizebox{\ifdim\width>\textwidth\textwidth\else\width\fi}{!}{%
        \begin{tabular}{lS[table-format=3.2]@{\,}S[table-text-alignment = left]lS[table-format=3.2]@{\,}S[table-text-alignment = left]lS[table-format=3.2]@{\,}S[table-text-alignment = left]lS[table-format=3.2]@{\,}S[table-text-alignment = left]l}
            \toprule
            & \multicolumn{3}{c}{\textbf{Öffentlich-Rechtlich}} & \multicolumn{3}{c}{\textbf{Privat}} & \multicolumn{3}{c}{\textbf{Kirchlich}} & \multicolumn{3}{c}{\textbf{Summe}}    \\
            \midrule
            \textbf{Universität}  & 53,37 & \si{\percent} & (87)  & 7,98 & \si{\percent} & (13) & 6,13 & \si{\percent} & (10) & 67,48  & \si{\percent} & (110) \\
            \textbf{\gls{fh} / \gls{haw}}     & 6,75  & \si{\percent} & (11)  & 0,00 & \si{\percent} & (0)  & 0,00 & \si{\percent} & (0)  & 6,75   & \si{\percent} & (11)  \\
            \textbf{\gls{kh}}          & 23,93 & \si{\percent} & (39)  & 0,61 & \si{\percent} & (1)  & 0,00 & \si{\percent} & (0)  & 24,54  & \si{\percent} & (40)  \\
            \textbf{\gls{hset}}         & 0,61  & \si{\percent} & (1)   & 0,00 & \si{\percent} & (0)  & 0,00 & \si{\percent} & (0)  & 0,61   & \si{\percent} & (1)   \\
            \textbf{\gls{vh}}          & 0,61  & \si{\percent} & (1)   & 0,00 & \si{\percent} & (0)  & 0,00 & \si{\percent} & (0)  & 0,61   & \si{\percent} & (1)   \\\midrule
            \textbf{Summe}        & 85,28 & \si{\percent} & (139) & 8,59 & \si{\percent} & (14) & 6,13 & \si{\percent} & (10) & 100,00 & \si{\percent} & (163) \\
            \bottomrule
        \end{tabular}
    }
	\label{tab:grundmenge-beschreibung-art}
\end{table}

\noindent Geografisch gesehen sind, in der gefilterten Liste, zu unterschiedlich hohen Anteilen, Institutionen aus allen deutschen Bundesländern vertreten.
Die genaue Verteilung ist in \cref{fig:DE-grundmenge-beschreibung} wiedergegeben.
\begin{figure}[!htbp]
    \centering
    \begin{tikzpicture}[y=1cm, x=1cm, yscale=.8,xscale=.8, every node/.append style={scale=1}, inner sep=0pt, outer sep=0pt]
  \footnotesize
  \drawgermany
  \drawbw{colorblindC1!93}{25}
  \drawbav{colorblindC1!78}{21}
  \drawbrandenburg{colorblindC1!15}{4}
  \drawhessen{colorblindC1!59}{16}
  \drawmecklenburg{colorblindC1!11}{3}
  \drawniedersachsen{colorblindC1!48}{13}
  \drawnrw{colorblindC1!100}{27}
  \drawrheinland{colorblindC1!30}{8}
  \drawsaarland{colorblindC1!11}{3}
  \drawsachsen{colorblindC1!33}{9}
  \drawsachsenanhalt{colorblindC1!26}{7}
  \drawschleswig{colorblindC1!19}{5}
  \drawthuringen{colorblindC1!19}{5}
  \drawbremen{colorblindC1!7}{2}
  \drawhamburg{colorblindC1!30}{8}
  \drawberlin{colorblindC1!26}{7}
\end{tikzpicture}
    \caption{Die absolute Anzahl promotionsberechtigter Institutionen nach Bundesland.}
    \label{fig:DE-grundmenge-beschreibung}
\end{figure}

\noindent Für die gesamte Liste promotionsberechtigter Institutionen, siehe \fxfatal*{ADD LINK TO DIGITAL APPENDIX!}{FIXME!}

\parsum{Stichprobenziehung}
Diese Liste promotionsberechtigter Institutionen bildete die Grundmenge für die Ziehung einer einfachen Zufallsstichprobe.
Bei der Auswahl der Stichprobe wurde ein Konfidenzintervall von \SI{95}{\percent} und eine Fehlerspanne von \SI{5}{\percent} zugrunde gelegt.
Diese Parameter gewährleisten, dass die Ergebnisse der Stichprobe mit hoher Wahrscheinlichkeit repräsentativ für die gesamte Population sind und die Unsicherheit der Schätzungen innerhalb akzeptabler Grenzen bleibt.
Um den Prozess der Stichprobenziehung zu automatisieren und eine zufällige Auswahl zu gewährleisten, wurde eine auf Python basierende Software \autocite{Krassnig2024-csv} genutzt, welche im Rahmen dieser Arbeit geschrieben wurde.%
\footnote{%
Die Software von \citeauthor{Krassnig2024-csv} \autocite{Krassnig2024-csv} nutzt standardmäßig die Anzahl an Nanosekunden seit dem Beginn der System-Epoche (1970-01-01T00:00:00Z) als Startwert für die Zufallsfunktion.
Der genutzte Startwert wird als begleitendes Metadatum der Stichprobe abgespeichert.
Die Ziehung ist somit wiederholbar und das Datum der Ziehung verifizierbar.} 

\parsum{Stichprobenbeschreibung}
Die so gezogene Stichprobe ($n=115$) besteht aus ca. \SI{71}{\percent} aller promotionsberechtigter Institutionen.
Die Stichprobe umfasst Institutionen aller Trägerschaften aus der Grundmenge: öffentlich-rechtliche, private sowie kirchliche Institutionen.
Darüber hinaus umfasst die Stichprobe von den Hochschultypen her Universitäten, \glspl{fh}, \glspl{haw} und eine \gls{hset}.
In der Stichprobe befindet sich nicht die \gls{vh}, die sich in der Grundmenge befindet.
Mit dieser Ausnahme sind somit alle anderen Hochschultypen vertreten.
Die relative sowie die absolute Distribution aller Institutionen in der Stichprobe nach Hochschultyp und Trägerschaft ist in \cref{tab:stichprobe-beschreibung-art} gegeben.
\begin{table}[!htbp]
	\caption{Die Verteilung der Institutionen in der Stichprobe nach $\text{\textit{Hochschultyp}}\times\text{\textit{Trägerschaft}}$ aufgegliedert. Absolute Werte in Klammern angegeben.}
    \resizebox{\ifdim\width>\textwidth\textwidth\else\width\fi}{!}{%
        \begin{tabular}{lS[table-format=3.2]@{\,}S[table-text-alignment = left]lS[table-format=3.2]@{\,}S[table-text-alignment = left]lS[table-format=3.2]@{\,}S[table-text-alignment = left]lS[table-format=3.2]@{\,}S[table-text-alignment = left]l}
            \toprule
            & \multicolumn{3}{c}{\textbf{Öffentlich-Rechtlich}} & \multicolumn{3}{c}{\textbf{Privat}} & \multicolumn{3}{c}{\textbf{Kirchlich}} & \multicolumn{3}{c}{\textbf{Summe}}    \\
            \midrule
            \textbf{Universität}  & 55,65 & \si{\percent} & (64)  & 8,70 & \si{\percent} & (10) & 4,35 & \si{\percent} & (5) & 68,70  & \si{\percent} & (79) \\
            \textbf{\gls{fh} / \gls{haw}}     & 6,96  & \si{\percent} & (8)  & 0,00 & \si{\percent} & (0)  & 0,00 & \si{\percent} & (0)  & 6,96   & \si{\percent} & (8)  \\
            \textbf{\gls{kh}}          & 22,61 & \si{\percent} & (26)  & 0,87 & \si{\percent} & (1)  & 0,00 & \si{\percent} & (0)  & 23,48  & \si{\percent} & (27)  \\
            \textbf{\gls{hset}}         & 0,87  & \si{\percent} & (1)   & 0,00 & \si{\percent} & (0)  & 0,00 & \si{\percent} & (0)  & 0,87   & \si{\percent} & (1)   \\
            \textbf{\gls{vh}}          & 0,00  & \si{\percent} & (0)   & 0,00 & \si{\percent} & (0)  & 0,00 & \si{\percent} & (0)  & 0,00   & \si{\percent} & (0)   \\\midrule
            \textbf{Summe}        & 86,09 & \si{\percent} & (99) & 9,57 & \si{\percent} & (11) & 4,35 & \si{\percent} & (5) & 100,00 & \si{\percent} & (115) \\
            \bottomrule
        \end{tabular}
    }
	\label{tab:stichprobe-beschreibung-art}
\end{table}

\noindent Geografisch gesehen sind, in der Stichprobe, zu unterschiedlich hohen Anteilen, Institutionen aus allen deutschen Bundesländern vertreten.
Die genaue Verteilung ist in \cref{fig:DE-stichprobe-beschreibung} wiedergegeben.
\begin{figure}[!htbp]
    \centering
    \begin{tikzpicture}[y=1cm, x=1cm, yscale=.8,xscale=.8, every node/.append style={scale=1}, inner sep=0pt, outer sep=0pt]
  \node[text=black,line width=0.0092cm,anchor=west] (title1) at (0,7){\textbf{(A)}};
  \footnotesize
  \drawgermany
  \drawbw{colorblindC1!77}{17}
  \drawbav{colorblindC1!55}{12}
  \drawbrandenburg{colorblindC1!18}{4}
  \drawhessen{colorblindC1!45}{10}
  \drawmecklenburg{colorblindC1!9}{2}
  \drawniedersachsen{colorblindC1!50}{11}
  \drawnrw{colorblindC1!100}{22}
  \drawrheinland{colorblindC1!23}{5}
  \drawsaarland{colorblindC1!9}{2}
  \drawsachsen{colorblindC1!23}{5}
  \drawsachsenanhalt{colorblindC1!27}{6}
  \drawschleswig{colorblindC1!9}{2}
  \drawthuringen{colorblindC1!9}{2}
  \drawbremen{colorblindC1!9}{2}
  \drawhamburg{colorblindC1!36}{8}
  \drawberlin{colorblindC1!23}{5}
\end{tikzpicture}\hfill%
\begin{tikzpicture}[y=1cm, x=1cm, yscale=.8,xscale=.8, every node/.append style={scale=1}, inner sep=0pt, outer sep=0pt]
  \node[text=black,line width=0.0092cm,anchor=west] (title1) at (-0.4,7){\textbf{(B)}};
  \footnotesize
  \drawgermany
  \drawbw{colorblindC1!68}{\SI{68}{\percent}}
  \drawbav{colorblindC1!57}{\SI{57}{\percent}}
  \drawbrandenburg{colorblindC1!100}{\SI{100}{\percent}}
  \drawhessen{colorblindC1!63}{\SI{63}{\percent}}
  \drawmecklenburg{colorblindC1!67}{\SI{67}{\percent}}
  \drawniedersachsen{colorblindC1!85}{\SI{85}{\percent}}
  \drawnrw{colorblindC1!81}{\SI{81}{\percent}}
  \drawrheinland{colorblindC1!63}{\SI{63}{\percent}}
  \drawsaarland{colorblindC1!67}{\SI{67}{\percent}}
  \drawsachsen{colorblindC1!56}{\SI{56}{\percent}}
  \drawsachsenanhalt{colorblindC1!86}{\SI{86}{\percent}}
  \drawschleswig{colorblindC1!40}{\SI{40}{\percent}}
  \drawthuringen{colorblindC1!40}{\SI{40}{\percent}}
  \drawbremen{colorblindC1!100}{\SI{100}{\percent}}
  \drawhamburg{colorblindC1!100}{\SI{100}{\percent}}
  \drawberlin{colorblindC1!71}{\SI{71}{\percent}}
\end{tikzpicture}

    \caption{Verteilung der Institutionen in der gezogenen Stichprobe nach Bundesland. \textbf{(A)}~Die absolute Anzahl der Institutionen nach Bundesland. \fxfatal*{Deutlicher formulieren!}{\textbf{(B)}~Der relative Anteil der aus der Grundmenge gezogenen Institutionen nach Bundesland.}}
    \label{fig:DE-stichprobe-beschreibung}
\end{figure}

\parsum{Dokumente der Stichprobe}
Für die Evaluation, inwiefern die Institutionen der Stichprobe verwaltungsrechtliche Dokumente besitzen, die entweder allgemeine \gls{fdm}-Richtlinien für alle Forschenden der Institution oder spezifische Regelungen für Promovierende beinhalten, wurde deren gesamte öffentlich zugängliche Online-Präsenz nach relevanten Dateien und Webseiten durchsucht.
Diese Suche fand im Allgemeinen auf zwei Wege statt: über die interne Suchfunktion der Institution und die externe Durchsuchung der Institutions-Domäne via der Suchmaschine \href{https://www.duckduckgo.com/}{DuckDuckGo}.

\parsum{Allgemeine Dokumente}
Für allgemeingültige Richtlinien wurden in erster Linie eigenständige \gls{fdm}-Richtlinien gesucht.
Hierfür wurde, jenseits der allgemeinen Suchmethode auch die dedizierte Forschungsdatenpolicies-Liste der Informationsplattform \citeauthor{Forschungsdaten2024} genutzt \autocite{Forschungsdaten2024}.%
\footnote{%
    Die \gls{fdm}-Richtlinien, welche sich nicht auf diesem Portal haben finden lassen, werden vom Autor nach Beendigung dieser Arbeit dort nach Möglichkeit eingetragen.%
}
Jenseits von spezifischen \gls{fdm}-Richtlinien wurden auch Richtlinien zur Sicherung guter wissenschaftlicher Praxis sowie andere Richtlinien gesucht, die anderweitige allgemeine wissenschaftliche Empfehlungen bzw. Auflagen für Forschende aussprechen.
Wenn es von einem Dokument sowohl eine HTML- wie auch eine PDF-Datei gab, so wurde nur die PDF-Datei zur Evaluation weitergenutzt.
Hierbei wurden insgesamt 142 Dokumente zur Weiterverarbeitung aufgenommen.

\parsum{Promotionsspezifische Dokumente}
Für promotionsspezifische Richtlinien wurden Promotions- und Prüfungsordnungen gesucht.
Dabei wurden sowohl fachspezifische Ordnungen wie auch verbindliche Rahmenbedingungen und anderweitige übergreifende Ordnungen aufgenommen.
Die heterogene Handhabung dieser Dokumente seitens der Institutionen führte zu folgenden Selektionsregeln bei der Auswahl:
Forschungsdaten.org
(i)~Wenn es eine aktuelle Lesefassung der Promotionsordnung gibt, so wird diese bevorzugt.
(ii)~Sollte es keine aktuelle Lesefassung der Promotionsordnung geben, so wird die aktuellste Gesamtversion der Promotionsordnung bevorzugt.
(iii)~Sollte es keine aktuelle Gesamtversion geben, so werden zusätzlich zu der letzten Version der Promotionsordnung auch alle seither erschienen relevanten verwaltungsrechtlichen Addenda aufgenommen.
(iv)~Wenn es von einem Dokument sowohl eine HTML- wie auch eine PDF-Datei gab, so wurde nur die PDF-Datei zur Evaluation weitergenutzt.
Hierbei wurden insgesamt 754 Dokumente zur Weiterverarbeitung aufgenommen.

\subsection{Methoden}\label{sec:policy-methods}


\section{Resultate}\label{sec:policy-results}

\section{Diskussion}\label{sec:policy-discussion}