\chapter{Datenmanagementplan}\label{appendix:dmp}
\section{Allgemein}
\subsection*{Thema}
\subsubsection*{Wie lautet die primäre Forschungsfrage der Abschlussarbeit?}
\subsubsection*{Bitte geben Sie einige Schlagwörter1 zur Forschungsfrage bzw. Fragestellung an.}
\subsubsection*{Welchen Regeln oder Richtlinien (HU) zum Umgang mit den in der Abschlussarbeit erhobenen Forschungsdaten folgen Sie für den DMP? Bitte referenzieren Sie diese hier inklusive Version bzw. Veröffentlichungsjahr.}

\section{Inhaltliche Einordnung}
NB: Bitte beschreiben Sie jeden Datensatztyp oder Datensammlung einzeln in dem jeweiligen Kapitel, wo sinnvoll.
\subsection*{Datensatz}
\subsubsection*{Um welche Arten von Daten handelt es sich? Bitte in wenigen Zeilen kurz beschreiben.}
\subsection*{Datenursprung}
\subsubsection*{Werden die Daten selbst erzeugt oder nachgenutzt?}
\subsubsection*{Wenn die Daten nachgenutzt werden, wer hat die Daten erzeugt? Bitte mit Angabe des Identifiers, falls vorhanden, z.B. DOI1.}
\subsection*{2.3. Reproduzierbarkeit}
\subsubsection*{Sind die Daten reproduzierbar, d. h. ließen sie sich, wenn sie verloren gingen, erneut erstellen oder erheben?}
\subsection*{2.4 Nachnutzung}
\subsubsection*{Für welche Personen, Gruppen oder Institutionen könnte dieser Datensatz (für die Nachnutzung) von Interesse sein? Für welche Szenarien ist dies denkbar?}

\section{Technische Einordnung}
\subsection*{Datenerhebung}
\subsubsection*{Wann erfolgt(e) die Erhebung bzw. Erstellung der Daten?}
\subsubsection*{Wann erfolgt(e) die Datenbereinigung / -aufbereitung bzw. Datenanalyse?}
\subsection*{Datengröße}
\subsubsection*{Was ist die tatsächliche oder erwartete Größe der Daten(typen)?}
\subsection*{Formate}
\subsubsection*{In welchen Formaten1 liegen die Daten vor?}
\subsection*{Werkzeuge}
\subsubsection*{Welche Instrumente, Software, Technologien oder Verfahren werden zur Erzeugung, Erfassung, Bereinigung, Analyse und/oder Visualisierung der Daten genutzt? Bitte (falls möglich) mit Versionsnummer und Referenz in Form einer Adresse jeweils angeben.}
\subsubsection*{Welche Software, Verfahren oder Technologien sind notwendig, um die Daten zu nutzen?}
\subsection*{Versionierung}
\subsubsection*{Werden verschiedene Versionen der Daten erzeugt (z.B. durch verschiedene Weiterbearbeitungsprozesse bzw. Bereinigung von Daten)?}

\section{Datennutzung}
\subsection*{Datenorganisation}
\subsubsection*{Gibt es eine Strategie zur Benennung der Daten? Wenn ja, bitte skizzieren Sie sie kurz.}
\subsection*{Datenspeicherung und -sicherheit}
\subsubsection*{Wer darf (zukünftig) auf die Daten zugreifen?}
\subsubsection*{Wie und wie oft werden Backups der Daten erstellt? }
\subsection*{Interoperabilität}
\subsubsection*{Sind die Datenformate im Sinne der FAIR-Prinzipien interoperabel, d.h. geeignet für den Datenaustausch und die Nachnutzung zwischen bzw. von unterschiedlichen Forschenden, Institutionen, Organisationen und Ländern?}
\subsection*{Weitergabe und Veröffentlichung}
\subsubsection*{Ist es geplant, die Daten nach Abgabe der Abschlussarbeit zu veröffentlichen oder zu teilen?}
\subsubsection*{Wenn nicht, skizzieren Sie kurz rechtliche und/oder vertragliche Gründe und freiwillige Einschränkungen.}
\subsubsection*{Wenn ja, unter welchen Nutzungsbedingungen oder welcher Lizenz sollen die Daten veröffentlicht bzw. geteilt werden?}
\subsection*{Qualitätssicherung}
\subsubsection*{Welche Maßnahmen zur Qualitätssicherung (z.B. Plausibilitätsprüfung von Datenwerten) werden für die Daten ergriffen?}
\subsection*{Datenintegration}
\subsubsection*{Falls Daten aus verschiedenen Quellen (z.B. Anpassung Skalierung, Zeiträume, Ortsangaben) integriert werden, wie wird dies gewährleistet?}

\section{Metadaten und Referenzierung}
\subsection*{Metadaten}
\subsubsection*{Welche Informationen sind für Außenstehende notwendig, um die Daten zu verstehen (d. h. ihre Erhebung bzw. Entstehung, Analyse sowie die auf ihrer Basis gewonnenen Forschungsergebnisse nachvollziehen) und nachnutzen zu können?}
\subsubsection*{Welche Standards, Ontologien, Klassifikationen etc. werden zur Beschreibung der Daten und Kontextinformation genutzt?}

\section{Rechtliche und ethische Fragen}
\subsection*{Personenbezogene Daten}
\subsubsection*{Enthalten die Daten personenbezogene Informationen?}
\subsection*{Sensible Daten}
\subsubsection*{Enthalten die Forschungsdaten besondere Kategorien personenbezogener Daten nach Artikel 9 der DSGVO (“Angaben über die rassische und ethnische Herkunft, politische Meinungen, religiöse oder philosophische Überzeugungen, Gewerkschaftszugehörigkeit, Gesundheit oder Sexualleben”)1?}
\subsubsection*{Werden die Daten anonymisiert oder pseudonymisiert?}
\subsubsection*{Haben Sie eine "informierte Einwilligung" der Betroffenen eingeholt? Fügen Sie bitte ein Template der Einverständniserklärung als Anlage bei.}
\subsubsection*{Wenn keine "informierte Einwilligung" eingeholt wird, begründen Sie dies bitte.}
\subsubsection*{Wo und wie sind die "informierten Einwilligungen" abgelegt?}
\subsubsection*{Bis wann werden die (un-anonymisierten bzw. un-pseudonymisierten) Originaldaten spätestens sicher vernichtet?}
\subsection*{Urheber- oder verwandte Schutzrechte}
\subsubsection*{Werden Daten genutzt und/oder erstellt, die durch Urheber- oder verwandte Schutzrechte geschützt sind?}

\section{Speicherung und Langzeitarchivierung}
\subsection*{Wo werden die Daten (einschließlich Metadaten, Dokumentation und ggf. relevantem Code bzw. relevanter Software) während Phase der Erarbeitung der Abschlussarbeit gespeichert?}
\subsection*{Wo werden die Daten (einschließlich Metadaten, Dokumentation und ggf. relevantem Code bzw. relevanter Software) nach dem Ende der Abschlussarbeit gespeichert bzw. archiviert?}
\subsection*{Handelt es sich dabei um ein zertifiziertes Repositorium oder Datenzentrum (z.B. durch das CoreTrustSeal, nestor-Siegel oder ISO~163634)? (Wurden mehrere Langzeitarchivierungsoptionen ausgewählt, kann die Frage bejaht werden, wenn dies auf mindestens eine der Optionen zutrifft).}