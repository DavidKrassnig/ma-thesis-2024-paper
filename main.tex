% !TeX TS-program = lualatex
% !TEX program = lualatex
% !TeX encoding = UTF-8
% TeX Live Version >=2022
% Diese Vorlage ist lizensiert unter GNU GENERAL PUBLIC LICENSE v3 (29 June 2007)
%
%% Anmerkungen %%
% Diese Vorlage wird weder von der HU-Berlin befürwortet, noch ist sie direkt mit ihr verbunden, autorisiert oder wird auf irgendeine Arte und Weise von ihr gesponsert. 
%
% Vor der Nutzung dieser Vorlage, sollte überprüft werden, ob sie noch den gültigen Richtlinien der HU-Berlin und des entsprechenden Studiengangs entspricht (Aktueller Stand: 2023-07-29 für MA(LIS) Bibliotheks- und Informationswissenschaft)
%
%% Dokumentkonfiguration %%
\directlua{pdf.setminorversion(7)} %Notwendig für korrekte Versionsanzeige wegen LuaLaTeX+pdfx Interaktion 
\documentclass[%
    fontsize=11pt,          %Schriftgröße festlegen
    a4paper,                %Seitenformat auf A4 setzen
    openright,              %Kapitel sollen auf der rechten Seite beginnen
    twoside=semi,           %Zweiseitiges Layout ohne abwechselnde Ränder
    headsepline,            %Trennlinie unterhalb des Headers hinzufügen
    titlepage=firstiscover, %\maketitle als Titelseite anzeigen
    numbers=noenddot,       %Nummerierung nach ISO 2145
    bibliography=totoc,     %Bibliographie zum Inhaltsverzeichnis hinzufügen
    listof=totoc,           %Abbildungs- und Tabellenverzeichnis im Inhaltsverzeichnis
    captions=tableheading,  %Setzt Formatierung für Tabellenüberschriften
    toc=index,              %Index zum Inhaltsverzeichnis hinzufügen
    hidelinks,				%Um Hyperlink-Boxen zu deaktivieren, falls Unterstrich gelöscht wird
    pdfa,                   %PDF/A Voreinstellungen
    headings=twolinechapter,%Zweizweilige Kapiteltitel
    %twocolumn,              %Für zweispaltige Darstellung
    draft,                  %Auskommentieren für den Entwurfsmodus
    %final                   %Auskommentieren für den endgültigen Druck
    ]{scrbook}              %Verwendung der KOMA-Script book-Klasse

%% Dokumentpaketeausschluss
\PreventPackageFromLoading{fontenc,lmodern} %Sicherstellen dass Standardfonts nicht geladen werden

%% Dokumentpakete
%\usepackage{showframe}%Anzeige der Layout-Boxen
\usepackage{lipsum}   %Generierung von Dummy-Text
\usepackage[a-2b,mathxmp]{pdfx} %PDF/A-2b erzeugen
\usepackage[automark,draft=false]{scrlayer-scrpage} %Detaillierte Konfiguration von Kopf- und Fußzeilen
\usepackage[main=ngerman,british]{babel}%Sprache auf Deutsch setzen
\usepackage{fontspec}                   %LuaLaTeX Standard-Fontpackage
\usepackage{amsmath}                    %Erweiterte Mathematik
\usepackage{unicode-math}               %Unicode-Mathematikmodus erzwingen
    \setmainfont{XCharter}       %Allgemeine Schriftart setzen
    \setmathfont{XCharter Math}  %Mathematische Schriftart setzen
    \setsansfont[Scale=MatchLowercase]{Tex Gyre Heros} %Serifenlose Schriftart setzen
    \setmonofont[Scale=MatchLowercase]{DejaVu Sans Mono} %Nicht-proportionale Schriftart setzen
\usepackage{xcharter-otf} %XCharter-Einstellungen (macht obige Einstellungen teils redundant)
\usepackage{xcolor}   %Verbessertes Farbmanagement
\usepackage[%
    english=british,  %Anführungszeichen für Englisch auf britischen Stil setzen
    autostyle=true    %Anführungszeichen automatisch gemäß Sprache setzen
    ]{csquotes}       %Automatische Formatierung von Anführungszeichen
%\input{config/citation_author-year} %Dekommentiere diese Linie für Autor-Jahr Zitate
\usepackage[backend=biber, style=config/iso690-numeric_2021, sorting=none, sortcites=true, bibnamessc=true, abbreviate=false, doi=true]{biblatex}
\setcounter{biburllcpenalty}{7000}
\setcounter{biburlucpenalty}{8000} %Dekommentiere diese Linie für numerische Zitate
\usepackage[%
    nameinlink, %Präfix in den Verlinkungen der Referenzen anzeigen
    ngerman,    %Sprache auf Deutsch setzen; unabhängig von Babel erforderlich
    capitalise, %Erster Buchstabe groß (nur wichtig für Englisch)
    noabbrev,   %Deaktiviert Abkürzungen wie Abb. für Abbildung
    ]{cleveref} %Automatische Präfix-Referenzierung
\usepackage[usetransparent=true]{svg} %Erlaubt Inkludierung von SVG Dateien
\usepackage{setspace}                  %Zeilenabstand festlegen
\usepackage{caption}                   %Anpassung der Beschriftungsformatierung
    %\captionsetup[table]{skip=10pt}    %Abstand zwischen Tabelle und Unterschrift
    \captionsetup{labelfont=bf} %Unterschrift in Fettschrift setzen
\usepackage{booktabs} %Bessere Tabellenästhetik
\usepackage{tabularx} %Tabellen mit Zeilenumbruch
\usepackage{array}    %Tabellenfeineinstellungen
\usepackage{multirow} %Tabellenzeilen kombinieren
\usepackage{multicol} %Mehrspaltige Dokumentdarstellung
\usepackage{enumitem}
    \setitemize[0]{leftmargin=*,itemsep=.05\baselineskip}
    \setenumerate[0]{leftmargin=*,itemsep=.05\baselineskip}
\usepackage{adjustbox}%Bessere Version von \resizebox
\usepackage[acronym,automake]{glossaries-extra}%Abkürzung- und Begriffverwaltung
    \makeglossaries
    \setabbreviationstyle[acronym]{long-short}
    \newacronym[\glslongpluralkey={Forschungsdaten}]{forschungsdaten}{FD}{Forschungsdatum}
\newacronym[\glslongpluralkey={Hochschulschriften}]{hochschulschriften}{HSS}{Hochschulschrift}
\newacronym[description={Institutionelles Repositorium der Leibniz Universität Hannover}]{luh-repo}{LUH-Repo}{institutionellen Repositorium der Leibniz Universität Hannover}
\newacronym[description={Technische Informationsbibliothek / Leibniz-Informationszentrum Technik und Naturwissenschaften und Universitätsbibliothek}]{tib}{TIB}{Technischen Informationsbibliothek}
\newacronym{luh}{LUH}{Leibniz Universität Hannover}
\newacronym{fdm}{FDM}{Forschungsdatenmanagement}
\newacronym{fohop}{FoHop!}{\enquote{Forschungsdaten von Hochschulschriften publizieren}}
\newacronym[\glslongpluralkey={Fachhochschulen}]{fh}{FH}{Fachhochschule}
\newacronym[\glslongpluralkey={Hochschulen für Angewandte Wissenschaften}]{haw}{HAW}{Hochschule für Angewandte Wissenschaften}
\newacronym[\glslongpluralkey={Künstlerische Hochschulen}]{kh}{KH}{Künstlerische Hochschule}
\newacronym[\glslongpluralkey={Hochschulen eigenen Typs}]{hset}{HSeT}{Hochschule eigenen Typs}
\newacronym[\glslongpluralkey={Verwaltungshochschule}]{vh}{VH}{Verwaltungshochschule}
\newacronym[description={Gute wissenschaftliche Praxis}]{gwp}{GWP}{guten wissenschaftlichen Praxis}
\newacronym{fair}{FAIR}{\textbf{F}indable, \textbf{A}ccessible, \textbf{I}nteroperable und \textbf{R}eusable}
\newacronym{oai-pmh}{OAI-PMH}{\textbf{O}pen \textbf{A}rchives \textbf{I}nitiative \textbf{P}rotocol for \textbf{M}etadata \textbf{H}arvesting}
\newacronym{fakultät2}{Fakultät~ARC}{Fakultät für Architektur und Landschaft}
\newacronym{fakultät3}{Fakultät~BAU}{Fakultät für Bauingenieurwesen und Geodäsie}
\newacronym{fakultät4}{Fakultät~INF}{Fakultät für Elektrotechnik und Informatik}
\newacronym{fakultät5}{Fakultät~JUR}{Juristische Fakultät}
\newacronym{fakultät6}{Fakultät~MAS}{Fakultät für Maschinenbau}
\newacronym{fakultät7}{Fakultät~MAT}{Fakultät für Mathematik und Physik}
\newacronym{fakultät8}{Fakultät~NAT}{Naturwissenschaftliche Fakultät}
\newacronym{fakultät9}{Fakultät~PHI}{Philosophische Fakultät}
\newacronym{fakultät10}{Fakultät~WIWI}{Wirtschaftswissenschaftliche Fakultät}
\newacronym{doi}{DOI}{\textbf{D}igital \textbf{O}bject \textbf{I}dentifier}
\newacronym[\glslongpluralkey={Primärdaten}]{pd}{PD}{Primärdatum}
\newacronym[\glslongpluralkey={Sekundärdaten}]{sd}{SD}{Sekundärdaten}
\newacronym[\glsshortpluralkey={PD-Dissertationen},\glslongpluralkey={Dissertationen mit produzierten Primärdaten (mit oder ohne Publikation dieser Primärdaten)}]{pdd}{PD-Dissertation}{Dissertation mit produzierten Primärdaten (mit oder ohne Publikation dieser Primärdaten)}
\newacronym[description={Künstliche Intelligenz}]{ki}{KI}{künstlichen Intelligenz}
\newacronym{dmp}{DMP}{\textbf{D}aten\textbf{m}anagement\textbf{p}lan}

\usepackage{siunitx}  %Bessere Einheitenimplementation
    \sisetup{group-minimum-digits=4,output-decimal-marker={,},locale=DE,}
\usepackage{soul}
\usepackage{pgfplots} %Akademische Diagramme
    \pgfplotsset{compat=1.18}
    \definecolor{colorblindA1}{RGB}{61,150,210}
    \definecolor{colorblindA2}{RGB}{166,206,227}
    \definecolor{colorblindA3}{RGB}{208,253,168}
    \definecolor{colorblindB1}{RGB}{102,194,165}
    \definecolor{colorblindB2}{RGB}{252,141,98}
    \definecolor{colorblindB3}{RGB}{141,160,203}
    \definecolor{colorblindC1}{RGB}{119,170,221} %From https://personal.sron.nl/~pault/#sec:qualitative
    \definecolor{colorblindC2}{RGB}{153,221,225}
    \definecolor{colorblindC3}{RGB}{68,187,153}
    \definecolor{colorblindC4}{RGB}{187,204,51}
    \definecolor{colorblindC5}{RGB}{170,170,0}
    \definecolor{colorblindC6}{RGB}{238,221,136}
    \definecolor{colorblindC7}{RGB}{238,136,102}
    \definecolor{colorblindC8}{RGB}{255,170,187}
    \definecolor{colorblindC9}{RGB}{221,221,221}
    \definecolor{cebebeb}{RGB}{235,235,235}
    \definecolor{cdddddd}{RGB}{221,221,221}
    \definecolor{cee8866}{RGB}{238,136,102}
    \definecolor{caaaa00}{RGB}{170,170,0}
    \definecolor{ceedd88}{RGB}{238,221,136}
    \definecolor{c99dde1}{RGB}{153,221,225}
    \definecolor{c77aadd}{RGB}{119,170,221}
    \definecolor{c333333}{RGB}{51,51,51}
    \definecolor{c4d4d4d}{RGB}{77,77,77}
    \definecolor{c1a1a1a}{RGB}{26,26,26}
    \definecolor{cb2c9e3}{RGB}{178,201,227}
    \definecolor{ce3ecf6}{RGB}{227,236,246}
    \definecolor{c98bfe6}{RGB}{152,191,230}
    \definecolor{c3574b7}{RGB}{53,116,183}
    \definecolor{c83addb}{RGB}{131,173,219}
    \def \globalscale {1.000000}
\usepackage{pgfplotstable}
\usepackage{pgf-pie}
\usepackage{tikz}     %Erweiterte Zeichnungs- und Grafikkapazitäten
\usepackage{pdfpages} %Einbinden von PDF-Dateien
\usepackage{orcidlink} %Einbinden von ORCID
\usepackage[nopatch=footnote]{microtype} %Mikrotypographische ästhetische Verbesserungen
\usepackage{pdfcomment}
\usepackage{listings}
    \input{config/listings.tex}
\usepackage[all]{nowidow}
\usepackage[nomargin,lang=ngerman,singleuser,author=DK]{fixme}
    \fxsetup{draft,layout=pdfcnote,theme=color}
    \definecolor{fxtarget}{rgb}{0.8000,0.0000,0.0000}

%% Neue Befehle und Variabeln
\input{config/possessive_citation} %\posscite-Befehl für besitzanzeigende Zitate im Englischen einbinden
\input{config/titlepage_commands}
\input{config/include_missing_file_error}
\newcommand{\parsum}[1]{\leavevmode\marginpar{\emph{\footnotesize{#1}}}\ignorespaces}
%Kartenbefehle
\definecolor{ce7e7e7}{RGB}{231,231,231}
\newcommand{\drawbw}[2]{%
%Baden-Württemberg
  \path[draw=black,fill=#1,line join=round,line width=0.0046cm] (2.3782, 
  2.1781) -- (2.3871, 2.1746) -- (2.3889, 2.1743) -- (2.4343, 2.1741) -- 
  (2.4462, 2.1764) -- (2.4517, 2.1776) -- (2.4625, 2.1809) -- (2.4678, 2.1839) 
  -- (2.49, 2.2121) -- (2.4893, 2.2143) -- (2.4841, 2.2216) -- (2.4905, 2.2319) 
  -- (2.5021, 2.2487) -- (2.5294, 2.2446) -- (2.5508, 2.2378) -- (2.5643, 
  2.2403) -- (2.5705, 2.248) -- (2.5725, 2.2529) -- (2.5676, 2.2575) -- (2.5746,
   2.2879) -- (2.5706, 2.3033) -- (2.5611, 2.3195) -- (2.5528, 2.3216) -- 
  (2.5495, 2.3163) -- (2.5509, 2.3115) -- (2.5505, 2.2966) -- (2.5476, 2.2958) 
  -- (2.5409, 2.2951) -- (2.5352, 2.2959) -- (2.5271, 2.3082) -- (2.5133, 
  2.3315) -- (2.5089, 2.3486) -- (2.5089, 2.3538) -- (2.5101, 2.356) -- (2.5125,
   2.3571) -- (2.5165, 2.3576) -- (2.5372, 2.3599) -- (2.5882, 2.3623) -- 
  (2.5996, 2.3577) -- (2.6057, 2.3673) -- (2.6094, 2.3721) -- (2.611, 2.3736) --
   (2.6229, 2.3748) -- (2.625, 2.3742) -- (2.6271, 2.3725) -- (2.6296, 2.3701) 
  -- (2.6304, 2.3682) -- (2.6303, 2.3658) -- (2.6287, 2.3605) -- (2.6284, 2.357)
   -- (2.6289, 2.355) -- (2.6311, 2.3514) -- (2.6337, 2.3491) -- (2.6572, 
  2.3338) -- (2.6597, 2.3328) -- (2.6618, 2.3327) -- (2.663, 2.3334) -- (2.7022,
   2.3714) -- (2.7134, 2.3766) -- (2.7147, 2.3676) -- (2.7089, 2.3127) -- 
  (2.7083, 2.3099) -- (2.7033, 2.3074) -- (2.7023, 2.3066) -- (2.6988, 2.2939) 
  -- (2.7034, 2.2872) -- (2.7166, 2.28) -- (2.7216, 2.2807) -- (2.7297, 2.2846) 
  -- (2.754, 2.2985) -- (2.7926, 2.3117) -- (2.8027, 2.3152) -- (2.8184, 2.2997)
   -- (2.8247, 2.2932) -- (2.8465, 2.2165) -- (2.8486, 2.2074) -- (2.8643, 
  2.1826) -- (2.867, 2.1831) -- (2.8713, 2.1835) -- (2.8737, 2.1829) -- (2.8748,
   2.1804) -- (2.8811, 2.1599) -- (2.8814, 2.1589) -- (2.8829, 2.1489) -- 
  (2.8814, 2.1104) -- (2.8811, 2.1093) -- (2.8796, 2.1058) -- (2.8758, 2.1017) 
  -- (2.8746, 2.0956) -- (2.882, 2.0881) -- (2.8837, 2.0871) -- (2.8977, 2.0838)
   -- (2.9349, 2.0862) -- (2.9382, 2.0873) -- (2.958, 2.1094) -- (2.9597, 
  2.1114) -- (2.9646, 2.1199) -- (2.9588, 2.1213) -- (2.9528, 2.1267) -- 
  (2.9505, 2.1297) -- (2.951, 2.1305) -- (2.965, 2.1437) -- (2.9681, 2.1446) -- 
  (2.9784, 2.1427) -- (2.9845, 2.1325) -- (2.9919, 2.1161) -- (2.9946, 2.1037) 
  -- (2.991, 2.1003) -- (2.9889, 2.096) -- (2.9885, 2.0942) -- (2.989, 2.0885) 
  -- (2.9983, 2.0764) -- (2.9996, 2.0759) -- (3.0105, 2.0399) -- (3.0209, 2.019)
   -- (3.0266, 2.0035) -- (3.0259, 2.0015) -- (3.0234, 1.9973) -- (3.0172, 
  1.9944) -- (2.9962, 1.9938) -- (2.9933, 1.9881) -- (2.9906, 1.9557) -- (2.991,
   1.9531) -- (2.9918, 1.95) -- (3.0099, 1.9102) -- (3.0148, 1.9023) -- (3.0061,
   1.8775) -- (3.0041, 1.8204) -- (3.0044, 1.8189) -- (3.0071, 1.8157) -- 
  (3.0308, 1.7994) -- (3.0445, 1.7923) -- (3.0603, 1.7856) -- (3.0793, 1.7513) 
  -- (3.0915, 1.6901) -- (3.0876, 1.6882) -- (3.0863, 1.6882) -- (3.0861, 
  1.6819) -- (3.0876, 1.6723) -- (3.126, 1.6561) -- (3.182, 1.6152) -- (3.1834, 
  1.612) -- (3.1899, 1.594) -- (3.2018, 1.5817) -- (3.2027, 1.5806) -- (3.2071, 
  1.5749) -- (3.2106, 1.5578) -- (3.2094, 1.5013) -- (3.2062, 1.4541) -- (3.206,
   1.4535) -- (3.195, 1.4247) -- (3.1886, 1.422) -- (3.191, 1.3919) -- (3.1929, 
  1.3921) -- (3.2162, 1.3845) -- (3.2168, 1.3766) -- (3.223, 1.3531) -- (3.2288,
   1.3474) -- (3.2308, 1.3475) -- (3.2337, 1.3417) -- (3.2306, 1.3377) -- 
  (3.2171, 1.3229) -- (3.1972, 1.3165) -- (3.1879, 1.3161) -- (3.1876, 1.3163) 
  -- (3.1871, 1.3185) -- (3.1881, 1.3203) -- (3.192, 1.3231) -- (3.2011, 1.343) 
  -- (3.2006, 1.3458) -- (3.1994, 1.3474) -- (3.1962, 1.3498) -- (3.1814, 1.349)
   -- (3.1376, 1.3395) -- (3.1319, 1.3413) -- (3.0948, 1.3531) -- (3.087, 
  1.3307) -- (3.0865, 1.3286) -- (3.086, 1.3163) -- (3.0912, 1.3089) -- (3.102, 
  1.2938) -- (3.114, 1.2773) -- (3.1229, 1.2247) -- (3.1234, 1.2191) -- (3.1226,
   1.1833) -- (3.1128, 1.1776) -- (3.1068, 1.1768) -- (3.1015, 1.1768) -- 
  (3.0929, 1.1806) -- (3.0884, 1.18) -- (3.0862, 1.179) -- (3.072, 1.1715) -- 
  (3.0741, 1.1669) -- (3.076, 1.1617) -- (3.0324, 1.1234) -- (3.0114, 1.119) -- 
  (3.0009, 1.1314) -- (2.9922, 1.1381) -- (2.9619, 1.1208) -- (2.9473, 1.1197) 
  -- (2.9502, 1.1153) -- (2.9516, 1.1088) -- (2.949, 1.0931) -- (2.9377, 1.0725)
   -- (2.9365, 1.0706) -- (2.9227, 1.0381) -- (2.9262, 1.0198) -- (2.9337, 
  1.0058) -- (2.9599, 0.9647) -- (2.9688, 0.9561) -- (2.98, 0.8749) -- (2.9814, 
  0.8677) -- (2.9845, 0.8516) -- (2.9873, 0.8454) -- (2.9873, 0.841) -- (2.9952,
   0.8153) -- (2.9975, 0.8094) -- (2.9983, 0.8081) -- (3.0061, 0.7997) -- 
  (3.0136, 0.7928) -- (3.0153, 0.7616) -- (3.0158, 0.7343) -- (3.0133, 0.7121) 
  -- (3.0119, 0.7059) -- (3.0068, 0.7008) -- (3.0031, 0.6945) -- (3.0009, 
  0.6774) -- (3.0003, 0.668) -- (2.9942, 0.6667) -- (2.9876, 0.6652) -- (2.9852,
   0.6638) -- (2.9837, 0.6621) -- (2.9824, 0.6599) -- (2.9813, 0.6555) -- 
  (2.981, 0.6504) -- (2.9823, 0.6448) -- (2.9843, 0.6411) -- (2.9865, 0.6386) --
   (2.9896, 0.6361) -- (2.997, 0.6313) -- (2.9935, 0.6182) -- (2.9916, 0.6017) 
  -- (2.9914, 0.5985) -- (2.9915, 0.5948) -- (2.9928, 0.5629) -- (2.9972, 
  0.5399) -- (3.002, 0.5299) -- (3.0068, 0.524) -- (3.0109, 0.5212) -- (3.0134, 
  0.514) -- (3.0012, 0.5031) -- (2.992, 0.5029) -- (2.9925, 0.469) -- (3.0158, 
  0.4118) -- (3.0157, 0.4062) -- (3.0095, 0.3857) -- (2.9994, 0.3763) -- 
  (2.9976, 0.3763) -- (2.9912, 0.3776) -- (2.9871, 0.3753) -- (2.9812, 0.3684) 
  -- (2.9791, 0.3647) -- (2.9753, 0.3542) -- (2.9759, 0.3515) -- (2.9679, 
  0.3839) -- (2.9677, 0.3849) -- (2.9646, 0.3901) -- (2.9644, 0.3904) -- 
  (2.9406, 0.3951) -- (2.9284, 0.393) -- (2.9269, 0.3922) -- (2.9247, 0.39) -- 
  (2.915, 0.3779) -- (2.9123, 0.3733) -- (2.8622, 0.3724) -- (2.8184, 0.37) -- 
  (2.8177, 0.3689) -- (2.7672, 0.3175) -- (2.7386, 0.3162) -- (2.7347, 0.3174) 
  -- (2.7324, 0.3192) -- (2.7314, 0.3212) -- (2.731, 0.3246) -- (2.7093, 0.3263)
   -- (2.6979, 0.3186) -- (2.6736, 0.3001) -- (2.6545, 0.3001) -- (2.6468, 
  0.3009) -- (2.6437, 0.3023) -- (2.6408, 0.304) -- (2.6276, 0.3184) -- (2.6203,
   0.3357) -- (2.619, 0.3437) -- (2.6155, 0.3508) -- (2.6095, 0.3591) -- 
  (2.6041, 0.3634) -- (2.559, 0.3777) -- (2.5501, 0.3802) -- (2.5447, 0.3805) --
   (2.5409, 0.38) -- (2.5287, 0.3781) -- (2.5265, 0.3777) -- (2.5216, 0.375) -- 
  (2.5169, 0.3735) -- (2.5136, 0.3737) -- (2.5112, 0.3748) -- (2.5029, 0.3792) 
  -- (2.4844, 0.3893) -- (2.4597, 0.406) -- (2.4502, 0.4144) -- (2.443, 0.4217) 
  -- (2.4382, 0.4299) -- (2.4365, 0.4345) -- (2.4352, 0.4424) -- (2.4361, 0.444)
   -- (2.4358, 0.4474) -- (2.4353, 0.4484) -- (2.4297, 0.4535) -- (2.3596, 
  0.4997) -- (2.3403, 0.512) -- (2.3378, 0.5131) -- (2.3346, 0.5187) -- (2.3319,
   0.5207) -- (2.3179, 0.5231) -- (2.3144, 0.5174) -- (2.3123, 0.5129) -- 
  (2.3122, 0.5117) -- (2.3135, 0.5095) -- (2.3362, 0.4859) -- (2.3432, 0.4803) 
  -- (2.3483, 0.4767) -- (2.3784, 0.4562) -- (2.3795, 0.4559) -- (2.3843, 
  0.4566) -- (2.3872, 0.4563) -- (2.4024, 0.4436) -- (2.4261, 0.3937) -- 
  (2.4297, 0.3791) -- (2.4152, 0.3702) -- (2.4027, 0.3682) -- (2.3794, 0.3777) 
  -- (2.3584, 0.389) -- (2.3548, 0.3911) -- (2.3503, 0.3924) -- (2.3277, 0.3976)
   -- (2.3236, 0.3982) -- (2.3072, 0.4) -- (2.3046, 0.4001) -- (2.3014, 0.3993) 
  -- (2.2906, 0.3951) -- (2.2798, 0.3893) -- (2.2758, 0.3866) -- (2.2713, 
  0.3823) -- (2.2642, 0.3781) -- (2.2528, 0.3725) -- (2.2261, 0.3649) -- 
  (2.2242, 0.3649) -- (2.2126, 0.3718) -- (2.1706, 0.4085) -- (2.1614, 0.3918) 
  -- (2.1588, 0.3914) -- (2.1535, 0.3936) -- (2.1184, 0.4089) -- (2.1176, 
  0.4116) -- (2.1084, 0.4443) -- (2.1106, 0.4522) -- (2.1165, 0.4588) -- (2.106,
   0.476) -- (2.0912, 0.489) -- (2.0699, 0.4958) -- (2.0722, 0.4855) -- (2.0688,
   0.4796) -- (2.0668, 0.4771) -- (2.0583, 0.4715) -- (2.0558, 0.4726) -- 
  (2.0507, 0.4804) -- (2.0478, 0.512) -- (2.022, 0.5172) -- (2.015, 0.5074) -- 
  (2.0111, 0.4962) -- (1.9815, 0.4898) -- (1.9679, 0.4874) -- (1.9579, 0.4792) 
  -- (1.9473, 0.4655) -- (1.9176, 0.4142) -- (1.9132, 0.3957) -- (1.9134, 
  0.3943) -- (1.9172, 0.3876) -- (1.9536, 0.3614) -- (1.9555, 0.3605) -- 
  (1.9914, 0.3752) -- (1.9901, 0.3806) -- (1.994, 0.3832) -- (2.0063, 0.3883) --
   (2.0118, 0.3896) -- (2.0354, 0.3914) -- (2.0375, 0.3913) -- (2.0408, 0.3907) 
  -- (2.0544, 0.3722) -- (2.0546, 0.3694) -- (2.0509, 0.361) -- (2.0373, 0.3326)
   -- (2.0292, 0.324) -- (2.0242, 0.3194) -- (2.0146, 0.3209) -- (2.0123, 
  0.3258) -- (2.0141, 0.3324) -- (2.0165, 0.3333) -- (2.0172, 0.3349) -- 
  (2.0167, 0.3373) -- (2.0156, 0.3391) -- (2.0088, 0.3465) -- (1.9971, 0.3526) 
  -- (1.9967, 0.3528) -- (1.9862, 0.3552) -- (1.9518, 0.3281) -- (1.9643, 
  0.3127) -- (1.9677, 0.3075) -- (1.9679, 0.3064) -- (1.9628, 0.3031) -- 
  (1.9488, 0.2984) -- (1.9319, 0.2939) -- (1.9273, 0.2933) -- (1.8662, 0.2984) 
  -- (1.8628, 0.2989) -- (1.8614, 0.2995) -- (1.8591, 0.3007) -- (1.8443, 
  0.3146) -- (1.8426, 0.3167) -- (1.8413, 0.3197) -- (1.8413, 0.323) -- (1.8431,
   0.3314) -- (1.8428, 0.3324) -- (1.8408, 0.3344) -- (1.8354, 0.3372) -- 
  (1.8184, 0.3417) -- (1.7889, 0.3473) -- (1.7844, 0.3479) -- (1.7819, 0.3473) 
  -- (1.7801, 0.3464) -- (1.7229, 0.3133) -- (1.7199, 0.3102) -- (1.7185, 
  0.3082) -- (1.7179, 0.3058) -- (1.717, 0.2979) -- (1.7152, 0.2941) -- (1.7112,
   0.2909) -- (1.7088, 0.29) -- (1.6665, 0.2845) -- (1.6649, 0.2845) -- (1.6267,
   0.2929) -- (1.5958, 0.3055) -- (1.5862, 0.3198) -- (1.5847, 0.3212) -- 
  (1.5814, 0.3226) -- (1.5792, 0.323) -- (1.5466, 0.3218) -- (1.5335, 0.3089) --
   (1.5317, 0.3052) -- (1.5223, 0.2951) -- (1.5176, 0.2926) -- (1.493, 0.2837) 
  -- (1.4578, 0.2737) -- (1.4553, 0.2735) -- (1.4477, 0.2741) -- (1.4406, 0.277)
   -- (1.4393, 0.2783) -- (1.4279, 0.2901) -- (1.4217, 0.3003) -- (1.424, 
  0.3016) -- (1.4284, 0.3002) -- (1.4512, 0.3095) -- (1.4514, 0.3127) -- 
  (1.4498, 0.321) -- (1.4465, 0.3299) -- (1.3934, 0.3439) -- (1.3535, 0.396) -- 
  (1.3517, 0.3989) -- (1.3503, 0.4028) -- (1.3497, 0.4061) -- (1.3457, 0.4321) 
  -- (1.3465, 0.4358) -- (1.3478, 0.4376) -- (1.3623, 0.4491) -- (1.3644, 
  0.4514) -- (1.3669, 0.4551) -- (1.3692, 0.4607) -- (1.3704, 0.4672) -- 
  (1.3703, 0.4704) -- (1.3691, 0.4753) -- (1.3602, 0.5018) -- (1.3597, 0.5057) 
  -- (1.3602, 0.5133) -- (1.3612, 0.5165) -- (1.3696, 0.5367) -- (1.3809, 
  0.5629) -- (1.383, 0.5692) -- (1.3835, 0.574) -- (1.383, 0.5792) -- (1.3799, 
  0.588) -- (1.3791, 0.5915) -- (1.3791, 0.5978) -- (1.3803, 0.6026) -- (1.3969,
   0.6464) -- (1.4137, 0.6697) -- (1.418, 0.6759) -- (1.4227, 0.6845) -- (1.424,
   0.689) -- (1.4238, 0.695) -- (1.421, 0.7063) -- (1.4185, 0.7121) -- (1.4143, 
  0.7193) -- (1.3948, 0.7438) -- (1.3933, 0.746) -- (1.3926, 0.7487) -- (1.3927,
   0.752) -- (1.3941, 0.7916) -- (1.3946, 0.7955) -- (1.4003, 0.8245) -- 
  (1.4004, 0.8288) -- (1.4013, 0.8313) -- (1.4157, 0.8623) -- (1.4173, 0.8652) 
  -- (1.4373, 0.8915) -- (1.459, 0.921) -- (1.4607, 0.9248) -- (1.4637, 0.9346) 
  -- (1.4699, 0.9557) -- (1.4763, 0.9896) -- (1.4771, 0.9928) -- (1.4791, 
  0.9973) -- (1.4828, 1.0016) -- (1.4871, 1.0049) -- (1.5025, 1.0105) -- 
  (1.5069, 1.0132) -- (1.5088, 1.0151) -- (1.5102, 1.0169) -- (1.5117, 1.0203) 
  -- (1.5125, 1.0232) -- (1.5126, 1.0261) -- (1.506, 1.0617) -- (1.5051, 1.07) 
  -- (1.5051, 1.0754) -- (1.5059, 1.0819) -- (1.5071, 1.0879) -- (1.509, 1.0938)
   -- (1.5286, 1.1416) -- (1.5497, 1.1847) -- (1.5524, 1.1886) -- (1.5553, 
  1.1949) -- (1.5562, 1.1997) -- (1.5563, 1.2035) -- (1.5567, 1.2363) -- 
  (1.5561, 1.2446) -- (1.5541, 1.2546) -- (1.554, 1.259) -- (1.5544, 1.2619) -- 
  (1.5558, 1.2658) -- (1.5804, 1.3139) -- (1.5817, 1.316) -- (1.584, 1.3181) -- 
  (1.595, 1.3228) -- (1.6085, 1.3303) -- (1.6131, 1.3339) -- (1.6569, 1.3843) --
   (1.6637, 1.4175) -- (1.665, 1.4205) -- (1.6664, 1.4218) -- (1.6709, 1.4238) 
  -- (1.6745, 1.424) -- (1.6974, 1.4298) -- (1.7293, 1.4524) -- (1.7364, 1.4588)
   -- (1.7428, 1.4666) -- (1.7468, 1.4727) -- (1.749, 1.4777) -- (1.7507, 1.485)
   -- (1.7532, 1.4961) -- (1.7571, 1.5079) -- (1.7625, 1.5232) -- (1.7657, 
  1.531) -- (1.7696, 1.5392) -- (1.7777, 1.5542) -- (1.8086, 1.6028) -- (1.8114,
   1.6058) -- (1.8142, 1.6073) -- (1.8325, 1.6132) -- (1.8363, 1.6154) -- 
  (1.8509, 1.6267) -- (1.8607, 1.6345) -- (1.8644, 1.6383) -- (1.8673, 1.6424) 
  -- (1.8699, 1.6473) -- (1.8738, 1.6559) -- (1.8746, 1.6588) -- (1.8753, 
  1.6615) -- (1.8772, 1.6744) -- (1.879, 1.6826) -- (1.8831, 1.6935) -- (1.8864,
   1.6995) -- (1.893, 1.7092) -- (1.9019, 1.7189) -- (1.902, 1.719) -- (1.9083, 
  1.7268) -- (1.9124, 1.7335) -- (1.9141, 1.7382) -- (1.9362, 1.8326) -- 
  (1.9348, 1.849) -- (1.9332, 1.8486) -- (1.9323, 1.8486) -- (1.9321, 1.849) -- 
  (1.9326, 1.8555) -- (1.9343, 1.8611) -- (1.9369, 1.8661) -- (1.9402, 1.8701) 
  -- (1.9449, 1.8741) -- (1.9499, 1.8773) -- (1.9632, 1.8858) -- (1.9677, 
  1.8899) -- (1.9838, 1.9079) -- (1.9963, 1.9143) -- (1.9972, 1.9162) -- 
  (1.9977, 1.9202) -- (1.9966, 1.9252) -- (1.9945, 1.9284) -- (1.9913, 1.9314) 
  -- (1.989, 1.9326) -- (1.9796, 1.9358) -- (1.9776, 1.9371) -- (1.9759, 1.9391)
   -- (1.9746, 1.9426) -- (1.9743, 1.9466) -- (1.9746, 1.9485) -- (1.9758, 
  1.9514) -- (1.9779, 1.9544) -- (1.9826, 1.9588) -- (1.9867, 1.9616) -- 
  (1.9908, 1.9647) -- (1.9954, 1.9695) -- (2.0007, 1.9774) -- (2.0042, 1.9852) 
  -- (2.0065, 1.9938) -- (2.0068, 1.999) -- (2.0064, 2.0068) -- (2.0048, 2.0279)
   -- (2.0119, 2.0471) -- (2.012, 2.05) -- (2.0111, 2.0529) -- (2.0079, 2.0566) 
  -- (2.0047, 2.058) -- (1.9995, 2.059) -- (1.9904, 2.0587) -- (1.9843, 2.0584) 
  -- (1.979, 2.0594) -- (1.9768, 2.0603) -- (1.9745, 2.0624) -- (1.973, 2.064) 
  -- (1.9718, 2.0662) -- (1.9717, 2.0695) -- (1.9728, 2.0723) -- (1.9749, 
  2.0745) -- (1.977, 2.0761) -- (1.9731, 2.1054) -- (1.9723, 2.111) -- (1.9623, 
  2.1517) -- (1.9584, 2.1594) -- (1.9574, 2.1622) -- (1.957, 2.1656) -- (1.9572,
   2.1695) -- (1.9584, 2.1741) -- (1.9621, 2.1805) -- (1.9629, 2.1909) -- 
  (1.9643, 2.1917) -- (1.9739, 2.1964) -- (1.9767, 2.1971) -- (1.9793, 2.1973) 
  -- (1.9885, 2.1962) -- (1.9935, 2.1919) -- (2.0069, 2.1803) -- (2.0256, 
  2.1562) -- (2.0291, 2.1454) -- (2.0349, 2.1375) -- (2.0432, 2.1312) -- 
  (2.0474, 2.1296) -- (2.0501, 2.1288) -- (2.0587, 2.1292) -- (2.0746, 2.1417) 
  -- (2.0807, 2.1516) -- (2.0817, 2.1566) -- (2.0817, 2.1612) -- (2.0735, 
  2.1985) -- (2.0677, 2.1976) -- (2.0657, 2.2012) -- (2.0654, 2.2087) -- 
  (2.0681, 2.2142) -- (2.0837, 2.2187) -- (2.1198, 2.2277) -- (2.1287, 2.2256) 
  -- (2.1299, 2.2213) -- (2.1251, 2.2122) -- (2.1242, 2.2111) -- (2.1202, 
  2.2125) -- (2.1193, 2.2143) -- (2.116, 2.2124) -- (2.1332, 2.1424) -- (2.1465,
   2.1333) -- (2.1676, 2.1239) -- (2.1718, 2.1225) -- (2.2314, 2.1107) -- 
  (2.2534, 2.1107) -- (2.2519, 2.1048) -- (2.2538, 2.0934) -- (2.2324, 2.0741) 
  -- (2.2113, 2.0667) -- (2.1951, 2.0324) -- (2.1917, 2.0244) -- (2.1944, 
  2.0166) -- (2.1999, 2.0107) -- (2.2043, 2.0098) -- (2.2214, 2.009) -- (2.2226,
   2.0093) -- (2.2249, 2.011) -- (2.2843, 2.0653) -- (2.2792, 2.073) -- (2.2724,
   2.0792) -- (2.2743, 2.0851) -- (2.2778, 2.0915) -- (2.2923, 2.1083) -- 
  (2.3019, 2.1151) -- (2.3041, 2.1162) -- (2.3558, 2.1328) -- (2.3663, 2.1303) 
  -- (2.3799, 2.1145) -- (2.3883, 2.1167) -- (2.391, 2.1206) -- (2.3911, 2.1245)
   -- (2.3859, 2.1372) -- (2.3827, 2.1395) -- (2.3679, 2.1475) -- (2.3729, 
  2.1515) -- (2.3742, 2.1535) -- (2.3778, 2.1625) -- (2.3769, 2.1633) -- 
  cycle(2.1107, 0.4111) -- (2.1039, 0.4284) -- (2.1018, 0.4302) -- (2.0784, 
  0.4289) -- (2.0805, 0.4088) -- (2.0864, 0.404) -- (2.1053, 0.406) -- (2.1118, 
  0.4068) -- cycle;

  \node[text=black,line width=0.0092cm,anchor=center] (text4) at (2.3711, 
  1.2664){\resizebox{\ifdim\width>2em 2em\else\width\fi}{!}{#2}};
  }
\newcommand{\drawbav}[2]{%
  %Bayern
  \path[draw=black,fill=#1,line join=round,line width=0.0046cm] (4.0809, 
  2.9865) -- (4.0894, 2.9848) -- (4.0957, 2.979) -- (4.1192, 2.9547) -- (4.1188,
   2.9508) -- (4.1167, 2.9351) -- (4.1139, 2.919) -- (4.1176, 2.9163) -- 
  (4.1282, 2.9145) -- (4.1692, 2.9024) -- (4.1874, 2.8955) -- (4.1916, 2.8896) 
  -- (4.2054, 2.8762) -- (4.2163, 2.8518) -- (4.2158, 2.8502) -- (4.2153, 
  2.8492) -- (4.2085, 2.844) -- (4.1999, 2.8419) -- (4.1961, 2.838) -- (4.1872, 
  2.8271) -- (4.1886, 2.8243) -- (4.2002, 2.8138) -- (4.2303, 2.7978) -- 
  (4.2326, 2.7959) -- (4.2429, 2.7849) -- (4.2557, 2.7692) -- (4.265, 2.7501) --
   (4.2644, 2.7467) -- (4.2626, 2.7413) -- (4.2564, 2.7352) -- (4.2534, 2.7262) 
  -- (4.2532, 2.7215) -- (4.255, 2.7073) -- (4.2573, 2.6958) -- (4.258, 2.6941) 
  -- (4.2631, 2.6883) -- (4.2639, 2.6879) -- (4.2703, 2.6899) -- (4.275, 2.6886)
   -- (4.2843, 2.6815) -- (4.2984, 2.6668) -- (4.2955, 2.6479) -- (4.3309, 
  2.6424) -- (4.3613, 2.6115) -- (4.399, 2.5893) -- (4.4144, 2.5889) -- (4.4234,
   2.5875) -- (4.4365, 2.5803) -- (4.4423, 2.572) -- (4.468, 2.5277) -- (4.4674,
   2.4957) -- (4.4647, 2.4933) -- (4.4593, 2.4869) -- (4.4388, 2.4486) -- 
  (4.4334, 2.4326) -- (4.4306, 2.4017) -- (4.4292, 2.3955) -- (4.3901, 2.3685) 
  -- (4.3897, 2.366) -- (4.3901, 2.3637) -- (4.3935, 2.3512) -- (4.3977, 2.3443)
   -- (4.4151, 2.3199) -- (4.4168, 2.3181) -- (4.4187, 2.3169) -- (4.4231, 
  2.3157) -- (4.4641, 2.304) -- (4.4656, 2.3033) -- (4.4693, 2.2882) -- (4.4764,
   2.2393) -- (4.5016, 2.1849) -- (4.5104, 2.1723) -- (4.5142, 2.1692) -- 
  (4.5111, 2.1656) -- (4.5197, 2.1602) -- (4.5311, 2.1561) -- (4.536, 2.1556) --
   (4.5372, 2.1595) -- (4.5381, 2.1604) -- (4.5409, 2.1602) -- (4.5437, 2.1559) 
  -- (4.5445, 2.154) -- (4.5459, 2.1413) -- (4.5467, 2.1218) -- (4.5467, 2.1167)
   -- (4.5463, 2.1156) -- (4.5407, 2.1082) -- (4.5566, 2.0846) -- (4.6193, 
  2.0348) -- (4.6221, 2.0232) -- (4.6205, 2.0163) -- (4.6342, 1.9911) -- 
  (4.6386, 1.9888) -- (4.6489, 1.9855) -- (4.695, 1.9788) -- (4.7011, 1.9789) --
   (4.7023, 1.9797) -- (4.7005, 1.985) -- (4.6971, 1.9891) -- (4.6964, 1.9956) 
  -- (4.7016, 1.9963) -- (4.7094, 1.9954) -- (4.7365, 1.991) -- (4.7394, 1.9903)
   -- (4.7536, 1.9823) -- (4.7724, 1.9647) -- (4.791, 1.9405) -- (4.8446, 
  1.8691) -- (4.8527, 1.8581) -- (4.8679, 1.8445) -- (4.8764, 1.8377) -- 
  (4.8787, 1.8365) -- (4.8819, 1.8364) -- (4.8826, 1.809) -- (4.8909, 1.8004) --
   (4.9007, 1.7936) -- (4.9085, 1.789) -- (4.9245, 1.7825) -- (4.9302, 1.7842) 
  -- (4.9366, 1.7869) -- (4.9491, 1.79) -- (4.9528, 1.7902) -- (4.9572, 1.7883) 
  -- (4.9798, 1.7722) -- (4.9925, 1.7622) -- (5.0218, 1.7296) -- (5.027, 1.7217)
   -- (5.0331, 1.7009) -- (5.0422, 1.6616) -- (5.0522, 1.6548) -- (5.0924, 
  1.6279) -- (5.0943, 1.6277) -- (5.1002, 1.6288) -- (5.0999, 1.637) -- (5.0986,
   1.6409) -- (5.0979, 1.6458) -- (5.0982, 1.6541) -- (5.1118, 1.6593) -- 
  (5.1433, 1.658) -- (5.146, 1.6564) -- (5.1477, 1.6549) -- (5.1708, 1.6293) -- 
  (5.1805, 1.617) -- (5.1874, 1.6011) -- (5.1934, 1.5877) -- (5.2039, 1.5756) --
   (5.2071, 1.5743) -- (5.2151, 1.5745) -- (5.2322, 1.5752) -- (5.235, 1.5766) 
  -- (5.2387, 1.5803) -- (5.2442, 1.5802) -- (5.2972, 1.502) -- (5.3091, 1.4776)
   -- (5.3038, 1.4572) -- (5.2975, 1.4376) -- (5.2893, 1.4228) -- (5.3013, 
  1.402) -- (5.3121, 1.3495) -- (5.3124, 1.3346) -- (5.3002, 1.293) -- (5.2581, 
  1.2337) -- (5.2562, 1.2322) -- (5.243, 1.2355) -- (5.2163, 1.2494) -- (5.2139,
   1.2514) -- (5.1679, 1.2804) -- (5.1187, 1.299) -- (5.1153, 1.2842) -- 
  (5.1137, 1.2812) -- (5.1066, 1.2733) -- (5.0967, 1.265) -- (5.0922, 1.2629) --
   (5.074, 1.257) -- (5.081, 1.2486) -- (5.0816, 1.2476) -- (5.0879, 1.223) -- 
  (5.0876, 1.2212) -- (5.0761, 1.1855) -- (5.0736, 1.1807) -- (5.0711, 1.1716) 
  -- (5.071, 1.1681) -- (5.0714, 1.1664) -- (5.0785, 1.155) -- (5.0797, 1.1507) 
  -- (5.0799, 1.148) -- (5.0791, 1.1445) -- (5.0675, 1.0989) -- (5.0657, 1.0944)
   -- (5.0643, 1.092) -- (5.0618, 1.0895) -- (5.0154, 1.0417) -- (5.0105, 
  1.0378) -- (4.9744, 1.0137) -- (4.971, 1.0119) -- (4.9667, 1.0111) -- (4.9212,
   1.0105) -- (4.8954, 0.996) -- (4.8908, 0.9941) -- (4.8839, 0.9928) -- 
  (4.8727, 0.9931) -- (4.8672, 0.9933) -- (4.862, 0.9921) -- (4.826, 0.9726) -- 
  (4.8229, 0.9705) -- (4.8199, 0.9674) -- (4.8183, 0.9647) -- (4.8152, 0.9579) 
  -- (4.8133, 0.9544) -- (4.7921, 0.9297) -- (4.7888, 0.9268) -- (4.7854, 
  0.9245) -- (4.7816, 0.9228) -- (4.7786, 0.9221) -- (4.7616, 0.9213) -- 
  (4.7354, 0.9169) -- (4.7335, 0.9163) -- (4.7309, 0.9145) -- (4.7289, 0.9116) 
  -- (4.7285, 0.9099) -- (4.7281, 0.9036) -- (4.7273, 0.901) -- (4.7185, 0.8821)
   -- (4.7104, 0.8721) -- (4.7088, 0.8701) -- (4.6889, 0.8545) -- (4.6651, 
  0.8349) -- (4.6605, 0.829) -- (4.6594, 0.8275) -- (4.6592, 0.8249) -- (4.6618,
   0.808) -- (4.6637, 0.8009) -- (4.6666, 0.7945) -- (4.6725, 0.7869) -- 
  (4.6988, 0.7651) -- (4.7172, 0.75) -- (4.7246, 0.7423) -- (4.7284, 0.7376) -- 
  (4.731, 0.7328) -- (4.7317, 0.7305) -- (4.74, 0.6971) -- (4.7422, 0.693) -- 
  (4.7446, 0.6906) -- (4.7666, 0.6866) -- (4.771, 0.6846) -- (4.7731, 0.6829) --
   (4.7815, 0.6708) -- (4.8022, 0.6384) -- (4.828, 0.592) -- (4.8297, 0.5877) --
   (4.8303, 0.5836) -- (4.8295, 0.5807) -- (4.8059, 0.5386) -- (4.7716, 0.4701) 
  -- (4.7715, 0.4684) -- (4.7722, 0.4664) -- (4.7772, 0.4609) -- (4.7851, 
  0.4551) -- (4.8158, 0.4528) -- (4.8247, 0.4534) -- (4.8273, 0.4547) -- 
  (4.8339, 0.4594) -- (4.8353, 0.4612) -- (4.836, 0.4628) -- (4.8353, 0.4638) --
   (4.8374, 0.4679) -- (4.8389, 0.4685) -- (4.8434, 0.4688) -- (4.8615, 0.46) --
   (4.864, 0.4582) -- (4.8775, 0.4409) -- (4.8972, 0.3927) -- (4.8982, 0.3886) 
  -- (4.8966, 0.3817) -- (4.8938, 0.3787) -- (4.882, 0.3648) -- (4.8651, 0.3384)
   -- (4.8642, 0.332) -- (4.859, 0.2912) -- (4.8592, 0.2899) -- (4.8605, 0.2882)
   -- (4.8633, 0.2858) -- (4.8663, 0.2848) -- (4.8691, 0.2826) -- (4.8698, 
  0.2803) -- (4.8712, 0.2593) -- (4.8711, 0.2526) -- (4.846, 0.2254) -- (4.8442,
   0.2244) -- (4.8426, 0.2249) -- (4.8184, 0.2347) -- (4.811, 0.2381) -- (4.784,
   0.2531) -- (4.7136, 0.301) -- (4.7073, 0.3072) -- (4.7003, 0.3182) -- 
  (4.6971, 0.3276) -- (4.6981, 0.3309) -- (4.7048, 0.3483) -- (4.7137, 0.3578) 
  -- (4.6891, 0.4089) -- (4.6755, 0.413) -- (4.6402, 0.4228) -- (4.6327, 0.4228)
   -- (4.6289, 0.4225) -- (4.6213, 0.4208) -- (4.5853, 0.413) -- (4.5825, 
  0.4122) -- (4.5777, 0.4054) -- (4.5694, 0.3877) -- (4.5251, 0.3651) -- 
  (4.5167, 0.3646) -- (4.4944, 0.3871) -- (4.4779, 0.4095) -- (4.4768, 0.4118) 
  -- (4.4288, 0.417) -- (4.4123, 0.4231) -- (4.4103, 0.4247) -- (4.4117, 0.4289)
   -- (4.3678, 0.4123) -- (4.3572, 0.445) -- (4.3605, 0.4508) -- (4.3646, 
  0.4609) -- (4.3633, 0.465) -- (4.3592, 0.4706) -- (4.3546, 0.4689) -- (4.3151,
   0.4303) -- (4.3138, 0.3985) -- (4.3244, 0.3753) -- (4.3272, 0.366) -- 
  (4.3317, 0.3482) -- (4.3318, 0.3463) -- (4.3312, 0.3439) -- (4.3301, 0.3421) 
  -- (4.3286, 0.341) -- (4.3161, 0.3382) -- (4.2819, 0.3391) -- (4.252, 0.3421) 
  -- (4.2379, 0.3502) -- (4.212, 0.343) -- (4.1791, 0.347) -- (4.1434, 0.3423) 
  -- (4.1138, 0.3308) -- (4.103, 0.3121) -- (4.1011, 0.3109) -- (4.0917, 0.3116)
   -- (4.0661, 0.3163) -- (3.9767, 0.3167) -- (3.9661, 0.3185) -- (3.9498, 
  0.308) -- (3.9487, 0.3067) -- (3.9383, 0.284) -- (3.9372, 0.2737) -- (3.9371, 
  0.2717) -- (3.9373, 0.2699) -- (3.9384, 0.2675) -- (3.9389, 0.2546) -- (3.929,
   0.2441) -- (3.9268, 0.2432) -- (3.8866, 0.2339) -- (3.8549, 0.237) -- 
  (3.8489, 0.2387) -- (3.8445, 0.2441) -- (3.8423, 0.2432) -- (3.8401, 0.2415) 
  -- (3.8275, 0.227) -- (3.826, 0.2236) -- (3.8103, 0.2056) -- (3.7602, 0.1611) 
  -- (3.7551, 0.152) -- (3.7535, 0.1485) -- (3.7505, 0.1394) -- (3.7468, 0.1347)
   -- (3.7396, 0.1305) -- (3.7378, 0.1299) -- (3.7112, 0.1272) -- (3.7089, 
  0.1274) -- (3.7074, 0.1292) -- (3.7083, 0.1332) -- (3.7184, 0.1505) -- 
  (3.7267, 0.1583) -- (3.7249, 0.1619) -- (3.7236, 0.1628) -- (3.6958, 0.1635) 
  -- (3.6644, 0.1511) -- (3.6611, 0.1494) -- (3.6493, 0.1426) -- (3.6176, 
  0.1268) -- (3.569, 0.1244) -- (3.5629, 0.1248) -- (3.552, 0.1269) -- (3.5462, 
  0.1296) -- (3.545, 0.1408) -- (3.5464, 0.1451) -- (3.5472, 0.1461) -- (3.5525,
   0.1471) -- (3.5551, 0.1494) -- (3.5542, 0.1579) -- (3.5395, 0.1751) -- 
  (3.5184, 0.1969) -- (3.4885, 0.2033) -- (3.4865, 0.2035) -- (3.482, 0.2057) --
   (3.4807, 0.2094) -- (3.4812, 0.2219) -- (3.4853, 0.2259) -- (3.5033, 0.2396) 
  -- (3.4909, 0.2578) -- (3.4707, 0.2554) -- (3.435, 0.2415) -- (3.4134, 0.2456)
   -- (3.3686, 0.2638) -- (3.3667, 0.2696) -- (3.3664, 0.2725) -- (3.367, 
  0.2742) -- (3.3662, 0.2767) -- (3.3197, 0.2843) -- (3.3085, 0.2853) -- 
  (3.3011, 0.2767) -- (3.2969, 0.2681) -- (3.2968, 0.2666) -- (3.2914, 0.2531) 
  -- (3.2886, 0.2528) -- (3.2602, 0.2541) -- (3.2377, 0.2585) -- (3.2196, 
  0.2701) -- (3.2154, 0.2732) -- (3.2194, 0.2791) -- (3.2251, 0.2861) -- 
  (3.2272, 0.2954) -- (3.2268, 0.3028) -- (3.2243, 0.3028) -- (3.2043, 0.2997) 
  -- (3.203, 0.2993) -- (3.1995, 0.2933) -- (3.2006, 0.2238) -- (3.2067, 0.2078)
   -- (3.2081, 0.2061) -- (3.21, 0.2055) -- (3.2138, 0.207) -- (3.2198, 0.2041) 
  -- (3.2242, 0.199) -- (3.2249, 0.192) -- (3.228, 0.1562) -- (3.1914, 0.1071) 
  -- (3.1469, 0.0502) -- (3.1331, 0.0369) -- (3.106, 0.0201) -- (3.0807, 0.0107)
   -- (3.0397, 0.0034) -- (3.0371, 0.0034) -- (3.036, 0.007) -- (3.0357, 0.0117)
   -- (3.0378, 0.02) -- (3.04, 0.0271) -- (3.0641, 0.0828) -- (3.07, 0.0902) -- 
  (3.0767, 0.0998) -- (3.0771, 0.1008) -- (3.0765, 0.1086) -- (3.0721, 0.1146) 
  -- (3.0707, 0.1152) -- (3.0417, 0.1184) -- (3.0338, 0.1122) -- (3.0335, 
  0.1022) -- (3.0339, 0.0987) -- (3.0314, 0.0963) -- (2.9893, 0.083) -- (2.975, 
  0.1291) -- (2.98, 0.139) -- (2.9855, 0.1413) -- (2.9865, 0.1422) -- (2.9929, 
  0.1524) -- (2.9846, 0.1667) -- (2.9767, 0.1774) -- (2.9592, 0.1973) -- 
  (2.9402, 0.2049) -- (2.9322, 0.2027) -- (2.9274, 0.2026) -- (2.9261, 0.2031) 
  -- (2.9225, 0.2069) -- (2.9032, 0.2399) -- (2.903, 0.2497) -- (2.9032, 0.2558)
   -- (2.905, 0.2633) -- (2.9029, 0.2615) -- (2.8828, 0.2505) -- (2.876, 0.2477)
   -- (2.8452, 0.2467) -- (2.8079, 0.2646) -- (2.8075, 0.2665) -- (2.8088, 0.27)
   -- (2.81, 0.271) -- (2.8113, 0.2713) -- (2.813, 0.2735) -- (2.8145, 0.2764) 
  -- (2.8155, 0.2937) -- (2.813, 0.3015) -- (2.7996, 0.3097) -- (2.7982, 0.31) 
  -- (2.7833, 0.3093) -- (2.7759, 0.3036) -- (2.77, 0.2951) -- (2.7471, 0.2636) 
  -- (2.7122, 0.2745) -- (2.6829, 0.2866) -- (2.6748, 0.2919) -- (2.6719, 
  0.2996) -- (2.6962, 0.318) -- (2.7076, 0.3258) -- (2.7292, 0.3241) -- (2.7297,
   0.3207) -- (2.7307, 0.3187) -- (2.733, 0.3169) -- (2.7368, 0.3156) -- 
  (2.7655, 0.317) -- (2.816, 0.3684) -- (2.8167, 0.3694) -- (2.8605, 0.3719) -- 
  (2.9106, 0.3726) -- (2.9133, 0.3774) -- (2.923, 0.3894) -- (2.9252, 0.3917) --
   (2.9267, 0.3924) -- (2.9389, 0.3945) -- (2.9627, 0.3899) -- (2.9629, 0.3897) 
  -- (2.9659, 0.3844) -- (2.9662, 0.3835) -- (2.9742, 0.3509) -- (2.9736, 
  0.3537) -- (2.9774, 0.3643) -- (2.9809, 0.3682) -- (2.9866, 0.3755) -- 
  (2.9907, 0.3778) -- (2.9971, 0.3765) -- (2.9988, 0.3765) -- (3.009, 0.3859) --
   (3.0152, 0.4065) -- (3.0153, 0.412) -- (2.9919, 0.4692) -- (2.9914, 0.5031) 
  -- (3.0007, 0.5033) -- (3.0128, 0.5143) -- (3.0103, 0.5214) -- (3.0062, 
  0.5241) -- (3.0015, 0.5301) -- (2.9966, 0.54) -- (2.9923, 0.5631) -- (2.9909, 
  0.595) -- (2.9908, 0.5987) -- (2.991, 0.6019) -- (2.9929, 0.6184) -- (2.9964, 
  0.6315) -- (2.9891, 0.6363) -- (2.986, 0.6388) -- (2.9837, 0.6413) -- (2.9817,
   0.645) -- (2.9804, 0.6506) -- (2.9808, 0.6558) -- (2.9818, 0.6601) -- 
  (2.9831, 0.6623) -- (2.9847, 0.664) -- (2.987, 0.6654) -- (2.9936, 0.6669) -- 
  (2.9997, 0.6681) -- (3.0003, 0.6776) -- (3.0025, 0.6947) -- (3.0062, 0.7009) 
  -- (3.0113, 0.7061) -- (3.0127, 0.7123) -- (3.0152, 0.7345) -- (3.0147, 
  0.7617) -- (3.013, 0.7929) -- (3.0055, 0.7998) -- (2.9978, 0.8082) -- (2.997, 
  0.8095) -- (2.9948, 0.8154) -- (2.9868, 0.8411) -- (2.9868, 0.8456) -- (2.984,
   0.8517) -- (2.9809, 0.8678) -- (2.9796, 0.875) -- (2.9684, 0.9562) -- 
  (2.9595, 0.9647) -- (2.9332, 1.006) -- (2.9258, 1.0199) -- (2.9222, 1.0382) --
   (2.936, 1.0707) -- (2.9373, 1.0727) -- (2.9486, 1.0932) -- (2.9512, 1.1089) 
  -- (2.9498, 1.1154) -- (2.9469, 1.1198) -- (2.9615, 1.121) -- (2.9918, 1.1382)
   -- (3.0004, 1.1315) -- (3.011, 1.1191) -- (3.032, 1.1236) -- (3.0755, 1.1619)
   -- (3.0737, 1.167) -- (3.0715, 1.1717) -- (3.0858, 1.1791) -- (3.088, 1.1801)
   -- (3.0925, 1.1807) -- (3.1011, 1.1769) -- (3.1063, 1.1769) -- (3.1123, 
  1.1777) -- (3.1222, 1.1834) -- (3.123, 1.2193) -- (3.1225, 1.2248) -- (3.1136,
   1.2774) -- (3.1016, 1.2939) -- (3.0908, 1.3091) -- (3.0856, 1.3164) -- 
  (3.0861, 1.3287) -- (3.0865, 1.3308) -- (3.0944, 1.3532) -- (3.1314, 1.3415) 
  -- (3.1371, 1.3396) -- (3.1809, 1.3491) -- (3.1957, 1.3499) -- (3.1989, 
  1.3475) -- (3.2002, 1.3459) -- (3.2007, 1.3431) -- (3.1916, 1.3232) -- 
  (3.1876, 1.3204) -- (3.1867, 1.3187) -- (3.1872, 1.3164) -- (3.1875, 1.3162) 
  -- (3.1967, 1.3166) -- (3.2167, 1.323) -- (3.2301, 1.3379) -- (3.2332, 1.3418)
   -- (3.2303, 1.3477) -- (3.2283, 1.3475) -- (3.2225, 1.3532) -- (3.2163, 
  1.3767) -- (3.2157, 1.3846) -- (3.1925, 1.3922) -- (3.1906, 1.392) -- (3.1882,
   1.4221) -- (3.1945, 1.4248) -- (3.2056, 1.4536) -- (3.2058, 1.4542) -- 
  (3.209, 1.5013) -- (3.2102, 1.5579) -- (3.2066, 1.575) -- (3.2022, 1.5808) -- 
  (3.2014, 1.5818) -- (3.1895, 1.5941) -- (3.1829, 1.6122) -- (3.1815, 1.6154) 
  -- (3.1255, 1.6562) -- (3.0871, 1.6725) -- (3.0857, 1.682) -- (3.0858, 1.6884)
   -- (3.0871, 1.6884) -- (3.091, 1.6902) -- (3.0789, 1.7514) -- (3.0599, 
  1.7858) -- (3.0441, 1.7924) -- (3.0304, 1.7996) -- (3.0066, 1.8158) -- (3.004,
   1.819) -- (3.0037, 1.8205) -- (3.0056, 1.8777) -- (3.0144, 1.9025) -- 
  (3.0094, 1.9103) -- (2.9914, 1.9501) -- (2.9905, 1.9532) -- (2.9901, 1.9558) 
  -- (2.9928, 1.9882) -- (2.9957, 1.9939) -- (3.0167, 1.9946) -- (3.023, 1.9974)
   -- (3.0255, 2.0017) -- (3.0261, 2.0036) -- (3.0204, 2.0191) -- (3.01, 2.0401)
   -- (2.9992, 2.076) -- (2.9978, 2.0765) -- (2.9886, 2.0886) -- (2.9881, 
  2.0944) -- (2.9885, 2.0961) -- (2.9905, 2.1005) -- (2.9941, 2.1039) -- 
  (2.9915, 2.1163) -- (2.984, 2.1327) -- (2.9779, 2.1428) -- (2.9676, 2.1448) --
   (2.9645, 2.1439) -- (2.9506, 2.1306) -- (2.9501, 2.1298) -- (2.9523, 2.1268) 
  -- (2.9583, 2.1214) -- (2.9641, 2.1201) -- (2.9592, 2.1115) -- (2.9575, 
  2.1096) -- (2.9377, 2.0875) -- (2.9344, 2.0863) -- (2.8973, 2.084) -- (2.8833,
   2.0873) -- (2.8815, 2.0882) -- (2.8741, 2.0957) -- (2.8754, 2.1019) -- 
  (2.8791, 2.1059) -- (2.8807, 2.1095) -- (2.8809, 2.1105) -- (2.8824, 2.149) --
   (2.8809, 2.159) -- (2.8807, 2.1601) -- (2.8743, 2.1805) -- (2.8732, 2.1831) 
  -- (2.8708, 2.1836) -- (2.8666, 2.1832) -- (2.8639, 2.1828) -- (2.8481, 
  2.2076) -- (2.846, 2.2167) -- (2.8243, 2.2934) -- (2.8179, 2.2999) -- (2.8023,
   2.3154) -- (2.7921, 2.3119) -- (2.7536, 2.2986) -- (2.7292, 2.2848) -- 
  (2.7211, 2.2809) -- (2.7162, 2.2802) -- (2.7029, 2.2874) -- (2.6983, 2.294) --
   (2.7018, 2.3067) -- (2.7028, 2.3075) -- (2.7079, 2.31) -- (2.7085, 2.3129) --
   (2.7143, 2.3677) -- (2.7129, 2.3767) -- (2.7017, 2.3715) -- (2.6626, 2.3336) 
  -- (2.6613, 2.3328) -- (2.6592, 2.3329) -- (2.6568, 2.334) -- (2.6333, 2.3493)
   -- (2.6306, 2.3515) -- (2.6285, 2.3552) -- (2.6279, 2.3572) -- (2.6283, 
  2.3606) -- (2.6299, 2.366) -- (2.63, 2.3684) -- (2.6292, 2.3702) -- (2.6266, 
  2.3727) -- (2.6246, 2.3744) -- (2.6224, 2.375) -- (2.6106, 2.3737) -- (2.6089,
   2.3723) -- (2.6052, 2.3674) -- (2.5992, 2.3578) -- (2.5878, 2.3625) -- 
  (2.5367, 2.36) -- (2.5161, 2.3577) -- (2.5121, 2.3572) -- (2.5097, 2.3561) -- 
  (2.5085, 2.3539) -- (2.5085, 2.3487) -- (2.5129, 2.3316) -- (2.5267, 2.3084) 
  -- (2.5348, 2.2961) -- (2.5405, 2.2953) -- (2.5472, 2.296) -- (2.5501, 2.2967)
   -- (2.5505, 2.3117) -- (2.5491, 2.3165) -- (2.5523, 2.3217) -- (2.5607, 
  2.3196) -- (2.5701, 2.3035) -- (2.5742, 2.288) -- (2.5672, 2.2577) -- (2.5721,
   2.2531) -- (2.5701, 2.2481) -- (2.5639, 2.2405) -- (2.5503, 2.238) -- (2.529,
   2.2447) -- (2.5016, 2.2488) -- (2.4901, 2.2321) -- (2.4837, 2.2218) -- 
  (2.4889, 2.2144) -- (2.4896, 2.2123) -- (2.4674, 2.1841) -- (2.4621, 2.1811) 
  -- (2.4513, 2.1778) -- (2.4457, 2.1765) -- (2.4338, 2.1743) -- (2.3885, 
  2.1744) -- (2.3867, 2.1747) -- (2.3778, 2.1782) -- (2.3696, 2.1933) -- (2.377,
   2.2572) -- (2.3895, 2.2913) -- (2.3941, 2.2935) -- (2.409, 2.3331) -- 
  (2.3981, 2.3818) -- (2.3965, 2.3829) -- (2.3944, 2.3836) -- (2.3872, 2.3852) 
  -- (2.3836, 2.3854) -- (2.3437, 2.4253) -- (2.3413, 2.4279) -- (2.3404, 
  2.4315) -- (2.3393, 2.4447) -- (2.339, 2.4505) -- (2.341, 2.451) -- (2.3494, 
  2.45) -- (2.3428, 2.4587) -- (2.3418, 2.4608) -- (2.3384, 2.4887) -- (2.3375, 
  2.5076) -- (2.3361, 2.5338) -- (2.3327, 2.5615) -- (2.3323, 2.5636) -- 
  (2.3291, 2.5679) -- (2.3436, 2.5761) -- (2.3403, 2.6118) -- (2.339, 2.6143) --
   (2.3361, 2.6168) -- (2.3349, 2.6174) -- (2.3302, 2.6188) -- (2.3248, 2.619) 
  -- (2.322, 2.6184) -- (2.3102, 2.6287) -- (2.3161, 2.6407) -- (2.3204, 2.6635)
   -- (2.3271, 2.6744) -- (2.3319, 2.68) -- (2.3331, 2.6809) -- (2.4566, 2.7148)
   -- (2.5455, 2.6974) -- (2.5483, 2.6967) -- (2.5493, 2.6955) -- (2.5664, 
  2.6559) -- (2.5713, 2.652) -- (2.5731, 2.652) -- (2.5855, 2.6533) -- (2.6323, 
  2.6618) -- (2.636, 2.6726) -- (2.636, 2.68) -- (2.632, 2.6895) -- (2.6298, 
  2.7008) -- (2.6236, 2.738) -- (2.6269, 2.7693) -- (2.6378, 2.7925) -- (2.6478,
   2.7891) -- (2.6757, 2.7819) -- (2.6829, 2.7821) -- (2.6975, 2.7831) -- 
  (2.6986, 2.7841) -- (2.6993, 2.7879) -- (2.7079, 2.8097) -- (2.7142, 2.8215) 
  -- (2.7172, 2.8264) -- (2.7184, 2.828) -- (2.7341, 2.8336) -- (2.7403, 2.8336)
   -- (2.7412, 2.8339) -- (2.7478, 2.8365) -- (2.7626, 2.8499) -- (2.7697, 
  2.8569) -- (2.7737, 2.8642) -- (2.7747, 2.8678) -- (2.7755, 2.878) -- (2.7747,
   2.893) -- (2.7724, 2.8982) -- (2.766, 2.9074) -- (2.7723, 2.9132) -- (2.7738,
   2.9162) -- (2.7789, 2.9359) -- (2.7821, 2.9504) -- (2.8148, 2.96) -- (2.8218,
   2.9544) -- (2.8423, 2.9459) -- (2.8565, 2.9475) -- (2.8675, 2.9521) -- 
  (2.8782, 2.9587) -- (2.8893, 2.9677) -- (2.9039, 2.9761) -- (2.9362, 3.0109) 
  -- (2.9447, 3.0235) -- (2.949, 3.0357) -- (2.9518, 3.0508) -- (2.9514, 3.0578)
   -- (2.954, 3.0585) -- (2.9574, 3.0602) -- (2.9664, 3.0692) -- (2.9831, 
  3.0876) -- (3.044, 3.0941) -- (3.0489, 3.0937) -- (3.0658, 3.0715) -- (3.0791,
   3.0535) -- (3.122, 3.0356) -- (3.127, 3.0315) -- (3.1316, 3.0267) -- (3.154, 
  2.9794) -- (3.1576, 2.9572) -- (3.1578, 2.9519) -- (3.1627, 2.945) -- (3.1642,
   2.9448) -- (3.1812, 2.943) -- (3.1897, 2.9469) -- (3.1978, 2.9509) -- 
  (3.2002, 2.9416) -- (3.2029, 2.9308) -- (3.2239, 2.9078) -- (3.2606, 2.8941) 
  -- (3.2885, 2.8877) -- (3.2861, 2.8619) -- (3.283, 2.8291) -- (3.289, 2.8028) 
  -- (3.2929, 2.7921) -- (3.2947, 2.788) -- (3.3114, 2.7839) -- (3.3594, 2.7663)
   -- (3.3619, 2.7667) -- (3.3637, 2.7675) -- (3.3678, 2.7745) -- (3.3665, 
  2.7905) -- (3.3652, 2.7911) -- (3.3648, 2.8067) -- (3.3655, 2.8087) -- 
  (3.3682, 2.8107) -- (3.3735, 2.8096) -- (3.3957, 2.8045) -- (3.4217, 2.7986) 
  -- (3.4243, 2.7982) -- (3.4306, 2.7998) -- (3.4369, 2.8041) -- (3.4382, 
  2.8058) -- (3.4402, 2.8113) -- (3.4398, 2.8239) -- (3.436, 2.8288) -- (3.4305,
   2.8316) -- (3.4188, 2.8354) -- (3.4131, 2.8368) -- (3.4115, 2.8373) -- 
  (3.3611, 2.873) -- (3.3577, 2.8758) -- (3.3569, 2.8792) -- (3.3552, 2.9072) --
   (3.3572, 2.9158) -- (3.426, 2.944) -- (3.4869, 2.9443) -- (3.4884, 2.9443) --
   (3.4901, 2.944) -- (3.4952, 2.9388) -- (3.5056, 2.9202) -- (3.5179, 2.9067) 
  -- (3.5509, 2.9009) -- (3.5545, 2.9004) -- (3.567, 2.9047) -- (3.5777, 2.9092)
   -- (3.5812, 2.9198) -- (3.5823, 2.9208) -- (3.5892, 2.9223) -- (3.5974, 
  2.9222) -- (3.6123, 2.9097) -- (3.6247, 2.888) -- (3.6243, 2.8824) -- (3.6205,
   2.8811) -- (3.6153, 2.8773) -- (3.6124, 2.8747) -- (3.6085, 2.8703) -- 
  (3.6075, 2.865) -- (3.6084, 2.8605) -- (3.6087, 2.8596) -- (3.615, 2.8503) -- 
  (3.6353, 2.8307) -- (3.6429, 2.831) -- (3.6437, 2.8322) -- (3.6523, 2.8496) --
   (3.6809, 2.8543) -- (3.6993, 2.9173) -- (3.6993, 2.9179) -- (3.6883, 2.9872) 
  -- (3.6855, 2.9949) -- (3.6815, 2.9963) -- (3.6751, 3.0007) -- (3.6741, 
  3.0203) -- (3.6741, 3.0221) -- (3.6754, 3.0278) -- (3.6775, 3.0302) -- 
  (3.6814, 3.0314) -- (3.6856, 3.0302) -- (3.6864, 3.0287) -- (3.7004, 3.0351) 
  -- (3.7205, 3.0514) -- (3.7226, 3.0583) -- (3.7325, 3.0688) -- (3.7341, 
  3.0694) -- (3.7485, 3.0714) -- (3.7804, 3.0652) -- (3.7828, 3.0632) -- 
  (3.7858, 3.0588) -- (3.791, 3.0513) -- (3.791, 3.0499) -- (3.7885, 3.0437) -- 
  (3.7878, 3.0427) -- (3.7838, 3.0428) -- (3.7797, 3.0368) -- (3.7784, 3.0224) 
  -- (3.7788, 3.0031) -- (3.7795, 2.9979) -- (3.7805, 2.9963) -- (3.7987, 
  2.9855) -- (3.8148, 2.9875) -- (3.8198, 2.9674) -- (3.8259, 2.9565) -- 
  (3.8412, 2.9393) -- (3.8466, 2.9367) -- (3.8639, 2.9421) -- (3.868, 2.9459) --
   (3.8681, 2.9478) -- (3.8696, 2.9549) -- (3.871, 2.9572) -- (3.8725, 2.9581) 
  -- (3.8846, 2.9607) -- (3.8881, 2.9589) -- (3.8907, 2.957) -- (3.8946, 2.9525)
   -- (3.8998, 2.9477) -- (3.9025, 2.9461) -- (3.9044, 2.9461) -- (4.0189, 
  2.9619) -- (4.0566, 2.9752) -- cycle;

  \node[text=black,line width=0.0092cm,anchor=center] (text4-3) at (3.8262, 
  1.6963){\resizebox{\ifdim\width>2em 2em\else\width\fi}{!}{#2}};
}
\newcommand{\drawberlin}[2]{%
 %Berlin
  \path[draw=black,fill=#1,line join=round,line width=0.0046cm] (4.8709, 
  5.1067) -- (4.8759, 5.1023) -- (4.8857, 5.097) -- (4.8911, 5.0987) -- (4.8951,
   5.1095) -- (4.9006, 5.1225) -- (4.906, 5.1298) -- (4.9133, 5.1354) -- 
  (4.9214, 5.1312) -- (4.9642, 5.0515) -- (4.9968, 5.0169) -- (5.0115, 4.9848) 
  -- (5.0184, 4.9559) -- (5.0332, 4.9517) -- (5.0565, 4.9365) -- (5.0647, 
  4.9359) -- (5.0788, 4.9212) -- (5.0713, 4.8985) -- (5.0651, 4.8828) -- 
  (5.0485, 4.869) -- (5.0484, 4.8666) -- (5.0532, 4.8652) -- (5.0554, 4.8651) --
   (5.0568, 4.8619) -- (5.0563, 4.8586) -- (5.0527, 4.8522) -- (5.0381, 4.8471) 
  -- (5.0265, 4.8496) -- (4.9619, 4.8679) -- (4.9271, 4.8789) -- (4.9202, 
  4.8959) -- (4.9154, 4.8945) -- (4.9088, 4.8915) -- (4.8938, 4.8844) -- 
  (4.8929, 4.8835) -- (4.8926, 4.882) -- (4.8947, 4.8524) -- (4.8763, 4.8537) --
   (4.8494, 4.8802) -- (4.832, 4.8715) -- (4.8229, 4.8834) -- (4.792, 4.89) -- 
  (4.7811, 4.8892) -- (4.7792, 4.8887) -- (4.7647, 4.8833) -- (4.7444, 4.8705) 
  -- (4.7436, 4.8646) -- (4.7486, 4.8635) -- (4.7426, 4.8629) -- (4.7245, 
  4.8641) -- (4.7056, 4.8759) -- (4.7035, 4.8777) -- (4.7019, 4.8847) -- (4.711,
   4.8897) -- (4.7157, 4.8943) -- (4.7212, 4.9039) -- (4.7175, 4.9103) -- 
  (4.7133, 4.9152) -- (4.7133, 4.9291) -- (4.7159, 4.9342) -- (4.7201, 4.9404) 
  -- (4.724, 4.9432) -- (4.7446, 4.9701) -- (4.7301, 4.9803) -- (4.7152, 4.9774)
   -- (4.7147, 4.9831) -- (4.7186, 5.0395) -- (4.719, 5.0432) -- (4.7343, 
  5.0534) -- (4.739, 5.0548) -- (4.7436, 5.0522) -- (4.755, 5.0472) -- (4.7596, 
  5.0464) -- (4.7696, 5.0456) -- (4.7702, 5.0491) -- (4.77, 5.0503) -- (4.7691, 
  5.0517) -- (4.7657, 5.0543) -- (4.761, 5.0598) -- (4.7597, 5.0627) -- (4.76, 
  5.0636) -- (4.7703, 5.0841) -- (4.8217, 5.0856) -- (4.8322, 5.0835) -- 
  (4.8405, 5.083) -- (4.8472, 5.0828) -- (4.8512, 5.0841) -- (4.8547, 5.086) -- 
  (4.859, 5.0893) -- (4.865, 5.0972) -- cycle;


  \node[text=black,line width=0.0092cm,anchor=south west] (text6) at (6.1791, 
  4.8901){\resizebox{\ifdim\width>2em 2em\else\width\fi}{!}{#2}};
  \path[draw=black,line width=0.0141cm] (4.8754, 4.9848) -- (text6.west);

  }

\newcommand{\drawbrandenburg}[2]{%
  %Brandenburg
  \path[draw=black,fill=#1,line join=round,line width=0.0046cm] (5.4119, 
  5.7771) -- (5.41, 5.7583) -- (5.4197, 5.7282) -- (5.4273, 5.7284) -- (5.4333, 
  5.7272) -- (5.4363, 5.7188) -- (5.4369, 5.7142) -- (5.428, 5.6944) -- (5.4057,
   5.6607) -- (5.4009, 5.6585) -- (5.3993, 5.6571) -- (5.3963, 5.6378) -- 
  (5.3949, 5.629) -- (5.3951, 5.6267) -- (5.3975, 5.617) -- (5.3987, 5.6134) -- 
  (5.4082, 5.601) -- (5.4059, 5.5775) -- (5.4052, 5.5726) -- (5.4001, 5.5434) --
   (5.394, 5.5226) -- (5.3886, 5.5136) -- (5.3863, 5.5107) -- (5.3564, 5.4794) 
  -- (5.3452, 5.4686) -- (5.3298, 5.4555) -- (5.3211, 5.451) -- (5.3138, 5.4485)
   -- (5.2943, 5.4353) -- (5.2809, 5.4225) -- (5.2784, 5.4141) -- (5.2787, 
  5.4127) -- (5.2813, 5.3975) -- (5.2913, 5.3616) -- (5.2945, 5.3551) -- 
  (5.2936, 5.343) -- (5.2867, 5.3318) -- (5.2795, 5.3218) -- (5.2751, 5.3116) --
   (5.2754, 5.3057) -- (5.276, 5.304) -- (5.2792, 5.2982) -- (5.2808, 5.2966) --
   (5.2871, 5.2937) -- (5.304, 5.2924) -- (5.3383, 5.2787) -- (5.41, 5.2322) -- 
  (5.424, 5.2188) -- (5.4387, 5.2018) -- (5.4543, 5.1831) -- (5.455, 5.1823) -- 
  (5.4572, 5.1786) -- (5.4588, 5.1722) -- (5.4781, 5.1539) -- (5.487, 5.1478) --
   (5.5037, 5.1367) -- (5.5113, 5.1329) -- (5.5244, 5.1264) -- (5.5323, 5.1224) 
  -- (5.5351, 5.1215) -- (5.5396, 5.1206) -- (5.5425, 5.119) -- (5.5584, 5.1083)
   -- (5.5623, 5.104) -- (5.5838, 5.0766) -- (5.5852, 5.0739) -- (5.5871, 
  5.0025) -- (5.5859, 4.9964) -- (5.5846, 4.994) -- (5.5688, 4.9663) -- (5.5619,
   4.957) -- (5.5475, 4.9469) -- (5.5449, 4.9451) -- (5.542, 4.9417) -- (5.5401,
   4.9375) -- (5.5392, 4.9328) -- (5.5358, 4.9118) -- (5.535, 4.9048) -- 
  (5.5446, 4.8901) -- (5.547, 4.885) -- (5.5502, 4.8605) -- (5.5639, 4.809) -- 
  (5.5645, 4.8067) -- (5.5663, 4.8047) -- (5.575, 4.795) -- (5.5768, 4.7932) -- 
  (5.5806, 4.7914) -- (5.5873, 4.7918) -- (5.5902, 4.7912) -- (5.6335, 4.7803) 
  -- (5.6344, 4.7799) -- (5.637, 4.7779) -- (5.639, 4.7754) -- (5.648, 4.7627) 
  -- (5.6491, 4.7609) -- (5.6493, 4.7588) -- (5.6472, 4.7001) -- (5.6467, 4.696)
   -- (5.6424, 4.6875) -- (5.6349, 4.6781) -- (5.6337, 4.6756) -- (5.6335, 
  4.6724) -- (5.6363, 4.6489) -- (5.64, 4.6412) -- (5.6429, 4.6371) -- (5.6475, 
  4.6324) -- (5.6495, 4.631) -- (5.6562, 4.6304) -- (5.6619, 4.6287) -- (5.6658,
   4.6267) -- (5.6713, 4.6227) -- (5.6751, 4.6189) -- (5.6837, 4.6049) -- 
  (5.684, 4.602) -- (5.6801, 4.5719) -- (5.6795, 4.5707) -- (5.6763, 4.564) -- 
  (5.6633, 4.5419) -- (5.6644, 4.5413) -- (5.6669, 4.5357) -- (5.6653, 4.4807) 
  -- (5.6564, 4.4496) -- (5.6295, 4.4127) -- (5.6189, 4.4037) -- (5.6033, 
  4.3876) -- (5.6, 4.384) -- (5.6, 4.3833) -- (5.6003, 4.3725) -- (5.6008, 
  4.3678) -- (5.6083, 4.3562) -- (5.6102, 4.3534) -- (5.6117, 4.3523) -- 
  (5.6211, 4.3496) -- (5.6472, 4.2919) -- (5.6518, 4.2818) -- (5.6659, 4.2661) 
  -- (5.6866, 4.2504) -- (5.6898, 4.2484) -- (5.6961, 4.2432) -- (5.7012, 
  4.2372) -- (5.7074, 4.2243) -- (5.712, 4.1857) -- (5.7047, 4.1609) -- (5.6965,
   4.1484) -- (5.6875, 4.1557) -- (5.68, 4.1487) -- (5.674, 4.1272) -- (5.673, 
  4.1248) -- (5.6716, 4.1226) -- (5.6673, 4.1176) -- (5.6657, 4.117) -- (5.6364,
   4.1141) -- (5.6315, 4.1148) -- (5.6277, 4.1157) -- (5.6263, 4.1172) -- 
  (5.6223, 4.1241) -- (5.621, 4.1302) -- (5.6007, 4.138) -- (5.537, 4.1155) -- 
  (5.5337, 4.1131) -- (5.5325, 4.1098) -- (5.5326, 4.1063) -- (5.5334, 4.1014) 
  -- (5.5304, 4.1031) -- (5.4988, 4.099) -- (5.459, 4.0765) -- (5.4321, 4.0858) 
  -- (5.3953, 4.0906) -- (5.3671, 4.0897) -- (5.3553, 4.0916) -- (5.3516, 
  4.0911) -- (5.3322, 4.0628) -- (5.3183, 4.0221) -- (5.3053, 3.9715) -- 
  (5.2808, 3.9394) -- (5.2119, 3.9253) -- (5.2089, 3.9248) -- (5.1833, 3.9255) 
  -- (5.1687, 3.9201) -- (5.1499, 3.9108) -- (5.1422, 3.9056) -- (5.1138, 
  3.9091) -- (5.1091, 3.91) -- (5.1015, 3.9142) -- (5.0984, 3.9182) -- (5.0941, 
  3.912) -- (5.0637, 3.9111) -- (5.0455, 3.9106) -- (5.0423, 3.911) -- (5.0157, 
  3.9189) -- (4.9706, 3.9466) -- (4.9509, 3.9626) -- (4.9297, 3.9814) -- 
  (4.9243, 3.9855) -- (4.8827, 3.9632) -- (4.8587, 3.9419) -- (4.8588, 3.9405) 
  -- (4.8597, 3.9308) -- (4.8441, 3.9264) -- (4.8394, 3.9252) -- (4.8184, 
  3.9257) -- (4.8165, 3.9294) -- (4.8144, 3.9352) -- (4.8004, 3.952) -- (4.8075,
   3.9858) -- (4.8096, 4.0336) -- (4.8081, 4.0398) -- (4.8061, 4.0461) -- 
  (4.7957, 4.0713) -- (4.7786, 4.1088) -- (4.7745, 4.1163) -- (4.7502, 4.1293) 
  -- (4.7485, 4.1264) -- (4.7455, 4.1235) -- (4.7388, 4.1192) -- (4.7335, 
  4.1221) -- (4.7114, 4.1576) -- (4.7705, 4.1969) -- (4.7753, 4.2021) -- 
  (4.7881, 4.2251) -- (4.7881, 4.2264) -- (4.7878, 4.2308) -- (4.7835, 4.2394) 
  -- (4.781, 4.2417) -- (4.7766, 4.2457) -- (4.7461, 4.3531) -- (4.746, 4.3554) 
  -- (4.7459, 4.3574) -- (4.7619, 4.3599) -- (4.7617, 4.3657) -- (4.7604, 
  4.3713) -- (4.7538, 4.3779) -- (4.75, 4.3803) -- (4.7417, 4.3807) -- (4.7423, 
  4.3746) -- (4.7392, 4.3726) -- (4.7204, 4.365) -- (4.6988, 4.3669) -- (4.6973,
   4.3671) -- (4.6946, 4.3688) -- (4.6668, 4.3975) -- (4.6221, 4.4213) -- 
  (4.6123, 4.4178) -- (4.6076, 4.418) -- (4.5854, 4.4236) -- (4.5402, 4.4496) --
   (4.5358, 4.4628) -- (4.5326, 4.463) -- (4.5159, 4.4716) -- (4.5013, 4.4798) 
  -- (4.4949, 4.4839) -- (4.4766, 4.4923) -- (4.4681, 4.4922) -- (4.4641, 
  4.4868) -- (4.4626, 4.4744) -- (4.446, 4.4625) -- (4.4389, 4.4614) -- (4.4026,
   4.4669) -- (4.3365, 4.4926) -- (4.2944, 4.5188) -- (4.2791, 4.536) -- 
  (4.2548, 4.5676) -- (4.2357, 4.5971) -- (4.2088, 4.6315) -- (4.2256, 4.6696) 
  -- (4.2235, 4.7065) -- (4.226, 4.7292) -- (4.2272, 4.7336) -- (4.2324, 4.75) 
  -- (4.2431, 4.7735) -- (4.2441, 4.7751) -- (4.2464, 4.7761) -- (4.2573, 
  4.7981) -- (4.2513, 4.8296) -- (4.2476, 4.8557) -- (4.2459, 4.8709) -- 
  (4.2539, 4.8906) -- (4.2591, 4.9) -- (4.262, 4.9086) -- (4.2585, 4.9386) -- 
  (4.2077, 4.9562) -- (4.206, 4.9476) -- (4.2056, 4.9462) -- (4.2027, 4.942) -- 
  (4.1925, 4.939) -- (4.1817, 4.9372) -- (4.1738, 4.9466) -- (4.157, 4.9665) -- 
  (4.1566, 4.9669) -- (4.157, 4.9738) -- (4.1733, 5.039) -- (4.1816, 5.054) -- 
  (4.1854, 5.0529) -- (4.1893, 5.0527) -- (4.2042, 5.0636) -- (4.208, 5.0974) --
   (4.2078, 5.1027) -- (4.2053, 5.1109) -- (4.1944, 5.1273) -- (4.183, 5.1459) 
  -- (4.1827, 5.1468) -- (4.1866, 5.1876) -- (4.1898, 5.198) -- (4.1959, 5.2099)
   -- (4.2094, 5.2166) -- (4.2112, 5.2331) -- (4.1999, 5.278) -- (4.1991, 
  5.2805) -- (4.1933, 5.2843) -- (4.1798, 5.2904) -- (4.1656, 5.284) -- (4.1572,
   5.2797) -- (4.1421, 5.2746) -- (4.1344, 5.2928) -- (4.1347, 5.2991) -- 
  (4.1369, 5.305) -- (4.1382, 5.3065) -- (4.1314, 5.3002) -- (4.056, 5.2913) -- 
  (4.0533, 5.2918) -- (4.0055, 5.3072) -- (3.9726, 5.3206) -- (3.9693, 5.3223) 
  -- (3.9673, 5.3243) -- (3.965, 5.3289) -- (3.9644, 5.3333) -- (3.9649, 5.3351)
   -- (3.9666, 5.3388) -- (3.9714, 5.3439) -- (3.9763, 5.348) -- (3.9776, 
  5.3498) -- (3.9784, 5.3523) -- (3.9786, 5.3545) -- (3.9783, 5.357) -- (3.9775,
   5.3595) -- (3.9752, 5.3623) -- (3.9278, 5.3898) -- (3.9257, 5.3909) -- 
  (3.9222, 5.3925) -- (3.9186, 5.3938) -- (3.9136, 5.3946) -- (3.9121, 5.3943) 
  -- (3.903, 5.389) -- (3.8962, 5.3842) -- (3.8934, 5.3834) -- (3.8885, 5.3843) 
  -- (3.8865, 5.3856) -- (3.8538, 5.4117) -- (3.8516, 5.414) -- (3.8503, 5.4166)
   -- (3.8498, 5.4201) -- (3.8503, 5.4229) -- (3.8582, 5.4335) -- (3.8592, 
  5.435) -- (3.8594, 5.4369) -- (3.8587, 5.4391) -- (3.8568, 5.4408) -- (3.8546,
   5.4419) -- (3.8527, 5.4424) -- (3.8483, 5.4428) -- (3.846, 5.4424) -- 
  (3.8331, 5.4372) -- (3.8276, 5.4374) -- (3.7874, 5.4471) -- (3.7816, 5.4488) 
  -- (3.7764, 5.4525) -- (3.7658, 5.4637) -- (3.7593, 5.471) -- (3.7559, 5.4736)
   -- (3.7521, 5.4755) -- (3.7474, 5.4759) -- (3.7436, 5.4755) -- (3.725, 
  5.4702) -- (3.7214, 5.4693) -- (3.7189, 5.4678) -- (3.7095, 5.4605) -- 
  (3.6939, 5.4522) -- (3.6897, 5.4521) -- (3.6876, 5.4525) -- (3.685, 5.454) -- 
  (3.6576, 5.4821) -- (3.6471, 5.4948) -- (3.6464, 5.4966) -- (3.6461, 5.4996) 
  -- (3.6458, 5.513) -- (3.6974, 5.51) -- (3.7227, 5.53) -- (3.7536, 5.5336) -- 
  (3.761, 5.5282) -- (3.7613, 5.5259) -- (3.7671, 5.5232) -- (3.7821, 5.5204) --
   (3.7962, 5.52) -- (3.8041, 5.5255) -- (3.8071, 5.5345) -- (3.813, 5.5578) -- 
  (3.8067, 5.5953) -- (3.8423, 5.6277) -- (3.8475, 5.6317) -- (3.8726, 5.632) --
   (3.8819, 5.6314) -- (3.8975, 5.6225) -- (3.9039, 5.6092) -- (3.9097, 5.6101) 
  -- (3.9604, 5.6215) -- (3.9624, 5.6236) -- (3.9626, 5.6247) -- (3.9614, 
  5.6279) -- (3.9512, 5.6449) -- (3.9807, 5.647) -- (4.0266, 5.6638) -- (4.0302,
   5.666) -- (4.0645, 5.6906) -- (4.0679, 5.6951) -- (4.0665, 5.7114) -- 
  (4.0855, 5.7386) -- (4.0996, 5.736) -- (4.1152, 5.7338) -- (4.1288, 5.7388) --
   (4.1795, 5.7492) -- (4.1826, 5.7493) -- (4.218, 5.7222) -- (4.2363, 5.7167) 
  -- (4.2413, 5.7133) -- (4.2469, 5.7117) -- (4.2537, 5.7085) -- (4.2707, 
  5.6993) -- (4.2797, 5.692) -- (4.2843, 5.6789) -- (4.3071, 5.6533) -- (4.3369,
   5.6569) -- (4.3387, 5.6589) -- (4.3383, 5.6636) -- (4.3545, 5.6671) -- 
  (4.4249, 5.6605) -- (4.4321, 5.6593) -- (4.4631, 5.6303) -- (4.4857, 5.6068) 
  -- (4.4892, 5.6005) -- (4.4999, 5.5993) -- (4.5076, 5.6021) -- (4.5182, 
  5.6062) -- (4.5183, 5.6075) -- (4.5191, 5.6081) -- (4.5355, 5.6135) -- 
  (4.6023, 5.6153) -- (4.6085, 5.6135) -- (4.6089, 5.6131) -- (4.6109, 5.6083) 
  -- (4.611, 5.6069) -- (4.6015, 5.6011) -- (4.593, 5.5945) -- (4.5905, 5.5907) 
  -- (4.5906, 5.5901) -- (4.5919, 5.5889) -- (4.6109, 5.5808) -- (4.6124, 5.581)
   -- (4.6151, 5.5825) -- (4.6794, 5.6526) -- (4.6834, 5.6562) -- (4.7042, 
  5.6663) -- (4.7058, 5.6666) -- (4.7236, 5.6689) -- (4.7264, 5.6637) -- 
  (4.7285, 5.6579) -- (4.736, 5.6418) -- (4.7377, 5.64) -- (4.7513, 5.6379) -- 
  (4.7558, 5.6459) -- (4.757, 5.6521) -- (4.7569, 5.6567) -- (4.7578, 5.668) -- 
  (4.7596, 5.6749) -- (4.7629, 5.6795) -- (4.7795, 5.694) -- (4.7896, 5.6987) --
   (4.8218, 5.6982) -- (4.8238, 5.6972) -- (4.8274, 5.6922) -- (4.828, 5.6904) 
  -- (4.8291, 5.6822) -- (4.839, 5.6673) -- (4.8424, 5.6663) -- (4.8488, 5.6692)
   -- (4.8922, 5.7148) -- (4.9118, 5.7412) -- (4.9124, 5.7421) -- (4.9123, 
  5.7431) -- (4.9096, 5.7597) -- (4.9073, 5.7717) -- (4.9116, 5.7864) -- 
  (4.9224, 5.8124) -- (4.9244, 5.8169) -- (4.9848, 5.8749) -- (4.9935, 5.8828) 
  -- (5.0095, 5.8955) -- (5.0342, 5.8973) -- (5.0594, 5.9196) -- (5.0468, 5.931)
   -- (5.0454, 5.9437) -- (5.0453, 5.9467) -- (5.0476, 5.9644) -- (5.0482, 
  5.9673) -- (5.0516, 5.9731) -- (5.0564, 5.9743) -- (5.1022, 5.9212) -- 
  (5.1111, 5.8908) -- (5.1246, 5.8619) -- (5.1395, 5.8559) -- (5.1448, 5.8582) 
  -- (5.176, 5.8638) -- (5.2006, 5.8599) -- (5.2362, 5.8611) -- (5.2428, 5.8739)
   -- (5.3009, 5.8703) -- (5.3076, 5.8658) -- (5.3114, 5.8595) -- (5.3136, 
  5.8334) -- (5.3135, 5.8286) -- (5.3086, 5.8105) -- (5.3073, 5.8062) -- 
  (5.3034, 5.7972) -- (5.2942, 5.782) -- (5.2869, 5.7724) -- (5.277, 5.7606) -- 
  (5.2678, 5.7508) -- (5.261, 5.7447) -- (5.2466, 5.7273) -- (5.2422, 5.7199) --
   (5.2413, 5.7155) -- (5.2396, 5.7044) -- (5.2498, 5.7032) -- (5.3032, 5.7014) 
  -- (5.3262, 5.705) -- (5.3337, 5.7075) -- (5.3522, 5.7335) -- (5.3579, 5.7499)
   -- (5.3593, 5.7585) -- (5.4054, 5.7779) -- (5.4083, 5.7781) -- cycle(4.8709, 
  5.1067) -- (4.8652, 5.097) -- (4.8591, 5.0892) -- (4.8549, 5.0859) -- (4.8513,
   5.0839) -- (4.8473, 5.0827) -- (4.8407, 5.0829) -- (4.8324, 5.0834) -- 
  (4.8218, 5.0855) -- (4.7705, 5.0839) -- (4.7602, 5.0635) -- (4.7599, 5.0625) 
  -- (4.7611, 5.0597) -- (4.7659, 5.0542) -- (4.7692, 5.0516) -- (4.7702, 
  5.0502) -- (4.7704, 5.049) -- (4.7698, 5.0455) -- (4.7598, 5.0462) -- (4.7552,
   5.0471) -- (4.7438, 5.0521) -- (4.7392, 5.0547) -- (4.7344, 5.0533) -- 
  (4.7191, 5.043) -- (4.7187, 5.0394) -- (4.7149, 4.9829) -- (4.7153, 4.9773) --
   (4.7302, 4.9802) -- (4.7447, 4.9699) -- (4.7241, 4.9431) -- (4.7203, 4.9403) 
  -- (4.7161, 4.934) -- (4.7135, 4.929) -- (4.7135, 4.915) -- (4.7177, 4.9102) 
  -- (4.7214, 4.9038) -- (4.7158, 4.8942) -- (4.7112, 4.8895) -- (4.7021, 
  4.8846) -- (4.7037, 4.8775) -- (4.7058, 4.8758) -- (4.7247, 4.864) -- (4.7428,
   4.8628) -- (4.7488, 4.8634) -- (4.7437, 4.8644) -- (4.7445, 4.8704) -- 
  (4.7649, 4.8832) -- (4.7794, 4.8886) -- (4.7813, 4.8891) -- (4.7922, 4.8899) 
  -- (4.8231, 4.8833) -- (4.8321, 4.8713) -- (4.8495, 4.8801) -- (4.8765, 
  4.8536) -- (4.8949, 4.8522) -- (4.8928, 4.8818) -- (4.8931, 4.8834) -- (4.894,
   4.8842) -- (4.909, 4.8913) -- (4.9156, 4.8943) -- (4.9204, 4.8958) -- 
  (4.9273, 4.8788) -- (4.962, 4.8677) -- (5.0267, 4.8494) -- (5.0383, 4.847) -- 
  (5.0529, 4.852) -- (5.0564, 4.8584) -- (5.0569, 4.8617) -- (5.0556, 4.8649) --
   (5.0534, 4.865) -- (5.0486, 4.8665) -- (5.0487, 4.8688) -- (5.0653, 4.8827) 
  -- (5.0715, 4.8984) -- (5.079, 4.9211) -- (5.0649, 4.9357) -- (5.0567, 4.9363)
   -- (5.0334, 4.9515) -- (5.0186, 4.9558) -- (5.0117, 4.9847) -- (4.997, 
  5.0168) -- (4.9644, 5.0513) -- (4.9216, 5.131) -- (4.9135, 5.1353) -- (4.9062,
   5.1297) -- (4.9008, 5.1224) -- (4.8953, 5.1094) -- (4.8913, 5.0985) -- 
  (4.8859, 5.0969) -- (4.8761, 5.1022) -- cycle;


  \node[text=black,line width=0.0092cm,anchor=center] (text19) at (5.1489, 
  4.5191){\resizebox{\ifdim\width>2em 2em\else\width\fi}{!}{#2}};
}

\newcommand{\drawbremen}[2]{%

  %Bremen
  \path[draw=black,fill=#1,line join=round,line width=0.0046cm] (2.1116, 
  5.8916) -- (2.1168, 5.895) -- (2.1185, 5.8991) -- (2.1192, 5.9016) -- (2.1188,
   5.9087) -- (2.1182, 5.9118) -- (2.117, 5.9147) -- (2.0974, 5.9493) -- 
  (2.0895, 5.9616) -- (2.0983, 5.9724) -- (2.1068, 5.9731) -- (2.1563, 5.9706) 
  -- (2.1717, 5.9682) -- (2.1723, 5.9673) -- (2.1625, 5.9217) -- (2.1707, 
  5.8861) -- (2.1698, 5.8814) -- (2.1464, 5.8566) -- (2.1447, 5.8565) -- (2.128,
   5.8583) -- (2.125, 5.8601) -- (2.1235, 5.8622) -- (2.1268, 5.8671) -- 
  (2.1274, 5.8688) -- (2.1274, 5.8706) -- (2.1149, 5.8872) -- cycle(2.3489, 
  5.521) -- (2.3363, 5.5363) -- (2.33, 5.5407) -- (2.3271, 5.5408) -- (2.3166, 
  5.5342) -- (2.2734, 5.5354) -- (2.2389, 5.553) -- (2.2001, 5.572) -- (2.1407, 
  5.579) -- (2.0718, 5.6162) -- (2.0693, 5.6241) -- (2.0692, 5.6303) -- (2.069, 
  5.6237) -- (2.071, 5.6116) -- (2.0746, 5.6038) -- (2.0775, 5.5986) -- (2.0843,
   5.5887) -- (2.0985, 5.5773) -- (2.1203, 5.5668) -- (2.1276, 5.5637) -- 
  (2.136, 5.5625) -- (2.143, 5.5607) -- (2.1463, 5.5579) -- (2.1486, 5.5532) -- 
  (2.1631, 5.5147) -- (2.1644, 5.5026) -- (2.1728, 5.4969) -- (2.1908, 5.4775) 
  -- (2.1953, 5.4713) -- (2.1959, 5.4655) -- (2.1965, 5.4415) -- (2.2009, 
  5.4369) -- (2.2059, 5.4326) -- (2.2087, 5.4319) -- (2.2184, 5.4352) -- 
  (2.2218, 5.4366) -- (2.2341, 5.4365) -- (2.2832, 5.4195) -- (2.3117, 5.4083) 
  -- (2.3415, 5.4338) -- (2.342, 5.4519) -- (2.3385, 5.4956) -- (2.3313, 5.5069)
   -- (2.332, 5.508) -- (2.3333, 5.5087) -- (2.3422, 5.512) -- (2.3514, 5.5155) 
  -- cycle;


  \node[text=black,line width=0.0092cm,anchor=south east] (text10) at (0.52, 
  5.3912){\resizebox{\ifdim\width>2em 2em\else\width\fi}{!}{#2}};
  \path[draw=black,line width=0.0141cm] (2.2604, 5.4885) -- (text10.east);
}

\newcommand{\drawgermany}{%
  %Deutschland
  \path[draw=black,line join=round,line width=0.0092cm] (0.7185, 2.203) -- 
  (0.7141, 2.2056) -- (0.7018, 2.1921) -- (0.6993, 2.1881) -- (0.6988, 2.1864) 
  -- (0.7025, 2.1291) -- (0.707, 2.1239) -- (0.7185, 2.1232) -- (0.7246, 2.1246)
   -- (0.7278, 2.1261) -- (0.7431, 2.1293) -- (0.7645, 2.1214) -- (0.8038, 
  2.091) -- (0.8083, 2.0876) -- (0.8132, 2.0824) -- (0.8169, 2.0767) -- (0.8114,
   2.0753) -- (0.805, 2.0703) -- (0.804, 2.0678) -- (0.8046, 2.0638) -- (0.8187,
   2.0091) -- (0.8194, 2.0075) -- (0.832, 1.9843) -- (0.8351, 1.98) -- (0.8437, 
  1.9691) -- (0.8465, 1.9663) -- (0.8613, 1.9581) -- (0.879, 1.9428) -- (0.8909,
   1.9134) -- (0.8913, 1.9117) -- (0.8906, 1.8949) -- (0.9102, 1.8837) -- 
  (0.9147, 1.8713) -- (0.9015, 1.8552) -- (0.9066, 1.8427) -- (0.9172, 1.8312) 
  -- (0.9679, 1.822) -- (0.9834, 1.8247) -- (0.992, 1.8383) -- (0.9936, 1.8422) 
  -- (0.994, 1.845) -- (0.9888, 1.8623) -- (0.9835, 1.8698) -- (0.9795, 1.8733) 
  -- (0.9807, 1.8753) -- (0.9939, 1.8836) -- (1.0307, 1.8794) -- (1.0435, 
  1.8758) -- (1.0649, 1.8681) -- (1.0851, 1.8527) -- (1.0965, 1.835) -- (1.0959,
   1.8302) -- (1.098, 1.8171) -- (1.1041, 1.7826) -- (1.106, 1.7781) -- (1.1087,
   1.7755) -- (1.1117, 1.7744) -- (1.1178, 1.7765) -- (1.1328, 1.7913) -- 
  (1.1404, 1.8146) -- (1.1492, 1.7994) -- (1.161, 1.7881) -- (1.1654, 1.7856) --
   (1.1731, 1.7811) -- (1.1753, 1.7801) -- (1.1807, 1.7864) -- (1.1868, 1.7888) 
  -- (1.1908, 1.7898) -- (1.2286, 1.7862) -- (1.242, 1.7807) -- (1.2509, 1.7739)
   -- (1.2576, 1.7717) -- (1.2609, 1.7738) -- (1.3009, 1.8059) -- (1.3035, 
  1.8156) -- (1.3029, 1.8183) -- (1.3033, 1.8229) -- (1.3133, 1.8255) -- 
  (1.3316, 1.8288) -- (1.3533, 1.8326) -- (1.3826, 1.8177) -- (1.3844, 1.8048) 
  -- (1.3826, 1.7934) -- (1.3825, 1.7861) -- (1.4017, 1.749) -- (1.4145, 1.7381)
   -- (1.4229, 1.7325) -- (1.4639, 1.7073) -- (1.4904, 1.6972) -- (1.526, 
  1.6952) -- (1.5438, 1.6961) -- (1.5465, 1.6967) -- (1.5779, 1.7039) -- 
  (1.6229, 1.6971) -- (1.6319, 1.6929) -- (1.6442, 1.6918) -- (1.6741, 1.6833) 
  -- (1.7198, 1.6612) -- (1.7264, 1.6546) -- (1.7325, 1.6472) -- (1.7336, 
  1.6446) -- (1.7485, 1.6377) -- (1.7761, 1.6265) -- (1.7839, 1.625) -- (1.793, 
  1.6254) -- (1.7959, 1.6266) -- (1.8279, 1.6228) -- (1.8292, 1.6219) -- 
  (1.8308, 1.6192) -- (1.8337, 1.6143) -- (1.8183, 1.6094) -- (1.8153, 1.6085) 
  -- (1.8126, 1.6069) -- (1.8097, 1.604) -- (1.7789, 1.5554) -- (1.7707, 1.5403)
   -- (1.7669, 1.5322) -- (1.7637, 1.5243) -- (1.7583, 1.5091) -- (1.7543, 
  1.4973) -- (1.7519, 1.4861) -- (1.7501, 1.4789) -- (1.7479, 1.4738) -- (1.744,
   1.4678) -- (1.7375, 1.46) -- (1.7305, 1.4536) -- (1.6985, 1.431) -- (1.6756, 
  1.4251) -- (1.672, 1.425) -- (1.6675, 1.423) -- (1.6662, 1.4217) -- (1.6648, 
  1.4187) -- (1.658, 1.3855) -- (1.6143, 1.3351) -- (1.6096, 1.3315) -- (1.5961,
   1.3239) -- (1.5851, 1.3193) -- (1.5829, 1.3171) -- (1.5815, 1.3151) -- 
  (1.5569, 1.267) -- (1.5555, 1.263) -- (1.5552, 1.2601) -- (1.5553, 1.2558) -- 
  (1.5572, 1.2458) -- (1.5579, 1.2374) -- (1.5575, 1.2047) -- (1.5574, 1.2009) 
  -- (1.5564, 1.196) -- (1.5535, 1.1897) -- (1.5509, 1.1858) -- (1.5297, 1.1428)
   -- (1.5102, 1.095) -- (1.5083, 1.0891) -- (1.507, 1.0831) -- (1.5062, 1.0766)
   -- (1.5062, 1.0712) -- (1.5071, 1.0629) -- (1.5138, 1.0272) -- (1.5137, 
  1.0244) -- (1.5129, 1.0215) -- (1.5113, 1.0181) -- (1.51, 1.0163) -- (1.5081, 
  1.0144) -- (1.5036, 1.0117) -- (1.4883, 1.0061) -- (1.4839, 1.0028) -- 
  (1.4803, 0.9984) -- (1.4782, 0.994) -- (1.4775, 0.9908) -- (1.471, 0.9568) -- 
  (1.4648, 0.9358) -- (1.4618, 0.926) -- (1.4601, 0.9221) -- (1.4385, 0.8927) --
   (1.4184, 0.8664) -- (1.4169, 0.8635) -- (1.4025, 0.8324) -- (1.4015, 0.8299) 
  -- (1.4014, 0.8257) -- (1.3957, 0.7967) -- (1.3952, 0.7928) -- (1.3938, 
  0.7531) -- (1.3937, 0.7499) -- (1.3945, 0.7471) -- (1.3959, 0.745) -- (1.4154,
   0.7205) -- (1.4197, 0.7133) -- (1.4222, 0.7075) -- (1.425, 0.6961) -- 
  (1.4252, 0.6902) -- (1.4239, 0.6857) -- (1.4192, 0.6771) -- (1.4148, 0.6709) 
  -- (1.398, 0.6475) -- (1.3814, 0.6038) -- (1.3802, 0.599) -- (1.3802, 0.5926) 
  -- (1.381, 0.5891) -- (1.3841, 0.5804) -- (1.3846, 0.5752) -- (1.3841, 0.5704)
   -- (1.3821, 0.564) -- (1.3708, 0.5379) -- (1.3623, 0.5177) -- (1.3613, 
  0.5144) -- (1.3608, 0.5069) -- (1.3613, 0.503) -- (1.3702, 0.4765) -- (1.3715,
   0.4716) -- (1.3716, 0.4684) -- (1.3703, 0.4618) -- (1.368, 0.4562) -- 
  (1.3655, 0.4526) -- (1.3634, 0.4503) -- (1.3489, 0.4388) -- (1.3476, 0.4369) 
  -- (1.3468, 0.4333) -- (1.3508, 0.4073) -- (1.3514, 0.404) -- (1.3529, 0.4) --
   (1.3546, 0.3972) -- (1.3946, 0.345) -- (1.4477, 0.3311) -- (1.4509, 0.3221) 
  -- (1.4525, 0.3138) -- (1.4524, 0.3107) -- (1.4295, 0.3014) -- (1.4251, 
  0.3027) -- (1.4228, 0.3015) -- (1.429, 0.2913) -- (1.4404, 0.2795) -- (1.4418,
   0.2781) -- (1.4488, 0.2752) -- (1.4564, 0.2747) -- (1.4589, 0.2748) -- 
  (1.4942, 0.2848) -- (1.5187, 0.2938) -- (1.5234, 0.2963) -- (1.5329, 0.3063) 
  -- (1.5346, 0.31) -- (1.5478, 0.323) -- (1.5803, 0.3242) -- (1.5825, 0.3238) 
  -- (1.5859, 0.3223) -- (1.5873, 0.321) -- (1.597, 0.3066) -- (1.6278, 0.294) 
  -- (1.666, 0.2856) -- (1.6677, 0.2856) -- (1.7099, 0.2912) -- (1.7124, 0.2921)
   -- (1.7164, 0.2953) -- (1.7181, 0.2991) -- (1.719, 0.307) -- (1.7197, 0.3094)
   -- (1.721, 0.3114) -- (1.724, 0.3144) -- (1.7812, 0.3476) -- (1.783, 0.3484) 
  -- (1.7855, 0.349) -- (1.7901, 0.3484) -- (1.8195, 0.3429) -- (1.8365, 0.3384)
   -- (1.842, 0.3355) -- (1.8439, 0.3336) -- (1.8442, 0.3325) -- (1.8425, 
  0.3242) -- (1.8425, 0.3209) -- (1.8437, 0.3179) -- (1.8454, 0.3158) -- 
  (1.8602, 0.3019) -- (1.8626, 0.3006) -- (1.8639, 0.3) -- (1.8673, 0.2996) -- 
  (1.9285, 0.2945) -- (1.933, 0.2951) -- (1.9499, 0.2996) -- (1.9639, 0.3042) --
   (1.969, 0.3076) -- (1.9688, 0.3087) -- (1.9655, 0.3138) -- (1.9529, 0.3293) 
  -- (1.9873, 0.3564) -- (1.9978, 0.354) -- (1.9982, 0.3538) -- (2.0099, 0.3477)
   -- (2.0167, 0.3403) -- (2.0179, 0.3385) -- (2.0184, 0.336) -- (2.0176, 
  0.3345) -- (2.0153, 0.3336) -- (2.0134, 0.327) -- (2.0158, 0.3221) -- (2.0253,
   0.3205) -- (2.0304, 0.3252) -- (2.0385, 0.3338) -- (2.0521, 0.3622) -- 
  (2.0557, 0.3705) -- (2.0555, 0.3734) -- (2.042, 0.3919) -- (2.0386, 0.3925) --
   (2.0365, 0.3926) -- (2.013, 0.3907) -- (2.0074, 0.3895) -- (1.9951, 0.3843) 
  -- (1.9912, 0.3818) -- (1.9925, 0.3764) -- (1.9567, 0.3617) -- (1.9547, 
  0.3625) -- (1.9184, 0.3888) -- (1.9145, 0.3955) -- (1.9143, 0.3968) -- 
  (1.9188, 0.4154) -- (1.9485, 0.4667) -- (1.9591, 0.4803) -- (1.969, 0.4886) --
   (1.9827, 0.491) -- (2.0123, 0.4974) -- (2.0162, 0.5085) -- (2.0231, 0.5183) 
  -- (2.0489, 0.5132) -- (2.0518, 0.4816) -- (2.057, 0.4737) -- (2.0594, 0.4727)
   -- (2.0679, 0.4783) -- (2.0699, 0.4808) -- (2.0733, 0.4866) -- (2.071, 0.497)
   -- (2.0923, 0.4901) -- (2.1072, 0.4771) -- (2.1176, 0.46) -- (2.1117, 0.4534)
   -- (2.1095, 0.4454) -- (2.1188, 0.4128) -- (2.1195, 0.41) -- (2.1547, 0.3948)
   -- (2.1599, 0.3926) -- (2.1625, 0.393) -- (2.1717, 0.4097) -- (2.2138, 0.373)
   -- (2.2254, 0.3661) -- (2.2273, 0.3661) -- (2.2539, 0.3737) -- (2.2653, 
  0.3793) -- (2.2724, 0.3835) -- (2.277, 0.3878) -- (2.2809, 0.3905) -- (2.2917,
   0.3962) -- (2.3026, 0.4005) -- (2.3058, 0.4012) -- (2.3084, 0.4012) -- 
  (2.3248, 0.3994) -- (2.3288, 0.3987) -- (2.3514, 0.3936) -- (2.3559, 0.3922) 
  -- (2.3595, 0.3901) -- (2.3806, 0.3789) -- (2.4038, 0.3694) -- (2.4163, 
  0.3714) -- (2.4309, 0.3802) -- (2.4272, 0.3949) -- (2.4036, 0.4448) -- 
  (2.3883, 0.4575) -- (2.3854, 0.4578) -- (2.3807, 0.457) -- (2.3795, 0.4573) --
   (2.3494, 0.4779) -- (2.3443, 0.4815) -- (2.3373, 0.4871) -- (2.3146, 0.5106) 
  -- (2.3134, 0.5128) -- (2.3135, 0.514) -- (2.3155, 0.5186) -- (2.3191, 0.5243)
   -- (2.333, 0.5219) -- (2.3357, 0.5198) -- (2.339, 0.5142) -- (2.3414, 0.5132)
   -- (2.3607, 0.5008) -- (2.4308, 0.4547) -- (2.4365, 0.4495) -- (2.437, 
  0.4486) -- (2.4373, 0.4452) -- (2.4364, 0.4435) -- (2.4377, 0.4357) -- 
  (2.4394, 0.431) -- (2.4442, 0.4229) -- (2.4514, 0.4156) -- (2.4608, 0.4072) --
   (2.4855, 0.3905) -- (2.5041, 0.3804) -- (2.5123, 0.376) -- (2.5148, 0.3749) 
  -- (2.5181, 0.3747) -- (2.5228, 0.3761) -- (2.5276, 0.3788) -- (2.5298, 
  0.3792) -- (2.5421, 0.3812) -- (2.5459, 0.3817) -- (2.5513, 0.3814) -- 
  (2.5602, 0.3789) -- (2.6052, 0.3646) -- (2.6106, 0.3602) -- (2.6166, 0.352) --
   (2.6202, 0.3448) -- (2.6214, 0.3369) -- (2.6288, 0.3195) -- (2.6419, 0.3052) 
  -- (2.6448, 0.3035) -- (2.6479, 0.3021) -- (2.6556, 0.3013) -- (2.6747, 
  0.3013) -- (2.676, 0.2931) -- (2.6841, 0.2878) -- (2.7134, 0.2757) -- (2.7483,
   0.2648) -- (2.7712, 0.2963) -- (2.7771, 0.3048) -- (2.7845, 0.3105) -- 
  (2.7994, 0.3112) -- (2.8007, 0.3109) -- (2.8142, 0.3027) -- (2.8167, 0.2949) 
  -- (2.8157, 0.2776) -- (2.8141, 0.2747) -- (2.8125, 0.2725) -- (2.8112, 
  0.2722) -- (2.81, 0.2712) -- (2.8087, 0.2677) -- (2.8091, 0.2658) -- (2.8464, 
  0.2479) -- (2.8772, 0.2489) -- (2.884, 0.2517) -- (2.9041, 0.2627) -- (2.9062,
   0.2645) -- (2.9044, 0.257) -- (2.9042, 0.2509) -- (2.9043, 0.2411) -- 
  (2.9236, 0.2081) -- (2.9273, 0.2043) -- (2.9286, 0.2038) -- (2.9334, 0.2039) 
  -- (2.9414, 0.2061) -- (2.9603, 0.1985) -- (2.9779, 0.1786) -- (2.9858, 
  0.1679) -- (2.9941, 0.1536) -- (2.9877, 0.1434) -- (2.9867, 0.1425) -- 
  (2.9812, 0.1402) -- (2.9761, 0.1303) -- (2.9905, 0.0842) -- (3.0326, 0.0975) 
  -- (3.0351, 0.0999) -- (3.0347, 0.1034) -- (3.0349, 0.1134) -- (3.0429, 
  0.1196) -- (3.0718, 0.1164) -- (3.0733, 0.1158) -- (3.0777, 0.1098) -- 
  (3.0783, 0.102) -- (3.0779, 0.101) -- (3.0712, 0.0914) -- (3.0652, 0.084) -- 
  (3.0412, 0.0283) -- (3.039, 0.0212) -- (3.0369, 0.0129) -- (3.0372, 0.0082) --
   (3.0382, 0.0046) -- (3.0409, 0.0046) -- (3.0819, 0.0119) -- (3.1071, 0.0213) 
  -- (3.1343, 0.0381) -- (3.1481, 0.0514) -- (3.1926, 0.1083) -- (3.2292, 
  0.1574) -- (3.2261, 0.1932) -- (3.2254, 0.2002) -- (3.221, 0.2053) -- (3.215, 
  0.2082) -- (3.2112, 0.2067) -- (3.2093, 0.2073) -- (3.2079, 0.209) -- (3.2018,
   0.225) -- (3.2007, 0.2945) -- (3.2042, 0.3005) -- (3.2055, 0.3009) -- 
  (3.2255, 0.304) -- (3.228, 0.304) -- (3.2283, 0.2966) -- (3.2263, 0.2873) -- 
  (3.2205, 0.2803) -- (3.2165, 0.2744) -- (3.2208, 0.2713) -- (3.2389, 0.2597) 
  -- (3.2614, 0.2553) -- (3.2898, 0.254) -- (3.2925, 0.2543) -- (3.298, 0.2678) 
  -- (3.2981, 0.2693) -- (3.3023, 0.2779) -- (3.3097, 0.2865) -- (3.3209, 
  0.2855) -- (3.3674, 0.2779) -- (3.3682, 0.2754) -- (3.3676, 0.2737) -- 
  (3.3679, 0.2708) -- (3.3698, 0.265) -- (3.4146, 0.2468) -- (3.4362, 0.2427) --
   (3.4719, 0.2566) -- (3.492, 0.259) -- (3.5045, 0.2408) -- (3.4865, 0.2271) --
   (3.4824, 0.2231) -- (3.4819, 0.2106) -- (3.4832, 0.2069) -- (3.4877, 0.2047) 
  -- (3.4897, 0.2045) -- (3.5196, 0.1981) -- (3.5407, 0.1763) -- (3.5554, 
  0.1591) -- (3.5563, 0.1506) -- (3.5537, 0.1483) -- (3.5483, 0.1473) -- 
  (3.5476, 0.1463) -- (3.5462, 0.142) -- (3.5474, 0.1308) -- (3.5531, 0.1281) --
   (3.564, 0.126) -- (3.5702, 0.1256) -- (3.6188, 0.128) -- (3.6504, 0.1438) -- 
  (3.6623, 0.1506) -- (3.6656, 0.1523) -- (3.697, 0.1647) -- (3.7248, 0.164) -- 
  (3.7261, 0.1631) -- (3.7279, 0.1595) -- (3.7196, 0.1517) -- (3.7095, 0.1344) 
  -- (3.7086, 0.1304) -- (3.7101, 0.1286) -- (3.7124, 0.1284) -- (3.739, 0.1311)
   -- (3.7408, 0.1317) -- (3.748, 0.1359) -- (3.7517, 0.1406) -- (3.7547, 
  0.1497) -- (3.7563, 0.1532) -- (3.7614, 0.1623) -- (3.8115, 0.2068) -- 
  (3.8272, 0.2248) -- (3.8287, 0.2282) -- (3.8413, 0.2427) -- (3.8435, 0.2444) 
  -- (3.8457, 0.2453) -- (3.8501, 0.2399) -- (3.8561, 0.2382) -- (3.8878, 
  0.2351) -- (3.928, 0.2444) -- (3.9302, 0.2453) -- (3.9401, 0.2558) -- (3.9396,
   0.2687) -- (3.9385, 0.2711) -- (3.9383, 0.2729) -- (3.9384, 0.2749) -- 
  (3.9395, 0.2852) -- (3.9499, 0.3079) -- (3.951, 0.3092) -- (3.9672, 0.3197) --
   (3.9779, 0.3179) -- (4.0673, 0.3174) -- (4.0929, 0.3128) -- (4.1022, 0.3121) 
  -- (4.1042, 0.3133) -- (4.115, 0.332) -- (4.1445, 0.3434) -- (4.1803, 0.3482) 
  -- (4.2132, 0.3441) -- (4.2391, 0.3514) -- (4.2532, 0.3433) -- (4.283, 0.3403)
   -- (4.3172, 0.3394) -- (4.3298, 0.3422) -- (4.3312, 0.3433) -- (4.3324, 
  0.3451) -- (4.333, 0.3474) -- (4.3328, 0.3494) -- (4.3283, 0.3672) -- (4.3256,
   0.3765) -- (4.315, 0.3997) -- (4.3163, 0.4315) -- (4.3557, 0.4701) -- 
  (4.3604, 0.4718) -- (4.3645, 0.4662) -- (4.3658, 0.462) -- (4.3617, 0.452) -- 
  (4.3584, 0.4461) -- (4.369, 0.4135) -- (4.4129, 0.4301) -- (4.4115, 0.4259) --
   (4.4135, 0.4243) -- (4.4299, 0.4182) -- (4.478, 0.413) -- (4.4791, 0.4107) --
   (4.4956, 0.3883) -- (4.5178, 0.3658) -- (4.5263, 0.3662) -- (4.5706, 0.3889) 
  -- (4.5789, 0.4066) -- (4.5837, 0.4134) -- (4.5865, 0.4142) -- (4.6225, 0.422)
   -- (4.63, 0.4237) -- (4.6339, 0.424) -- (4.6414, 0.424) -- (4.6767, 0.4142) 
  -- (4.6903, 0.4101) -- (4.7148, 0.359) -- (4.706, 0.3495) -- (4.6993, 0.3321) 
  -- (4.6982, 0.3288) -- (4.7015, 0.3194) -- (4.7085, 0.3083) -- (4.7148, 
  0.3022) -- (4.7852, 0.2543) -- (4.8122, 0.2393) -- (4.8195, 0.2359) -- 
  (4.8437, 0.2261) -- (4.8454, 0.2256) -- (4.8472, 0.2266) -- (4.8723, 0.2538) 
  -- (4.8724, 0.2605) -- (4.871, 0.2815) -- (4.8702, 0.2838) -- (4.8675, 0.286) 
  -- (4.8645, 0.287) -- (4.8616, 0.2894) -- (4.8604, 0.2911) -- (4.8602, 0.2924)
   -- (4.8654, 0.3332) -- (4.8663, 0.3396) -- (4.8832, 0.366) -- (4.895, 0.3799)
   -- (4.8978, 0.3829) -- (4.8994, 0.3898) -- (4.8983, 0.3939) -- (4.8787, 
  0.4421) -- (4.8652, 0.4594) -- (4.8627, 0.4612) -- (4.8446, 0.47) -- (4.8401, 
  0.4697) -- (4.8386, 0.4691) -- (4.8365, 0.465) -- (4.8372, 0.464) -- (4.8365, 
  0.4624) -- (4.835, 0.4606) -- (4.8285, 0.4559) -- (4.8259, 0.4546) -- (4.817, 
  0.454) -- (4.7863, 0.4563) -- (4.7784, 0.4621) -- (4.7734, 0.4676) -- (4.7727,
   0.4696) -- (4.7728, 0.4713) -- (4.8071, 0.5398) -- (4.8307, 0.5819) -- 
  (4.8315, 0.5848) -- (4.8309, 0.5889) -- (4.8292, 0.5932) -- (4.8034, 0.6396) 
  -- (4.7827, 0.672) -- (4.7742, 0.6841) -- (4.7722, 0.6858) -- (4.7678, 0.6878)
   -- (4.7458, 0.6918) -- (4.7434, 0.6942) -- (4.7412, 0.6983) -- (4.7328, 
  0.7317) -- (4.7322, 0.734) -- (4.7296, 0.7388) -- (4.7258, 0.7435) -- (4.7184,
   0.7512) -- (4.6999, 0.7663) -- (4.6736, 0.7881) -- (4.6678, 0.7957) -- 
  (4.6649, 0.8021) -- (4.6629, 0.8092) -- (4.6604, 0.8261) -- (4.6606, 0.8286) 
  -- (4.6617, 0.8302) -- (4.6663, 0.8361) -- (4.6901, 0.8557) -- (4.71, 0.8713) 
  -- (4.7116, 0.8733) -- (4.7197, 0.8833) -- (4.7285, 0.9022) -- (4.7293, 
  0.9048) -- (4.7296, 0.9111) -- (4.7301, 0.9128) -- (4.7321, 0.9157) -- 
  (4.7347, 0.9175) -- (4.7366, 0.9181) -- (4.7628, 0.9225) -- (4.7797, 0.9233) 
  -- (4.7827, 0.924) -- (4.7866, 0.9257) -- (4.79, 0.928) -- (4.7933, 0.9309) --
   (4.8145, 0.9556) -- (4.8164, 0.9591) -- (4.8194, 0.9659) -- (4.8211, 0.9686) 
  -- (4.8241, 0.9717) -- (4.8272, 0.9738) -- (4.8632, 0.9933) -- (4.8684, 
  0.9945) -- (4.8739, 0.9943) -- (4.8851, 0.994) -- (4.8919, 0.9953) -- (4.8966,
   0.9972) -- (4.9224, 1.0117) -- (4.9679, 1.0123) -- (4.9722, 1.0131) -- 
  (4.9756, 1.0149) -- (5.0117, 1.039) -- (5.0166, 1.0428) -- (5.063, 1.0907) -- 
  (5.0655, 1.0932) -- (5.0669, 1.0956) -- (5.0686, 1.1001) -- (5.0803, 1.1457) 
  -- (5.0811, 1.1492) -- (5.0809, 1.1519) -- (5.0797, 1.1562) -- (5.0726, 
  1.1676) -- (5.0722, 1.1693) -- (5.0723, 1.1728) -- (5.0748, 1.1819) -- 
  (5.0772, 1.1867) -- (5.0887, 1.2224) -- (5.0891, 1.2242) -- (5.0828, 1.2488) 
  -- (5.0822, 1.2498) -- (5.0752, 1.2582) -- (5.0934, 1.2641) -- (5.0979, 
  1.2662) -- (5.1078, 1.2745) -- (5.1149, 1.2824) -- (5.1165, 1.2854) -- 
  (5.1199, 1.3002) -- (5.169, 1.2816) -- (5.2151, 1.2526) -- (5.2175, 1.2506) --
   (5.2442, 1.2367) -- (5.2573, 1.2334) -- (5.2593, 1.2349) -- (5.3014, 1.2942) 
  -- (5.3135, 1.3358) -- (5.3133, 1.3507) -- (5.3024, 1.4032) -- (5.2905, 1.424)
   -- (5.2987, 1.4388) -- (5.305, 1.4583) -- (5.3103, 1.4788) -- (5.2984, 
  1.5032) -- (5.2454, 1.5814) -- (5.2399, 1.5815) -- (5.2362, 1.5778) -- 
  (5.2333, 1.5764) -- (5.2163, 1.5757) -- (5.2083, 1.5755) -- (5.2051, 1.5768) 
  -- (5.1946, 1.5889) -- (5.1885, 1.6023) -- (5.1817, 1.6182) -- (5.172, 1.6305)
   -- (5.1488, 1.6561) -- (5.1472, 1.6575) -- (5.1445, 1.6592) -- (5.1129, 
  1.6605) -- (5.0994, 1.6553) -- (5.099, 1.647) -- (5.0998, 1.6421) -- (5.1011, 
  1.6382) -- (5.1014, 1.63) -- (5.0955, 1.6288) -- (5.0936, 1.6291) -- (5.0534, 
  1.656) -- (5.0434, 1.6628) -- (5.0342, 1.7021) -- (5.0282, 1.7229) -- (5.023, 
  1.7308) -- (4.9936, 1.7634) -- (4.981, 1.7734) -- (4.9584, 1.7895) -- (4.954, 
  1.7913) -- (4.9503, 1.7912) -- (4.9378, 1.7881) -- (4.9314, 1.7854) -- 
  (4.9256, 1.7837) -- (4.9097, 1.7902) -- (4.9019, 1.7948) -- (4.892, 1.8016) --
   (4.8838, 1.8102) -- (4.8831, 1.8376) -- (4.8799, 1.8377) -- (4.8776, 1.8389) 
  -- (4.8691, 1.8457) -- (4.8539, 1.8593) -- (4.8458, 1.8703) -- (4.7922, 
  1.9416) -- (4.7736, 1.9659) -- (4.7548, 1.9835) -- (4.7406, 1.9915) -- 
  (4.7377, 1.9921) -- (4.7106, 1.9966) -- (4.7028, 1.9975) -- (4.6975, 1.9968) 
  -- (4.6983, 1.9903) -- (4.7017, 1.9862) -- (4.7035, 1.9809) -- (4.7023, 1.98) 
  -- (4.6962, 1.98) -- (4.6501, 1.9867) -- (4.6398, 1.99) -- (4.6354, 1.9923) --
   (4.6216, 2.0175) -- (4.6233, 2.0244) -- (4.6205, 2.036) -- (4.5577, 2.0858) 
  -- (4.5419, 2.1094) -- (4.5475, 2.1168) -- (4.5479, 2.1178) -- (4.5479, 2.123)
   -- (4.547, 2.1425) -- (4.5457, 2.1552) -- (4.5449, 2.1571) -- (4.5421, 
  2.1614) -- (4.5393, 2.1616) -- (4.5384, 2.1607) -- (4.5372, 2.1568) -- 
  (4.5323, 2.1573) -- (4.5209, 2.1614) -- (4.5123, 2.1668) -- (4.5154, 2.1704) 
  -- (4.5116, 2.1735) -- (4.5028, 2.1861) -- (4.4776, 2.2405) -- (4.4705, 
  2.2894) -- (4.4668, 2.3045) -- (4.4652, 2.3052) -- (4.4243, 2.3169) -- 
  (4.4199, 2.3181) -- (4.418, 2.3192) -- (4.4163, 2.3211) -- (4.3989, 2.3455) --
   (4.3947, 2.3524) -- (4.3913, 2.3649) -- (4.3909, 2.3671) -- (4.3913, 2.3697) 
  -- (4.4304, 2.3967) -- (4.4318, 2.4029) -- (4.4346, 2.4337) -- (4.44, 2.4497) 
  -- (4.4605, 2.488) -- (4.4658, 2.4945) -- (4.4686, 2.4969) -- (4.4691, 2.5289)
   -- (4.4434, 2.5732) -- (4.4377, 2.5815) -- (4.4246, 2.5887) -- (4.4155, 
  2.5901) -- (4.4002, 2.5905) -- (4.3624, 2.6127) -- (4.3321, 2.6435) -- 
  (4.2967, 2.6491) -- (4.2996, 2.668) -- (4.2855, 2.6826) -- (4.2761, 2.6898) --
   (4.2715, 2.6911) -- (4.265, 2.6891) -- (4.2643, 2.6895) -- (4.2592, 2.6953) 
  -- (4.2584, 2.697) -- (4.2562, 2.7084) -- (4.2544, 2.7227) -- (4.2546, 2.7273)
   -- (4.2576, 2.7364) -- (4.2638, 2.7425) -- (4.2656, 2.7479) -- (4.2662, 
  2.7513) -- (4.2569, 2.7704) -- (4.2441, 2.7861) -- (4.2338, 2.7971) -- 
  (4.2314, 2.799) -- (4.2014, 2.815) -- (4.1973, 2.8392) -- (4.201, 2.843) -- 
  (4.2097, 2.8452) -- (4.2165, 2.8504) -- (4.2169, 2.8514) -- (4.2175, 2.853) --
   (4.2065, 2.8773) -- (4.1927, 2.8905) -- (4.1963, 2.8933) -- (4.2212, 2.8942) 
  -- (4.2392, 2.8944) -- (4.2436, 2.8942) -- (4.254, 2.8823) -- (4.3073, 2.8057)
   -- (4.3095, 2.7907) -- (4.3076, 2.7834) -- (4.306, 2.7822) -- (4.3011, 
  2.7801) -- (4.3007, 2.7778) -- (4.3032, 2.7726) -- (4.3105, 2.7608) -- 
  (4.3116, 2.7596) -- (4.3134, 2.7586) -- (4.3224, 2.7567) -- (4.3295, 2.7558) 
  -- (4.3366, 2.7558) -- (4.3391, 2.7576) -- (4.3406, 2.7749) -- (4.3374, 
  2.7815) -- (4.3355, 2.7833) -- (4.3334, 2.7844) -- (4.3322, 2.7856) -- 
  (4.3308, 2.7898) -- (4.3333, 2.8151) -- (4.3352, 2.8221) -- (4.3529, 2.8568) 
  -- (4.3722, 2.8852) -- (4.44, 2.9703) -- (4.481, 2.9817) -- (4.5272, 2.9876) 
  -- (4.5374, 2.989) -- (4.5713, 3.0081) -- (4.6197, 3.0372) -- (4.6203, 3.0375)
   -- (4.6263, 3.0343) -- (4.6646, 3.0097) -- (4.6913, 2.9923) -- (4.6982, 
  2.9899) -- (4.7175, 3.0017) -- (4.7211, 3.0043) -- (4.7396, 3.0272) -- 
  (4.7416, 3.0306) -- (4.7424, 3.034) -- (4.7425, 3.0344) -- (4.7427, 3.039) -- 
  (4.7413, 3.0425) -- (4.7401, 3.0491) -- (4.7414, 3.0679) -- (4.7453, 3.0875) 
  -- (4.7457, 3.089) -- (4.7471, 3.0897) -- (4.7527, 3.0918) -- (4.8032, 3.0963)
   -- (4.8315, 3.0873) -- (4.8452, 3.0904) -- (4.8622, 3.1318) -- (4.8614, 
  3.1381) -- (4.8923, 3.1721) -- (4.9461, 3.2134) -- (4.9657, 3.2116) -- 
  (5.0013, 3.1941) -- (5.0187, 3.2165) -- (5.0271, 3.2218) -- (5.0439, 3.2599) 
  -- (5.0443, 3.2611) -- (5.0422, 3.2696) -- (5.0404, 3.2742) -- (5.1069, 3.314)
   -- (5.1158, 3.3152) -- (5.1797, 3.3237) -- (5.1811, 3.3235) -- (5.1984, 
  3.3197) -- (5.2012, 3.3187) -- (5.2063, 3.3151) -- (5.2102, 3.3139) -- 
  (5.2115, 3.3137) -- (5.2223, 3.3154) -- (5.2279, 3.3172) -- (5.2506, 3.3324) 
  -- (5.2535, 3.3397) -- (5.2541, 3.3436) -- (5.2521, 3.3727) -- (5.2497, 
  3.3769) -- (5.2667, 3.3782) -- (5.2741, 3.3789) -- (5.3513, 3.4035) -- 
  (5.3566, 3.4056) -- (5.3669, 3.418) -- (5.3847, 3.4278) -- (5.4872, 3.4798) --
   (5.5355, 3.4973) -- (5.542, 3.5203) -- (5.5407, 3.5287) -- (5.5382, 3.5325) 
  -- (5.5354, 3.5344) -- (5.532, 3.536) -- (5.5258, 3.5342) -- (5.4936, 3.5428) 
  -- (5.4895, 3.5445) -- (5.4874, 3.5462) -- (5.4541, 3.5755) -- (5.4533, 
  3.5772) -- (5.4703, 3.6307) -- (5.4729, 3.6364) -- (5.4762, 3.6406) -- 
  (5.4827, 3.6411) -- (5.484, 3.6408) -- (5.4859, 3.6393) -- (5.4948, 3.6288) --
   (5.4965, 3.6272) -- (5.4982, 3.6263) -- (5.5403, 3.6105) -- (5.5419, 3.6104) 
  -- (5.5464, 3.6109) -- (5.59, 3.6172) -- (5.5908, 3.6304) -- (5.5894, 3.6366) 
  -- (5.5942, 3.6398) -- (5.6428, 3.5987) -- (5.655, 3.587) -- (5.6744, 3.5305) 
  -- (5.6821, 3.496) -- (5.6744, 3.4737) -- (5.6753, 3.4666) -- (5.6941, 3.4591)
   -- (5.7442, 3.4401) -- (5.7637, 3.4369) -- (5.7681, 3.437) -- (5.7812, 
  3.4381) -- (5.7844, 3.4401) -- (5.7857, 3.4466) -- (5.7899, 3.4688) -- 
  (5.7928, 3.4909) -- (5.8056, 3.516) -- (5.8183, 3.532) -- (5.8323, 3.5499) -- 
  (5.8351, 3.5546) -- (5.8461, 3.5881) -- (5.8528, 3.6177) -- (5.8548, 3.6275) 
  -- (5.8546, 3.6283) -- (5.8523, 3.6305) -- (5.8586, 3.6533) -- (5.8665, 
  3.6644) -- (5.8685, 3.6795) -- (5.8865, 3.8111) -- (5.8885, 3.822) -- (5.8991,
   3.8467) -- (5.8994, 3.8733) -- (5.8934, 3.8918) -- (5.8487, 3.9876) -- 
  (5.8492, 4.0246) -- (5.849, 4.0283) -- (5.8374, 4.0479) -- (5.8334, 4.0543) --
   (5.8324, 4.0552) -- (5.8173, 4.0632) -- (5.8153, 4.0643) -- (5.7401, 4.0923) 
  -- (5.7379, 4.0925) -- (5.7292, 4.0918) -- (5.7212, 4.0924) -- (5.7133, 
  4.0947) -- (5.7035, 4.0982) -- (5.6995, 4.1031) -- (5.6878, 4.1305) -- 
  (5.6881, 4.132) -- (5.6972, 4.1499) -- (5.7059, 4.1621) -- (5.7132, 4.1868) --
   (5.7086, 4.2255) -- (5.7024, 4.2384) -- (5.6973, 4.2444) -- (5.691, 4.2496) 
  -- (5.6879, 4.2516) -- (5.6671, 4.2672) -- (5.653, 4.2829) -- (5.6484, 4.2931)
   -- (5.6223, 4.3508) -- (5.6129, 4.3535) -- (5.6114, 4.3546) -- (5.6095, 
  4.3574) -- (5.602, 4.3689) -- (5.6015, 4.3737) -- (5.6012, 4.3845) -- (5.6012,
   4.3851) -- (5.6045, 4.3888) -- (5.6201, 4.4049) -- (5.6307, 4.4138) -- 
  (5.6576, 4.4508) -- (5.6665, 4.4819) -- (5.6681, 4.5369) -- (5.6656, 4.5424) 
  -- (5.6645, 4.5431) -- (5.6775, 4.5652) -- (5.6808, 4.5719) -- (5.6813, 
  4.5731) -- (5.6852, 4.6032) -- (5.685, 4.6061) -- (5.6763, 4.6201) -- (5.6725,
   4.6239) -- (5.667, 4.6278) -- (5.6631, 4.6299) -- (5.6574, 4.6316) -- 
  (5.6508, 4.6322) -- (5.6487, 4.6335) -- (5.6441, 4.6383) -- (5.6412, 4.6424) 
  -- (5.6375, 4.65) -- (5.6348, 4.6736) -- (5.6349, 4.6768) -- (5.6361, 4.6792) 
  -- (5.6436, 4.6887) -- (5.648, 4.6972) -- (5.6485, 4.7012) -- (5.6505, 4.76) 
  -- (5.6503, 4.7621) -- (5.6492, 4.7638) -- (5.6402, 4.7765) -- (5.6382, 
  4.7791) -- (5.6356, 4.7811) -- (5.6347, 4.7815) -- (5.5914, 4.7923) -- 
  (5.5885, 4.7929) -- (5.5818, 4.7925) -- (5.578, 4.7944) -- (5.5762, 4.7961) --
   (5.5675, 4.8058) -- (5.5658, 4.8079) -- (5.5651, 4.8102) -- (5.5514, 4.8617) 
  -- (5.5482, 4.8862) -- (5.5458, 4.8913) -- (5.5362, 4.906) -- (5.537, 4.913) 
  -- (5.5404, 4.934) -- (5.5413, 4.9387) -- (5.5432, 4.9429) -- (5.5461, 4.9462)
   -- (5.5487, 4.9481) -- (5.5631, 4.9582) -- (5.57, 4.9675) -- (5.5858, 4.9951)
   -- (5.5871, 4.9976) -- (5.5884, 5.0037) -- (5.5864, 5.075) -- (5.585, 5.0777)
   -- (5.5635, 5.1052) -- (5.5596, 5.1094) -- (5.5437, 5.1202) -- (5.5408, 
  5.1218) -- (5.5363, 5.1226) -- (5.5335, 5.1236) -- (5.5256, 5.1276) -- 
  (5.5125, 5.1341) -- (5.505, 5.1379) -- (5.4882, 5.1489) -- (5.4793, 5.155) -- 
  (5.4601, 5.1733) -- (5.4584, 5.1797) -- (5.4562, 5.1834) -- (5.4555, 5.1843) 
  -- (5.4399, 5.2029) -- (5.4252, 5.22) -- (5.4112, 5.2334) -- (5.3396, 5.2799) 
  -- (5.3052, 5.2936) -- (5.2883, 5.2949) -- (5.282, 5.2977) -- (5.2805, 5.2994)
   -- (5.2772, 5.3052) -- (5.2766, 5.3069) -- (5.2763, 5.3128) -- (5.2807, 
  5.323) -- (5.2879, 5.3329) -- (5.2948, 5.3442) -- (5.2957, 5.3562) -- (5.2925,
   5.3627) -- (5.2825, 5.3987) -- (5.2799, 5.4138) -- (5.2796, 5.4153) -- 
  (5.2821, 5.4237) -- (5.2955, 5.4365) -- (5.315, 5.4497) -- (5.3224, 5.4521) --
   (5.331, 5.4567) -- (5.3464, 5.4698) -- (5.3576, 5.4805) -- (5.3875, 5.5119) 
  -- (5.3898, 5.5148) -- (5.3952, 5.5238) -- (5.4014, 5.5445) -- (5.4064, 
  5.5738) -- (5.4071, 5.5787) -- (5.4095, 5.6021) -- (5.3999, 5.6145) -- 
  (5.3987, 5.6182) -- (5.3963, 5.6279) -- (5.3962, 5.6301) -- (5.3975, 5.639) --
   (5.4006, 5.6583) -- (5.4021, 5.6596) -- (5.4069, 5.6619) -- (5.4292, 5.6956) 
  -- (5.4382, 5.7154) -- (5.4375, 5.7199) -- (5.4345, 5.7284) -- (5.4285, 
  5.7295) -- (5.421, 5.7293) -- (5.4112, 5.7595) -- (5.4132, 5.7783) -- (5.4112,
   5.7842) -- (5.3878, 5.8516) -- (5.3764, 5.8968) -- (5.3731, 5.9171) -- 
  (5.3582, 5.938) -- (5.3566, 5.9391) -- (5.349, 5.9586) -- (5.3425, 5.9766) -- 
  (5.3402, 5.9856) -- (5.3403, 5.9866) -- (5.3439, 6.0126) -- (5.3406, 6.0473) 
  -- (5.3224, 6.1058) -- (5.3126, 6.1581) -- (5.3087, 6.1686) -- (5.3011, 
  6.1753) -- (5.2976, 6.1768) -- (5.2914, 6.1782) -- (5.2875, 6.178) -- (5.2862,
   6.1767) -- (5.2825, 6.1711) -- (5.2748, 6.1624) -- (5.265, 6.1537) -- 
  (5.2631, 6.1525) -- (5.2599, 6.1523) -- (5.2291, 6.1529) -- (5.2195, 6.1544) 
  -- (5.2072, 6.1584) -- (5.145, 6.1806) -- (5.1415, 6.1821) -- (5.1271, 6.1953)
   -- (5.1083, 6.2093) -- (5.0549, 6.2422) -- (5.0531, 6.244) -- (5.0478, 
  6.2545) -- (5.047, 6.2581) -- (5.0603, 6.2776) -- (5.0769, 6.2967) -- (5.0844,
   6.3048) -- (5.0876, 6.3076) -- (5.0911, 6.3091) -- (5.0946, 6.3093) -- 
  (5.0963, 6.3088) -- (5.0972, 6.3093) -- (5.1002, 6.3125) -- (5.1037, 6.318) --
   (5.104, 6.3202) -- (5.1028, 6.3233) -- (5.099, 6.3296) -- (5.0921, 6.3392) --
   (5.0562, 6.3769) -- (5.0258, 6.4395) -- (5.0257, 6.4779) -- (5.0253, 6.4811) 
  -- (5.0234, 6.4871) -- (5.0183, 6.4952) -- (5.0126, 6.5017) -- (4.9768, 
  6.5373) -- (4.9444, 6.5275) -- (4.8634, 6.5006) -- (4.8604, 6.4995) -- 
  (4.8519, 6.4925) -- (4.852, 6.4907) -- (4.8581, 6.4763) -- (4.8624, 6.4703) --
   (4.8664, 6.4649) -- (4.8671, 6.4627) -- (4.8668, 6.4621) -- (4.8645, 6.4606) 
  -- (4.8628, 6.4601) -- (4.8491, 6.4626) -- (4.8457, 6.4641) -- (4.8424, 6.467)
   -- (4.8321, 6.4839) -- (4.8312, 6.4865) -- (4.8311, 6.495) -- (4.8251, 
  6.5079) -- (4.8237, 6.5106) -- (4.8229, 6.5112) -- (4.8116, 6.5183) -- 
  (4.8095, 6.5181) -- (4.8074, 6.516) -- (4.8026, 6.5129) -- (4.7998, 6.5132) --
   (4.7981, 6.515) -- (4.7968, 6.5212) -- (4.8123, 6.5349) -- (4.8121, 6.5358) 
  -- (4.7918, 6.5435) -- (4.7716, 6.5433) -- (4.7691, 6.5489) -- (4.7674, 
  6.5488) -- (4.7658, 6.5498) -- (4.7637, 6.5534) -- (4.7614, 6.5588) -- (4.747,
   6.589) -- (4.743, 6.5957) -- (4.7417, 6.5971) -- (4.6855, 6.6239) -- (4.6839,
   6.6239) -- (4.6776, 6.6208) -- (4.6618, 6.6119) -- (4.6387, 6.6377) -- 
  (4.6229, 6.6889) -- (4.6214, 6.6961) -- (4.6286, 6.7126) -- (4.5909, 6.7677) 
  -- (4.5705, 6.7698) -- (4.5361, 6.7568) -- (4.5338, 6.755) -- (4.5211, 6.7438)
   -- (4.5206, 6.7424) -- (4.5152, 6.7283) -- (4.5151, 6.7194) -- (4.5143, 
  6.7153) -- (4.5124, 6.7115) -- (4.5104, 6.7096) -- (4.498, 6.6985) -- (4.4945,
   6.6963) -- (4.4765, 6.693) -- (4.4336, 6.7126) -- (4.3872, 6.722) -- (4.3323,
   6.6954) -- (4.2895, 6.6624) -- (4.2603, 6.6291) -- (4.2562, 6.62) -- (4.2568,
   6.6121) -- (4.2622, 6.6016) -- (4.2639, 6.6009) -- (4.2681, 6.6009) -- 
  (4.2795, 6.5986) -- (4.2827, 6.5969) -- (4.2854, 6.5947) -- (4.2853, 6.5929) 
  -- (4.282, 6.591) -- (4.2728, 6.5894) -- (4.262, 6.59) -- (4.2523, 6.5936) -- 
  (4.2318, 6.6076) -- (4.2315, 6.6083) -- (4.2312, 6.6312) -- (4.234, 6.6484) --
   (4.2591, 6.7075) -- (4.2677, 6.7146) -- (4.3497, 6.7445) -- (4.3781, 6.7623) 
  -- (4.4079, 6.77) -- (4.4221, 6.7706) -- (4.4226, 6.7706) -- (4.425, 6.7694) 
  -- (4.4274, 6.7632) -- (4.4315, 6.7569) -- (4.4335, 6.7551) -- (4.4347, 
  6.7548) -- (4.4441, 6.7541) -- (4.4709, 6.7549) -- (4.5099, 6.7579) -- 
  (4.5271, 6.7623) -- (4.5307, 6.7643) -- (4.5542, 6.7801) -- (4.5549, 6.781) --
   (4.5546, 6.7829) -- (4.554, 6.7839) -- (4.5529, 6.7845) -- (4.5406, 6.7858) 
  -- (4.5241, 6.7855) -- (4.512, 6.784) -- (4.4724, 6.7812) -- (4.4058, 6.7795) 
  -- (4.3731, 6.7832) -- (4.3357, 6.7881) -- (4.3315, 6.7891) -- (4.3237, 
  6.7924) -- (4.3076, 6.8056) -- (4.2874, 6.7708) -- (4.2623, 6.7213) -- 
  (4.2237, 6.6486) -- (4.2183, 6.64) -- (4.2141, 6.6348) -- (4.2083, 6.6289) -- 
  (4.2034, 6.6249) -- (4.1989, 6.6216) -- (4.1884, 6.6152) -- (4.1563, 6.5951) 
  -- (4.1415, 6.5847) -- (4.1323, 6.5764) -- (4.1294, 6.5722) -- (4.1113, 
  6.5374) -- (4.0976, 6.5063) -- (4.0706, 6.5098) -- (4.0326, 6.5155) -- 
  (3.9685, 6.49) -- (3.9568, 6.4863) -- (3.9517, 6.485) -- (3.9436, 6.4845) -- 
  (3.9343, 6.4848) -- (3.9237, 6.4861) -- (3.9149, 6.4882) -- (3.9082, 6.4902) 
  -- (3.9025, 6.4928) -- (3.9009, 6.493) -- (3.8697, 6.4919) -- (3.8647, 6.4917)
   -- (3.8597, 6.4912) -- (3.8578, 6.4906) -- (3.7884, 6.4335) -- (3.7724, 
  6.4115) -- (3.7666, 6.4002) -- (3.7655, 6.3955) -- (3.7657, 6.3907) -- (3.767,
   6.3827) -- (3.7677, 6.3821) -- (3.7683, 6.3824) -- (3.7841, 6.4016) -- 
  (3.791, 6.4102) -- (3.7979, 6.4197) -- (3.8032, 6.4291) -- (3.8152, 6.4411) --
   (3.8179, 6.4436) -- (3.8201, 6.4436) -- (3.8221, 6.4414) -- (3.828, 6.4318) 
  -- (3.8285, 6.4295) -- (3.8277, 6.4188) -- (3.8025, 6.3797) -- (3.8009, 
  6.3774) -- (3.7965, 6.3734) -- (3.7935, 6.3714) -- (3.7857, 6.3688) -- 
  (3.7692, 6.3593) -- (3.7656, 6.356) -- (3.7644, 6.3543) -- (3.7561, 6.3375) --
   (3.7509, 6.2862) -- (3.7477, 6.277) -- (3.7341, 6.2536) -- (3.7269, 6.2573) 
  -- (3.7148, 6.2645) -- (3.7092, 6.2689) -- (3.7081, 6.271) -- (3.7087, 6.2792)
   -- (3.6972, 6.2845) -- (3.6756, 6.2938) -- (3.6607, 6.2904) -- (3.6591, 
  6.2879) -- (3.6549, 6.2848) -- (3.644, 6.2774) -- (3.6427, 6.2768) -- (3.6377,
   6.2763) -- (3.6326, 6.278) -- (3.6271, 6.2809) -- (3.6216, 6.2859) -- 
  (3.6185, 6.2933) -- (3.6182, 6.2983) -- (3.6202, 6.3157) -- (3.5893, 6.3504) 
  -- (3.586, 6.3538) -- (3.5835, 6.3553) -- (3.5382, 6.3532) -- (3.5322, 6.3527)
   -- (3.5122, 6.3479) -- (3.5082, 6.3469) -- (3.4989, 6.3429) -- (3.4718, 
  6.3267) -- (3.43, 6.2982) -- (3.4262, 6.2984) -- (3.4234, 6.2988) -- (3.4186, 
  6.3009) -- (3.4175, 6.3082) -- (3.4187, 6.3177) -- (3.4186, 6.3195) -- 
  (3.4168, 6.3243) -- (3.4139, 6.328) -- (3.4101, 6.3311) -- (3.4054, 6.3334) --
   (3.4025, 6.3337) -- (3.3944, 6.3317) -- (3.3923, 6.3314) -- (3.3749, 6.3314) 
  -- (3.3707, 6.3321) -- (3.3659, 6.3351) -- (3.3638, 6.3372) -- (3.3599, 
  6.3421) -- (3.3564, 6.3476) -- (3.3511, 6.3577) -- (3.3447, 6.373) -- (3.3438,
   6.3777) -- (3.3436, 6.3803) -- (3.3442, 6.3847) -- (3.3458, 6.3902) -- 
  (3.3474, 6.3934) -- (3.3498, 6.3959) -- (3.3788, 6.4268) -- (3.3831, 6.4237) 
  -- (3.386, 6.4229) -- (3.4065, 6.4192) -- (3.4111, 6.4209) -- (3.4156, 6.4238)
   -- (3.4197, 6.4271) -- (3.4437, 6.4573) -- (3.471, 6.4824) -- (3.5006, 
  6.5032) -- (3.5066, 6.5074) -- (3.5305, 6.5269) -- (3.5313, 6.5276) -- 
  (3.5315, 6.5286) -- (3.5247, 6.5805) -- (3.5191, 6.6421) -- (3.5178, 6.6633) 
  -- (3.5184, 6.6663) -- (3.5226, 6.6736) -- (3.5249, 6.7064) -- (3.4501, 6.703)
   -- (3.4392, 6.7002) -- (3.4161, 6.6824) -- (3.4, 6.6682) -- (3.3895, 6.6557) 
  -- (3.3828, 6.6489) -- (3.3658, 6.6351) -- (3.3609, 6.6322) -- (3.3541, 
  6.6294) -- (3.3491, 6.6278) -- (3.3455, 6.6273) -- (3.321, 6.6262) -- (3.3153,
   6.6261) -- (3.3101, 6.6275) -- (3.3079, 6.6288) -- (3.304, 6.632) -- (3.2853,
   6.6522) -- (3.2842, 6.6548) -- (3.2848, 6.6553) -- (3.2827, 6.66) -- (3.2783,
   6.6667) -- (3.2749, 6.6697) -- (3.2613, 6.6785) -- (3.2546, 6.6822) -- 
  (3.1406, 6.7392) -- (3.125, 6.7436) -- (3.1141, 6.7466) -- (3.1083, 6.7476) --
   (3.1032, 6.7474) -- (3.0938, 6.7455) -- (3.0519, 6.7249) -- (3.0477, 6.7202) 
  -- (3.0265, 6.6738) -- (3.0234, 6.6625) -- (3.011, 6.6788) -- (3.0122, 6.6844)
   -- (3.0173, 6.6948) -- (3.0287, 6.7044) -- (3.0218, 6.739) -- (3.0202, 
  6.7445) -- (3.0298, 6.7589) -- (3.034, 6.7662) -- (3.021, 6.7791) -- (3.0148, 
  6.784) -- (3.0012, 6.793) -- (2.9969, 6.7937) -- (2.9921, 6.7936) -- (2.9888, 
  6.7931) -- (2.9537, 6.7862) -- (2.9069, 6.775) -- (2.8999, 6.7727) -- (2.8999,
   6.772) -- (2.8825, 6.7644) -- (2.8797, 6.7627) -- (2.8735, 6.7613) -- 
  (2.8551, 6.7585) -- (2.8526, 6.7592) -- (2.8482, 6.7619) -- (2.8448, 6.7656) 
  -- (2.8423, 6.7699) -- (2.8358, 6.7828) -- (2.8369, 6.784) -- (2.8393, 6.7854)
   -- (2.8423, 6.786) -- (2.848, 6.7859) -- (2.8496, 6.7836) -- (2.8519, 6.7821)
   -- (2.8556, 6.783) -- (2.8717, 6.7909) -- (2.8954, 6.803) -- (2.9101, 6.8116)
   -- (2.9196, 6.8196) -- (2.9254, 6.8263) -- (2.9272, 6.8287) -- (2.9425, 
  6.8551) -- (2.9429, 6.8592) -- (2.9431, 6.9085) -- (2.9447, 6.9364) -- 
  (2.9453, 6.9562) -- (2.9143, 7.0534) -- (2.9044, 7.0714) -- (2.8885, 7.0822) 
  -- (2.8789, 7.0882) -- (2.8779, 7.0874) -- (2.8763, 7.0846) -- (2.8746, 7.08) 
  -- (2.8754, 7.0781) -- (2.8606, 7.0462) -- (2.8597, 7.0451) -- (2.8588, 
  7.0451) -- (2.8541, 7.0456) -- (2.8458, 7.048) -- (2.8366, 7.0529) -- (2.7824,
   7.1142) -- (2.752, 7.1266) -- (2.7128, 7.1427) -- (2.7222, 7.1601) -- 
  (2.7259, 7.1661) -- (2.7062, 7.1732) -- (2.7018, 7.1723) -- (2.6547, 7.1303) 
  -- (2.6462, 7.1263) -- (2.634, 7.1229) -- (2.6215, 7.1179) -- (2.6114, 7.1222)
   -- (2.5875, 7.1176) -- (2.5868, 7.1118) -- (2.5852, 7.1101) -- (2.5718, 
  7.0983) -- (2.5625, 7.0974) -- (2.5193, 7.1016) -- (2.5131, 7.1179) -- 
  (2.5118, 7.1219) -- (2.5115, 7.1258) -- (2.5134, 7.1294) -- (2.5168, 7.1339) 
  -- (2.5143, 7.1397) -- (2.4952, 7.1485) -- (2.4681, 7.1597) -- (2.4615, 7.162)
   -- (2.4578, 7.1627) -- (2.4453, 7.1628) -- (2.4333, 7.1595) -- (2.4182, 
  7.1606) -- (2.4101, 7.1615) -- (2.4037, 7.1652) -- (2.3997, 7.1676) -- (2.4, 
  7.1695) -- (2.3919, 7.1754) -- (2.3635, 7.1885) -- (2.3612, 7.1895) -- (2.356,
   7.1911) -- (2.3311, 7.1932) -- (2.2938, 7.1957) -- (2.2889, 7.196) -- 
  (2.2819, 7.1952) -- (2.2754, 7.1941) -- (2.2256, 7.1915) -- (2.2146, 7.1991) 
  -- (2.2108, 7.2008) -- (2.2045, 7.2016) -- (2.1877, 7.2019) -- (2.1741, 
  7.1711) -- (2.1705, 7.1379) -- (2.1708, 7.1361) -- (2.1902, 7.1145) -- 
  (2.2109, 7.09) -- (2.2119, 7.0886) -- (2.2199, 7.0719) -- (2.227, 7.0372) -- 
  (2.2473, 6.9846) -- (2.2504, 6.9807) -- (2.2612, 6.97) -- (2.2779, 6.9621) -- 
  (2.2887, 6.9556) -- (2.316, 6.9169) -- (2.3748, 6.8333) -- (2.3864, 6.8075) --
   (2.3883, 6.8003) -- (2.3885, 6.7971) -- (2.387, 6.787) -- (2.381, 6.7772) -- 
  (2.3798, 6.7759) -- (2.3296, 6.7335) -- (2.3196, 6.7283) -- (2.3081, 6.7231) 
  -- (2.298, 6.7203) -- (2.2854, 6.7263) -- (2.2735, 6.7276) -- (2.2127, 6.717) 
  -- (2.2077, 6.6949) -- (2.2083, 6.6899) -- (2.2085, 6.6882) -- (2.2085, 
  6.6799) -- (2.2078, 6.678) -- (2.2066, 6.677) -- (2.1827, 6.6625) -- (2.1811, 
  6.6619) -- (2.176, 6.6614) -- (2.1647, 6.6694) -- (2.1641, 6.6778) -- (2.1557,
   6.6734) -- (2.1548, 6.6723) -- (2.1529, 6.6689) -- (2.1498, 6.6608) -- 
  (2.1477, 6.6534) -- (2.1454, 6.6402) -- (2.1449, 6.6335) -- (2.1459, 6.6286) 
  -- (2.1509, 6.6191) -- (2.156, 6.6156) -- (2.1967, 6.5942) -- (2.2004, 6.5937)
   -- (2.2037, 6.5941) -- (2.2051, 6.5947) -- (2.2062, 6.5958) -- (2.2067, 
  6.5969) -- (2.2064, 6.5971) -- (2.2056, 6.5965) -- (2.2043, 6.5966) -- 
  (2.1998, 6.5977) -- (2.1993, 6.5983) -- (2.2051, 6.6051) -- (2.2144, 6.6142) 
  -- (2.2367, 6.6159) -- (2.2437, 6.6164) -- (2.2785, 6.6163) -- (2.2813, 
  6.6155) -- (2.2829, 6.6143) -- (2.2862, 6.61) -- (2.2888, 6.6052) -- (2.2899, 
  6.6013) -- (2.2905, 6.596) -- (2.2905, 6.5933) -- (2.2888, 6.5919) -- (2.291, 
  6.5914) -- (2.291, 6.5896) -- (2.2905, 6.583) -- (2.2877, 6.551) -- (2.2868, 
  6.5471) -- (2.2809, 6.5346) -- (2.2711, 6.5141) -- (2.2689, 6.5085) -- 
  (2.2687, 6.507) -- (2.2697, 6.5026) -- (2.2791, 6.477) -- (2.284, 6.4677) -- 
  (2.2853, 6.466) -- (2.2952, 6.4597) -- (2.3, 6.4567) -- (2.3115, 6.4642) -- 
  (2.3151, 6.4665) -- (2.3239, 6.4683) -- (2.3262, 6.4678) -- (2.3325, 6.4659) 
  -- (2.3347, 6.4631) -- (2.3643, 6.402) -- (2.3653, 6.3988) -- (2.3656, 6.3959)
   -- (2.3655, 6.3936) -- (2.3639, 6.3875) -- (2.3626, 6.3823) -- (2.3605, 
  6.3788) -- (2.3582, 6.3762) -- (2.3559, 6.3744) -- (2.3513, 6.3725) -- 
  (2.3435, 6.3704) -- (2.3375, 6.3713) -- (2.3335, 6.3758) -- (2.3292, 6.3788) 
  -- (2.3063, 6.3842) -- (2.2975, 6.3824) -- (2.2867, 6.3764) -- (2.2729, 
  6.3649) -- (2.2733, 6.363) -- (2.2762, 6.3625) -- (2.2796, 6.363) -- (2.2987, 
  6.3326) -- (2.311, 6.3032) -- (2.314, 6.2932) -- (2.3199, 6.2775) -- (2.3285, 
  6.2653) -- (2.3459, 6.2511) -- (2.3555, 6.2432) -- (2.3589, 6.2419) -- 
  (2.3631, 6.2418) -- (2.366, 6.2428) -- (2.3796, 6.2496) -- (2.3811, 6.25) -- 
  (2.3836, 6.2261) -- (2.3759, 6.2226) -- (2.3684, 6.2192) -- (2.3722, 6.2079) 
  -- (2.3792, 6.1899) -- (2.3837, 6.1841) -- (2.364, 6.1888) -- (2.3591, 6.1891)
   -- (2.34, 6.185) -- (2.3374, 6.1841) -- (2.3373, 6.1827) -- (2.3367, 6.182) 
  -- (2.3341, 6.1811) -- (2.3191, 6.1781) -- (2.2726, 6.1818) -- (2.2583, 
  6.1837) -- (2.2515, 6.1855) -- (2.2456, 6.1883) -- (2.2367, 6.1945) -- 
  (2.2109, 6.2135) -- (2.1974, 6.2297) -- (2.1961, 6.2326) -- (2.1955, 6.2354) 
  -- (2.1958, 6.2403) -- (2.1857, 6.2397) -- (2.1739, 6.2367) -- (2.1573, 
  6.2318) -- (2.1496, 6.2245) -- (2.1295, 6.1983) -- (2.12, 6.1742) -- (2.1152, 
  6.1615) -- (2.0806, 6.0574) -- (2.0801, 6.0514) -- (2.0801, 6.0397) -- 
  (2.0802, 6.0212) -- (2.091, 5.9935) -- (2.0994, 5.9735) -- (2.1082, 5.9737) --
   (2.1577, 5.9712) -- (2.1731, 5.9688) -- (2.1737, 5.9679) -- (2.1639, 5.9223) 
  -- (2.1721, 5.8867) -- (2.1712, 5.882) -- (2.1477, 5.8571) -- (2.1461, 5.857) 
  -- (2.1294, 5.8589) -- (2.1264, 5.8607) -- (2.1249, 5.8628) -- (2.1282, 
  5.8677) -- (2.1288, 5.8695) -- (2.1288, 5.8712) -- (2.1163, 5.8878) -- 
  (2.1132, 5.8922) -- (2.1164, 5.8946) -- (2.1181, 5.8961) -- (2.1183, 5.8968) 
  -- (2.1168, 5.9056) -- (2.1159, 5.908) -- (2.1147, 5.9093) -- (2.0971, 5.9202)
   -- (2.0902, 5.9213) -- (2.0772, 5.9236) -- (2.073, 5.9234) -- (2.0694, 5.923)
   -- (2.0674, 5.9234) -- (2.0332, 5.9388) -- (2.023, 5.9444) -- (2.0146, 
  5.9569) -- (2.0078, 5.9697) -- (2.0038, 5.9754) -- (1.9909, 5.9839) -- 
  (1.9799, 5.9849) -- (1.9679, 5.9805) -- (1.9593, 5.9737) -- (1.9507, 5.9652) 
  -- (1.9477, 5.9609) -- (1.9439, 5.9527) -- (1.9432, 5.9499) -- (1.9409, 
  5.9347) -- (1.9364, 5.8965) -- (1.956, 5.8928) -- (1.961, 5.8943) -- (1.9754, 
  5.8999) -- (1.9795, 5.8976) -- (1.9808, 5.896) -- (1.9831, 5.8914) -- (1.9837,
   5.8894) -- (1.9844, 5.8751) -- (1.9823, 5.841) -- (1.9811, 5.8337) -- 
  (1.9781, 5.8267) -- (1.9714, 5.8108) -- (1.9671, 5.8032) -- (1.9618, 5.7958) 
  -- (1.9556, 5.7888) -- (1.954, 5.787) -- (1.9493, 5.7835) -- (1.9437, 5.7806) 
  -- (1.9403, 5.7802) -- (1.9331, 5.7811) -- (1.9232, 5.7834) -- (1.9172, 5.786)
   -- (1.9159, 5.7873) -- (1.914, 5.7912) -- (1.9142, 5.7934) -- (1.9137, 
  5.7944) -- (1.9078, 5.8037) -- (1.8967, 5.8198) -- (1.8876, 5.8315) -- 
  (1.8812, 5.8334) -- (1.873, 5.8331) -- (1.8712, 5.8327) -- (1.8636, 5.8276) --
   (1.86, 5.8262) -- (1.8527, 5.8341) -- (1.8471, 5.8416) -- (1.8442, 5.8478) --
   (1.842, 5.8539) -- (1.8408, 5.8588) -- (1.8399, 5.8673) -- (1.8392, 5.8862) 
  -- (1.8483, 5.8791) -- (1.8567, 5.8789) -- (1.8604, 5.8798) -- (1.8801, 5.887)
   -- (1.8883, 5.8901) -- (1.8966, 5.8944) -- (1.8992, 5.8997) -- (1.9016, 
  5.9079) -- (1.9026, 5.9111) -- (1.9031, 5.9142) -- (1.9031, 5.9151) -- 
  (1.8723, 5.9874) -- (1.867, 5.9976) -- (1.8622, 6.0048) -- (1.8596, 6.008) -- 
  (1.8332, 6.0117) -- (1.8283, 6.0244) -- (1.8263, 6.0323) -- (1.8231, 6.0532) 
  -- (1.824, 6.0599) -- (1.8281, 6.0657) -- (1.8289, 6.0678) -- (1.8293, 6.0722)
   -- (1.8271, 6.0749) -- (1.8181, 6.0795) -- (1.7919, 6.0836) -- (1.7627, 
  6.0868) -- (1.7059, 6.0818) -- (1.677, 6.0731) -- (1.6657, 6.072) -- (1.6552, 
  6.0723) -- (1.65, 6.0736) -- (1.6499, 6.0751) -- (1.6458, 6.0755) -- (1.6403, 
  6.0747) -- (1.6367, 6.074) -- (1.598, 6.0631) -- (1.5618, 6.0513) -- (1.5457, 
  6.0483) -- (1.5415, 6.0484) -- (1.5349, 6.0517) -- (1.5302, 6.0549) -- 
  (1.5289, 6.0567) -- (1.5234, 6.0599) -- (1.5201, 6.061) -- (1.4517, 6.0667) --
   (1.4365, 6.0653) -- (1.4252, 6.064) -- (1.4128, 6.0611) -- (1.387, 6.052) -- 
  (1.3769, 6.045) -- (1.3519, 6.0268) -- (1.3408, 6.0158) -- (1.2987, 5.9721) --
   (1.2961, 5.9639) -- (1.3008, 5.9579) -- (1.3159, 5.9478) -- (1.3221, 5.9441) 
  -- (1.3313, 5.9277) -- (1.3215, 5.9173) -- (1.3039, 5.8996) -- (1.295, 5.9123)
   -- (1.2642, 5.9263) -- (1.2498, 5.8093) -- (1.2486, 5.7854) -- (1.2396, 
  5.7699) -- (1.2388, 5.7673) -- (1.2405, 5.7626) -- (1.2473, 5.75) -- (1.2499, 
  5.7467) -- (1.2535, 5.7445) -- (1.2721, 5.7419) -- (1.3362, 5.7398) -- 
  (1.3647, 5.7373) -- (1.3726, 5.736) -- (1.3839, 5.7334) -- (1.4047, 5.7275) --
   (1.408, 5.7262) -- (1.3816, 5.7262) -- (1.3767, 5.7245) -- (1.3758, 5.7236) 
  -- (1.3751, 5.7213) -- (1.3689, 5.7019) -- (1.3586, 5.6584) -- (1.3551, 
  5.6082) -- (1.347, 5.5893) -- (1.3397, 5.5763) -- (1.3318, 5.5546) -- (1.3322,
   5.5468) -- (1.3335, 5.5392) -- (1.3441, 5.5306) -- (1.3412, 5.4989) -- 
  (1.346, 5.4353) -- (1.344, 5.4248) -- (1.3388, 5.4085) -- (1.3262, 5.3699) -- 
  (1.2884, 5.3126) -- (1.2691, 5.2864) -- (1.259, 5.2496) -- (1.254, 5.1967) -- 
  (1.2503, 5.159) -- (1.2427, 5.093) -- (1.2372, 5.0854) -- (1.235, 5.0832) -- 
  (1.2343, 5.083) -- (1.2183, 5.0872) -- (1.2149, 5.0892) -- (1.2147, 5.0898) --
   (1.213, 5.092) -- (1.2014, 5.096) -- (1.1528, 5.1041) -- (1.0929, 5.1077) -- 
  (1.0889, 5.1078) -- (1.0742, 5.1054) -- (1.0645, 5.1023) -- (1.0451, 5.0864) 
  -- (1.0491, 5.0495) -- (1.0453, 5.0119) -- (1.0381, 4.9855) -- (1.0359, 
  4.9743) -- (1.0326, 4.9564) -- (1.0322, 4.9536) -- (1.063, 4.9312) -- (1.0753,
   4.9266) -- (1.1704, 4.8997) -- (1.1733, 4.9008) -- (1.1826, 4.907) -- 
  (1.1873, 4.9171) -- (1.1923, 4.9267) -- (1.1981, 4.9303) -- (1.2004, 4.9283) 
  -- (1.2012, 4.9269) -- (1.2385, 4.8551) -- (1.2431, 4.838) -- (1.2421, 4.8184)
   -- (1.2324, 4.8053) -- (1.2263, 4.7918) -- (1.2137, 4.7631) -- (1.2132, 
  4.7587) -- (1.2145, 4.7509) -- (1.2159, 4.7438) -- (1.2214, 4.7288) -- 
  (1.2342, 4.7143) -- (1.2109, 4.7008) -- (1.1405, 4.6568) -- (1.1236, 4.6387) 
  -- (1.1222, 4.6312) -- (1.1184, 4.6163) -- (1.1148, 4.6126) -- (1.1077, 
  4.6058) -- (1.1026, 4.6051) -- (1.0503, 4.5884) -- (1.0362, 4.5659) -- 
  (1.0227, 4.5338) -- (1.0446, 4.522) -- (1.0859, 4.4869) -- (1.0877, 4.4799) --
   (1.0886, 4.4674) -- (1.0884, 4.4653) -- (1.086, 4.4596) -- (1.0616, 4.4282) 
  -- (1.05, 4.4161) -- (1.0387, 4.4082) -- (1.0275, 4.4008) -- (1.0212, 4.3985) 
  -- (1.0209, 4.3987) -- (1.0078, 4.4119) -- (1.0049, 4.4166) -- (1.0003, 
  4.4194) -- (0.9946, 4.418) -- (0.9418, 4.4) -- (0.9152, 4.3896) -- (0.9016, 
  4.3825) -- (0.8932, 4.378) -- (0.8487, 4.3785) -- (0.8293, 4.3868) -- (0.8278,
   4.387) -- (0.8252, 4.3785) -- (0.8256, 4.3756) -- (0.8354, 4.3433) -- 
  (0.8314, 4.3438) -- (0.8198, 4.3505) -- (0.8019, 4.3661) -- (0.7746, 4.3837) 
  -- (0.7684, 4.3877) -- (0.762, 4.3908) -- (0.7597, 4.3913) -- (0.7144, 4.3932)
   -- (0.6965, 4.3815) -- (0.6517, 4.3755) -- (0.6316, 4.3764) -- (0.5912, 
  4.3583) -- (0.5772, 4.3354) -- (0.5908, 4.271) -- (0.608, 4.2137) -- (0.6085, 
  4.2127) -- (0.6118, 4.2086) -- (0.6411, 4.195) -- (0.6426, 4.1537) -- (0.6404,
   4.1439) -- (0.6703, 4.1095) -- (0.6766, 4.1049) -- (0.7003, 4.066) -- 
  (0.7064, 4.0534) -- (0.7108, 4.017) -- (0.7107, 4.011) -- (0.7077, 3.9903) -- 
  (0.703, 3.9758) -- (0.6972, 3.9466) -- (0.7063, 3.9138) -- (0.6848, 3.8905) --
   (0.6186, 3.8071) -- (0.6077, 3.7755) -- (0.6085, 3.746) -- (0.6135, 3.7362) 
  -- (0.6179, 3.7349) -- (0.6197, 3.7347) -- (0.6238, 3.7348) -- (0.6268, 
  3.7359) -- (0.6302, 3.7374) -- (0.632, 3.7392) -- (0.6331, 3.7409) -- (0.6376,
   3.744) -- (0.6532, 3.7509) -- (0.6631, 3.7548) -- (0.6718, 3.747) -- (0.6681,
   3.7454) -- (0.6463, 3.7362) -- (0.6372, 3.7099) -- (0.565, 3.6614) -- 
  (0.5077, 3.6429) -- (0.4799, 3.6306) -- (0.4793, 3.6299) -- (0.4795, 3.6251) 
  -- (0.4851, 3.5975) -- (0.498, 3.5689) -- (0.5062, 3.5594) -- (0.5295, 3.565) 
  -- (0.5656, 3.5621) -- (0.5704, 3.5613) -- (0.5706, 3.5604) -- (0.568, 3.5524)
   -- (0.5626, 3.5411) -- (0.5621, 3.5282) -- (0.5632, 3.5156) -- (0.5853, 
  3.5076) -- (0.5977, 3.5025) -- (0.6048, 3.4994) -- (0.6043, 3.496) -- (0.6032,
   3.4935) -- (0.5998, 3.4903) -- (0.599, 3.4884) -- (0.595, 3.4748) -- (0.5956,
   3.4702) -- (0.598, 3.4654) -- (0.6008, 3.4621) -- (0.6016, 3.4553) -- 
  (0.5919, 3.4316) -- (0.5824, 3.4364) -- (0.5817, 3.4389) -- (0.578, 3.4393) --
   (0.5685, 3.4365) -- (0.5593, 3.4329) -- (0.5572, 3.426) -- (0.5424, 3.3785) 
  -- (0.5669, 3.3375) -- (0.6078, 3.3147) -- (0.6095, 3.3136) -- (0.6125, 
  3.3108) -- (0.6248, 3.2795) -- (0.6259, 3.2753) -- (0.6282, 3.2705) -- 
  (0.6373, 3.2556) -- (0.6416, 3.2495) -- (0.6488, 3.2427) -- (0.6386, 3.2387) 
  -- (0.6371, 3.2382) -- (0.6453, 3.2277) -- (0.677, 3.2193) -- (0.6696, 3.1862)
   -- (0.6394, 3.1577) -- (0.6387, 3.1554) -- (0.6392, 3.1415) -- (0.6437, 
  3.127) -- (0.6466, 3.122) -- (0.6647, 3.0966) -- (0.6656, 3.0962) -- (0.6665, 
  3.0964) -- (0.6753, 3.1002) -- (0.68, 3.1034) -- (0.6812, 3.1038) -- (0.6922, 
  3.104) -- (0.7156, 3.1) -- (0.7179, 3.0979) -- (0.7197, 3.0954) -- (0.7299, 
  3.0888) -- (0.7344, 3.082) -- (0.75, 3.0555) -- (0.7517, 3.0513) -- (0.7525, 
  3.0411) -- (0.7517, 3.0346) -- (0.7468, 3.0116) -- (0.7437, 3.0044) -- 
  (0.7384, 2.995) -- (0.7394, 2.9765) -- (0.7618, 2.9502) -- (0.7654, 2.9513) --
   (0.7697, 2.9434) -- (0.7709, 2.939) -- (0.7774, 2.9291) -- (0.7642, 2.9301) 
  -- (0.7382, 2.9234) -- (0.7181, 2.9337) -- (0.7093, 2.9314) -- (0.7053, 2.93) 
  -- (0.6944, 2.9068) -- (0.6911, 2.8985) -- (0.6227, 2.8545) -- (0.6213, 
  2.8524) -- (0.6212, 2.8516) -- (0.6195, 2.8393) -- (0.6224, 2.8005) -- 
  (0.5998, 2.7816) -- (0.5814, 2.7158) -- (0.5785, 2.6883) -- (0.5924, 2.6325) 
  -- (0.6032, 2.6058) -- (0.6063, 2.5987) -- (0.6103, 2.5899) -- (0.641, 2.5388)
   -- (0.6573, 2.5198) -- (0.7087, 2.4884) -- (0.7372, 2.4601) -- (0.7393, 
  2.4581) -- (0.7559, 2.4485) -- (0.7571, 2.4482) -- (0.7706, 2.4484) -- 
  (0.7841, 2.457) -- (0.8022, 2.445) -- (0.8075, 2.4063) -- (0.8077, 2.4004) -- 
  (0.794, 2.3533) -- (0.7406, 2.267) -- (0.7295, 2.2552) -- (0.7272, 2.2529) -- 
  (0.7247, 2.2517) -- (0.7201, 2.2515) -- (0.718, 2.2505) -- (0.7141, 2.2463) --
   (0.7017, 2.2284) -- (0.701, 2.2268) -- (0.701, 2.2258) -- (0.7138, 2.2058) --
   (0.7182, 2.2032)(1.7387, 6.1396) -- (1.7292, 6.1512) -- (1.7283, 6.1522) -- 
  (1.7283, 6.1539) -- (1.7322, 6.1574) -- (1.7357, 6.1588) -- (1.748, 6.1593) --
   (1.7548, 6.1591) -- (1.7611, 6.1582) -- (1.7772, 6.1546) -- (1.7829, 6.1529) 
  -- (1.785, 6.1519) -- (1.7962, 6.144) -- (1.7967, 6.1396) -- (1.7952, 6.1393) 
  -- (1.7803, 6.1448) -- (1.745, 6.1538) -- (1.7436, 6.1537) -- (1.7428, 6.153) 
  -- (1.7404, 6.149) -- cycle(1.8238, 6.1106) -- (1.8228, 6.1107) -- (1.8198, 
  6.112) -- (1.8175, 6.1135) -- (1.8155, 6.1158) -- (1.8147, 6.1179) -- (1.8107,
   6.1333) -- (1.8108, 6.1368) -- (1.8175, 6.1448) -- (1.8208, 6.1407) -- 
  (1.8245, 6.1213) -- (1.8243, 6.1115) -- cycle(1.6344, 6.1227) -- (1.6296, 
  6.1236) -- (1.628, 6.1245) -- (1.626, 6.1282) -- (1.6261, 6.1318) -- (1.6287, 
  6.1369) -- (1.6342, 6.1439) -- (1.6368, 6.1464) -- (1.6393, 6.148) -- (1.6448,
   6.1497) -- (1.6545, 6.1501) -- (1.6683, 6.1497) -- (1.6901, 6.1478) -- 
  (1.6941, 6.1472) -- (1.7065, 6.1438) -- (1.7101, 6.14) -- (1.7106, 6.1387) -- 
  (1.7101, 6.1383) -- (1.6764, 6.132) -- (1.6425, 6.1341) -- cycle(1.5181, 
  6.0967) -- (1.5161, 6.0971) -- (1.5128, 6.0998) -- (1.5114, 6.108) -- (1.5116,
   6.1147) -- (1.5123, 6.1173) -- (1.5189, 6.1263) -- (1.5207, 6.1279) -- 
  (1.5264, 6.1308) -- (1.5304, 6.1315) -- (1.5615, 6.1317) -- (1.5861, 6.1304) 
  -- (1.595, 6.1293) -- (1.5973, 6.1287) -- (1.6009, 6.1273) -- (1.6042, 6.1254)
   -- (1.6054, 6.1236) -- (1.6057, 6.1192) -- (1.6055, 6.1182) -- (1.6051, 
  6.1181) -- (1.5978, 6.1176) -- (1.5977, 6.1181) -- (1.5989, 6.1193) -- 
  (1.5988, 6.1198) -- (1.5982, 6.12) -- (1.5787, 6.1212) -- (1.5766, 6.1212) -- 
  (1.562, 6.12) -- (1.5407, 6.1179) -- (1.5381, 6.117) -- cycle(1.4969, 6.0955) 
  -- (1.4842, 6.096) -- (1.4694, 6.0978) -- (1.4567, 6.1007) -- (1.4537, 6.1051)
   -- (1.4538, 6.1061) -- (1.4546, 6.107) -- (1.4681, 6.1125) -- (1.4704, 6.113)
   -- (1.4734, 6.113) -- (1.4925, 6.1088) -- (1.4938, 6.1081) -- (1.4967, 
  6.0979) -- (1.4971, 6.0964) -- cycle(1.3373, 6.0808) -- (1.3314, 6.0857) -- 
  (1.3289, 6.0892) -- (1.3285, 6.091) -- (1.3296, 6.0931) -- (1.3324, 6.0951) --
   (1.3359, 6.0969) -- (1.3488, 6.1027) -- (1.3573, 6.1049) -- (1.3624, 6.1056) 
  -- (1.4279, 6.1053) -- (1.4377, 6.105) -- (1.4426, 6.1037) -- (1.4449, 6.1026)
   -- (1.4465, 6.0979) -- (1.4461, 6.0966) -- (1.438, 6.0937) -- (1.4325, 
  6.0919) -- (1.3622, 6.0832) -- (1.344, 6.0817) -- cycle(1.1948, 6.0562) -- 
  (1.1895, 6.0566) -- (1.1884, 6.0567) -- (1.1853, 6.0577) -- (1.1807, 6.0607) 
  -- (1.1801, 6.0623) -- (1.1807, 6.0635) -- (1.1838, 6.0644) -- (1.2031, 
  6.0663) -- (1.2666, 6.0726) -- (1.2953, 6.0726) -- (1.3033, 6.0708) -- 
  (1.3044, 6.0694) -- (1.3049, 6.068) -- (1.3048, 6.0657) -- (1.3033, 6.0636) --
   (1.3027, 6.065) -- (1.302, 6.0654) -- (1.2882, 6.0663) -- (1.2825, 6.0665) --
   (1.2757, 6.0663) -- (1.2432, 6.0651) -- (1.2004, 6.0585) -- (1.1952, 6.057) 
  -- cycle(1.106, 5.9595) -- (1.1012, 5.9694) -- (1.0984, 5.9712) -- (1.0936, 
  5.9709) -- (1.0896, 5.9695) -- (1.089, 5.9682) -- (1.089, 5.9655) -- (1.09, 
  5.9627) -- (1.0912, 5.9618) -- (1.0911, 5.9614) -- (1.09, 5.9611) -- (1.084, 
  5.9621) -- (1.0602, 5.9811) -- (1.0561, 5.9873) -- (1.0546, 5.9951) -- 
  (1.0553, 5.9989) -- (1.0568, 6.0012) -- (1.0577, 6.0021) -- (1.0603, 6.0031) 
  -- (1.1083, 6.0176) -- (1.1108, 6.0177) -- (1.1159, 6.017) -- (1.122, 6.0144) 
  -- (1.1381, 6.0023) -- (1.1429, 5.9973) -- (1.1434, 5.9958) -- (1.095, 5.9822)
   -- (1.1065, 5.9718) -- (1.1127, 5.9643) -- (1.1122, 5.9635) -- (1.1071, 
  5.9596) -- cycle(2.0205, 7.2497) -- (2.0102, 7.2234) -- (2.0019, 7.1997) -- 
  (1.9992, 7.1867) -- (1.9969, 7.1681) -- (1.9949, 7.1462) -- (1.9891, 7.0623) 
  -- (1.989, 7.0558) -- (1.9941, 7.0441) -- (1.9969, 7.0404) -- (1.9981, 7.0407)
   -- (1.9991, 7.0424) -- (1.9993, 7.0462) -- (2.0015, 7.1178) -- (2.0003, 
  7.1199) -- (1.999, 7.1242) -- (1.999, 7.1258) -- (2.0005, 7.1395) -- (2.0045, 
  7.1522) -- (2.0068, 7.1553) -- (2.0263, 7.1679) -- (2.0318, 7.1631) -- 
  (2.0329, 7.1617) -- (2.0461, 7.1542) -- (2.059, 7.1449) -- (2.0632, 7.1434) --
   (2.0652, 7.1434) -- (2.0904, 7.1562) -- (2.1064, 7.1696) -- (2.1056, 7.1703) 
  -- (2.1024, 7.1711) -- (2.0976, 7.1716) -- (2.0778, 7.1708) -- (2.0729, 
  7.1716) -- (2.0704, 7.1724) -- (2.0535, 7.1875) -- (2.0531, 7.1879) -- 
  (2.0381, 7.2034) -- (2.0343, 7.2579) -- (2.0447, 7.2787) -- (2.0494, 7.2846) 
  -- (2.0582, 7.2916) -- (2.059, 7.2921) -- (2.0701, 7.2947) -- (2.0818, 7.302) 
  -- (2.0962, 7.3301) -- (2.096, 7.3309) -- (2.095, 7.3318) -- (2.0755, 7.3398) 
  -- (2.0702, 7.3414) -- (2.0671, 7.3416) -- (2.0655, 7.341) -- (2.0608, 7.338) 
  -- (2.0595, 7.3367) -- (2.0556, 7.3304) -- (2.0382, 7.2893) -- (2.0348, 
  7.2807) -- cycle(2.0978, 7.053) -- (2.0931, 7.0531) -- (2.0753, 7.0497) -- 
  (2.0701, 7.0477) -- (2.0663, 7.0447) -- (2.0613, 7.0396) -- (2.0576, 7.033) --
   (2.0527, 7.0213) -- (2.052, 7.0138) -- (2.0528, 7.0099) -- (2.0544, 7.0071) 
  -- (2.0562, 7.0055) -- (2.0929, 6.9875) -- (2.1005, 6.9854) -- (2.1094, 
  6.9848) -- (2.1394, 6.9836) -- (2.1432, 6.9838) -- (2.1452, 6.9847) -- 
  (2.1622, 7.0202) -- (2.1567, 7.0421) -- (2.1553, 7.0446) -- (2.1469, 7.0504) 
  -- (2.1368, 7.054) -- (2.1326, 7.055) -- (2.1299, 7.0557) -- (2.1234, 7.0562) 
  -- (2.1158, 7.0534) -- (2.1129, 7.0523) -- cycle(1.9966, 6.9698) -- (1.9967, 
  6.9677) -- (1.9994, 6.9607) -- (2.0107, 6.9426) -- (2.0166, 6.9335) -- 
  (2.0233, 6.9231) -- (2.0249, 6.9224) -- (2.0281, 6.9219) -- (2.0385, 6.922) --
   (2.0392, 6.9222) -- (2.0501, 6.9275) -- (2.0511, 6.9286) -- (2.0539, 6.9355) 
  -- (2.0535, 6.9365) -- (2.0527, 6.9369) -- (2.0418, 6.9424) -- (2.0323, 
  6.9571) -- (2.0207, 6.9894) -- (2.0194, 6.9962) -- (2.021, 7.0008) -- (2.0235,
   7.0054) -- (2.0329, 7.0147) -- (2.0321, 7.0156) -- (2.0306, 7.0161) -- 
  (2.0269, 7.0145) -- (2.0044, 6.994) -- (2.0015, 6.9902) -- (1.998, 6.9817) -- 
  (1.9977, 6.974) -- cycle(2.1548, 6.9366) -- (2.1883, 6.9449) -- (2.1927, 
  6.9475) -- (2.1958, 6.9516) -- (2.1975, 6.9572) -- (2.1992, 6.9665) -- 
  (2.1977, 6.9672) -- (2.14, 6.9447) -- (2.1249, 6.9384) -- (2.1235, 6.9347) -- 
  (2.124, 6.9333) -- (2.1337, 6.9255) -- cycle(2.1156, 6.8894) -- (2.1132, 
  6.8891) -- (2.1113, 6.8875) -- (2.1118, 6.8859) -- (2.1142, 6.8823) -- 
  (2.1157, 6.8811) -- (2.1304, 6.8702) -- (2.1331, 6.8684) -- (2.1409, 6.8683) 
  -- (2.1456, 6.8697) -- (2.1467, 6.8716) -- (2.1466, 6.8757) -- (2.1397, 
  6.8863) -- (2.1387, 6.8878) -- (2.1377, 6.8885) -- (2.1347, 6.8894) -- 
  (2.1287, 6.8876) -- cycle(2.1663, 6.8092) -- (2.1713, 6.8046) -- (2.1742, 
  6.8038) -- (2.1803, 6.804) -- (2.1897, 6.806) -- (2.1977, 6.8083) -- (2.1999, 
  6.8096) -- (2.2031, 6.8133) -- (2.2132, 6.8356) -- (2.2187, 6.8485) -- 
  (2.2195, 6.8621) -- (2.2174, 6.8634) -- (2.2117, 6.8643) -- (2.2065, 6.8637) 
  -- (2.1862, 6.8573) -- (2.1833, 6.8564) -- (2.1595, 6.8465) -- (2.1562, 
  6.8448) -- (2.1542, 6.8423) -- (2.1537, 6.8398) -- (2.154, 6.8293) -- (2.1552,
   6.8249) -- (2.1565, 6.822) -- (2.1587, 6.8191) -- (2.1599, 6.8181) -- 
  cycle(2.3059, 6.8338) -- (2.277, 6.8151) -- (2.2732, 6.8104) -- (2.2732, 
  6.8079) -- (2.2731, 6.7843) -- (2.2793, 6.7794) -- (2.2847, 6.7767) -- 
  (2.2918, 6.7751) -- (2.3118, 6.7731) -- (2.3202, 6.7732) -- (2.3319, 6.7829) 
  -- (2.3369, 6.7892) -- (2.3377, 6.7903) -- (2.3514, 6.8113) -- (2.3517, 
  6.8128) -- (2.3495, 6.8247) -- (2.3466, 6.828) -- (2.338, 6.8299) -- (2.3174, 
  6.8307) -- cycle(3.5202, 6.8405) -- (3.5159, 6.842) -- (3.5141, 6.8436) -- 
  (3.5132, 6.8451) -- (3.5118, 6.8445) -- (3.5088, 6.8421) -- (3.5016, 6.8339) 
  -- (3.496, 6.8262) -- (3.4927, 6.8204) -- (3.4863, 6.8062) -- (3.4831, 6.7971)
   -- (3.4823, 6.7941) -- (3.4795, 6.7628) -- (3.4806, 6.7577) -- (3.4819, 
  6.7559) -- (3.4963, 6.7461) -- (3.4986, 6.7451) -- (3.5022, 6.7445) -- 
  (3.5026, 6.745) -- (3.5022, 6.7454) -- (3.502, 6.745) -- (3.5014, 6.745) -- 
  (3.4978, 6.7463) -- (3.489, 6.7518) -- (3.4891, 6.7575) -- (3.4909, 6.7592) --
   (3.5121, 6.7666) -- (3.5217, 6.7635) -- (3.5253, 6.762) -- (3.5291, 6.7604) 
  -- (3.5303, 6.758) -- (3.5329, 6.7503) -- (3.5322, 6.7418) -- (3.5308, 6.7301)
   -- (3.5772, 6.7214) -- (3.6488, 6.7208) -- (3.6494, 6.721) -- (3.6497, 
  6.7218) -- (3.6493, 6.7273) -- (3.6378, 6.7521) -- (3.6283, 6.7721) -- 
  (3.5995, 6.8163) -- (3.5698, 6.8317) -- cycle(2.2015, 6.3829) -- (2.1999, 
  6.384) -- (2.1967, 6.3873) -- (2.192, 6.3929) -- (2.1921, 6.3966) -- (2.1932, 
  6.401) -- (2.1951, 6.4049) -- (2.1967, 6.407) -- (2.1987, 6.409) -- (2.2005, 
  6.4101) -- (2.2029, 6.4111) -- (2.2051, 6.4113) -- (2.208, 6.409) -- (2.2082, 
  6.4084) -- (2.2068, 6.385) -- (2.2042, 6.3832) -- cycle(1.7681, 6.5298) -- 
  (1.7789, 6.5309) -- (1.7795, 6.5253) -- (1.7716, 6.5113) -- (1.7649, 6.5119) 
  -- (1.7622, 6.5125) -- (1.7542, 6.5273) -- (1.7545, 6.5295) -- (1.7594, 
  6.5329) -- (1.7634, 6.5343) -- cycle(4.7234, 7.011) -- (4.7181, 7.0077) -- 
  (4.7019, 6.9945) -- (4.6901, 6.9809) -- (4.6881, 6.9777) -- (4.6606, 6.9094) 
  -- (4.6599, 6.9075) -- (4.6591, 6.8995) -- (4.6603, 6.8993) -- (4.6647, 
  6.9007) -- (4.6668, 6.9025) -- (4.6942, 6.9288) -- (4.7002, 6.9443) -- 
  (4.6988, 6.9499) -- (4.6963, 6.958) -- (4.696, 6.9593) -- (4.6965, 6.9604) -- 
  (4.7117, 6.9819) -- (4.7212, 6.9846) -- (4.7222, 6.9843) -- (4.7241, 6.9804) 
  -- (4.7254, 6.9692) -- (4.7253, 6.9664) -- (4.7243, 6.9609) -- (4.7212, 
  6.9517) -- (4.7145, 6.9396) -- (4.7119, 6.9361) -- (4.7097, 6.9309) -- 
  (4.7064, 6.9122) -- (4.7068, 6.9027) -- (4.7325, 6.9149) -- (4.7515, 6.9258) 
  -- (4.8141, 6.9) -- (4.8156, 6.8996) -- (4.8166, 6.8997) -- (4.8339, 6.9019) 
  -- (4.8509, 6.9088) -- (4.852, 6.8612) -- (4.8507, 6.8354) -- (4.8494, 6.8346)
   -- (4.846, 6.8345) -- (4.8305, 6.8341) -- (4.8059, 6.8419) -- (4.7996, 6.845)
   -- (4.763, 6.8664) -- (4.7609, 6.8688) -- (4.761, 6.8718) -- (4.764, 6.9005) 
  -- (4.7354, 6.8994) -- (4.7048, 6.8982) -- (4.6945, 6.8958) -- (4.6643, 
  6.8872) -- (4.6561, 6.8825) -- (4.6617, 6.858) -- (4.6658, 6.85) -- (4.6844, 
  6.851) -- (4.6989, 6.8497) -- (4.7219, 6.8288) -- (4.714, 6.8138) -- (4.7118, 
  6.8125) -- (4.6913, 6.8054) -- (4.6807, 6.8034) -- (4.669, 6.8011) -- (4.6584,
   6.7774) -- (4.6585, 6.7724) -- (4.6589, 6.7711) -- (4.662, 6.7694) -- 
  (4.7107, 6.7587) -- (4.7169, 6.7575) -- (4.7036, 6.7493) -- (4.6492, 6.7229) 
  -- (4.6478, 6.7209) -- (4.6465, 6.7172) -- (4.6424, 6.6947) -- (4.6428, 
  6.6851) -- (4.6582, 6.6395) -- (4.6593, 6.6373) -- (4.8015, 6.5896) -- 
  (4.8132, 6.5984) -- (4.8156, 6.6014) -- (4.8169, 6.6039) -- (4.8172, 6.605) --
   (4.8165, 6.6105) -- (4.8127, 6.6191) -- (4.8049, 6.6282) -- (4.804, 6.6286) 
  -- (4.7989, 6.6286) -- (4.7974, 6.6267) -- (4.7966, 6.6252) -- (4.7971, 6.623)
   -- (4.7967, 6.6218) -- (4.7908, 6.6167) -- (4.7836, 6.6131) -- (4.7802, 
  6.6128) -- (4.7769, 6.6167) -- (4.7766, 6.6277) -- (4.8149, 6.6684) -- 
  (4.8451, 6.6974) -- (4.8548, 6.7045) -- (4.8578, 6.7057) -- (4.9499, 6.717) --
   (4.9538, 6.7163) -- (4.9665, 6.6952) -- (4.9669, 6.6942) -- (4.967, 6.6899) 
  -- (4.9653, 6.6859) -- (4.9646, 6.6848) -- (4.9573, 6.6792) -- (4.9567, 
  6.6788) -- (4.948, 6.6744) -- (4.9479, 6.6602) -- (4.9675, 6.6654) -- (4.9757,
   6.6817) -- (4.9868, 6.7045) -- (4.9805, 6.7198) -- (4.9643, 6.7461) -- 
  (4.9527, 6.7556) -- (4.9409, 6.7639) -- (4.9399, 6.7626) -- (4.9378, 6.7615) 
  -- (4.9319, 6.7609) -- (4.9205, 6.7603) -- (4.9187, 6.7605) -- (4.9161, 
  6.7621) -- (4.9123, 6.7655) -- (4.9031, 6.7765) -- (4.8929, 6.7956) -- 
  (4.8906, 6.8016) -- (4.8877, 6.8119) -- (4.8867, 6.8252) -- (4.8877, 6.8304) 
  -- (4.8894, 6.8335) -- (4.8909, 6.8355) -- (4.8918, 6.8366) -- (4.9063, 6.853)
   -- (4.9256, 6.8692) -- (4.9307, 6.8718) -- (4.9326, 6.8722) -- (4.9382, 
  6.8763) -- (4.9434, 6.8869) -- (4.9447, 6.8966) -- (4.9425, 6.916) -- (4.9389,
   6.9207) -- (4.9326, 6.9265) -- (4.9195, 6.9351) -- (4.9172, 6.9359) -- 
  (4.913, 6.9359) -- (4.8967, 6.9343) -- (4.8928, 6.9339) -- (4.872, 6.9296) -- 
  (4.8691, 6.9283) -- (4.8504, 6.9224) -- (4.8239, 6.9176) -- (4.818, 6.918) -- 
  (4.8086, 6.9203) -- (4.805, 6.9215) -- (4.8006, 6.9245) -- (4.7978, 6.9263) --
   (4.7902, 6.934) -- (4.7835, 6.943) -- (4.7786, 6.953) -- (4.7748, 6.9669) -- 
  (4.7735, 6.9748) -- (4.7739, 6.9778) -- (4.7771, 6.9873) -- (4.7803, 6.9916) 
  -- (4.7818, 6.9927) -- (4.7867, 6.995) -- (4.7904, 6.996) -- (4.794, 6.9963) 
  -- (4.7966, 6.9974) -- (4.8004, 7.0007) -- (4.8026, 7.0051) -- (4.8059, 
  7.0176) -- (4.8058, 7.0182) -- (4.8024, 7.0229) -- (4.8002, 7.0253) -- 
  (4.7722, 7.0228) -- (4.731, 7.0111) -- cycle(4.6091, 6.8074) -- (4.6102, 
  6.8046) -- (4.6111, 6.8046) -- (4.6128, 6.8069) -- (4.6144, 6.8133) -- 
  (4.6245, 6.8691) -- (4.6317, 6.9048) -- (4.6526, 6.9154) -- (4.6253, 6.9251) 
  -- (4.6087, 6.8299) -- (4.6074, 6.8203) -- (4.6073, 6.8153) -- (4.608, 6.8122)
   -- cycle(4.6536, 6.7937) -- (4.6926, 6.8152) -- (4.6973, 6.819) -- (4.6984, 
  6.8203) -- (4.6958, 6.8288) -- (4.6934, 6.8317) -- (4.6815, 6.8372) -- 
  (4.6793, 6.8382) -- (4.673, 6.8389) -- (4.6668, 6.8374) -- (4.6608, 6.8333) --
   (4.6531, 6.823) -- (4.6516, 6.8198) -- (4.6419, 6.797) -- (4.6416, 6.7909) --
   (4.6422, 6.7894) -- (4.6435, 6.7888) -- cycle(4.5658, 6.785) -- (4.5645, 
  6.7853) -- (4.5625, 6.7852) -- (4.5608, 6.7847) -- (4.5587, 6.7833) -- 
  (4.5576, 6.7821) -- (4.5578, 6.7809) -- (4.5603, 6.7785) -- (4.5625, 6.7764) 
  -- (4.5686, 6.7768) -- (4.5983, 6.7873) -- (4.6044, 6.7926) -- (4.6076, 
  6.7961) -- (4.603, 6.8011) -- (4.601, 6.8007) -- (4.575, 6.7909) -- (4.5717, 
  6.787) -- (4.5669, 6.7813) -- cycle(5.0356, 6.5466) -- (5.031, 6.5517) -- 
  (5.0287, 6.5532) -- (5.024, 6.5544) -- (5.021, 6.5541) -- (5.015, 6.5524) -- 
  (5.0083, 6.5488) -- (5.0039, 6.5453) -- (5.0025, 6.5434) -- (5.001, 6.5401) --
   (5.0014, 6.5278) -- (5.0046, 6.5181) -- (5.0065, 6.5157) -- (5.0247, 6.4953) 
  -- (5.0267, 6.494) -- (5.0314, 6.448) -- (5.0654, 6.3951) -- (5.0663, 6.3937) 
  -- (5.068, 6.3927) -- (5.0697, 6.3924) -- (5.1046, 6.4093) -- (5.105, 6.4096) 
  -- (5.1087, 6.4197) -- (5.1085, 6.4207) -- (5.097, 6.4379) -- (5.0936, 6.453) 
  -- (5.0935, 6.4535) -- (5.0942, 6.4545) -- (5.1261, 6.4517) -- (5.1403, 
  6.4394) -- (5.165, 6.4119) -- (5.1673, 6.4089) -- (5.169, 6.4054) -- (5.1729, 
  6.3952) -- (5.1722, 6.348) -- (5.139, 6.3538) -- (5.1322, 6.3679) -- (5.1267, 
  6.386) -- (5.1253, 6.3873) -- (5.1211, 6.3891) -- (5.1143, 6.39) -- (5.097, 
  6.3855) -- (5.0957, 6.385) -- (5.0946, 6.3841) -- (5.0912, 6.3771) -- (5.0897,
   6.3646) -- (5.09, 6.3624) -- (5.0916, 6.3586) -- (5.0944, 6.3563) -- (5.1057,
   6.3426) -- (5.109, 6.3347) -- (5.1162, 6.3136) -- (5.1165, 6.3092) -- 
  (5.1125, 6.2989) -- (5.1095, 6.2913) -- (5.1074, 6.2895) -- (5.0977, 6.2836) 
  -- (5.0862, 6.2773) -- (5.0744, 6.2714) -- (5.0706, 6.2706) -- (5.0669, 
  6.2708) -- (5.0644, 6.2712) -- (5.0575, 6.2571) -- (5.0643, 6.2505) -- 
  (5.0753, 6.2457) -- (5.0881, 6.2425) -- (5.0976, 6.2435) -- (5.1215, 6.2476) 
  -- (5.1467, 6.2545) -- (5.1497, 6.2558) -- (5.1878, 6.2754) -- (5.2566, 
  6.2825) -- (5.2595, 6.2824) -- (5.2615, 6.2821) -- (5.2738, 6.3373) -- (5.238,
   6.3667) -- (5.1988, 6.4028) -- (5.1516, 6.4471) -- (5.1416, 6.453) -- (5.128,
   6.46) -- (5.1247, 6.4615) -- (5.1215, 6.4623) -- (5.1121, 6.4639) -- (5.1029,
   6.4672) -- (5.086, 6.4767) -- (5.0811, 6.4796) -- (5.071, 6.4881) -- (5.061, 
  6.4984) -- (5.0538, 6.5066) -- (5.0516, 6.5099) -- cycle(3.7503, 6.3261) -- 
  (3.7568, 6.3463) -- (3.7588, 6.3573) -- (3.7583, 6.3622) -- (3.7576, 6.3639) 
  -- (3.7552, 6.3651) -- (3.7466, 6.3671) -- (3.7369, 6.3657) -- (3.7337, 
  6.3648) -- (3.7292, 6.3629) -- (3.7076, 6.3516) -- (3.6941, 6.3412) -- 
  (3.6917, 6.3358) -- (3.6922, 6.3263) -- (3.6935, 6.3203) -- (3.6958, 6.3154) 
  -- (3.6979, 6.3126) -- (3.7012, 6.3099) -- (3.7045, 6.3092) -- (3.7369, 
  6.3032) -- cycle(2.1118, 0.4122) -- (2.1051, 0.4295) -- (2.103, 0.4313) -- 
  (2.0796, 0.43) -- (2.0816, 0.4099) -- (2.0875, 0.4051) -- (2.1065, 0.4071) -- 
  (2.1129, 0.4079) -- cycle;

}

\newcommand{\drawhamburg}[2]{%
  %Hamburg
  \path[draw=black,fill=#1,line join=round,line width=0.0046cm] (2.7756, 
  5.9184) -- (2.7797, 5.9393) -- (2.7836, 5.9535) -- (2.7859, 5.9618) -- 
  (2.7925, 5.981) -- (2.7947, 5.9839) -- (2.7983, 5.984) -- (2.7974, 5.9793) -- 
  (2.7986, 5.9732) -- (2.8142, 5.9542) -- (2.8238, 5.9452) -- (2.8287, 5.9431) 
  -- (2.8356, 5.9465) -- (2.8679, 5.9871) -- (2.8713, 5.9922) -- (2.8743, 
  5.9973) -- (2.8742, 6.0003) -- (2.8749, 6.0073) -- (2.8852, 6.0079) -- 
  (2.8966, 6.0076) -- (2.8989, 6.0058) -- (2.9077, 6.0039) -- (2.9123, 6.0034) 
  -- (2.9192, 6.0034) -- (2.964, 6.038) -- (2.9714, 6.0558) -- (2.9668, 6.0615) 
  -- (2.9799, 6.0701) -- (3.0192, 6.0892) -- (3.0218, 6.0888) -- (3.0236, 6.088)
   -- (3.0354, 6.0813) -- (3.0343, 6.0735) -- (3.0287, 6.0605) -- (3.03, 6.0176)
   -- (3.0369, 6.0094) -- (3.0516, 5.9896) -- (3.0495, 5.9835) -- (3.0491, 
  5.9825) -- (3.0376, 5.9581) -- (3.0187, 5.9429) -- (3.0128, 5.937) -- (3.01, 
  5.9262) -- (3.0106, 5.9071) -- (3.0184, 5.8848) -- (3.0204, 5.883) -- (3.0524,
   5.8606) -- (3.0559, 5.8599) -- (3.0603, 5.8604) -- (3.059, 5.8547) -- 
  (3.0612, 5.8474) -- (3.0955, 5.8107) -- (3.1004, 5.8012) -- (3.0991, 5.8011) 
  -- (3.0973, 5.8002) -- (3.077, 5.7891) -- (3.0736, 5.7872) -- (3.0713, 5.7849)
   -- (3.0695, 5.7824) -- (3.064, 5.7687) -- (3.0618, 5.7664) -- (3.0601, 
  5.7652) -- (3.0567, 5.7647) -- (3.0273, 5.766) -- (3.0238, 5.7668) -- (3.0197,
   5.7695) -- (2.9964, 5.7915) -- (2.9802, 5.8099) -- (2.9768, 5.8146) -- 
  (2.9721, 5.8199) -- (2.956, 5.8295) -- (2.9509, 5.8227) -- (2.9454, 5.8108) --
   (2.9437, 5.8025) -- (2.9179, 5.7834) -- (2.914, 5.7832) -- (2.8995, 5.7953) 
  -- (2.8801, 5.8137) -- (2.8733, 5.8221) -- (2.8703, 5.8242) -- (2.8679, 
  5.8227) -- (2.8562, 5.8096) -- (2.8486, 5.8039) -- (2.8462, 5.8029) -- 
  (2.8149, 5.8351) -- (2.8143, 5.8405) -- (2.8055, 5.8609) -- (2.7979, 5.8688) 
  -- (2.8007, 5.8892) -- (2.8007, 5.915) -- cycle;



  \node[text=black,line width=0.0092cm,anchor=south east] (text7) at (0.52, 
  5.8462){\resizebox{\ifdim\width>2em 2em\else\width\fi}{!}{#2}};
  \path[draw=black,line width=0.0141cm] (2.9766, 5.9423) -- (text7.east);



}

\newcommand{\drawhessen}[2]{%
  %Hessen
  \path[draw=black,fill=#1,line join=round,line width=0.0046cm] (2.883, 
  3.8664) -- (2.8812, 3.858) -- (2.8825, 3.8488) -- (2.8878, 3.8289) -- (2.901, 
  3.7973) -- (2.9121, 3.7834) -- (2.9434, 3.7719) -- (2.9607, 3.7646) -- 
  (2.9691, 3.738) -- (3.0109, 3.7168) -- (3.0261, 3.7116) -- (3.0336, 3.7071) --
   (3.0389, 3.7027) -- (3.051, 3.6909) -- (3.0624, 3.6881) -- (3.065, 3.6825) --
   (3.062, 3.6752) -- (3.0564, 3.6712) -- (3.053, 3.6695) -- (3.05, 3.6667) -- 
  (3.0461, 3.6545) -- (3.0464, 3.6522) -- (3.0497, 3.6472) -- (3.0515, 3.6475) 
  -- (3.0519, 3.6312) -- (3.0515, 3.6239) -- (3.0498, 3.6219) -- (3.0446, 
  3.6219) -- (3.0412, 3.6283) -- (3.0413, 3.6287) -- (3.0388, 3.6328) -- 
  (3.0355, 3.6449) -- (3.0258, 3.6548) -- (3.0019, 3.6489) -- (2.9977, 3.6437) 
  -- (2.9987, 3.6413) -- (3.011, 3.615) -- (3.0167, 3.6038) -- (3.0186, 3.5642) 
  -- (3.0351, 3.5582) -- (3.0433, 3.5527) -- (3.0451, 3.5507) -- (3.0498, 
  3.5459) -- (3.0546, 3.5372) -- (3.053, 3.5327) -- (3.0476, 3.5221) -- (3.049, 
  3.5204) -- (3.0439, 3.5144) -- (3.0414, 3.5129) -- (3.0367, 3.5111) -- 
  (3.0351, 3.5115) -- (3.0328, 3.5098) -- (3.0171, 3.5077) -- (3.0133, 3.5072) 
  -- (3.0091, 3.5085) -- (3.0064, 3.5105) -- (3.0037, 3.5138) -- (2.9919, 
  3.5202) -- (2.9709, 3.5231) -- (2.9577, 3.5245) -- (2.9498, 3.5235) -- 
  (2.9474, 3.5223) -- (2.946, 3.5214) -- (2.9447, 3.5187) -- (2.9391, 3.507) -- 
  (2.9526, 3.4629) -- (2.9614, 3.4087) -- (2.9616, 3.4014) -- (2.9467, 3.3638) 
  -- (2.9402, 3.3532) -- (2.9275, 3.3528) -- (2.9126, 3.3548) -- (2.9084, 
  3.3518) -- (2.8984, 3.3436) -- (2.8971, 3.3404) -- (2.8938, 3.3239) -- 
  (2.8939, 3.3188) -- (2.8974, 3.3123) -- (2.8983, 3.3033) -- (2.8984, 3.2991) 
  -- (2.8972, 3.2977) -- (2.8948, 3.2969) -- (2.8825, 3.293) -- (2.8909, 3.2826)
   -- (2.8925, 3.2691) -- (2.8925, 3.2661) -- (2.8881, 3.2548) -- (2.8767, 
  3.2302) -- (2.8639, 3.2145) -- (2.8577, 3.2056) -- (2.8549, 3.2015) -- 
  (2.8543, 3.1772) -- (2.8548, 3.1687) -- (2.8848, 3.1633) -- (2.8948, 3.1617) 
  -- (2.9033, 3.1634) -- (2.9082, 3.1748) -- (2.8969, 3.1888) -- (2.8965, 
  3.1944) -- (2.8987, 3.1973) -- (2.921, 3.206) -- (2.9255, 3.2072) -- (2.9452, 
  3.206) -- (2.9569, 3.2049) -- (2.9596, 3.2027) -- (2.9668, 3.1902) -- (2.9735,
   3.1713) -- (2.957, 3.1574) -- (2.9551, 3.1553) -- (2.9514, 3.1495) -- 
  (2.9499, 3.1461) -- (2.9517, 3.0571) -- (2.9521, 3.0501) -- (2.9493, 3.035) --
   (2.945, 3.023) -- (2.9365, 3.0102) -- (2.9042, 2.9755) -- (2.8896, 2.967) -- 
  (2.8785, 2.958) -- (2.8678, 2.9514) -- (2.8568, 2.9468) -- (2.8426, 2.9452) --
   (2.8221, 2.9537) -- (2.8151, 2.9593) -- (2.7825, 2.9498) -- (2.7793, 2.9352) 
  -- (2.7741, 2.9155) -- (2.7726, 2.9125) -- (2.7663, 2.9067) -- (2.7727, 
  2.8975) -- (2.775, 2.8923) -- (2.7758, 2.8773) -- (2.775, 2.8671) -- (2.774, 
  2.8635) -- (2.77, 2.8562) -- (2.7629, 2.8492) -- (2.7482, 2.8358) -- (2.7415, 
  2.8332) -- (2.7406, 2.8329) -- (2.7344, 2.8329) -- (2.7187, 2.8273) -- 
  (2.7175, 2.8257) -- (2.7145, 2.8208) -- (2.7082, 2.809) -- (2.6996, 2.7872) --
   (2.6989, 2.7834) -- (2.6978, 2.7824) -- (2.6832, 2.7814) -- (2.676, 2.7812) 
  -- (2.6481, 2.7884) -- (2.6381, 2.7918) -- (2.6272, 2.7686) -- (2.6239, 
  2.7373) -- (2.6301, 2.7) -- (2.6323, 2.6888) -- (2.6363, 2.6793) -- (2.6363, 
  2.6719) -- (2.6326, 2.661) -- (2.5858, 2.6526) -- (2.5734, 2.6513) -- (2.5716,
   2.6513) -- (2.5667, 2.6552) -- (2.5496, 2.6948) -- (2.5486, 2.696) -- 
  (2.5458, 2.6967) -- (2.4569, 2.7141) -- (2.3334, 2.6802) -- (2.3322, 2.6793) 
  -- (2.3274, 2.6737) -- (2.3207, 2.6628) -- (2.3164, 2.64) -- (2.3105, 2.628) 
  -- (2.3224, 2.6177) -- (2.3251, 2.6183) -- (2.3305, 2.6181) -- (2.3352, 
  2.6167) -- (2.3364, 2.6161) -- (2.3393, 2.6136) -- (2.3406, 2.6111) -- 
  (2.3439, 2.5754) -- (2.3294, 2.5672) -- (2.3326, 2.5629) -- (2.333, 2.5608) --
   (2.3364, 2.5331) -- (2.3378, 2.5069) -- (2.3387, 2.488) -- (2.3422, 2.4601) 
  -- (2.3431, 2.458) -- (2.3497, 2.4493) -- (2.3413, 2.4503) -- (2.3393, 2.4498)
   -- (2.3396, 2.444) -- (2.3407, 2.4308) -- (2.3416, 2.4272) -- (2.344, 2.4246)
   -- (2.3839, 2.3846) -- (2.3875, 2.3844) -- (2.3947, 2.3829) -- (2.3968, 
  2.3822) -- (2.3984, 2.3811) -- (2.4093, 2.3324) -- (2.3944, 2.2928) -- 
  (2.3898, 2.2906) -- (2.3774, 2.2564) -- (2.3699, 2.1925) -- (2.3781, 2.1775) 
  -- (2.3765, 2.1625) -- (2.3773, 2.1617) -- (2.3737, 2.1527) -- (2.3724, 
  2.1507) -- (2.3674, 2.1467) -- (2.3822, 2.1387) -- (2.3854, 2.1364) -- 
  (2.3906, 2.1237) -- (2.3905, 2.1198) -- (2.3878, 2.1159) -- (2.3794, 2.1137) 
  -- (2.3658, 2.1295) -- (2.3553, 2.132) -- (2.3036, 2.1154) -- (2.3014, 2.1143)
   -- (2.2918, 2.1075) -- (2.2773, 2.0907) -- (2.2738, 2.0842) -- (2.2719, 
  2.0784) -- (2.2787, 2.0722) -- (2.2838, 2.0645) -- (2.2244, 2.0102) -- 
  (2.2221, 2.0085) -- (2.2209, 2.0082) -- (2.2038, 2.009) -- (2.1994, 2.0098) --
   (2.194, 2.0158) -- (2.1912, 2.0237) -- (2.1946, 2.0315) -- (2.2108, 2.0659) 
  -- (2.2319, 2.0733) -- (2.2532, 2.0926) -- (2.2514, 2.104) -- (2.2529, 2.1099)
   -- (2.2309, 2.1099) -- (2.1713, 2.1216) -- (2.1671, 2.1231) -- (2.146, 
  2.1325) -- (2.1326, 2.1416) -- (2.1155, 2.2116) -- (2.1188, 2.2135) -- 
  (2.1197, 2.2117) -- (2.1237, 2.2103) -- (2.1246, 2.2113) -- (2.1293, 2.2204) 
  -- (2.1282, 2.2248) -- (2.1193, 2.2269) -- (2.0832, 2.2178) -- (2.0675, 
  2.2134) -- (2.0648, 2.2078) -- (2.0652, 2.2004) -- (2.0672, 2.1968) -- 
  (2.0729, 2.1977) -- (2.0812, 2.1604) -- (2.0812, 2.1558) -- (2.0802, 2.1508) 
  -- (2.0741, 2.1408) -- (2.0582, 2.1284) -- (2.0496, 2.128) -- (2.0469, 2.1287)
   -- (2.0427, 2.1304) -- (2.0343, 2.1367) -- (2.0286, 2.1445) -- (2.0251, 
  2.1554) -- (2.0064, 2.1795) -- (1.993, 2.1911) -- (1.988, 2.1954) -- (1.9788, 
  2.1964) -- (1.9762, 2.1962) -- (1.9733, 2.1956) -- (1.9638, 2.1909) -- 
  (1.9623, 2.1901) -- (1.9577, 2.201) -- (1.9431, 2.2229) -- (1.9385, 2.2314) --
   (1.9238, 2.2786) -- (1.9228, 2.2828) -- (1.9244, 2.2917) -- (1.925, 2.2939) 
  -- (1.9269, 2.2973) -- (1.928, 2.2985) -- (1.9337, 2.3029) -- (1.9388, 2.3061)
   -- (1.9432, 2.3082) -- (1.9496, 2.3107) -- (1.9634, 2.3137) -- (1.9669, 
  2.3151) -- (1.9707, 2.3173) -- (1.9751, 2.3207) -- (1.9788, 2.3263) -- 
  (1.9796, 2.328) -- (1.9809, 2.3307) -- (1.9866, 2.3401) -- (1.9834, 2.3615) --
   (1.9578, 2.3882) -- (1.9531, 2.3973) -- (1.9439, 2.4104) -- (1.9411, 2.415) 
  -- (1.9206, 2.4667) -- (1.9189, 2.474) -- (1.9189, 2.4776) -- (1.9199, 2.4824)
   -- (1.9212, 2.4856) -- (1.9244, 2.4908) -- (1.9272, 2.4968) -- (1.9278, 
  2.4998) -- (1.928, 2.5044) -- (1.9268, 2.5094) -- (1.9214, 2.5269) -- (1.9211,
   2.5356) -- (1.9167, 2.5464) -- (1.9124, 2.5538) -- (1.909, 2.559) -- (1.9024,
   2.5668) -- (1.8907, 2.5774) -- (1.8836, 2.5881) -- (1.8795, 2.594) -- 
  (1.8732, 2.6009) -- (1.8625, 2.6111) -- (1.8572, 2.6133) -- (1.8402, 2.6169) 
  -- (1.8306, 2.6182) -- (1.8222, 2.6176) -- (1.799, 2.6102) -- (1.761, 2.5928) 
  -- (1.6966, 2.5682) -- (1.6939, 2.5674) -- (1.6757, 2.5645) -- (1.6451, 
  2.5635) -- (1.6419, 2.5639) -- (1.6394, 2.5647) -- (1.6351, 2.5677) -- 
  (1.5854, 2.6371) -- (1.5792, 2.6549) -- (1.5878, 2.6567) -- (1.6061, 2.667) --
   (1.6442, 2.6988) -- (1.6477, 2.6974) -- (1.6689, 2.6915) -- (1.6717, 2.6952) 
  -- (1.6752, 2.7227) -- (1.674, 2.7259) -- (1.6649, 2.7342) -- (1.6576, 2.7397)
   -- (1.6529, 2.7426) -- (1.6495, 2.7474) -- (1.649, 2.7595) -- (1.6496, 
  2.7607) -- (1.6636, 2.7791) -- (1.6689, 2.783) -- (1.6721, 2.7846) -- (1.7119,
   2.8019) -- (1.7288, 2.792) -- (1.7379, 2.7898) -- (1.7407, 2.7899) -- (1.756,
   2.8069) -- (1.7552, 2.8085) -- (1.7518, 2.8109) -- (1.7364, 2.8258) -- 
  (1.7343, 2.8302) -- (1.7434, 2.8421) -- (1.7516, 2.8447) -- (1.7577, 2.8439) 
  -- (1.7769, 2.836) -- (1.7952, 2.8466) -- (1.795, 2.8642) -- (1.7944, 2.8679) 
  -- (1.7876, 2.8885) -- (1.787, 2.89) -- (1.7851, 2.8937) -- (1.7608, 2.9382) 
  -- (1.7124, 2.9705) -- (1.7077, 2.9706) -- (1.7103, 2.9777) -- (1.7264, 
  3.0231) -- (1.7197, 3.032) -- (1.7148, 3.0438) -- (1.7147, 3.0468) -- (1.719, 
  3.0691) -- (1.7238, 3.0789) -- (1.7458, 3.105) -- (1.7492, 3.1063) -- (1.7547,
   3.1078) -- (1.7576, 3.1075) -- (1.7687, 3.1031) -- (1.7733, 3.0935) -- 
  (1.7826, 3.0903) -- (1.7925, 3.0884) -- (1.7995, 3.0893) -- (1.8012, 3.0933) 
  -- (1.8083, 3.1103) -- (1.8108, 3.1164) -- (1.8176, 3.1329) -- (1.8211, 
  3.1437) -- (1.821, 3.1492) -- (1.8102, 3.1553) -- (1.8069, 3.1572) -- (1.8058,
   3.1595) -- (1.7973, 3.2041) -- (1.7974, 3.2127) -- (1.798, 3.2177) -- 
  (1.7993, 3.2215) -- (1.8074, 3.2307) -- (1.8141, 3.2398) -- (1.8325, 3.274) --
   (1.833, 3.2772) -- (1.8314, 3.2859) -- (1.824, 3.2962) -- (1.817, 3.3101) -- 
  (1.8134, 3.3176) -- (1.8095, 3.3268) -- (1.8965, 3.4118) -- (1.8983, 3.4133) 
  -- (1.9001, 3.4136) -- (1.9091, 3.4133) -- (1.9108, 3.4127) -- (1.9482, 3.393)
   -- (1.9499, 3.3925) -- (1.9549, 3.3936) -- (1.956, 3.3947) -- (1.9663, 
  3.4058) -- (2.0056, 3.4651) -- (2.0111, 3.4889) -- (2.0116, 3.49) -- (2.0153, 
  3.4922) -- (2.0161, 3.4926) -- (2.0216, 3.4927) -- (2.0231, 3.4922) -- 
  (2.0206, 3.4956) -- (2.0205, 3.4978) -- (2.0213, 3.4992) -- (2.0292, 3.5078) 
  -- (2.0424, 3.522) -- (2.0514, 3.53) -- (2.055, 3.5334) -- (2.0594, 3.54) -- 
  (2.056, 3.5481) -- (2.0498, 3.5615) -- (2.0401, 3.5915) -- (2.0392, 3.5944) --
   (2.0392, 3.5961) -- (2.0432, 3.6019) -- (2.0481, 3.6061) -- (2.0551, 3.612) 
  -- (2.0563, 3.6127) -- (2.0679, 3.617) -- (2.0692, 3.6181) -- (2.095, 3.6165) 
  -- (2.1106, 3.6139) -- (2.1176, 3.6111) -- (2.1278, 3.6087) -- (2.1319, 
  3.6089) -- (2.1611, 3.6201) -- (2.1871, 3.6772) -- (2.1896, 3.6832) -- 
  (2.1903, 3.6867) -- (2.1888, 3.7054) -- (2.182, 3.7642) -- (2.1809, 3.7666) --
   (2.1731, 3.7772) -- (2.1647, 3.7774) -- (2.1583, 3.7768) -- (2.1164, 3.758) 
  -- (2.1076, 3.752) -- (2.1066, 3.7517) -- (2.0996, 3.7521) -- (2.0968, 3.7531)
   -- (2.0948, 3.755) -- (2.0759, 3.7815) -- (2.0811, 3.7894) -- (2.1051, 
  3.8238) -- (2.1151, 3.8362) -- (2.1251, 3.8427) -- (2.1522, 3.8599) -- 
  (2.1978, 3.874) -- (2.2324, 3.8832) -- (2.2706, 3.8841) -- (2.3022, 3.882) -- 
  (2.3035, 3.8825) -- (2.3111, 3.905) -- (2.3091, 3.9175) -- (2.3087, 3.9195) --
   (2.3059, 3.9211) -- (2.3009, 3.9222) -- (2.2779, 3.9602) -- (2.2762, 3.9662) 
  -- (2.2762, 3.9703) -- (2.2825, 3.9766) -- (2.3086, 3.9878) -- (2.3232, 
  3.9927) -- (2.3436, 3.9977) -- (2.3543, 3.9982) -- (2.3811, 3.9869) -- (2.392,
   3.969) -- (2.3854, 3.9527) -- (2.3906, 3.9377) -- (2.3922, 3.9355) -- 
  (2.3947, 3.9339) -- (2.4003, 3.932) -- (2.4256, 3.9304) -- (2.4311, 3.9312) --
   (2.4366, 3.9405) -- (2.4425, 3.9501) -- (2.4821, 3.982) -- (2.5231, 4.014) --
   (2.5528, 4.0627) -- (2.588, 4.105) -- (2.5999, 4.125) -- (2.6299, 4.1315) -- 
  (2.6324, 4.131) -- (2.6338, 4.129) -- (2.6341, 4.1255) -- (2.6318, 4.1206) -- 
  (2.6299, 4.1131) -- (2.6303, 4.1099) -- (2.6331, 4.1063) -- (2.6398, 4.1022) 
  -- (2.6461, 4.1013) -- (2.6755, 4.1008) -- (2.6843, 4.101) -- (2.7156, 4.085) 
  -- (2.7163, 4.0841) -- (2.7294, 4.0698) -- (2.7357, 4.0645) -- (2.7433, 
  4.0596) -- (2.7449, 4.0579) -- (2.7465, 4.0542) -- (2.7436, 4.0466) -- 
  (2.7425, 4.045) -- (2.7415, 4.045) -- (2.7355, 4.0467) -- (2.7288, 4.0412) -- 
  (2.7234, 4.0339) -- (2.7217, 4.0306) -- (2.7077, 4.0282) -- (2.6854, 4.0033) 
  -- (2.6841, 4.0004) -- (2.684, 3.9995) -- (2.6845, 3.998) -- (2.7034, 3.9658) 
  -- (2.7084, 3.9602) -- (2.7116, 3.9457) -- (2.7186, 3.9062) -- (2.7177, 
  3.9043) -- (2.7109, 3.8938) -- (2.6991, 3.8943) -- (2.6843, 3.8918) -- 
  (2.6796, 3.891) -- (2.6698, 3.8755) -- (2.6679, 3.8715) -- (2.6633, 3.8532) --
   (2.6626, 3.8503) -- (2.6672, 3.8431) -- (2.6733, 3.8325) -- (2.6892, 3.8274) 
  -- (2.7377, 3.8072) -- (2.7685, 3.792) -- (2.7697, 3.7899) -- (2.791, 3.8008) 
  -- (2.7935, 3.8117) -- (2.7944, 3.8161) -- (2.7937, 3.8286) -- (2.793, 3.8299)
   -- (2.7883, 3.8297) -- (2.7725, 3.8222) -- (2.7562, 3.8477) -- (2.7662, 
  3.8614) -- (2.8065, 3.8954) -- (2.8105, 3.8979) -- (2.8121, 3.8983) -- 
  (2.8415, 3.9025) -- (2.8699, 3.9062) -- (2.8725, 3.9047) -- (2.884, 3.8861) --
   (2.8871, 3.8806) -- (2.8875, 3.8699) -- (2.8871, 3.8686) -- cycle;


   \node[text=black,line width=0.0092cm,anchor=center] (text9) at (2.3588, 
   3.102){\resizebox{\ifdim\width>1.75em 1.75em\else\width\fi}{!}{#2}};
   }

\newcommand{\drawmecklenburg}[2]{%
  %Mecklenburg-Vorpommern
  \path[draw=black,fill=#1,line join=round,line width=0.0046cm] (3.4288, 
  6.2971) -- (3.4706, 6.3256) -- (3.4977, 6.3418) -- (3.507, 6.3458) -- (3.5109,
   6.3467) -- (3.5309, 6.3515) -- (3.5369, 6.352) -- (3.5822, 6.3542) -- 
  (3.5848, 6.3526) -- (3.5881, 6.3492) -- (3.6189, 6.3145) -- (3.617, 6.2971) --
   (3.6173, 6.2921) -- (3.6204, 6.2847) -- (3.6258, 6.2797) -- (3.6314, 6.2768) 
  -- (3.6364, 6.2751) -- (3.6414, 6.2756) -- (3.6428, 6.2762) -- (3.6537, 
  6.2837) -- (3.6579, 6.2868) -- (3.6595, 6.2892) -- (3.6743, 6.2927) -- (3.696,
   6.2833) -- (3.7075, 6.2781) -- (3.7069, 6.2698) -- (3.708, 6.2677) -- 
  (3.7135, 6.2633) -- (3.7257, 6.2561) -- (3.7329, 6.2525) -- (3.7464, 6.2758) 
  -- (3.7496, 6.2851) -- (3.7549, 6.3364) -- (3.7631, 6.3531) -- (3.7643, 
  6.3549) -- (3.768, 6.3582) -- (3.7845, 6.3677) -- (3.7923, 6.3703) -- (3.7953,
   6.3722) -- (3.7997, 6.3763) -- (3.8013, 6.3785) -- (3.8265, 6.4177) -- 
  (3.8273, 6.4283) -- (3.8268, 6.4306) -- (3.8208, 6.4403) -- (3.8189, 6.4424) 
  -- (3.8167, 6.4424) -- (3.8139, 6.44) -- (3.802, 6.4279) -- (3.7967, 6.4185) 
  -- (3.7898, 6.409) -- (3.7829, 6.4004) -- (3.767, 6.3812) -- (3.7665, 6.3809) 
  -- (3.7658, 6.3815) -- (3.7645, 6.3895) -- (3.7643, 6.3944) -- (3.7653, 6.399)
   -- (3.7712, 6.4104) -- (3.7871, 6.4323) -- (3.8565, 6.4894) -- (3.8585, 6.49)
   -- (3.8634, 6.4905) -- (3.8685, 6.4908) -- (3.8997, 6.4918) -- (3.9012, 
  6.4916) -- (3.907, 6.489) -- (3.9137, 6.487) -- (3.9225, 6.4849) -- (3.9331, 
  6.4836) -- (3.9423, 6.4833) -- (3.9505, 6.4839) -- (3.9556, 6.4852) -- 
  (3.9673, 6.4888) -- (4.0314, 6.5143) -- (4.0693, 6.5086) -- (4.0963, 6.5051) 
  -- (4.1101, 6.5362) -- (4.1282, 6.5711) -- (4.1311, 6.5752) -- (4.1403, 
  6.5836) -- (4.1551, 6.5939) -- (4.1872, 6.614) -- (4.1977, 6.6204) -- (4.2022,
   6.6237) -- (4.2071, 6.6277) -- (4.2128, 6.6336) -- (4.2171, 6.6388) -- 
  (4.2224, 6.6475) -- (4.261, 6.7201) -- (4.2861, 6.7696) -- (4.3064, 6.8044) --
   (4.3225, 6.7913) -- (4.3303, 6.788) -- (4.3345, 6.7869) -- (4.3719, 6.7821) 
  -- (4.4046, 6.7783) -- (4.4711, 6.78) -- (4.5108, 6.7828) -- (4.5228, 6.7843) 
  -- (4.5393, 6.7847) -- (4.5517, 6.7833) -- (4.5528, 6.7827) -- (4.5533, 
  6.7817) -- (4.5536, 6.7798) -- (4.553, 6.7789) -- (4.5295, 6.7632) -- (4.5259,
   6.7611) -- (4.5086, 6.7567) -- (4.4697, 6.7538) -- (4.4429, 6.7529) -- 
  (4.4334, 6.7536) -- (4.4323, 6.754) -- (4.4302, 6.7557) -- (4.4261, 6.762) -- 
  (4.4238, 6.7682) -- (4.4214, 6.7695) -- (4.4209, 6.7695) -- (4.4067, 6.7688) 
  -- (4.3768, 6.7612) -- (4.3485, 6.7433) -- (4.2664, 6.7134) -- (4.2579, 
  6.7064) -- (4.2327, 6.6472) -- (4.2299, 6.6301) -- (4.2302, 6.6071) -- 
  (4.2306, 6.6064) -- (4.251, 6.5925) -- (4.2608, 6.5889) -- (4.2716, 6.5882) --
   (4.2808, 6.5898) -- (4.2841, 6.5917) -- (4.2842, 6.5935) -- (4.2814, 6.5957) 
  -- (4.2782, 6.5974) -- (4.2669, 6.5997) -- (4.2626, 6.5997) -- (4.261, 6.6004)
   -- (4.2556, 6.611) -- (4.2549, 6.6188) -- (4.259, 6.6279) -- (4.2882, 6.6612)
   -- (4.3311, 6.6942) -- (4.3859, 6.7208) -- (4.4323, 6.7114) -- (4.4753, 
  6.6918) -- (4.4933, 6.6951) -- (4.4967, 6.6973) -- (4.5092, 6.7084) -- 
  (4.5111, 6.7103) -- (4.5131, 6.7141) -- (4.5139, 6.7183) -- (4.5139, 6.7271) 
  -- (4.5193, 6.7413) -- (4.5198, 6.7426) -- (4.5326, 6.7539) -- (4.5348, 
  6.7556) -- (4.5693, 6.7686) -- (4.5896, 6.7666) -- (4.6274, 6.7115) -- 
  (4.6201, 6.695) -- (4.6217, 6.6877) -- (4.6375, 6.6365) -- (4.6605, 6.6107) --
   (4.6764, 6.6196) -- (4.6827, 6.6227) -- (4.6843, 6.6227) -- (4.7405, 6.5959) 
  -- (4.7418, 6.5945) -- (4.7458, 6.5878) -- (4.7601, 6.5577) -- (4.7625, 
  6.5522) -- (4.7645, 6.5487) -- (4.7662, 6.5476) -- (4.7679, 6.5477) -- 
  (4.7704, 6.5422) -- (4.7905, 6.5424) -- (4.8109, 6.5346) -- (4.8111, 6.5337) 
  -- (4.7956, 6.52) -- (4.7968, 6.5138) -- (4.7986, 6.512) -- (4.8013, 6.5117) 
  -- (4.8061, 6.5148) -- (4.8082, 6.5169) -- (4.8103, 6.5171) -- (4.8216, 
  6.5101) -- (4.8225, 6.5094) -- (4.8239, 6.5068) -- (4.8299, 6.4938) -- (4.83, 
  6.4853) -- (4.8308, 6.4827) -- (4.8412, 6.4659) -- (4.8445, 6.4629) -- 
  (4.8479, 6.4614) -- (4.8615, 6.459) -- (4.8633, 6.4595) -- (4.8655, 6.4609) --
   (4.8658, 6.4615) -- (4.8652, 6.4637) -- (4.8611, 6.4691) -- (4.8569, 6.4752) 
  -- (4.8508, 6.4895) -- (4.8507, 6.4913) -- (4.8592, 6.4983) -- (4.8622, 
  6.4994) -- (4.9432, 6.5263) -- (4.9756, 6.5361) -- (5.0113, 6.5005) -- 
  (5.0171, 6.494) -- (5.0221, 6.486) -- (5.0241, 6.48) -- (5.0245, 6.4767) -- 
  (5.0246, 6.4383) -- (5.055, 6.3758) -- (5.0909, 6.338) -- (5.0977, 6.3284) -- 
  (5.1016, 6.3221) -- (5.1028, 6.319) -- (5.1025, 6.3168) -- (5.099, 6.3114) -- 
  (5.096, 6.3082) -- (5.095, 6.3076) -- (5.0934, 6.3081) -- (5.0898, 6.308) -- 
  (5.0863, 6.3064) -- (5.0831, 6.3036) -- (5.0757, 6.2956) -- (5.0591, 6.2765) 
  -- (5.0458, 6.257) -- (5.0465, 6.2534) -- (5.0518, 6.2428) -- (5.0537, 6.2411)
   -- (5.1071, 6.2081) -- (5.1259, 6.1942) -- (5.1403, 6.181) -- (5.1437, 
  6.1794) -- (5.206, 6.1572) -- (5.2183, 6.1532) -- (5.2278, 6.1518) -- (5.2587,
   6.1511) -- (5.2618, 6.1513) -- (5.2638, 6.1526) -- (5.2736, 6.1613) -- 
  (5.2813, 6.1699) -- (5.285, 6.1755) -- (5.2862, 6.1769) -- (5.2902, 6.1771) --
   (5.2964, 6.1756) -- (5.2998, 6.1742) -- (5.3075, 6.1675) -- (5.3113, 6.157) 
  -- (5.3212, 6.1046) -- (5.3393, 6.0461) -- (5.3427, 6.0114) -- (5.3391, 
  5.9855) -- (5.339, 5.9844) -- (5.3413, 5.9754) -- (5.3478, 5.9574) -- (5.3554,
   5.938) -- (5.357, 5.9368) -- (5.3719, 5.916) -- (5.3752, 5.8956) -- (5.3866, 
  5.8504) -- (5.41, 5.783) -- (5.4126, 5.7766) -- (5.4085, 5.7775) -- (5.4057, 
  5.7773) -- (5.3596, 5.7579) -- (5.3581, 5.7494) -- (5.3525, 5.7329) -- 
  (5.3339, 5.7069) -- (5.3265, 5.7045) -- (5.3035, 5.7008) -- (5.25, 5.7026) -- 
  (5.2399, 5.7038) -- (5.2416, 5.7149) -- (5.2425, 5.7193) -- (5.2469, 5.7267) 
  -- (5.2613, 5.7441) -- (5.268, 5.7502) -- (5.2774, 5.76) -- (5.2872, 5.7718) 
  -- (5.2944, 5.7815) -- (5.3037, 5.7966) -- (5.3076, 5.8056) -- (5.3088, 5.81) 
  -- (5.3137, 5.828) -- (5.3139, 5.8328) -- (5.3117, 5.8589) -- (5.3078, 5.8652)
   -- (5.3012, 5.8697) -- (5.2431, 5.8733) -- (5.2365, 5.8605) -- (5.2009, 
  5.8593) -- (5.1762, 5.8632) -- (5.1451, 5.8577) -- (5.1397, 5.8553) -- 
  (5.1249, 5.8613) -- (5.1114, 5.8902) -- (5.1025, 5.9206) -- (5.0566, 5.9737) 
  -- (5.0519, 5.9726) -- (5.0485, 5.9667) -- (5.0478, 5.9638) -- (5.0455, 
  5.9461) -- (5.0456, 5.9431) -- (5.0471, 5.9304) -- (5.0596, 5.919) -- (5.0344,
   5.8967) -- (5.0098, 5.8949) -- (4.9938, 5.8822) -- (4.985, 5.8743) -- 
  (4.9246, 5.8164) -- (4.9227, 5.8118) -- (4.9119, 5.7858) -- (4.9075, 5.7711) 
  -- (4.9098, 5.7591) -- (4.9125, 5.7426) -- (4.9126, 5.7415) -- (4.9121, 
  5.7406) -- (4.8925, 5.7142) -- (4.849, 5.6686) -- (4.8426, 5.6658) -- (4.8393,
   5.6667) -- (4.8294, 5.6817) -- (4.8282, 5.6898) -- (4.8276, 5.6917) -- 
  (4.824, 5.6966) -- (4.8221, 5.6977) -- (4.7898, 5.6982) -- (4.7798, 5.6934) --
   (4.7631, 5.6789) -- (4.7598, 5.6743) -- (4.7581, 5.6674) -- (4.7571, 5.6561) 
  -- (4.7572, 5.6515) -- (4.7561, 5.6453) -- (4.7516, 5.6373) -- (4.7379, 
  5.6394) -- (4.7363, 5.6412) -- (4.7288, 5.6573) -- (4.7267, 5.6631) -- 
  (4.7238, 5.6684) -- (4.7061, 5.666) -- (4.7044, 5.6657) -- (4.6837, 5.6556) --
   (4.6797, 5.652) -- (4.6154, 5.5819) -- (4.6126, 5.5804) -- (4.6112, 5.5802) 
  -- (4.5922, 5.5884) -- (4.5908, 5.5895) -- (4.5907, 5.5901) -- (4.5932, 
  5.5939) -- (4.6017, 5.6006) -- (4.6113, 5.6063) -- (4.6111, 5.6078) -- 
  (4.6091, 5.6125) -- (4.6088, 5.6129) -- (4.6025, 5.6147) -- (4.5357, 5.6129) 
  -- (4.5193, 5.6076) -- (4.5186, 5.607) -- (4.5184, 5.6056) -- (4.5078, 5.6016)
   -- (4.5002, 5.5987) -- (4.4895, 5.5999) -- (4.486, 5.6062) -- (4.4634, 
  5.6297) -- (4.4323, 5.6587) -- (4.4252, 5.6599) -- (4.3548, 5.6665) -- 
  (4.3386, 5.663) -- (4.3389, 5.6583) -- (4.3371, 5.6563) -- (4.3073, 5.6527) --
   (4.2846, 5.6783) -- (4.2799, 5.6914) -- (4.2709, 5.6987) -- (4.2539, 5.7079) 
  -- (4.2471, 5.7111) -- (4.2415, 5.7127) -- (4.2365, 5.7161) -- (4.2183, 
  5.7216) -- (4.1828, 5.7488) -- (4.1798, 5.7486) -- (4.1291, 5.7382) -- 
  (4.1154, 5.7332) -- (4.0998, 5.7354) -- (4.0857, 5.738) -- (4.0667, 5.7108) --
   (4.0682, 5.6945) -- (4.0648, 5.69) -- (4.0304, 5.6655) -- (4.0269, 5.6632) --
   (3.9809, 5.6464) -- (3.9515, 5.6443) -- (3.9616, 5.6273) -- (3.9628, 5.6241) 
  -- (3.9626, 5.623) -- (3.9606, 5.6209) -- (3.9099, 5.6095) -- (3.9041, 5.6086)
   -- (3.8978, 5.622) -- (3.8821, 5.6308) -- (3.8728, 5.6314) -- (3.8477, 
  5.6311) -- (3.8425, 5.6271) -- (3.8069, 5.5947) -- (3.8132, 5.5572) -- 
  (3.8073, 5.5339) -- (3.8043, 5.5249) -- (3.7964, 5.5194) -- (3.7823, 5.5198) 
  -- (3.7673, 5.5226) -- (3.7614, 5.5253) -- (3.7612, 5.5276) -- (3.7538, 5.533)
   -- (3.7228, 5.5294) -- (3.6975, 5.5094) -- (3.646, 5.5124) -- (3.6443, 
  5.5156) -- (3.6424, 5.5182) -- (3.6338, 5.5275) -- (3.6286, 5.5312) -- 
  (3.6229, 5.534) -- (3.5903, 5.5471) -- (3.5953, 5.5489) -- (3.6003, 5.554) -- 
  (3.6009, 5.5558) -- (3.6036, 5.5645) -- (3.6032, 5.5677) -- (3.6019, 5.5698) 
  -- (3.5947, 5.5738) -- (3.5895, 5.5743) -- (3.5813, 5.5762) -- (3.5706, 
  5.5788) -- (3.5444, 5.5983) -- (3.5382, 5.6045) -- (3.5277, 5.6188) -- 
  (3.5067, 5.6594) -- (3.4843, 5.7104) -- (3.4679, 5.7148) -- (3.4442, 5.7209) 
  -- (3.4315, 5.711) -- (3.4279, 5.7041) -- (3.4283, 5.7033) -- (3.4279, 5.7007)
   -- (3.4176, 5.694) -- (3.4075, 5.6873) -- (3.3996, 5.6828) -- (3.3752, 
  5.6921) -- (3.3691, 5.6958) -- (3.3591, 5.7049) -- (3.3559, 5.71) -- (3.3513, 
  5.716) -- (3.3401, 5.7294) -- (3.3343, 5.7349) -- (3.325, 5.7423) -- (3.3228, 
  5.7436) -- (3.3211, 5.7444) -- (3.2848, 5.7442) -- (3.2751, 5.7432) -- 
  (3.2689, 5.741) -- (3.2628, 5.7372) -- (3.2791, 5.8115) -- (3.2847, 5.8232) --
   (3.2939, 5.8286) -- (3.2986, 5.8283) -- (3.3045, 5.8258) -- (3.3172, 5.823) 
  -- (3.3523, 5.8497) -- (3.3551, 5.8526) -- (3.3643, 5.8627) -- (3.384, 5.8859)
   -- (3.4441, 5.9656) -- (3.4478, 5.9879) -- (3.4607, 6.0057) -- (3.4542, 
  6.0404) -- (3.442, 6.0537) -- (3.4338, 6.0577) -- (3.4234, 6.0598) -- (3.4219,
   6.0594) -- (3.4199, 6.0583) -- (3.4196, 6.0566) -- (3.4159, 6.0555) -- 
  (3.4044, 6.055) -- (3.4024, 6.0559) -- (3.3884, 6.0644) -- (3.3535, 6.0983) --
   (3.3573, 6.1269) -- (3.359, 6.1407) -- (3.3584, 6.1454) -- (3.3553, 6.1536) 
  -- (3.3535, 6.1571) -- (3.3444, 6.1798) -- (3.3442, 6.1964) -- (3.3573, 
  6.2172) -- (3.4052, 6.2509) -- (3.4292, 6.2491) -- (3.4379, 6.2687) -- 
  cycle(4.7223, 7.0099) -- (4.7169, 7.0066) -- (4.7007, 6.9934) -- (4.6889, 
  6.9798) -- (4.6869, 6.9766) -- (4.6595, 6.9083) -- (4.6588, 6.9064) -- (4.658,
   6.8984) -- (4.6592, 6.8982) -- (4.6636, 6.8996) -- (4.6657, 6.9014) -- 
  (4.6931, 6.9277) -- (4.6991, 6.9432) -- (4.6976, 6.9488) -- (4.6952, 6.9569) 
  -- (4.6949, 6.9582) -- (4.6954, 6.9593) -- (4.7105, 6.9808) -- (4.7201, 
  6.9835) -- (4.7211, 6.9832) -- (4.7229, 6.9792) -- (4.7243, 6.9681) -- 
  (4.7241, 6.9653) -- (4.7231, 6.9598) -- (4.7201, 6.9506) -- (4.7133, 6.9385) 
  -- (4.7108, 6.935) -- (4.7085, 6.9298) -- (4.7052, 6.9111) -- (4.7056, 6.9016)
   -- (4.7313, 6.9138) -- (4.7503, 6.9247) -- (4.813, 6.8989) -- (4.8144, 
  6.8985) -- (4.8155, 6.8986) -- (4.8327, 6.9008) -- (4.8498, 6.9077) -- 
  (4.8508, 6.8601) -- (4.8496, 6.8343) -- (4.8482, 6.8335) -- (4.8448, 6.8333) 
  -- (4.8293, 6.833) -- (4.8048, 6.8408) -- (4.7984, 6.8439) -- (4.7618, 6.8653)
   -- (4.7597, 6.8677) -- (4.7599, 6.8707) -- (4.7629, 6.8994) -- (4.7342, 
  6.8983) -- (4.7037, 6.8971) -- (4.6934, 6.8947) -- (4.6632, 6.8861) -- (4.655,
   6.8813) -- (4.6605, 6.8569) -- (4.6647, 6.8489) -- (4.6832, 6.8499) -- 
  (4.6977, 6.8486) -- (4.7207, 6.8277) -- (4.7128, 6.8127) -- (4.7107, 6.8114) 
  -- (4.6901, 6.8042) -- (4.6795, 6.8023) -- (4.6678, 6.8) -- (4.6572, 6.7763) 
  -- (4.6573, 6.7713) -- (4.6577, 6.77) -- (4.6609, 6.7683) -- (4.7096, 6.7576) 
  -- (4.7157, 6.7563) -- (4.7025, 6.7482) -- (4.648, 6.7218) -- (4.6466, 6.7198)
   -- (4.6454, 6.7161) -- (4.6413, 6.6936) -- (4.6417, 6.684) -- (4.6571, 
  6.6384) -- (4.6582, 6.6362) -- (4.8004, 6.5885) -- (4.8121, 6.5973) -- 
  (4.8145, 6.6003) -- (4.8158, 6.6028) -- (4.816, 6.6039) -- (4.8154, 6.6094) --
   (4.8116, 6.618) -- (4.8038, 6.6271) -- (4.8029, 6.6275) -- (4.7977, 6.6275) 
  -- (4.7963, 6.6256) -- (4.7955, 6.6241) -- (4.796, 6.6219) -- (4.7956, 6.6207)
   -- (4.7897, 6.6156) -- (4.7824, 6.6119) -- (4.779, 6.6117) -- (4.7757, 
  6.6156) -- (4.7754, 6.6266) -- (4.8138, 6.6673) -- (4.844, 6.6963) -- (4.8536,
   6.7034) -- (4.8566, 6.7046) -- (4.9488, 6.7158) -- (4.9526, 6.7152) -- 
  (4.9654, 6.6941) -- (4.9657, 6.6931) -- (4.9659, 6.6888) -- (4.9641, 6.6848) 
  -- (4.9635, 6.6836) -- (4.9561, 6.6781) -- (4.9555, 6.6777) -- (4.9469, 
  6.6733) -- (4.9468, 6.6591) -- (4.9664, 6.6643) -- (4.9746, 6.6805) -- 
  (4.9857, 6.7034) -- (4.9793, 6.7187) -- (4.9631, 6.745) -- (4.9515, 6.7545) --
   (4.9398, 6.7628) -- (4.9387, 6.7615) -- (4.9367, 6.7604) -- (4.9308, 6.7598) 
  -- (4.9194, 6.7592) -- (4.9176, 6.7594) -- (4.915, 6.761) -- (4.9112, 6.7644) 
  -- (4.9019, 6.7754) -- (4.8918, 6.7945) -- (4.8895, 6.8005) -- (4.8865, 
  6.8107) -- (4.8855, 6.8241) -- (4.8865, 6.8293) -- (4.8883, 6.8324) -- 
  (4.8898, 6.8344) -- (4.8906, 6.8355) -- (4.9051, 6.8519) -- (4.9244, 6.8681) 
  -- (4.9296, 6.8707) -- (4.9315, 6.8711) -- (4.9371, 6.8751) -- (4.9422, 
  6.8858) -- (4.9436, 6.8955) -- (4.9414, 6.9149) -- (4.9377, 6.9196) -- 
  (4.9314, 6.9254) -- (4.9183, 6.934) -- (4.916, 6.9348) -- (4.9118, 6.9348) -- 
  (4.8956, 6.9332) -- (4.8917, 6.9328) -- (4.8708, 6.9285) -- (4.8679, 6.9272) 
  -- (4.8492, 6.9213) -- (4.8227, 6.9165) -- (4.8169, 6.9169) -- (4.8074, 
  6.9192) -- (4.8038, 6.9204) -- (4.7994, 6.9233) -- (4.7967, 6.9252) -- (4.789,
   6.9329) -- (4.7823, 6.9419) -- (4.7775, 6.9518) -- (4.7736, 6.9658) -- 
  (4.7724, 6.9737) -- (4.7728, 6.9766) -- (4.7759, 6.9862) -- (4.7791, 6.9905) 
  -- (4.7807, 6.9916) -- (4.7856, 6.9939) -- (4.7892, 6.9949) -- (4.7928, 
  6.9952) -- (4.7954, 6.9963) -- (4.7992, 6.9996) -- (4.8015, 7.004) -- (4.8047,
   7.0165) -- (4.8046, 7.0171) -- (4.8013, 7.0218) -- (4.799, 7.0242) -- 
  (4.7711, 7.0217) -- (4.7299, 7.01) -- cycle(4.6079, 6.8063) -- (4.6091, 
  6.8034) -- (4.61, 6.8034) -- (4.6116, 6.8058) -- (4.6133, 6.8122) -- (4.6233, 
  6.868) -- (4.6306, 6.9037) -- (4.6515, 6.9143) -- (4.6242, 6.924) -- (4.6076, 
  6.8287) -- (4.6063, 6.8191) -- (4.6062, 6.8142) -- (4.6069, 6.8111) -- 
  cycle(4.6524, 6.7926) -- (4.6914, 6.8141) -- (4.6962, 6.8179) -- (4.6972, 
  6.8192) -- (4.6946, 6.8277) -- (4.6923, 6.8306) -- (4.6803, 6.8361) -- 
  (4.6781, 6.8371) -- (4.6718, 6.8378) -- (4.6657, 6.8363) -- (4.6597, 6.8322) 
  -- (4.6519, 6.8219) -- (4.6505, 6.8187) -- (4.6407, 6.7959) -- (4.6405, 
  6.7898) -- (4.641, 6.7883) -- (4.6423, 6.7877) -- cycle(4.5647, 6.7838) -- 
  (4.5633, 6.7842) -- (4.5614, 6.7841) -- (4.5597, 6.7835) -- (4.5576, 6.7822) 
  -- (4.5565, 6.781) -- (4.5566, 6.7798) -- (4.5591, 6.7773) -- (4.5614, 6.7753)
   -- (4.5675, 6.7757) -- (4.5971, 6.7862) -- (4.6032, 6.7915) -- (4.6065, 
  6.795) -- (4.6019, 6.8) -- (4.5998, 6.7995) -- (4.5738, 6.7897) -- (4.5705, 
  6.7859) -- (4.5657, 6.7802) -- cycle(5.0345, 6.5455) -- (5.0298, 6.5505) -- 
  (5.0276, 6.5521) -- (5.0229, 6.5532) -- (5.0199, 6.553) -- (5.0138, 6.5513) --
   (5.0071, 6.5477) -- (5.0027, 6.5442) -- (5.0014, 6.5423) -- (4.9998, 6.539) 
  -- (5.0002, 6.5267) -- (5.0034, 6.517) -- (5.0054, 6.5145) -- (5.0236, 6.4942)
   -- (5.0255, 6.4929) -- (5.0302, 6.4469) -- (5.0642, 6.394) -- (5.0652, 
  6.3926) -- (5.0668, 6.3916) -- (5.0686, 6.3913) -- (5.1034, 6.4082) -- 
  (5.1039, 6.4085) -- (5.1075, 6.4186) -- (5.1074, 6.4196) -- (5.0959, 6.4368) 
  -- (5.0925, 6.4519) -- (5.0924, 6.4524) -- (5.093, 6.4534) -- (5.1249, 6.4506)
   -- (5.1391, 6.4383) -- (5.1638, 6.4108) -- (5.1661, 6.4078) -- (5.1679, 
  6.4043) -- (5.1717, 6.3941) -- (5.171, 6.3469) -- (5.1378, 6.3527) -- (5.131, 
  6.3668) -- (5.1256, 6.3849) -- (5.1242, 6.3862) -- (5.12, 6.388) -- (5.1132, 
  6.3889) -- (5.0959, 6.3843) -- (5.0946, 6.3839) -- (5.0934, 6.383) -- (5.0901,
   6.376) -- (5.0885, 6.3635) -- (5.0888, 6.3613) -- (5.0905, 6.3575) -- 
  (5.0932, 6.3552) -- (5.1045, 6.3415) -- (5.1078, 6.3336) -- (5.115, 6.3125) --
   (5.1153, 6.308) -- (5.1114, 6.2978) -- (5.1084, 6.2902) -- (5.1063, 6.2884) 
  -- (5.0965, 6.2824) -- (5.085, 6.2762) -- (5.0732, 6.2703) -- (5.0694, 6.2695)
   -- (5.0657, 6.2697) -- (5.0633, 6.2701) -- (5.0564, 6.2559) -- (5.0632, 
  6.2494) -- (5.0741, 6.2446) -- (5.0869, 6.2414) -- (5.0965, 6.2424) -- 
  (5.1204, 6.2465) -- (5.1456, 6.2534) -- (5.1485, 6.2547) -- (5.1867, 6.2743) 
  -- (5.2554, 6.2814) -- (5.2584, 6.2813) -- (5.2603, 6.2809) -- (5.2727, 
  6.3362) -- (5.2369, 6.3656) -- (5.1976, 6.4017) -- (5.1505, 6.446) -- (5.1404,
   6.4519) -- (5.1269, 6.4589) -- (5.1235, 6.4604) -- (5.1203, 6.4612) -- 
  (5.111, 6.4628) -- (5.1017, 6.4661) -- (5.0849, 6.4756) -- (5.08, 6.4785) -- 
  (5.0699, 6.487) -- (5.0598, 6.4973) -- (5.0526, 6.5055) -- (5.0504, 6.5088) --
   cycle(3.7492, 6.325) -- (3.7557, 6.3452) -- (3.7577, 6.3562) -- (3.7572, 
  6.361) -- (3.7565, 6.3628) -- (3.754, 6.364) -- (3.7454, 6.366) -- (3.7357, 
  6.3645) -- (3.7326, 6.3637) -- (3.728, 6.3618) -- (3.7065, 6.3505) -- (3.693, 
  6.3401) -- (3.6906, 6.3347) -- (3.6911, 6.3252) -- (3.6924, 6.3192) -- 
  (3.6946, 6.3143) -- (3.6967, 6.3115) -- (3.7, 6.3088) -- (3.7034, 6.3081) -- 
  (3.7358, 6.3021) -- cycle;


  \node[text=black,line width=0.0092cm,anchor=center] (text12) at (4.2954, 
  6.1405){\resizebox{\ifdim\width>1.75em 1.75em\else\width\fi}{!}{#2}};
}

\newcommand{\drawniedersachsen}[2]{%
  %Niedersachsen
  \path[draw=black,fill=#1,line join=round,line width=0.0046cm] (2.0982, 
  5.9723) -- (2.0898, 5.9923) -- (2.079, 6.02) -- (2.0789, 6.0386) -- (2.0789, 
  6.0502) -- (2.0795, 6.0562) -- (2.1141, 6.1604) -- (2.1189, 6.1731) -- 
  (2.1283, 6.1971) -- (2.1484, 6.2234) -- (2.1561, 6.2307) -- (2.1727, 6.2356) 
  -- (2.1845, 6.2386) -- (2.1946, 6.2392) -- (2.1944, 6.2342) -- (2.1949, 
  6.2314) -- (2.1963, 6.2286) -- (2.2097, 6.2123) -- (2.2356, 6.1933) -- 
  (2.2445, 6.1871) -- (2.2504, 6.1844) -- (2.2571, 6.1825) -- (2.2714, 6.1807) 
  -- (2.3179, 6.1769) -- (2.3329, 6.1799) -- (2.3356, 6.1809) -- (2.3362, 
  6.1816) -- (2.3363, 6.1829) -- (2.3389, 6.1839) -- (2.358, 6.1879) -- (2.3628,
   6.1877) -- (2.3825, 6.1829) -- (2.3781, 6.1887) -- (2.371, 6.2068) -- 
  (2.3673, 6.218) -- (2.3747, 6.2214) -- (2.3829, 6.2242) -- (2.3906, 6.2256) --
   (2.3999, 6.2259) -- (2.4292, 6.2239) -- (2.4625, 6.2234) -- (2.4815, 6.2237) 
  -- (2.4989, 6.2214) -- (2.5314, 6.2073) -- (2.5479, 6.1954) -- (2.5695, 
  6.1642) -- (2.5743, 6.1554) -- (2.5911, 6.1194) -- (2.5981, 6.1024) -- 
  (2.6049, 6.0907) -- (2.6103, 6.0831) -- (2.6181, 6.0759) -- (2.6396, 6.0599) 
  -- (2.6486, 6.0499) -- (2.6518, 6.0444) -- (2.6589, 6.0272) -- (2.6628, 
  6.0178) -- (2.6637, 6.0135) -- (2.6645, 5.9989) -- (2.6677, 5.9873) -- 
  (2.6714, 5.9793) -- (2.6754, 5.9736) -- (2.6962, 5.957) -- (2.7453, 5.9249) --
   (2.7605, 5.9198) -- (2.7758, 5.9182) -- (2.8005, 5.9148) -- (2.8005, 5.8889) 
  -- (2.7977, 5.8685) -- (2.8053, 5.8607) -- (2.8141, 5.8402) -- (2.8148, 
  5.8348) -- (2.8461, 5.8026) -- (2.8484, 5.8037) -- (2.856, 5.8094) -- (2.8678,
   5.8225) -- (2.8701, 5.8239) -- (2.8731, 5.8219) -- (2.8799, 5.8134) -- 
  (2.8993, 5.795) -- (2.9138, 5.7829) -- (2.9177, 5.7832) -- (2.9436, 5.8022) --
   (2.9452, 5.8105) -- (2.9507, 5.8225) -- (2.9558, 5.8292) -- (2.972, 5.8196) 
  -- (2.9767, 5.8143) -- (2.9801, 5.8096) -- (2.9962, 5.7912) -- (3.0196, 
  5.7692) -- (3.0236, 5.7665) -- (3.0271, 5.7657) -- (3.0566, 5.7645) -- (3.06, 
  5.7649) -- (3.0617, 5.7661) -- (3.0639, 5.7684) -- (3.0694, 5.7821) -- 
  (3.0711, 5.7846) -- (3.0734, 5.787) -- (3.0768, 5.7888) -- (3.0972, 5.7999) --
   (3.0989, 5.8008) -- (3.1003, 5.8009) -- (3.115, 5.7929) -- (3.1691, 5.7682) 
  -- (3.1912, 5.7567) -- (3.1988, 5.7514) -- (3.2072, 5.7464) -- (3.2125, 
  5.7443) -- (3.2504, 5.7332) -- (3.2549, 5.7328) -- (3.2578, 5.7341) -- 
  (3.2618, 5.7365) -- (3.2626, 5.7369) -- (3.2621, 5.7373) -- (3.2681, 5.7411) 
  -- (3.2744, 5.7433) -- (3.284, 5.7443) -- (3.3203, 5.7445) -- (3.3222, 5.7437)
   -- (3.3243, 5.7423) -- (3.3336, 5.735) -- (3.3393, 5.7296) -- (3.3506, 
  5.7161) -- (3.3552, 5.7101) -- (3.3584, 5.705) -- (3.3684, 5.6959) -- (3.3745,
   5.6922) -- (3.3989, 5.6829) -- (3.4068, 5.6874) -- (3.4169, 5.694) -- 
  (3.4272, 5.7008) -- (3.4276, 5.7033) -- (3.4272, 5.7042) -- (3.4308, 5.7111) 
  -- (3.4435, 5.721) -- (3.4672, 5.7149) -- (3.4836, 5.7105) -- (3.506, 5.6595) 
  -- (3.527, 5.6189) -- (3.5375, 5.6046) -- (3.5436, 5.5984) -- (3.5699, 5.5789)
   -- (3.5806, 5.5763) -- (3.5888, 5.5744) -- (3.594, 5.5739) -- (3.6012, 
  5.5699) -- (3.6025, 5.5678) -- (3.6029, 5.5645) -- (3.6002, 5.5559) -- 
  (3.5996, 5.5541) -- (3.5946, 5.549) -- (3.5896, 5.5472) -- (3.6222, 5.5341) --
   (3.6279, 5.5313) -- (3.6331, 5.5276) -- (3.6417, 5.5183) -- (3.6437, 5.5157) 
  -- (3.6453, 5.5125) -- (3.6456, 5.4992) -- (3.6459, 5.4961) -- (3.6466, 
  5.4943) -- (3.6571, 5.4815) -- (3.6845, 5.4534) -- (3.6871, 5.452) -- (3.6892,
   5.4516) -- (3.6934, 5.4517) -- (3.709, 5.46) -- (3.7184, 5.4673) -- (3.721, 
  5.4688) -- (3.7245, 5.4697) -- (3.7432, 5.475) -- (3.7469, 5.4753) -- (3.7516,
   5.475) -- (3.7554, 5.4731) -- (3.7588, 5.4705) -- (3.7653, 5.4632) -- 
  (3.7759, 5.452) -- (3.7811, 5.4483) -- (3.7869, 5.4466) -- (3.8271, 5.4369) --
   (3.8327, 5.4367) -- (3.8335, 5.4354) -- (3.8334, 5.4346) -- (3.8107, 5.4013) 
  -- (3.8104, 5.4008) -- (3.8036, 5.4014) -- (3.7976, 5.4047) -- (3.7941, 
  5.4096) -- (3.7934, 5.41) -- (3.7857, 5.4085) -- (3.7826, 5.402) -- (3.7806, 
  5.3927) -- (3.7753, 5.3633) -- (3.7826, 5.3462) -- (3.7747, 5.3463) -- 
  (3.7634, 5.3436) -- (3.7532, 5.3378) -- (3.7371, 5.3222) -- (3.7289, 5.3116) 
  -- (3.7275, 5.3101) -- (3.6753, 5.2846) -- (3.665, 5.2818) -- (3.636, 5.2847) 
  -- (3.6331, 5.2853) -- (3.6322, 5.2859) -- (3.6318, 5.2871) -- (3.6318, 
  5.2896) -- (3.6306, 5.2925) -- (3.6277, 5.2955) -- (3.6209, 5.3026) -- 
  (3.5852, 5.3096) -- (3.5002, 5.3133) -- (3.4971, 5.313) -- (3.4947, 5.312) -- 
  (3.4917, 5.3085) -- (3.4902, 5.3025) -- (3.4898, 5.2956) -- (3.4907, 5.2915) 
  -- (3.4905, 5.29) -- (3.4886, 5.2861) -- (3.4625, 5.2577) -- (3.4069, 5.2562) 
  -- (3.3823, 5.2512) -- (3.3634, 5.2459) -- (3.3588, 5.2258) -- (3.3586, 5.224)
   -- (3.3606, 5.2031) -- (3.3593, 5.1976) -- (3.3585, 5.1924) -- (3.3826, 
  5.1264) -- (3.4066, 5.1088) -- (3.4408, 5.06) -- (3.4664, 5.0103) -- (3.4663, 
  5.0099) -- (3.4647, 4.9898) -- (3.4678, 4.9775) -- (3.4709, 4.9705) -- 
  (3.4778, 4.9587) -- (3.4899, 4.9433) -- (3.5066, 4.9239) -- (3.4907, 4.9245) 
  -- (3.4701, 4.9135) -- (3.4644, 4.8992) -- (3.4722, 4.8824) -- (3.5137, 
  4.8294) -- (3.5256, 4.822) -- (3.5285, 4.8216) -- (3.5306, 4.8206) -- (3.5377,
   4.8145) -- (3.5433, 4.809) -- (3.5438, 4.8054) -- (3.5434, 4.7925) -- (3.543,
   4.7908) -- (3.5355, 4.7849) -- (3.5252, 4.7801) -- (3.5236, 4.7796) -- 
  (3.5184, 4.7809) -- (3.5071, 4.7652) -- (3.511, 4.7276) -- (3.5115, 4.7267) --
   (3.5246, 4.7123) -- (3.5371, 4.6997) -- (3.5502, 4.6814) -- (3.5542, 4.6726) 
  -- (3.5544, 4.6718) -- (3.5541, 4.6712) -- (3.5487, 4.6641) -- (3.5388, 4.662)
   -- (3.5324, 4.6604) -- (3.5233, 4.6561) -- (3.5203, 4.6542) -- (3.5176, 
  4.6492) -- (3.5138, 4.6396) -- (3.5127, 4.6327) -- (3.5133, 4.6289) -- 
  (3.5146, 4.626) -- (3.5162, 4.6243) -- (3.5197, 4.6216) -- (3.5258, 4.6179) --
   (3.5302, 4.6178) -- (3.5382, 4.6189) -- (3.5398, 4.6187) -- (3.5403, 4.618) 
  -- (3.5413, 4.6147) -- (3.5408, 4.6013) -- (3.5393, 4.5988) -- (3.5284, 
  4.5825) -- (3.5112, 4.5685) -- (3.4926, 4.5552) -- (3.4891, 4.5375) -- 
  (3.4863, 4.5092) -- (3.4852, 4.5101) -- (3.4696, 4.5117) -- (3.4381, 4.5107) 
  -- (3.4333, 4.5098) -- (3.4248, 4.5037) -- (3.4215, 4.5024) -- (3.4088, 
  4.4998) -- (3.392, 4.4992) -- (3.3834, 4.5012) -- (3.381, 4.5012) -- (3.3349, 
  4.5006) -- (3.3255, 4.4988) -- (3.3079, 4.4935) -- (3.3069, 4.4848) -- 
  (3.3084, 4.4807) -- (3.308, 4.4782) -- (3.3075, 4.4768) -- (3.2913, 4.4608) --
   (3.2527, 4.4568) -- (3.2681, 4.4349) -- (3.3024, 4.4092) -- (3.2907, 4.4054) 
  -- (3.2838, 4.3953) -- (3.2836, 4.3936) -- (3.2838, 4.3787) -- (3.2919, 
  4.3757) -- (3.2993, 4.3727) -- (3.3041, 4.3668) -- (3.3051, 4.3641) -- 
  (3.3058, 4.3581) -- (3.3058, 4.3513) -- (3.3048, 4.3481) -- (3.2954, 4.3335) 
  -- (3.2938, 4.3323) -- (3.2903, 4.3321) -- (3.2867, 4.3307) -- (3.2671, 
  4.3144) -- (3.2638, 4.3101) -- (3.2624, 4.3055) -- (3.2625, 4.2738) -- (3.265,
   4.2479) -- (3.2714, 4.2344) -- (3.2808, 4.2292) -- (3.3085, 4.1883) -- 
  (3.3171, 4.1679) -- (3.3151, 4.1631) -- (3.3159, 4.1552) -- (3.318, 4.1435) --
   (3.3192, 4.1399) -- (3.3308, 4.1213) -- (3.3336, 4.1175) -- (3.3341, 4.117) 
  -- (3.3375, 4.1159) -- (3.3365, 4.1143) -- (3.3259, 4.1128) -- (3.3234, 
  4.1132) -- (3.3043, 4.0625) -- (3.2958, 4.0463) -- (3.2539, 4.0357) -- 
  (3.2316, 4.0328) -- (3.2262, 4.0357) -- (3.2151, 4.0522) -- (3.2116, 4.0545) 
  -- (3.2048, 4.0587) -- (3.201, 4.0605) -- (3.1869, 4.0665) -- (3.1792, 4.069) 
  -- (3.1774, 4.0688) -- (3.1557, 4.0631) -- (3.1493, 4.0524) -- (3.141, 4.0347)
   -- (3.145, 4.0328) -- (3.1454, 4.032) -- (3.1433, 4.0245) -- (3.1419, 4.0216)
   -- (3.1281, 4.0009) -- (3.1268, 3.9999) -- (3.103, 3.9808) -- (3.0294, 
  3.9309) -- (2.9807, 3.9211) -- (2.9196, 3.8856) -- (2.8914, 3.8666) -- 
  (2.8869, 3.8648) -- (2.8849, 3.8643) -- (2.8831, 3.8655) -- (2.8864, 3.8676) 
  -- (2.8868, 3.8689) -- (2.8864, 3.8797) -- (2.8833, 3.8851) -- (2.8718, 
  3.9037) -- (2.8692, 3.9053) -- (2.8408, 3.9015) -- (2.8115, 3.8973) -- 
  (2.8098, 3.8969) -- (2.8058, 3.8944) -- (2.7655, 3.8605) -- (2.7556, 3.8467) 
  -- (2.7718, 3.8212) -- (2.7876, 3.8288) -- (2.7923, 3.829) -- (2.793, 3.8276) 
  -- (2.7937, 3.8151) -- (2.7928, 3.8107) -- (2.7903, 3.7999) -- (2.769, 3.7889)
   -- (2.7678, 3.791) -- (2.737, 3.8063) -- (2.6884, 3.8265) -- (2.6726, 3.8316)
   -- (2.6665, 3.8421) -- (2.6619, 3.8493) -- (2.6625, 3.8522) -- (2.6672, 
  3.8705) -- (2.669, 3.8745) -- (2.6789, 3.89) -- (2.6836, 3.8909) -- (2.6984, 
  3.8933) -- (2.7101, 3.8929) -- (2.717, 3.9033) -- (2.7179, 3.9053) -- (2.7109,
   3.9447) -- (2.7077, 3.9593) -- (2.7027, 3.9648) -- (2.6838, 3.9971) -- 
  (2.6833, 3.9986) -- (2.6834, 3.9994) -- (2.6847, 4.0023) -- (2.707, 4.0273) --
   (2.721, 4.0297) -- (2.7227, 4.033) -- (2.7281, 4.0403) -- (2.7348, 4.0458) --
   (2.7408, 4.044) -- (2.7418, 4.044) -- (2.7429, 4.0456) -- (2.7459, 4.0533) --
   (2.7442, 4.057) -- (2.7425, 4.0587) -- (2.735, 4.0635) -- (2.7286, 4.0689) --
   (2.7156, 4.0831) -- (2.7149, 4.0841) -- (2.6837, 4.1001) -- (2.6748, 4.0999) 
  -- (2.6454, 4.1004) -- (2.6391, 4.1013) -- (2.6324, 4.1053) -- (2.6296, 
  4.1091) -- (2.6292, 4.1121) -- (2.6311, 4.1197) -- (2.6334, 4.1246) -- 
  (2.6331, 4.1281) -- (2.6317, 4.1301) -- (2.6292, 4.1306) -- (2.5992, 4.124) --
   (2.5979, 4.1248) -- (2.5958, 4.1247) -- (2.5904, 4.123) -- (2.5899, 4.1228) 
  -- (2.5857, 4.1206) -- (2.5844, 4.1204) -- (2.5697, 4.12) -- (2.5677, 4.12) --
   (2.5652, 4.1207) -- (2.561, 4.1238) -- (2.5689, 4.1901) -- (2.5855, 4.2412) 
  -- (2.5969, 4.2691) -- (2.5971, 4.3003) -- (2.5808, 4.3166) -- (2.5757, 
  4.3196) -- (2.5621, 4.3247) -- (2.558, 4.3258) -- (2.5561, 4.3259) -- (2.5504,
   4.3239) -- (2.5473, 4.322) -- (2.5381, 4.3162) -- (2.5402, 4.3176) -- 
  (2.5449, 4.3398) -- (2.5462, 4.3491) -- (2.5466, 4.3546) -- (2.5435, 4.3665) 
  -- (2.5392, 4.3741) -- (2.5386, 4.3749) -- (2.5289, 4.3801) -- (2.5258, 
  4.3806) -- (2.5252, 4.3803) -- (2.5159, 4.3789) -- (2.5079, 4.3833) -- (2.456,
   4.4208) -- (2.4503, 4.4349) -- (2.4477, 4.4918) -- (2.448, 4.5083) -- 
  (2.4483, 4.5105) -- (2.4526, 4.5166) -- (2.4572, 4.5191) -- (2.4585, 4.5201) 
  -- (2.4501, 4.5347) -- (2.4377, 4.5459) -- (2.4403, 4.5663) -- (2.4372, 
  4.5752) -- (2.4324, 4.5806) -- (2.4303, 4.5815) -- (2.4217, 4.5845) -- (2.39, 
  4.5906) -- (2.3583, 4.615) -- (2.3408, 4.6382) -- (2.3618, 4.6272) -- (2.3752,
   4.627) -- (2.3934, 4.6721) -- (2.394, 4.6742) -- (2.3706, 4.6897) -- (2.3522,
   4.6971) -- (2.3379, 4.7011) -- (2.3352, 4.7021) -- (2.3337, 4.7033) -- 
  (2.3291, 4.7176) -- (2.3498, 4.7614) -- (2.3559, 4.7656) -- (2.3653, 4.7717) 
  -- (2.3742, 4.7761) -- (2.3852, 4.7777) -- (2.3924, 4.7819) -- (2.3993, 
  4.7882) -- (2.4003, 4.7894) -- (2.4077, 4.8036) -- (2.4213, 4.8315) -- 
  (2.4231, 4.836) -- (2.4239, 4.8424) -- (2.4155, 4.8638) -- (2.432, 4.9013) -- 
  (2.4315, 4.9033) -- (2.4296, 4.9095) -- (2.4048, 4.9207) -- (2.3833, 4.9258) 
  -- (2.3688, 4.9069) -- (2.3619, 4.894) -- (2.3424, 4.8685) -- (2.3158, 4.8339)
   -- (2.3141, 4.8333) -- (2.2672, 4.8228) -- (2.2055, 4.8222) -- (2.1886, 
  4.8295) -- (2.1837, 4.8331) -- (2.1821, 4.8466) -- (2.1798, 4.9165) -- 
  (2.1823, 4.9288) -- (2.1819, 4.9316) -- (2.1798, 4.9368) -- (2.1732, 4.9432) 
  -- (2.1537, 4.9585) -- (2.0995, 4.9303) -- (2.0712, 4.9442) -- (2.0427, 
  4.9223) -- (2.0415, 4.9094) -- (2.0397, 4.8974) -- (2.0387, 4.8943) -- 
  (2.0246, 4.8822) -- (2.0156, 4.881) -- (1.9848, 4.8786) -- (1.9803, 4.8833) --
   (1.9806, 4.885) -- (1.9777, 4.8879) -- (1.956, 4.8936) -- (1.952, 4.8945) -- 
  (1.9482, 4.892) -- (1.9528, 4.8821) -- (1.9588, 4.8528) -- (2.0066, 4.8168) --
   (2.0265, 4.8014) -- (2.0409, 4.759) -- (2.0395, 4.7429) -- (2.0417, 4.7124) 
  -- (2.0373, 4.7034) -- (2.027, 4.6626) -- (2.0278, 4.6605) -- (2.0328, 4.6512)
   -- (2.04, 4.6449) -- (2.0426, 4.6438) -- (2.0459, 4.6436) -- (2.057, 4.64) --
   (2.0683, 4.6341) -- (2.0625, 4.6263) -- (2.0581, 4.6209) -- (2.0488, 4.6112) 
  -- (2.0467, 4.6094) -- (2.0419, 4.6071) -- (2.0385, 4.6065) -- (2.0326, 
  4.6069) -- (2.031, 4.6064) -- (2.0108, 4.5929) -- (2.0068, 4.5892) -- (2.0037,
   4.5847) -- (2.0032, 4.5816) -- (2.0043, 4.5764) -- (2.0065, 4.5698) -- 
  (1.9789, 4.5727) -- (1.9565, 4.5809) -- (1.9297, 4.5867) -- (1.9242, 4.5874) 
  -- (1.9115, 4.5779) -- (1.8959, 4.563) -- (1.859, 4.5364) -- (1.8443, 4.533) 
  -- (1.8236, 4.5201) -- (1.8208, 4.5258) -- (1.8107, 4.5284) -- (1.8012, 
  4.5298) -- (1.7872, 4.5308) -- (1.7856, 4.5307) -- (1.7648, 4.5148) -- 
  (1.7612, 4.51) -- (1.7554, 4.5014) -- (1.7449, 4.5054) -- (1.7284, 4.5124) -- 
  (1.7189, 4.5174) -- (1.7167, 4.5266) -- (1.7038, 4.5476) -- (1.7013, 4.5476) 
  -- (1.7017, 4.5489) -- (1.7103, 4.5554) -- (1.7441, 4.5666) -- (1.7734, 
  4.5756) -- (1.7716, 4.6136) -- (1.7536, 4.6304) -- (1.7323, 4.6345) -- 
  (1.7317, 4.6348) -- (1.7204, 4.6429) -- (1.7178, 4.6473) -- (1.715, 4.6564) --
   (1.7135, 4.6635) -- (1.7136, 4.6646) -- (1.7386, 4.7179) -- (1.7483, 4.7244) 
  -- (1.7358, 4.7372) -- (1.7343, 4.739) -- (1.7336, 4.7444) -- (1.734, 4.7487) 
  -- (1.7349, 4.7522) -- (1.7542, 4.7728) -- (1.7478, 4.796) -- (1.739, 4.8114) 
  -- (1.7135, 4.827) -- (1.6991, 4.8236) -- (1.6967, 4.822) -- (1.6845, 4.8166) 
  -- (1.6642, 4.8197) -- (1.6177, 4.848) -- (1.6034, 4.8888) -- (1.5986, 4.8991)
   -- (1.5967, 4.9015) -- (1.5953, 4.903) -- (1.5531, 4.9196) -- (1.5509, 
  4.9205) -- (1.5438, 4.9116) -- (1.5383, 4.9021) -- (1.5348, 4.8947) -- 
  (1.5305, 4.8808) -- (1.5367, 4.8766) -- (1.5494, 4.8698) -- (1.5482, 4.8636) 
  -- (1.5317, 4.8344) -- (1.5105, 4.8238) -- (1.5075, 4.8236) -- (1.5051, 
  4.8233) -- (1.4895, 4.8144) -- (1.4509, 4.7907) -- (1.4168, 4.7655) -- 
  (1.4112, 4.7607) -- (1.4081, 4.7544) -- (1.4056, 4.7498) -- (1.3961, 4.7448) 
  -- (1.3867, 4.7438) -- (1.3793, 4.743) -- (1.3756, 4.7377) -- (1.3675, 4.7282)
   -- (1.3215, 4.7316) -- (1.3096, 4.7324) -- (1.2859, 4.735) -- (1.2663, 
  4.7306) -- (1.2608, 4.7255) -- (1.2587, 4.723) -- (1.2524, 4.7139) -- (1.2331,
   4.7132) -- (1.2203, 4.7277) -- (1.2147, 4.7427) -- (1.2134, 4.7498) -- 
  (1.212, 4.7576) -- (1.2125, 4.762) -- (1.2252, 4.7907) -- (1.2312, 4.8042) -- 
  (1.241, 4.8173) -- (1.2419, 4.8369) -- (1.2374, 4.8541) -- (1.2001, 4.9258) --
   (1.1993, 4.9272) -- (1.1969, 4.9292) -- (1.1912, 4.9256) -- (1.1861, 4.916) 
  -- (1.1814, 4.9059) -- (1.1722, 4.8997) -- (1.1693, 4.8986) -- (1.0741, 
  4.9255) -- (1.0619, 4.9301) -- (1.0311, 4.9525) -- (1.0315, 4.9553) -- 
  (1.0348, 4.9732) -- (1.037, 4.9844) -- (1.0442, 5.0108) -- (1.048, 5.0484) -- 
  (1.044, 5.0853) -- (1.0634, 5.1012) -- (1.0731, 5.1043) -- (1.0877, 5.1067) --
   (1.0917, 5.1066) -- (1.1517, 5.103) -- (1.2002, 5.0949) -- (1.2119, 5.0909) 
  -- (1.2136, 5.0887) -- (1.2138, 5.0881) -- (1.2171, 5.0861) -- (1.2332, 
  5.0819) -- (1.2338, 5.0821) -- (1.2361, 5.0843) -- (1.2416, 5.0919) -- 
  (1.2492, 5.1579) -- (1.2528, 5.1956) -- (1.2579, 5.2485) -- (1.2679, 5.2853) 
  -- (1.2872, 5.3115) -- (1.325, 5.3688) -- (1.3377, 5.4074) -- (1.3429, 5.4237)
   -- (1.3449, 5.4342) -- (1.3401, 5.4978) -- (1.343, 5.5295) -- (1.3323, 
  5.5381) -- (1.3311, 5.5457) -- (1.3307, 5.5535) -- (1.3386, 5.5752) -- 
  (1.3459, 5.5882) -- (1.354, 5.6071) -- (1.3575, 5.6573) -- (1.3677, 5.7008) --
   (1.3739, 5.7202) -- (1.3747, 5.7225) -- (1.3756, 5.7234) -- (1.3805, 5.7251) 
  -- (1.4069, 5.7251) -- (1.4036, 5.7264) -- (1.3828, 5.7323) -- (1.3714, 5.735)
   -- (1.3635, 5.7362) -- (1.3351, 5.7387) -- (1.271, 5.7408) -- (1.2524, 
  5.7434) -- (1.2487, 5.7456) -- (1.2462, 5.7489) -- (1.2394, 5.7615) -- 
  (1.2376, 5.7662) -- (1.2384, 5.7688) -- (1.2474, 5.7843) -- (1.2486, 5.8082) 
  -- (1.263, 5.9252) -- (1.2938, 5.9112) -- (1.3028, 5.8985) -- (1.3204, 5.9162)
   -- (1.3301, 5.9266) -- (1.321, 5.943) -- (1.3148, 5.9467) -- (1.2996, 5.9568)
   -- (1.295, 5.9628) -- (1.2976, 5.971) -- (1.3397, 6.0147) -- (1.3508, 6.0257)
   -- (1.3757, 6.0439) -- (1.3858, 6.0509) -- (1.4117, 6.0601) -- (1.4241, 
  6.0629) -- (1.4353, 6.0642) -- (1.4505, 6.0656) -- (1.519, 6.0599) -- (1.5222,
   6.0588) -- (1.5278, 6.0556) -- (1.529, 6.0538) -- (1.5337, 6.0506) -- 
  (1.5403, 6.0473) -- (1.5446, 6.0472) -- (1.5607, 6.0502) -- (1.5969, 6.062) --
   (1.6355, 6.0729) -- (1.6392, 6.0736) -- (1.6447, 6.0744) -- (1.6488, 6.074) 
  -- (1.6489, 6.0725) -- (1.654, 6.0712) -- (1.6646, 6.0709) -- (1.6758, 6.072) 
  -- (1.7048, 6.0807) -- (1.7616, 6.0857) -- (1.7908, 6.0825) -- (1.817, 6.0784)
   -- (1.8259, 6.0738) -- (1.8282, 6.0711) -- (1.8278, 6.0667) -- (1.827, 
  6.0646) -- (1.8229, 6.0588) -- (1.822, 6.0521) -- (1.8252, 6.0312) -- (1.8272,
   6.0233) -- (1.8321, 6.0106) -- (1.8585, 6.0069) -- (1.8611, 6.0037) -- 
  (1.8659, 5.9965) -- (1.8711, 5.9863) -- (1.9019, 5.914) -- (1.9019, 5.9131) --
   (1.9014, 5.91) -- (1.9005, 5.9068) -- (1.8981, 5.8986) -- (1.8955, 5.8933) --
   (1.8872, 5.889) -- (1.879, 5.8859) -- (1.8593, 5.8787) -- (1.8556, 5.8778) --
   (1.8471, 5.878) -- (1.8381, 5.8851) -- (1.8387, 5.8662) -- (1.8397, 5.8577) 
  -- (1.8408, 5.8528) -- (1.8431, 5.8467) -- (1.8459, 5.8405) -- (1.8516, 5.833)
   -- (1.8589, 5.8251) -- (1.8625, 5.8265) -- (1.8701, 5.8316) -- (1.8718, 
  5.832) -- (1.88, 5.8323) -- (1.8864, 5.8304) -- (1.8956, 5.8187) -- (1.9067, 
  5.8026) -- (1.9126, 5.7933) -- (1.9131, 5.7923) -- (1.9129, 5.7901) -- 
  (1.9148, 5.7862) -- (1.9161, 5.7849) -- (1.922, 5.7823) -- (1.9319, 5.78) -- 
  (1.9391, 5.7791) -- (1.9425, 5.7795) -- (1.9482, 5.7824) -- (1.9529, 5.7859) 
  -- (1.9545, 5.7877) -- (1.9607, 5.7947) -- (1.966, 5.8021) -- (1.9702, 5.8097)
   -- (1.977, 5.8256) -- (1.98, 5.8326) -- (1.9811, 5.8399) -- (1.9833, 5.874) 
  -- (1.9826, 5.8883) -- (1.9819, 5.8903) -- (1.9797, 5.8949) -- (1.9783, 
  5.8965) -- (1.9743, 5.8988) -- (1.9599, 5.8932) -- (1.9549, 5.8917) -- 
  (1.9353, 5.8954) -- (1.9397, 5.9336) -- (1.9421, 5.9488) -- (1.9427, 5.9516) 
  -- (1.9466, 5.9598) -- (1.9496, 5.9641) -- (1.9581, 5.9726) -- (1.9668, 
  5.9794) -- (1.9788, 5.9838) -- (1.9898, 5.9828) -- (2.0026, 5.9743) -- 
  (2.0066, 5.9686) -- (2.0135, 5.9558) -- (2.0219, 5.9433) -- (2.0321, 5.9377) 
  -- (2.0663, 5.9223) -- (2.0682, 5.9219) -- (2.0719, 5.9223) -- (2.0761, 
  5.9225) -- (2.0891, 5.9202) -- (2.096, 5.9191) -- (2.1136, 5.9082) -- (2.1148,
   5.9069) -- (2.1156, 5.9045) -- (2.1171, 5.8957) -- (2.117, 5.895) -- (2.1152,
   5.8935) -- (2.112, 5.8911) -- (2.1151, 5.8868) -- (2.1276, 5.8701) -- 
  (2.1276, 5.8684) -- (2.1271, 5.8666) -- (2.1237, 5.8617) -- (2.1253, 5.8596) 
  -- (2.1282, 5.8578) -- (2.1449, 5.8559) -- (2.1466, 5.856) -- (2.1701, 5.8809)
   -- (2.171, 5.8856) -- (2.1627, 5.9212) -- (2.1726, 5.9668) -- (2.172, 5.9677)
   -- (2.1565, 5.9701) -- (2.1071, 5.9726) -- cycle(2.3515, 5.5155) -- (2.349, 
  5.5208) -- (2.3364, 5.5361) -- (2.3301, 5.5405) -- (2.3272, 5.5406) -- 
  (2.3167, 5.534) -- (2.2735, 5.5352) -- (2.239, 5.5528) -- (2.2002, 5.5718) -- 
  (2.1408, 5.5788) -- (2.0719, 5.616) -- (2.0694, 5.6239) -- (2.0693, 5.6302) --
   (2.0691, 5.6235) -- (2.0711, 5.6114) -- (2.0747, 5.6036) -- (2.0777, 5.5984) 
  -- (2.0844, 5.5886) -- (2.0986, 5.5771) -- (2.1204, 5.5666) -- (2.1277, 
  5.5635) -- (2.1361, 5.5624) -- (2.1431, 5.5605) -- (2.1464, 5.5577) -- 
  (2.1487, 5.553) -- (2.1632, 5.5146) -- (2.1645, 5.5024) -- (2.1729, 5.4967) --
   (2.1909, 5.4773) -- (2.1954, 5.4711) -- (2.196, 5.4653) -- (2.1966, 5.4414) 
  -- (2.201, 5.4367) -- (2.206, 5.4324) -- (2.2088, 5.4318) -- (2.2185, 5.435) 
  -- (2.2219, 5.4364) -- (2.2343, 5.4363) -- (2.2833, 5.4193) -- (2.3118, 
  5.4082) -- (2.3416, 5.4337) -- (2.3421, 5.4517) -- (2.3386, 5.4954) -- 
  (2.3314, 5.5067) -- (2.3322, 5.5079) -- (2.3334, 5.5085) -- (2.3423, 5.5118) 
  -- cycle(1.7375, 6.1385) -- (1.7281, 6.1501) -- (1.7271, 6.1511) -- (1.7271, 
  6.1528) -- (1.7311, 6.1563) -- (1.7346, 6.1577) -- (1.7468, 6.1582) -- 
  (1.7537, 6.1579) -- (1.7599, 6.1571) -- (1.7761, 6.1535) -- (1.7818, 6.1518) 
  -- (1.7838, 6.1508) -- (1.7951, 6.1429) -- (1.7956, 6.1385) -- (1.794, 6.1381)
   -- (1.7791, 6.1437) -- (1.7438, 6.1527) -- (1.7425, 6.1526) -- (1.7416, 
  6.1519) -- (1.7393, 6.1479) -- cycle(1.8226, 6.1094) -- (1.8217, 6.1095) -- 
  (1.8187, 6.1109) -- (1.8163, 6.1124) -- (1.8144, 6.1147) -- (1.8135, 6.1168) 
  -- (1.8096, 6.1322) -- (1.8097, 6.1357) -- (1.8163, 6.1437) -- (1.8196, 
  6.1396) -- (1.8234, 6.1202) -- (1.8232, 6.1104) -- cycle(1.6333, 6.1216) -- 
  (1.6284, 6.1224) -- (1.6268, 6.1234) -- (1.6249, 6.1271) -- (1.625, 6.1307) --
   (1.6275, 6.1358) -- (1.6331, 6.1428) -- (1.6357, 6.1453) -- (1.6381, 6.1469) 
  -- (1.6437, 6.1486) -- (1.6533, 6.149) -- (1.6672, 6.1486) -- (1.689, 6.1467) 
  -- (1.693, 6.1461) -- (1.7054, 6.1427) -- (1.709, 6.1389) -- (1.7095, 6.1376) 
  -- (1.709, 6.1372) -- (1.6753, 6.1309) -- (1.6414, 6.133) -- cycle(1.517, 
  6.0956) -- (1.5149, 6.096) -- (1.5116, 6.0987) -- (1.5103, 6.1069) -- (1.5105,
   6.1136) -- (1.5111, 6.1162) -- (1.5177, 6.1252) -- (1.5195, 6.1268) -- 
  (1.5253, 6.1297) -- (1.5292, 6.1304) -- (1.5603, 6.1306) -- (1.5849, 6.1293) 
  -- (1.5938, 6.1282) -- (1.5961, 6.1276) -- (1.5998, 6.1262) -- (1.6031, 
  6.1243) -- (1.6043, 6.1225) -- (1.6045, 6.1181) -- (1.6043, 6.1171) -- (1.604,
   6.117) -- (1.5967, 6.1165) -- (1.5966, 6.117) -- (1.5977, 6.1182) -- (1.5976,
   6.1187) -- (1.5971, 6.1189) -- (1.5776, 6.1201) -- (1.5754, 6.1201) -- 
  (1.5609, 6.1189) -- (1.5395, 6.1168) -- (1.5369, 6.1159) -- cycle(1.4957, 
  6.0944) -- (1.4831, 6.0949) -- (1.4683, 6.0966) -- (1.4555, 6.0995) -- 
  (1.4526, 6.104) -- (1.4527, 6.105) -- (1.4535, 6.1058) -- (1.4669, 6.1114) -- 
  (1.4693, 6.1119) -- (1.4723, 6.1119) -- (1.4914, 6.1077) -- (1.4926, 6.107) --
   (1.4955, 6.0968) -- (1.4959, 6.0953) -- cycle(1.3361, 6.0797) -- (1.3302, 
  6.0846) -- (1.3278, 6.0881) -- (1.3273, 6.0899) -- (1.3285, 6.092) -- (1.3313,
   6.094) -- (1.3347, 6.0957) -- (1.3477, 6.1016) -- (1.3562, 6.1038) -- 
  (1.3612, 6.1045) -- (1.4268, 6.1042) -- (1.4366, 6.1039) -- (1.4414, 6.1026) 
  -- (1.4437, 6.1015) -- (1.4454, 6.0968) -- (1.445, 6.0955) -- (1.4369, 6.0926)
   -- (1.4314, 6.0908) -- (1.3611, 6.0821) -- (1.3429, 6.0805) -- cycle(1.1937, 
  6.0551) -- (1.1883, 6.0555) -- (1.1873, 6.0555) -- (1.1841, 6.0566) -- 
  (1.1795, 6.0596) -- (1.179, 6.0612) -- (1.1795, 6.0624) -- (1.1827, 6.0633) --
   (1.202, 6.0652) -- (1.2655, 6.0715) -- (1.2941, 6.0715) -- (1.3021, 6.0697) 
  -- (1.3033, 6.0683) -- (1.3038, 6.0669) -- (1.3037, 6.0646) -- (1.3021, 
  6.0624) -- (1.3016, 6.0639) -- (1.3008, 6.0643) -- (1.2871, 6.0652) -- 
  (1.2814, 6.0654) -- (1.2745, 6.0652) -- (1.2421, 6.064) -- (1.1992, 6.0574) --
   (1.1941, 6.0558) -- cycle(1.1048, 5.9584) -- (1.1001, 5.9683) -- (1.0972, 
  5.9701) -- (1.0925, 5.9698) -- (1.0885, 5.9684) -- (1.0879, 5.9671) -- 
  (1.0879, 5.9644) -- (1.0889, 5.9616) -- (1.09, 5.9607) -- (1.0899, 5.9603) -- 
  (1.0889, 5.96) -- (1.0829, 5.961) -- (1.059, 5.98) -- (1.055, 5.9862) -- 
  (1.0534, 5.994) -- (1.0542, 5.9978) -- (1.0556, 6.0001) -- (1.0565, 6.001) -- 
  (1.0591, 6.002) -- (1.1072, 6.0165) -- (1.1097, 6.0166) -- (1.1147, 6.0159) --
   (1.1208, 6.0133) -- (1.137, 6.0012) -- (1.1417, 5.9962) -- (1.1422, 5.9947) 
  -- (1.0938, 5.981) -- (1.1054, 5.9707) -- (1.1116, 5.9632) -- (1.1111, 5.9623)
   -- (1.106, 5.9585) -- cycle;

  \node[text=black,line width=0.0092cm,anchor=center] (text11) at (2.9296, 
  5.0111){\resizebox{\ifdim\width>1.9em 1.9em\else\width\fi}{!}{#2}};
}
\newcommand{\drawnrw}[2]{%
%Nordrhein-Westfalen
  \path[draw=black,fill=#1,line join=round,line width=0.0046cm] (2.5993, 
  4.1253) -- (2.5874, 4.1053) -- (2.5521, 4.063) -- (2.5224, 4.0142) -- (2.4815,
   3.9822) -- (2.4419, 3.9503) -- (2.4359, 3.9407) -- (2.4304, 3.9315) -- 
  (2.425, 3.9306) -- (2.3997, 3.9323) -- (2.394, 3.9341) -- (2.3915, 3.9358) -- 
  (2.39, 3.938) -- (2.3848, 3.953) -- (2.3913, 3.9692) -- (2.3805, 3.9872) -- 
  (2.3537, 3.9985) -- (2.343, 3.998) -- (2.3226, 3.993) -- (2.308, 3.988) -- 
  (2.2818, 3.9769) -- (2.2756, 3.9706) -- (2.2756, 3.9665) -- (2.2772, 3.9605) 
  -- (2.3003, 3.9225) -- (2.3053, 3.9213) -- (2.3081, 3.9198) -- (2.3085, 
  3.9177) -- (2.3105, 3.9053) -- (2.3029, 3.8828) -- (2.3015, 3.8823) -- (2.27, 
  3.8843) -- (2.2318, 3.8835) -- (2.1972, 3.8743) -- (2.1515, 3.8602) -- 
  (2.1244, 3.8429) -- (2.1145, 3.8364) -- (2.1045, 3.824) -- (2.0805, 3.7896) --
   (2.0753, 3.7818) -- (2.0942, 3.7553) -- (2.0962, 3.7533) -- (2.099, 3.7524) 
  -- (2.106, 3.752) -- (2.1069, 3.7523) -- (2.1158, 3.7583) -- (2.1576, 3.7771) 
  -- (2.1641, 3.7777) -- (2.1725, 3.7775) -- (2.1803, 3.7668) -- (2.1814, 
  3.7645) -- (2.1882, 3.7056) -- (2.1897, 3.6869) -- (2.189, 3.6834) -- (2.1865,
   3.6774) -- (2.1605, 3.6204) -- (2.1312, 3.6091) -- (2.1272, 3.609) -- (2.117,
   3.6114) -- (2.11, 3.6142) -- (2.0944, 3.6167) -- (2.0686, 3.6184) -- (2.0673,
   3.6173) -- (2.0557, 3.613) -- (2.0545, 3.6123) -- (2.0475, 3.6064) -- 
  (2.0426, 3.6022) -- (2.0386, 3.5964) -- (2.0386, 3.5946) -- (2.0395, 3.5918) 
  -- (2.0492, 3.5618) -- (2.0554, 3.5484) -- (2.0588, 3.5403) -- (2.0544, 
  3.5337) -- (2.0509, 3.5302) -- (2.0418, 3.5223) -- (2.0286, 3.5081) -- 
  (2.0207, 3.4995) -- (2.02, 3.4981) -- (2.0201, 3.4958) -- (2.0225, 3.4924) -- 
  (2.021, 3.493) -- (2.0156, 3.4929) -- (2.0147, 3.4924) -- (2.011, 3.4903) -- 
  (2.0105, 3.4892) -- (2.005, 3.4654) -- (1.9657, 3.406) -- (1.9554, 3.395) -- 
  (1.9543, 3.3938) -- (1.9493, 3.3927) -- (1.9477, 3.3932) -- (1.9102, 3.413) --
   (1.9085, 3.4136) -- (1.8996, 3.4139) -- (1.8978, 3.4136) -- (1.8959, 3.412) 
  -- (1.809, 3.3271) -- (1.8128, 3.3179) -- (1.8164, 3.3103) -- (1.8235, 3.2964)
   -- (1.8308, 3.2862) -- (1.8325, 3.2775) -- (1.832, 3.2742) -- (1.8136, 
  3.2401) -- (1.8068, 3.231) -- (1.7976, 3.2424) -- (1.7913, 3.2481) -- (1.7796,
   3.2513) -- (1.7553, 3.2428) -- (1.7581, 3.2567) -- (1.7568, 3.271) -- 
  (1.7532, 3.275) -- (1.7398, 3.2892) -- (1.7264, 3.3022) -- (1.7239, 3.305) -- 
  (1.7183, 3.312) -- (1.7157, 3.3162) -- (1.7138, 3.325) -- (1.7145, 3.3334) -- 
  (1.7149, 3.3394) -- (1.7183, 3.3429) -- (1.719, 3.3447) -- (1.7211, 3.3641) --
   (1.7212, 3.3693) -- (1.7161, 3.3821) -- (1.7138, 3.3848) -- (1.7114, 3.3868) 
  -- (1.6662, 3.4124) -- (1.656, 3.415) -- (1.6349, 3.4206) -- (1.6342, 3.4249) 
  -- (1.6343, 3.4276) -- (1.6351, 3.4312) -- (1.6401, 3.4454) -- (1.6441, 
  3.4534) -- (1.6486, 3.4603) -- (1.6272, 3.4756) -- (1.6153, 3.4779) -- 
  (1.6112, 3.4758) -- (1.5892, 3.4568) -- (1.585, 3.4517) -- (1.5902, 3.4061) --
   (1.5787, 3.3897) -- (1.5568, 3.37) -- (1.5557, 3.3696) -- (1.5414, 3.3643) --
   (1.5385, 3.3643) -- (1.5325, 3.3656) -- (1.5414, 3.3582) -- (1.544, 3.3553) 
  -- (1.5456, 3.3407) -- (1.5424, 3.3329) -- (1.5325, 3.3203) -- (1.5283, 
  3.3165) -- (1.4081, 3.2737) -- (1.4, 3.27) -- (1.3976, 3.2679) -- (1.3785, 
  3.2742) -- (1.3608, 3.2736) -- (1.3563, 3.2692) -- (1.352, 3.2599) -- (1.3528,
   3.2465) -- (1.3526, 3.232) -- (1.3478, 3.2127) -- (1.3461, 3.2096) -- 
  (1.3338, 3.203) -- (1.3083, 3.1952) -- (1.3008, 3.193) -- (1.2861, 3.1897) -- 
  (1.2589, 3.1899) -- (1.2592, 3.1936) -- (1.2598, 3.2079) -- (1.2598, 3.2121) 
  -- (1.2592, 3.2146) -- (1.2547, 3.2148) -- (1.2523, 3.2136) -- (1.2497, 
  3.2083) -- (1.2488, 3.2039) -- (1.2454, 3.1907) -- (1.2448, 3.1899) -- 
  (1.2355, 3.1806) -- (1.2204, 3.1764) -- (1.1721, 3.1655) -- (1.1483, 3.1615) 
  -- (1.1403, 3.1514) -- (1.1303, 3.1411) -- (1.1286, 3.1398) -- (1.1129, 
  3.1322) -- (1.1064, 3.1327) -- (1.0865, 3.1358) -- (1.0844, 3.1233) -- 
  (1.0672, 3.0523) -- (1.0452, 3.0422) -- (1.0389, 3.0402) -- (1.0169, 3.055) --
   (1.0135, 3.0575) -- (0.9838, 3.0384) -- (0.9848, 2.9836) -- (0.9936, 2.9673) 
  -- (1.0019, 2.9532) -- (0.9968, 2.9543) -- (0.9898, 2.9556) -- (0.9622, 
  2.9471) -- (0.9331, 2.9551) -- (0.898, 2.9697) -- (0.8968, 2.9715) -- (0.885, 
  2.984) -- (0.8421, 2.9694) -- (0.7978, 2.9626) -- (0.7913, 2.9723) -- (0.7905,
   2.973) -- (0.7706, 2.9908) -- (0.7534, 2.9867) -- (0.7564, 2.9745) -- 
  (0.7629, 2.9501) -- (0.7647, 2.9452) -- (0.7689, 2.9378) -- (0.7754, 2.9279) 
  -- (0.7622, 2.9289) -- (0.765, 2.9399) -- (0.7649, 2.941) -- (0.7606, 2.9491) 
  -- (0.7382, 2.9754) -- (0.7372, 2.9939) -- (0.7426, 3.0032) -- (0.7456, 
  3.0105) -- (0.7506, 3.0335) -- (0.7514, 3.04) -- (0.7506, 3.0501) -- (0.7488, 
  3.0543) -- (0.7332, 3.0809) -- (0.7288, 3.0877) -- (0.7186, 3.0943) -- 
  (0.7167, 3.0968) -- (0.7144, 3.0988) -- (0.6911, 3.1029) -- (0.6801, 3.1027) 
  -- (0.6789, 3.1023) -- (0.6741, 3.099) -- (0.6654, 3.0952) -- (0.6645, 3.0951)
   -- (0.6635, 3.0955) -- (0.6455, 3.1208) -- (0.6426, 3.1259) -- (0.6381, 
  3.1404) -- (0.6375, 3.1542) -- (0.6383, 3.1566) -- (0.6685, 3.1851) -- 
  (0.6758, 3.2182) -- (0.6441, 3.2265) -- (0.636, 3.237) -- (0.6374, 3.2376) -- 
  (0.6476, 3.2416) -- (0.6405, 3.2484) -- (0.6361, 3.2545) -- (0.6271, 3.2694) 
  -- (0.6247, 3.2742) -- (0.6237, 3.2783) -- (0.6113, 3.3097) -- (0.6083, 
  3.3125) -- (0.6066, 3.3135) -- (0.5657, 3.3364) -- (0.5412, 3.3773) -- 
  (0.5561, 3.4249) -- (0.5582, 3.4317) -- (0.5674, 3.4353) -- (0.5768, 3.4382) 
  -- (0.5805, 3.4378) -- (0.5813, 3.4353) -- (0.5907, 3.4305) -- (0.6005, 
  3.4541) -- (0.5997, 3.4609) -- (0.5969, 3.4642) -- (0.5945, 3.4691) -- 
  (0.5938, 3.4737) -- (0.5979, 3.4873) -- (0.5986, 3.4892) -- (0.602, 3.4924) --
   (0.6032, 3.4949) -- (0.6037, 3.4983) -- (0.5965, 3.5013) -- (0.5842, 3.5065) 
  -- (0.5621, 3.5144) -- (0.5609, 3.5271) -- (0.5615, 3.54) -- (0.5669, 3.5513) 
  -- (0.5695, 3.5593) -- (0.5692, 3.5602) -- (0.5644, 3.561) -- (0.5284, 3.5639)
   -- (0.5051, 3.5583) -- (0.4969, 3.5678) -- (0.4839, 3.5964) -- (0.4783, 
  3.624) -- (0.4781, 3.6288) -- (0.4787, 3.6294) -- (0.5065, 3.6418) -- (0.5638,
   3.6603) -- (0.6361, 3.7088) -- (0.6452, 3.735) -- (0.667, 3.7443) -- (0.6707,
   3.7459) -- (0.6619, 3.7537) -- (0.6521, 3.7498) -- (0.6365, 3.7429) -- 
  (0.6319, 3.7398) -- (0.6309, 3.7381) -- (0.629, 3.7362) -- (0.6257, 3.7348) --
   (0.6227, 3.7337) -- (0.6186, 3.7336) -- (0.6167, 3.7338) -- (0.6123, 3.735) 
  -- (0.6074, 3.7448) -- (0.6066, 3.7744) -- (0.6175, 3.806) -- (0.6837, 3.8894)
   -- (0.7051, 3.9127) -- (0.696, 3.9455) -- (0.7019, 3.9746) -- (0.7066, 
  3.9892) -- (0.7096, 4.0098) -- (0.7097, 4.0158) -- (0.7052, 4.0522) -- 
  (0.6992, 4.0648) -- (0.6755, 4.1038) -- (0.6692, 4.1084) -- (0.6392, 4.1428) 
  -- (0.6414, 4.1526) -- (0.64, 4.1939) -- (0.6106, 4.2074) -- (0.6073, 4.2116) 
  -- (0.6069, 4.2126) -- (0.5897, 4.2699) -- (0.576, 4.3343) -- (0.59, 4.3572) 
  -- (0.6305, 4.3752) -- (0.6505, 4.3743) -- (0.6953, 4.3804) -- (0.7132, 
  4.3921) -- (0.7585, 4.3901) -- (0.7608, 4.3897) -- (0.7673, 4.3865) -- 
  (0.7734, 4.3826) -- (0.8008, 4.3649) -- (0.8186, 4.3494) -- (0.8303, 4.3427) 
  -- (0.8343, 4.3421) -- (0.8245, 4.3744) -- (0.8241, 4.3773) -- (0.8266, 
  4.3859) -- (0.8282, 4.3857) -- (0.8476, 4.3774) -- (0.8921, 4.3769) -- 
  (0.9005, 4.3813) -- (0.914, 4.3885) -- (0.9407, 4.3989) -- (0.9935, 4.4169) --
   (0.9992, 4.4183) -- (1.0037, 4.4154) -- (1.0066, 4.4108) -- (1.0198, 4.3976) 
  -- (1.0201, 4.3974) -- (1.0263, 4.3996) -- (1.0376, 4.4071) -- (1.0489, 
  4.4149) -- (1.0605, 4.427) -- (1.0849, 4.4585) -- (1.0873, 4.4642) -- (1.0875,
   4.4662) -- (1.0865, 4.4787) -- (1.0847, 4.4858) -- (1.0435, 4.5209) -- 
  (1.0215, 4.5326) -- (1.0351, 4.5647) -- (1.0492, 4.5872) -- (1.1014, 4.604) --
   (1.1065, 4.6047) -- (1.1137, 4.6115) -- (1.1172, 4.6151) -- (1.121, 4.6301) 
  -- (1.1225, 4.6375) -- (1.1393, 4.6557) -- (1.2097, 4.6997) -- (1.233, 4.7135)
   -- (1.2523, 4.7141) -- (1.2587, 4.7232) -- (1.2607, 4.7257) -- (1.2662, 
  4.7308) -- (1.2858, 4.7352) -- (1.3096, 4.7326) -- (1.3215, 4.7318) -- 
  (1.3673, 4.7294) -- (1.3754, 4.7389) -- (1.3792, 4.7442) -- (1.3865, 4.745) --
   (1.3959, 4.746) -- (1.4054, 4.751) -- (1.4079, 4.7556) -- (1.411, 4.7618) -- 
  (1.4167, 4.7666) -- (1.4508, 4.7918) -- (1.4893, 4.8156) -- (1.5049, 4.8245) 
  -- (1.5073, 4.8248) -- (1.5103, 4.825) -- (1.5315, 4.8355) -- (1.5481, 4.8647)
   -- (1.5492, 4.8709) -- (1.5366, 4.8777) -- (1.5304, 4.882) -- (1.5346, 
  4.8959) -- (1.5381, 4.9032) -- (1.5436, 4.9127) -- (1.5508, 4.9217) -- 
  (1.5529, 4.9207) -- (1.5952, 4.9041) -- (1.5966, 4.9027) -- (1.5984, 4.9002) 
  -- (1.6033, 4.89) -- (1.6175, 4.8491) -- (1.6641, 4.8208) -- (1.6843, 4.8177) 
  -- (1.6965, 4.8231) -- (1.6989, 4.8247) -- (1.7133, 4.8282) -- (1.7389, 
  4.8126) -- (1.7476, 4.7972) -- (1.754, 4.7739) -- (1.7347, 4.7533) -- (1.7339,
   4.7498) -- (1.7335, 4.7456) -- (1.7342, 4.7401) -- (1.7356, 4.7383) -- 
  (1.7481, 4.7255) -- (1.7385, 4.719) -- (1.7134, 4.6657) -- (1.7133, 4.6646) --
   (1.7148, 4.6576) -- (1.7176, 4.6484) -- (1.7202, 4.644) -- (1.7315, 4.6359) 
  -- (1.7321, 4.6357) -- (1.7534, 4.6315) -- (1.7715, 4.6147) -- (1.7732, 
  4.5767) -- (1.744, 4.5678) -- (1.7101, 4.5565) -- (1.7015, 4.55) -- (1.7011, 
  4.5488) -- (1.7036, 4.5488) -- (1.7165, 4.5277) -- (1.7187, 4.5185) -- 
  (1.7282, 4.5136) -- (1.7447, 4.5066) -- (1.7552, 4.5026) -- (1.761, 4.5111) --
   (1.7646, 4.516) -- (1.7854, 4.5319) -- (1.787, 4.532) -- (1.801, 4.5309) -- 
  (1.8106, 4.5296) -- (1.8206, 4.5269) -- (1.8234, 4.5212) -- (1.8442, 4.5341) 
  -- (1.8588, 4.5375) -- (1.8957, 4.5642) -- (1.9113, 4.5791) -- (1.924, 4.5885)
   -- (1.9295, 4.5877) -- (1.9563, 4.582) -- (1.9787, 4.5739) -- (2.0064, 4.571)
   -- (2.0041, 4.5775) -- (2.0031, 4.5828) -- (2.0036, 4.5859) -- (2.0067, 
  4.5903) -- (2.0106, 4.594) -- (2.0309, 4.6075) -- (2.0324, 4.6081) -- (2.0384,
   4.6077) -- (2.0417, 4.6083) -- (2.0465, 4.6106) -- (2.0487, 4.6123) -- 
  (2.0579, 4.622) -- (2.0623, 4.6275) -- (2.0682, 4.6352) -- (2.0568, 4.6412) --
   (2.0457, 4.6447) -- (2.0424, 4.6449) -- (2.0398, 4.646) -- (2.0327, 4.6523) 
  -- (2.0277, 4.6616) -- (2.0269, 4.6637) -- (2.0371, 4.7046) -- (2.0415, 
  4.7135) -- (2.0393, 4.744) -- (2.0407, 4.7601) -- (2.0264, 4.8025) -- (2.0064,
   4.8179) -- (1.9586, 4.8539) -- (1.9526, 4.8833) -- (1.9481, 4.8932) -- 
  (1.9518, 4.8956) -- (1.9559, 4.8947) -- (1.9776, 4.889) -- (1.9804, 4.8862) --
   (1.9801, 4.8844) -- (1.9847, 4.8798) -- (2.0155, 4.8821) -- (2.0244, 4.8834) 
  -- (2.0386, 4.8954) -- (2.0395, 4.8985) -- (2.0413, 4.9105) -- (2.0426, 
  4.9234) -- (2.0711, 4.9454) -- (2.0993, 4.9315) -- (2.1535, 4.9596) -- (2.173,
   4.9444) -- (2.1797, 4.938) -- (2.1817, 4.9328) -- (2.1821, 4.9299) -- 
  (2.1797, 4.9176) -- (2.1818, 4.8477) -- (2.1835, 4.8342) -- (2.1884, 4.8306) 
  -- (2.2053, 4.8234) -- (2.267, 4.8239) -- (2.3139, 4.8344) -- (2.3157, 4.8351)
   -- (2.3422, 4.8696) -- (2.3617, 4.8951) -- (2.3687, 4.908) -- (2.3832, 
  4.9269) -- (2.4046, 4.9219) -- (2.4294, 4.9106) -- (2.4313, 4.9044) -- 
  (2.4318, 4.9025) -- (2.4153, 4.8649) -- (2.4237, 4.8435) -- (2.4229, 4.8371) 
  -- (2.4212, 4.8327) -- (2.4075, 4.8046) -- (2.4001, 4.7906) -- (2.399, 4.7895)
   -- (2.3921, 4.7832) -- (2.3849, 4.779) -- (2.3739, 4.7775) -- (2.365, 4.7731)
   -- (2.3556, 4.7669) -- (2.3496, 4.7627) -- (2.3288, 4.7189) -- (2.3335, 
  4.7047) -- (2.335, 4.7035) -- (2.3376, 4.7024) -- (2.3519, 4.6985) -- (2.3703,
   4.6911) -- (2.3937, 4.6756) -- (2.3931, 4.6734) -- (2.3749, 4.6283) -- 
  (2.3616, 4.6285) -- (2.3405, 4.6396) -- (2.3581, 4.6163) -- (2.3897, 4.592) --
   (2.4215, 4.5859) -- (2.43, 4.5829) -- (2.4321, 4.5819) -- (2.437, 4.5766) -- 
  (2.4401, 4.5676) -- (2.4375, 4.5473) -- (2.4498, 4.536) -- (2.4582, 4.5215) --
   (2.4569, 4.5204) -- (2.4523, 4.518) -- (2.4481, 4.5119) -- (2.4478, 4.5097) 
  -- (2.4475, 4.4932) -- (2.45, 4.4363) -- (2.4558, 4.4222) -- (2.5076, 4.3847) 
  -- (2.5156, 4.3802) -- (2.5249, 4.3817) -- (2.5255, 4.382) -- (2.5286, 4.3815)
   -- (2.5383, 4.3762) -- (2.5389, 4.3755) -- (2.5432, 4.3679) -- (2.5463, 
  4.356) -- (2.5459, 4.3504) -- (2.5446, 4.3411) -- (2.5399, 4.3189) -- (2.5378,
   4.3176) -- (2.547, 4.3234) -- (2.5501, 4.3252) -- (2.5558, 4.3272) -- 
  (2.5578, 4.3271) -- (2.5618, 4.326) -- (2.5754, 4.3209) -- (2.5805, 4.3179) --
   (2.5968, 4.3017) -- (2.5966, 4.2705) -- (2.5852, 4.2425) -- (2.5687, 4.1914) 
  -- (2.5608, 4.1251) -- (2.565, 4.122) -- (2.5675, 4.1213) -- (2.5694, 4.1213) 
  -- (2.5842, 4.1217) -- (2.5854, 4.1219) -- (2.5896, 4.1241) -- (2.5902, 
  4.1243) -- (2.5956, 4.126) -- (2.5977, 4.1262) -- cycle;

  \node[text=black,line width=0.0092cm,anchor=center] (text15) at (1.4443, 
  3.9743){\resizebox{\ifdim\width>2em 2em\else\width\fi}{!}{#2}};
}

\newcommand{\drawrheinland}[2]{%
  %Rheinland-Pfalz
  \path[draw=black,fill=#1,line join=round,line width=0.0046cm] (1.8075, 
  3.231) -- (1.7994, 3.2218) -- (1.7981, 3.2181) -- (1.7975, 3.213) -- (1.7974, 
  3.2045) -- (1.8059, 3.1599) -- (1.807, 3.1575) -- (1.8103, 3.1556) -- (1.8211,
   3.1495) -- (1.8212, 3.144) -- (1.8177, 3.1332) -- (1.8109, 3.1167) -- 
  (1.8084, 3.1106) -- (1.8013, 3.0936) -- (1.7996, 3.0896) -- (1.7926, 3.0887) 
  -- (1.7827, 3.0906) -- (1.7734, 3.0938) -- (1.7688, 3.1034) -- (1.7577, 
  3.1079) -- (1.7549, 3.1082) -- (1.7493, 3.1066) -- (1.7459, 3.1054) -- (1.724,
   3.0793) -- (1.7191, 3.0695) -- (1.7148, 3.0472) -- (1.7148, 3.0442) -- 
  (1.7198, 3.0323) -- (1.7265, 3.0234) -- (1.7104, 2.978) -- (1.7078, 2.9709) --
   (1.7125, 2.9709) -- (1.7609, 2.9386) -- (1.7852, 2.8941) -- (1.7871, 2.8904) 
  -- (1.7878, 2.8888) -- (1.7945, 2.8683) -- (1.7952, 2.8646) -- (1.7953, 2.847)
   -- (1.777, 2.8364) -- (1.7578, 2.8442) -- (1.7517, 2.845) -- (1.7435, 2.8425)
   -- (1.7344, 2.8306) -- (1.7365, 2.8262) -- (1.7519, 2.8112) -- (1.7553, 
  2.8088) -- (1.7562, 2.8073) -- (1.7408, 2.7902) -- (1.7381, 2.7901) -- 
  (1.7289, 2.7923) -- (1.712, 2.8022) -- (1.6722, 2.785) -- (1.6691, 2.7833) -- 
  (1.6637, 2.7794) -- (1.6497, 2.761) -- (1.6491, 2.7598) -- (1.6497, 2.7477) --
   (1.653, 2.743) -- (1.6578, 2.7401) -- (1.665, 2.7345) -- (1.6742, 2.7263) -- 
  (1.6754, 2.723) -- (1.6719, 2.6956) -- (1.6691, 2.6919) -- (1.6479, 2.6978) --
   (1.6444, 2.6992) -- (1.6062, 2.6674) -- (1.588, 2.6571) -- (1.5793, 2.6553) 
  -- (1.5856, 2.6375) -- (1.6352, 2.5681) -- (1.6395, 2.5651) -- (1.6421, 
  2.5643) -- (1.6453, 2.5639) -- (1.6759, 2.5649) -- (1.6941, 2.5678) -- 
  (1.6968, 2.5686) -- (1.7612, 2.5932) -- (1.7992, 2.6105) -- (1.8223, 2.618) --
   (1.8307, 2.6186) -- (1.8404, 2.6173) -- (1.8574, 2.6137) -- (1.8627, 2.6114) 
  -- (1.8733, 2.6013) -- (1.8797, 2.5944) -- (1.8837, 2.5885) -- (1.8908, 
  2.5778) -- (1.9025, 2.5672) -- (1.9091, 2.5594) -- (1.9126, 2.5542) -- 
  (1.9168, 2.5468) -- (1.9212, 2.536) -- (1.9215, 2.5273) -- (1.927, 2.5098) -- 
  (1.9281, 2.5048) -- (1.9279, 2.5002) -- (1.9273, 2.4971) -- (1.9245, 2.4912) 
  -- (1.9213, 2.486) -- (1.9201, 2.4828) -- (1.9191, 2.4779) -- (1.9191, 2.4743)
   -- (1.9208, 2.4671) -- (1.9413, 2.4154) -- (1.944, 2.4108) -- (1.9532, 
  2.3976) -- (1.958, 2.3886) -- (1.9836, 2.3619) -- (1.9868, 2.3404) -- (1.9811,
   2.3311) -- (1.9798, 2.3284) -- (1.979, 2.3267) -- (1.9753, 2.3211) -- 
  (1.9709, 2.3176) -- (1.9671, 2.3155) -- (1.9636, 2.314) -- (1.9497, 2.311) -- 
  (1.9433, 2.3086) -- (1.9389, 2.3064) -- (1.9339, 2.3033) -- (1.9281, 2.2988) 
  -- (1.9271, 2.2977) -- (1.9252, 2.2943) -- (1.9245, 2.292) -- (1.923, 2.2831) 
  -- (1.9239, 2.2789) -- (1.9387, 2.2317) -- (1.9433, 2.2233) -- (1.9579, 
  2.2014) -- (1.9625, 2.1905) -- (1.9617, 2.18) -- (1.9581, 2.1736) -- (1.9569, 
  2.1691) -- (1.9567, 2.1652) -- (1.9571, 2.1618) -- (1.958, 2.159) -- (1.9619, 
  2.1512) -- (1.972, 2.1107) -- (1.9727, 2.105) -- (1.9767, 2.0757) -- (1.9745, 
  2.074) -- (1.9725, 2.0719) -- (1.9714, 2.0691) -- (1.9715, 2.0658) -- (1.9727,
   2.0636) -- (1.9741, 2.0619) -- (1.9765, 2.0599) -- (1.9786, 2.0589) -- 
  (1.984, 2.058) -- (1.99, 2.0582) -- (1.9992, 2.0585) -- (2.0043, 2.0576) -- 
  (2.0076, 2.0561) -- (2.0108, 2.0524) -- (2.0117, 2.0495) -- (2.0116, 2.0466) 
  -- (2.0044, 2.0274) -- (2.006, 2.0064) -- (2.0065, 1.9985) -- (2.0061, 1.9933)
   -- (2.0038, 1.9848) -- (2.0004, 1.977) -- (1.995, 1.969) -- (1.9904, 1.9643) 
  -- (1.9864, 1.9612) -- (1.9822, 1.9584) -- (1.9775, 1.9539) -- (1.9754, 1.951)
   -- (1.9742, 1.948) -- (1.9739, 1.9462) -- (1.9742, 1.9422) -- (1.9755, 
  1.9387) -- (1.9773, 1.9366) -- (1.9792, 1.9354) -- (1.9887, 1.9322) -- 
  (1.9909, 1.9309) -- (1.9941, 1.928) -- (1.9962, 1.9247) -- (1.9973, 1.9198) --
   (1.9969, 1.9159) -- (1.9959, 1.9138) -- (1.9835, 1.9074) -- (1.9673, 1.8894) 
  -- (1.9628, 1.8853) -- (1.9496, 1.8768) -- (1.9445, 1.8737) -- (1.9398, 
  1.8696) -- (1.9365, 1.8657) -- (1.9339, 1.8607) -- (1.9323, 1.8551) -- 
  (1.9318, 1.8486) -- (1.932, 1.8482) -- (1.9329, 1.8482) -- (1.9344, 1.8486) --
   (1.9359, 1.8321) -- (1.9138, 1.7377) -- (1.9121, 1.733) -- (1.908, 1.7263) --
   (1.9017, 1.7186) -- (1.9016, 1.7185) -- (1.8926, 1.7087) -- (1.8861, 1.699) 
  -- (1.8828, 1.693) -- (1.8786, 1.6822) -- (1.8768, 1.6739) -- (1.875, 1.661) 
  -- (1.8743, 1.6583) -- (1.8734, 1.6554) -- (1.8696, 1.6468) -- (1.8669, 
  1.6419) -- (1.8641, 1.6378) -- (1.8604, 1.634) -- (1.8505, 1.6262) -- (1.836, 
  1.6149) -- (1.8322, 1.6127) -- (1.8296, 1.6181) -- (1.8281, 1.6207) -- 
  (1.8267, 1.6216) -- (1.7948, 1.6255) -- (1.7919, 1.6242) -- (1.7828, 1.6238) 
  -- (1.775, 1.6254) -- (1.7473, 1.6365) -- (1.7324, 1.6434) -- (1.7313, 1.646) 
  -- (1.7253, 1.6534) -- (1.7187, 1.66) -- (1.6729, 1.6821) -- (1.6431, 1.6907) 
  -- (1.6307, 1.6917) -- (1.6217, 1.696) -- (1.5768, 1.7028) -- (1.5454, 1.6956)
   -- (1.5427, 1.6949) -- (1.5248, 1.6941) -- (1.4893, 1.6961) -- (1.4628, 
  1.7062) -- (1.4218, 1.7314) -- (1.4134, 1.7369) -- (1.4006, 1.7479) -- 
  (1.3814, 1.785) -- (1.3815, 1.7923) -- (1.3832, 1.8036) -- (1.3815, 1.8166) --
   (1.3522, 1.8314) -- (1.3304, 1.8276) -- (1.3121, 1.8244) -- (1.3021, 1.8218) 
  -- (1.3018, 1.8171) -- (1.3023, 1.8145) -- (1.3041, 1.8119) -- (1.2951, 
  1.8118) -- (1.2927, 1.8137) -- (1.2912, 1.8157) -- (1.2747, 1.8395) -- 
  (1.2665, 1.8506) -- (1.2572, 1.8785) -- (1.2574, 1.8839) -- (1.2575, 1.8854) 
  -- (1.2667, 1.8975) -- (1.2929, 1.9152) -- (1.3056, 1.9215) -- (1.3111, 1.924)
   -- (1.3146, 1.9279) -- (1.3199, 1.9357) -- (1.3248, 1.9567) -- (1.3198, 
  1.9611) -- (1.3196, 1.962) -- (1.3261, 2.0018) -- (1.3317, 2.0058) -- (1.3276,
   2.0093) -- (1.3162, 2.0042) -- (1.3096, 2.0024) -- (1.2635, 2.0288) -- 
  (1.2637, 2.0322) -- (1.2662, 2.0458) -- (1.2653, 2.0534) -- (1.2422, 2.0689) 
  -- (1.2408, 2.0724) -- (1.2392, 2.0816) -- (1.239, 2.0829) -- (1.2425, 2.0854)
   -- (1.2577, 2.0901) -- (1.2621, 2.138) -- (1.261, 2.1784) -- (1.2604, 2.1861)
   -- (1.2602, 2.1879) -- (1.256, 2.1972) -- (1.2527, 2.2039) -- (1.251, 2.2067)
   -- (1.2471, 2.2029) -- (1.2369, 2.1975) -- (1.2295, 2.195) -- (1.2225, 
  2.1964) -- (1.1666, 2.2265) -- (1.1363, 2.254) -- (1.1122, 2.2697) -- (1.1072,
   2.264) -- (1.1022, 2.2598) -- (1.0872, 2.254) -- (1.0858, 2.2536) -- (1.0857,
   2.2553) -- (1.0873, 2.2617) -- (1.0879, 2.2645) -- (1.0793, 2.2662) -- 
  (1.061, 2.2685) -- (1.0577, 2.2672) -- (1.0297, 2.2509) -- (1.0285, 2.2499) --
   (1.028, 2.2493) -- (1.0281, 2.2485) -- (1.0241, 2.2485) -- (1.0067, 2.2418) 
  -- (0.979, 2.2306) -- (0.9719, 2.2273) -- (0.9694, 2.2256) -- (0.9642, 2.22) 
  -- (0.9623, 2.2167) -- (0.9574, 2.2118) -- (0.9558, 2.2107) -- (0.9435, 
  2.2036) -- (0.9378, 2.2005) -- (0.8987, 2.1891) -- (0.8337, 2.1777) -- 
  (0.7938, 2.1804) -- (0.7816, 2.1818) -- (0.717, 2.202) -- (0.7127, 2.2046) -- 
  (0.6998, 2.2246) -- (0.6998, 2.2257) -- (0.7006, 2.2272) -- (0.7129, 2.2452) 
  -- (0.7169, 2.2493) -- (0.7189, 2.2504) -- (0.7236, 2.2506) -- (0.7261, 
  2.2517) -- (0.7284, 2.2541) -- (0.7395, 2.2658) -- (0.7928, 2.3522) -- 
  (0.8066, 2.3992) -- (0.8064, 2.4051) -- (0.8011, 2.4438) -- (0.7829, 2.4559) 
  -- (0.7694, 2.4472) -- (0.756, 2.4471) -- (0.7547, 2.4474) -- (0.7382, 2.4569)
   -- (0.736, 2.459) -- (0.7076, 2.4872) -- (0.6562, 2.5186) -- (0.6398, 2.5376)
   -- (0.6091, 2.5887) -- (0.6052, 2.5975) -- (0.6021, 2.6046) -- (0.5913, 
  2.6314) -- (0.5773, 2.6871) -- (0.5802, 2.7147) -- (0.5987, 2.7804) -- 
  (0.6213, 2.7994) -- (0.6184, 2.8382) -- (0.62, 2.8504) -- (0.6201, 2.8513) -- 
  (0.6216, 2.8533) -- (0.6899, 2.8973) -- (0.6932, 2.9057) -- (0.7041, 2.9288) 
  -- (0.7081, 2.9303) -- (0.7169, 2.9325) -- (0.7371, 2.9222) -- (0.763, 2.9289)
   -- (0.7762, 2.9279) -- (0.7698, 2.9378) -- (0.7655, 2.9453) -- (0.7638, 
  2.9501) -- (0.7572, 2.9746) -- (0.7543, 2.9868) -- (0.7715, 2.9909) -- 
  (0.7913, 2.9731) -- (0.7921, 2.9723) -- (0.7987, 2.9626) -- (0.843, 2.9694) --
   (0.8858, 2.9841) -- (0.8976, 2.9716) -- (0.8989, 2.9697) -- (0.934, 2.9552) 
  -- (0.9631, 2.9471) -- (0.9907, 2.9556) -- (0.9977, 2.9544) -- (1.0028, 
  2.9532) -- (0.9945, 2.9674) -- (0.9857, 2.9837) -- (0.9846, 3.0385) -- 
  (1.0143, 3.0576) -- (1.0177, 3.0551) -- (1.0398, 3.0403) -- (1.046, 3.0423) --
   (1.068, 3.0523) -- (1.0852, 3.1234) -- (1.0873, 3.1358) -- (1.1072, 3.1327) 
  -- (1.1137, 3.1323) -- (1.1294, 3.1399) -- (1.1311, 3.1412) -- (1.1411, 
  3.1515) -- (1.1491, 3.1615) -- (1.1729, 3.1655) -- (1.2212, 3.1765) -- 
  (1.2363, 3.1807) -- (1.2456, 3.19) -- (1.2463, 3.1907) -- (1.2496, 3.204) -- 
  (1.2506, 3.2084) -- (1.2531, 3.2137) -- (1.2555, 3.2149) -- (1.26, 3.2147) -- 
  (1.2606, 3.2122) -- (1.2606, 3.208) -- (1.26, 3.1937) -- (1.2597, 3.19) -- 
  (1.2869, 3.1898) -- (1.3016, 3.193) -- (1.3091, 3.1952) -- (1.3345, 3.2031) --
   (1.3469, 3.2097) -- (1.3486, 3.2128) -- (1.3534, 3.2321) -- (1.3536, 3.2465) 
  -- (1.3528, 3.26) -- (1.3571, 3.2692) -- (1.3616, 3.2737) -- (1.3793, 3.2743) 
  -- (1.3984, 3.268) -- (1.4008, 3.2701) -- (1.4089, 3.2738) -- (1.5291, 3.3165)
   -- (1.5333, 3.3203) -- (1.5432, 3.3329) -- (1.5464, 3.3408) -- (1.5448, 
  3.3553) -- (1.5422, 3.3582) -- (1.5333, 3.3657) -- (1.5393, 3.3643) -- 
  (1.5422, 3.3643) -- (1.5565, 3.3697) -- (1.5576, 3.37) -- (1.5795, 3.3897) -- 
  (1.591, 3.4062) -- (1.5859, 3.4518) -- (1.59, 3.4568) -- (1.612, 3.4758) -- 
  (1.6161, 3.478) -- (1.6279, 3.4756) -- (1.6494, 3.4603) -- (1.6448, 3.4534) --
   (1.6409, 3.4455) -- (1.6359, 3.4312) -- (1.6351, 3.4276) -- (1.635, 3.425) --
   (1.6357, 3.4207) -- (1.6568, 3.4151) -- (1.667, 3.4125) -- (1.7122, 3.3869) 
  -- (1.7146, 3.3849) -- (1.7169, 3.3821) -- (1.722, 3.3693) -- (1.7219, 3.3642)
   -- (1.7198, 3.3448) -- (1.7191, 3.343) -- (1.7157, 3.3395) -- (1.7152, 
  3.3335) -- (1.7146, 3.325) -- (1.7165, 3.3163) -- (1.7191, 3.312) -- (1.7247, 
  3.305) -- (1.7272, 3.3023) -- (1.7406, 3.2892) -- (1.7539, 3.2751) -- (1.7576,
   3.2711) -- (1.7588, 3.2567) -- (1.7561, 3.2429) -- (1.7804, 3.2513) -- 
  (1.792, 3.2482) -- (1.7983, 3.2425) -- cycle;

  \node[text=black,line width=0.0092cm,anchor=center,align=center] (text14) at (1.183, 
  2.5762){\resizebox{\ifdim\width>1.5em 1.5em\else\width\fi}{!}{#2}};
}

\newcommand{\drawsaarland}[2]{%
  %Saarland
  \path[draw=black,fill=#1,line join=round,line width=0.0046cm] (0.713, 
  2.2045) -- (0.7174, 2.2019) -- (0.7819, 2.1817) -- (0.7941, 2.1802) -- (0.834,
   2.1775) -- (0.899, 2.189) -- (0.9382, 2.2003) -- (0.9437, 2.2034) -- (0.9561,
   2.2105) -- (0.9578, 2.2117) -- (0.9626, 2.2166) -- (0.9645, 2.2198) -- 
  (0.9697, 2.2254) -- (0.9722, 2.2271) -- (0.9793, 2.2305) -- (1.007, 2.2416) --
   (1.0245, 2.2484) -- (1.0284, 2.2484) -- (1.0283, 2.2492) -- (1.0288, 2.2497) 
  -- (1.03, 2.2507) -- (1.058, 2.2671) -- (1.0614, 2.2684) -- (1.0796, 2.266) --
   (1.0883, 2.2644) -- (1.0876, 2.2616) -- (1.086, 2.2552) -- (1.0861, 2.2534) 
  -- (1.0875, 2.2538) -- (1.1026, 2.2596) -- (1.1075, 2.2638) -- (1.1125, 
  2.2695) -- (1.1367, 2.2538) -- (1.1669, 2.2264) -- (1.2229, 2.1962) -- 
  (1.2298, 2.1949) -- (1.2372, 2.1974) -- (1.2474, 2.2027) -- (1.2513, 2.2066) 
  -- (1.253, 2.2038) -- (1.2563, 2.1971) -- (1.2605, 2.1878) -- (1.2607, 2.186) 
  -- (1.2613, 2.1783) -- (1.2624, 2.1378) -- (1.258, 2.0899) -- (1.2427, 2.0853)
   -- (1.2392, 2.0827) -- (1.2394, 2.0814) -- (1.2411, 2.0723) -- (1.2424, 
  2.0687) -- (1.2655, 2.0533) -- (1.2665, 2.0457) -- (1.264, 2.0321) -- (1.2638,
   2.0286) -- (1.3098, 2.0023) -- (1.3165, 2.004) -- (1.3279, 2.0092) -- 
  (1.3319, 2.0057) -- (1.3264, 2.0017) -- (1.3199, 1.9618) -- (1.3201, 1.961) --
   (1.3254, 1.9574) -- (1.3205, 1.9364) -- (1.3151, 1.9287) -- (1.3117, 1.9247) 
  -- (1.3062, 1.9223) -- (1.2935, 1.916) -- (1.2673, 1.8982) -- (1.2581, 1.8861)
   -- (1.258, 1.8846) -- (1.2577, 1.8792) -- (1.2671, 1.8512) -- (1.2753, 
  1.8402) -- (1.2918, 1.8165) -- (1.2932, 1.8144) -- (1.2957, 1.8125) -- 
  (1.3046, 1.8126) -- (1.2997, 1.8047) -- (1.2597, 1.7726) -- (1.2565, 1.7706) 
  -- (1.2498, 1.7727) -- (1.2408, 1.7795) -- (1.2275, 1.785) -- (1.1897, 1.7886)
   -- (1.1856, 1.7877) -- (1.1795, 1.7852) -- (1.1741, 1.7789) -- (1.172, 1.78) 
  -- (1.1643, 1.7844) -- (1.1599, 1.7869) -- (1.1481, 1.7983) -- (1.1393, 
  1.8135) -- (1.1317, 1.7901) -- (1.1166, 1.7753) -- (1.1105, 1.7732) -- 
  (1.1075, 1.7743) -- (1.1048, 1.7769) -- (1.103, 1.7815) -- (1.0969, 1.8159) --
   (1.0947, 1.829) -- (1.0954, 1.8338) -- (1.084, 1.8515) -- (1.0638, 1.8669) --
   (1.0424, 1.8747) -- (1.0296, 1.8783) -- (0.9927, 1.8824) -- (0.9796, 1.8742) 
  -- (0.9783, 1.8722) -- (0.9824, 1.8687) -- (0.9877, 1.8612) -- (0.9928, 
  1.8439) -- (0.9924, 1.841) -- (0.9909, 1.8372) -- (0.9823, 1.8236) -- (0.9667,
   1.8209) -- (0.9161, 1.8301) -- (0.9054, 1.8416) -- (0.9003, 1.854) -- 
  (0.9135, 1.8701) -- (0.909, 1.8825) -- (0.8895, 1.8938) -- (0.8902, 1.9106) --
   (0.8898, 1.9123) -- (0.8778, 1.9417) -- (0.8601, 1.9569) -- (0.8453, 1.9651) 
  -- (0.8426, 1.968) -- (0.8339, 1.9788) -- (0.8308, 1.9832) -- (0.8183, 2.0064)
   -- (0.8176, 2.0079) -- (0.8035, 2.0627) -- (0.8029, 2.0667) -- (0.8038, 
  2.0691) -- (0.8102, 2.0741) -- (0.8157, 2.0755) -- (0.812, 2.0812) -- (0.8072,
   2.0865) -- (0.8027, 2.0899) -- (0.7633, 2.1202) -- (0.742, 2.1282) -- 
  (0.7266, 2.125) -- (0.7235, 2.1235) -- (0.7174, 2.122) -- (0.7059, 2.1227) -- 
  (0.7014, 2.128) -- (0.6977, 2.1853) -- (0.6982, 2.187) -- (0.7006, 2.191) -- 
  cycle;

  \node[text=black,line width=0.0092cm,anchor=south east] (text13) at (0.52, 
  1.8635){\resizebox{\ifdim\width>2em 2em\else\width\fi}{!}{#2}};
  \path[draw=black,line width=0.0141cm] (1.0639, 2.0209) -- (text13.east);


}
\newcommand{\drawsachsen}[2]{%
  %Sachsen
  \path[draw=black,fill=#1,line join=round,line width=0.0046cm] (4.7109, 
  4.1579) -- (4.733, 4.1223) -- (4.7383, 4.1194) -- (4.745, 4.1238) -- (4.748, 
  4.1267) -- (4.7497, 4.1295) -- (4.774, 4.1165) -- (4.7781, 4.1091) -- (4.7952,
   4.0715) -- (4.8055, 4.0463) -- (4.8076, 4.0401) -- (4.809, 4.0339) -- (4.807,
   3.9861) -- (4.7999, 3.9523) -- (4.8139, 3.9355) -- (4.816, 3.9297) -- 
  (4.8179, 3.926) -- (4.8388, 3.9255) -- (4.8435, 3.9267) -- (4.8592, 3.9311) --
   (4.8583, 3.9408) -- (4.8582, 3.9422) -- (4.8822, 3.9635) -- (4.9238, 3.9858) 
  -- (4.9292, 3.9817) -- (4.9504, 3.9628) -- (4.9701, 3.9469) -- (5.0151, 
  3.9192) -- (5.0418, 3.9113) -- (5.045, 3.9109) -- (5.0632, 3.9114) -- (5.0936,
   3.9122) -- (5.0979, 3.9184) -- (5.101, 3.9145) -- (5.1086, 3.9103) -- 
  (5.1132, 3.9093) -- (5.1417, 3.9058) -- (5.1494, 3.9111) -- (5.1683, 3.9204) 
  -- (5.1827, 3.9258) -- (5.2083, 3.925) -- (5.2113, 3.9255) -- (5.2803, 3.9397)
   -- (5.3048, 3.9718) -- (5.3178, 4.0224) -- (5.3317, 4.0631) -- (5.3511, 
  4.0913) -- (5.3548, 4.0919) -- (5.3666, 4.09) -- (5.3948, 4.0909) -- (5.4316, 
  4.0861) -- (5.4585, 4.0768) -- (5.4983, 4.0993) -- (5.5299, 4.1033) -- 
  (5.5329, 4.1017) -- (5.5321, 4.1066) -- (5.532, 4.1101) -- (5.5332, 4.1134) --
   (5.5365, 4.1158) -- (5.6002, 4.1384) -- (5.6204, 4.1305) -- (5.6218, 4.1244) 
  -- (5.6258, 4.1175) -- (5.6272, 4.116) -- (5.6311, 4.1151) -- (5.636, 4.1144) 
  -- (5.6652, 4.1173) -- (5.6668, 4.1179) -- (5.6711, 4.1229) -- (5.6726, 
  4.1251) -- (5.6735, 4.1275) -- (5.6795, 4.149) -- (5.687, 4.156) -- (5.696, 
  4.1487) -- (5.6869, 4.1308) -- (5.6866, 4.1294) -- (5.6982, 4.1019) -- 
  (5.7023, 4.0971) -- (5.7121, 4.0935) -- (5.72, 4.0913) -- (5.728, 4.0907) -- 
  (5.7366, 4.0913) -- (5.7389, 4.0912) -- (5.8141, 4.0631) -- (5.8161, 4.0621) 
  -- (5.8312, 4.054) -- (5.8322, 4.0531) -- (5.8362, 4.0468) -- (5.8478, 4.0272)
   -- (5.848, 4.0234) -- (5.8475, 3.9864) -- (5.8922, 3.8906) -- (5.8982, 
  3.8721) -- (5.8979, 3.8456) -- (5.8873, 3.8208) -- (5.8852, 3.8099) -- 
  (5.8673, 3.6783) -- (5.8653, 3.6633) -- (5.8574, 3.6521) -- (5.851, 3.6293) --
   (5.8534, 3.6271) -- (5.8536, 3.6263) -- (5.8516, 3.6165) -- (5.8448, 3.587) 
  -- (5.8338, 3.5534) -- (5.8311, 3.5488) -- (5.8171, 3.5308) -- (5.8044, 
  3.5148) -- (5.7916, 3.4897) -- (5.7887, 3.4676) -- (5.7845, 3.4454) -- 
  (5.7832, 3.4389) -- (5.78, 3.437) -- (5.7668, 3.4358) -- (5.7625, 3.4357) -- 
  (5.743, 3.4389) -- (5.6929, 3.4579) -- (5.6741, 3.4655) -- (5.6732, 3.4725) --
   (5.6809, 3.4949) -- (5.6732, 3.5293) -- (5.6538, 3.5858) -- (5.6416, 3.5976) 
  -- (5.5929, 3.6387) -- (5.5882, 3.6354) -- (5.5896, 3.6292) -- (5.5888, 
  3.6161) -- (5.5452, 3.6098) -- (5.5406, 3.6093) -- (5.5391, 3.6094) -- (5.497,
   3.6252) -- (5.4953, 3.6261) -- (5.4936, 3.6276) -- (5.4847, 3.6381) -- 
  (5.4828, 3.6396) -- (5.4815, 3.6399) -- (5.475, 3.6394) -- (5.4716, 3.6353) --
   (5.4691, 3.6295) -- (5.4521, 3.576) -- (5.4529, 3.5743) -- (5.4862, 3.545) --
   (5.4883, 3.5434) -- (5.4924, 3.5416) -- (5.5246, 3.5331) -- (5.5308, 3.5348) 
  -- (5.5342, 3.5333) -- (5.537, 3.5313) -- (5.5395, 3.5275) -- (5.5408, 3.5191)
   -- (5.5343, 3.4961) -- (5.4859, 3.4787) -- (5.3835, 3.4266) -- (5.3657, 
  3.4168) -- (5.3554, 3.4044) -- (5.35, 3.4024) -- (5.2729, 3.3777) -- (5.2655, 
  3.3771) -- (5.2485, 3.3758) -- (5.2509, 3.3715) -- (5.2529, 3.3424) -- 
  (5.2523, 3.3385) -- (5.2493, 3.3313) -- (5.2267, 3.316) -- (5.2211, 3.3142) --
   (5.2103, 3.3125) -- (5.209, 3.3127) -- (5.2051, 3.3139) -- (5.2, 3.3175) -- 
  (5.1972, 3.3185) -- (5.1799, 3.3223) -- (5.1785, 3.3225) -- (5.1146, 3.3141) 
  -- (5.1056, 3.3128) -- (5.0392, 3.2731) -- (5.041, 3.2684) -- (5.0431, 3.2599)
   -- (5.0427, 3.2587) -- (5.0259, 3.2206) -- (5.0175, 3.2154) -- (5.0, 3.193) 
  -- (4.9645, 3.2104) -- (4.9449, 3.2122) -- (4.891, 3.1709) -- (4.8602, 3.1369)
   -- (4.861, 3.1306) -- (4.844, 3.0892) -- (4.8302, 3.0862) -- (4.802, 3.0951) 
  -- (4.7514, 3.0906) -- (4.7459, 3.0885) -- (4.7445, 3.0879) -- (4.7441, 
  3.0863) -- (4.7402, 3.0667) -- (4.7389, 3.0479) -- (4.74, 3.0413) -- (4.7415, 
  3.0378) -- (4.7413, 3.0332) -- (4.7412, 3.0329) -- (4.7404, 3.0295) -- 
  (4.7384, 3.0261) -- (4.7199, 3.0031) -- (4.7163, 3.0006) -- (4.697, 2.9887) --
   (4.6901, 2.9912) -- (4.6634, 3.0085) -- (4.6251, 3.0331) -- (4.619, 3.0363) 
  -- (4.6185, 3.036) -- (4.5701, 3.0069) -- (4.5361, 2.9878) -- (4.526, 2.9865) 
  -- (4.4797, 2.9805) -- (4.4388, 2.9691) -- (4.371, 2.8841) -- (4.3517, 2.8556)
   -- (4.334, 2.8209) -- (4.3321, 2.814) -- (4.3296, 2.7886) -- (4.331, 2.7845) 
  -- (4.3322, 2.7832) -- (4.3343, 2.7821) -- (4.3362, 2.7803) -- (4.3394, 
  2.7737) -- (4.3379, 2.7564) -- (4.3354, 2.7546) -- (4.3282, 2.7546) -- 
  (4.3212, 2.7555) -- (4.3122, 2.7574) -- (4.3104, 2.7584) -- (4.3093, 2.7596) 
  -- (4.302, 2.7714) -- (4.2995, 2.7766) -- (4.2999, 2.7789) -- (4.3047, 2.781) 
  -- (4.3064, 2.7822) -- (4.3082, 2.7896) -- (4.3061, 2.8045) -- (4.2528, 
  2.8811) -- (4.2424, 2.893) -- (4.238, 2.8932) -- (4.22, 2.893) -- (4.1951, 
  2.8922) -- (4.1915, 2.8893) -- (4.1872, 2.8952) -- (4.169, 2.9021) -- (4.128, 
  2.9142) -- (4.1174, 2.9159) -- (4.1137, 2.9187) -- (4.1166, 2.9347) -- 
  (4.1186, 2.9505) -- (4.119, 2.9543) -- (4.0955, 2.9786) -- (4.0893, 2.9844) --
   (4.0808, 2.9861) -- (4.0664, 2.9962) -- (4.0619, 3.0007) -- (4.0607, 3.0025) 
  -- (4.0647, 3.0089) -- (4.0703, 3.0135) -- (4.0853, 3.0214) -- (4.1022, 
  3.0373) -- (4.1053, 3.0422) -- (4.1049, 3.0446) -- (4.1027, 3.0492) -- 
  (4.0845, 3.0662) -- (4.0769, 3.0732) -- (4.0666, 3.0719) -- (4.0582, 3.0721) 
  -- (4.0554, 3.0768) -- (4.0485, 3.0933) -- (4.0487, 3.0941) -- (4.0498, 
  3.0972) -- (4.0785, 3.1376) -- (4.082, 3.1396) -- (4.0886, 3.1427) -- (4.0906,
   3.1435) -- (4.1111, 3.1647) -- (4.1116, 3.1722) -- (4.124, 3.1835) -- 
  (4.1252, 3.1843) -- (4.1287, 3.1853) -- (4.1356, 3.178) -- (4.1396, 3.1498) --
   (4.1354, 3.124) -- (4.1717, 3.1258) -- (4.1944, 3.123) -- (4.205, 3.13) -- 
  (4.2082, 3.1287) -- (4.216, 3.1296) -- (4.2179, 3.1306) -- (4.2189, 3.1314) --
   (4.2202, 3.1363) -- (4.2181, 3.14) -- (4.2172, 3.1404) -- (4.2109, 3.1406) --
   (4.2077, 3.1376) -- (4.1897, 3.143) -- (4.1924, 3.1507) -- (4.1943, 3.1521) 
  -- (4.1991, 3.1541) -- (4.2287, 3.1671) -- (4.2621, 3.1815) -- (4.2635, 
  3.1813) -- (4.2812, 3.1853) -- (4.3, 3.2049) -- (4.3008, 3.2059) -- (4.3107, 
  3.2191) -- (4.3128, 3.2309) -- (4.2857, 3.2382) -- (4.2784, 3.2454) -- 
  (4.2556, 3.286) -- (4.2554, 3.2886) -- (4.2551, 3.2933) -- (4.2559, 3.3029) --
   (4.2628, 3.3528) -- (4.2672, 3.3616) -- (4.2724, 3.3548) -- (4.2845, 3.3535) 
  -- (4.3247, 3.3679) -- (4.3245, 3.3806) -- (4.3246, 3.3812) -- (4.3396, 
  3.3929) -- (4.3423, 3.3937) -- (4.3567, 3.3937) -- (4.3622, 3.3917) -- 
  (4.3673, 3.385) -- (4.3677, 3.3844) -- (4.3793, 3.3878) -- (4.3811, 3.3886) --
   (4.3896, 3.4127) -- (4.3899, 3.4241) -- (4.4032, 3.4355) -- (4.4056, 3.4368) 
  -- (4.4351, 3.4443) -- (4.4524, 3.447) -- (4.4576, 3.4457) -- (4.4706, 3.448) 
  -- (4.4963, 3.4564) -- (4.5037, 3.4677) -- (4.4988, 3.4711) -- (4.4964, 
  3.4734) -- (4.4903, 3.4835) -- (4.4889, 3.4858) -- (4.4836, 3.5067) -- 
  (4.4804, 3.5218) -- (4.4582, 3.5274) -- (4.4364, 3.5357) -- (4.4137, 3.5527) 
  -- (4.4109, 3.5689) -- (4.4116, 3.5804) -- (4.4127, 3.5864) -- (4.4105, 
  3.5945) -- (4.4088, 3.597) -- (4.4053, 3.5993) -- (4.3788, 3.6091) -- (4.3082,
   3.631) -- (4.3073, 3.6309) -- (4.2799, 3.6175) -- (4.2672, 3.6264) -- 
  (4.2461, 3.6449) -- (4.2522, 3.6501) -- (4.2527, 3.6537) -- (4.2477, 3.659) --
   (4.2449, 3.6604) -- (4.2262, 3.6644) -- (4.2264, 3.7307) -- (4.223, 3.7529) 
  -- (4.2226, 3.7554) -- (4.2189, 3.7728) -- (4.2073, 3.7896) -- (4.1919, 
  3.8308) -- (4.2066, 3.8388) -- (4.2188, 3.8434) -- (4.2171, 3.858) -- (4.2082,
   3.9081) -- (4.2062, 3.9144) -- (4.2058, 3.9152) -- (4.2045, 3.9158) -- 
  (4.1953, 3.9468) -- (4.1917, 3.96) -- (4.2141, 4.0087) -- (4.2159, 4.0306) -- 
  (4.2477, 4.0589) -- (4.296, 4.0843) -- (4.302, 4.0851) -- (4.3253, 4.0856) -- 
  (4.3464, 4.0876) -- (4.4154, 4.1074) -- (4.4336, 4.1164) -- (4.4366, 4.1267) 
  -- (4.4552, 4.114) -- (4.4696, 4.121) -- (4.4772, 4.1266) -- (4.4847, 4.1324) 
  -- (4.4867, 4.1341) -- (4.493, 4.151) -- (4.494, 4.1565) -- (4.4951, 4.1586) 
  -- (4.4976, 4.1615) -- (4.5009, 4.1635) -- (4.5102, 4.1635) -- (4.515, 4.1627)
   -- (4.5182, 4.1617) -- (4.5274, 4.1584) -- (4.5339, 4.1537) -- (4.5348, 
  4.1531) -- (4.5433, 4.1499) -- (4.6306, 4.1516) -- (4.6312, 4.1516) -- 
  (4.6913, 4.1545) -- cycle;

  \node[text=black,line width=0.0092cm,anchor=center] (text18) at (4.8893, 
  3.5789){\resizebox{\ifdim\width>1.5em 1.5em\else\width\fi}{!}{#2}};
}
\newcommand{\drawsachsenanhalt}[2]{%
  %Sachsen-Anhalt
  \path[draw=black,fill=#1,line join=round,line width=0.0046cm] (3.8323, 
  5.4372) -- (3.8451, 5.4424) -- (3.8475, 5.4428) -- (3.8519, 5.4424) -- 
  (3.8539, 5.4419) -- (3.856, 5.4408) -- (3.8579, 5.4391) -- (3.8586, 5.4369) --
   (3.8584, 5.435) -- (3.8574, 5.4335) -- (3.8495, 5.4229) -- (3.849, 5.4201) --
   (3.8495, 5.4166) -- (3.8508, 5.414) -- (3.853, 5.4117) -- (3.8858, 5.3856) --
   (3.8877, 5.3843) -- (3.8926, 5.3834) -- (3.8954, 5.3842) -- (3.9022, 5.389) 
  -- (3.9113, 5.3943) -- (3.9129, 5.3946) -- (3.9178, 5.3938) -- (3.9214, 
  5.3925) -- (3.9249, 5.3909) -- (3.927, 5.3898) -- (3.9744, 5.3623) -- (3.9767,
   5.3595) -- (3.9775, 5.3571) -- (3.9778, 5.3545) -- (3.9776, 5.3523) -- 
  (3.9768, 5.3498) -- (3.9755, 5.348) -- (3.9706, 5.3439) -- (3.9658, 5.3388) --
   (3.9641, 5.3352) -- (3.9636, 5.3333) -- (3.9642, 5.3289) -- (3.9665, 5.3243) 
  -- (3.9686, 5.3223) -- (3.9718, 5.3206) -- (4.0047, 5.3072) -- (4.0525, 
  5.2918) -- (4.0552, 5.2913) -- (4.1306, 5.3002) -- (4.1374, 5.3066) -- 
  (4.1362, 5.305) -- (4.1339, 5.2991) -- (4.1336, 5.2929) -- (4.1413, 5.2746) --
   (4.1564, 5.2797) -- (4.1648, 5.2841) -- (4.179, 5.2905) -- (4.1925, 5.2844) 
  -- (4.1983, 5.2806) -- (4.1991, 5.2781) -- (4.2104, 5.2332) -- (4.2086, 
  5.2166) -- (4.1952, 5.2099) -- (4.189, 5.1981) -- (4.1858, 5.1876) -- (4.1819,
   5.1468) -- (4.1822, 5.1459) -- (4.1936, 5.1273) -- (4.2045, 5.1109) -- 
  (4.207, 5.1027) -- (4.2072, 5.0974) -- (4.2034, 5.0637) -- (4.1885, 5.0527) --
   (4.1846, 5.0529) -- (4.1808, 5.054) -- (4.1725, 5.0391) -- (4.1562, 4.9738) 
  -- (4.1559, 4.9669) -- (4.1562, 4.9665) -- (4.173, 4.9466) -- (4.1809, 4.9372)
   -- (4.1917, 4.939) -- (4.202, 4.942) -- (4.2049, 4.9463) -- (4.2052, 4.9476) 
  -- (4.2069, 4.9562) -- (4.2578, 4.9386) -- (4.2612, 4.9086) -- (4.2583, 
  4.9001) -- (4.253, 4.8907) -- (4.2451, 4.8709) -- (4.2468, 4.8558) -- (4.2505,
   4.8296) -- (4.2565, 4.7982) -- (4.2456, 4.7762) -- (4.2433, 4.7751) -- 
  (4.2423, 4.7735) -- (4.2316, 4.75) -- (4.2264, 4.7336) -- (4.2252, 4.7292) -- 
  (4.2227, 4.7065) -- (4.2248, 4.6696) -- (4.2081, 4.6316) -- (4.2349, 4.5972) 
  -- (4.254, 4.5677) -- (4.2783, 4.5361) -- (4.2936, 4.5188) -- (4.3357, 4.4926)
   -- (4.4018, 4.4669) -- (4.4381, 4.4614) -- (4.4452, 4.4626) -- (4.4619, 
  4.4744) -- (4.4633, 4.4868) -- (4.4674, 4.4922) -- (4.4758, 4.4924) -- 
  (4.4941, 4.4839) -- (4.5005, 4.4798) -- (4.5152, 4.4716) -- (4.5318, 4.463) --
   (4.535, 4.4628) -- (4.5394, 4.4496) -- (4.5846, 4.4236) -- (4.6068, 4.418) --
   (4.6115, 4.4178) -- (4.6213, 4.4213) -- (4.666, 4.3976) -- (4.6937, 4.3689) 
  -- (4.6965, 4.3671) -- (4.6979, 4.3669) -- (4.7195, 4.365) -- (4.7384, 4.3727)
   -- (4.7415, 4.3746) -- (4.7409, 4.3807) -- (4.7492, 4.3803) -- (4.7529, 
  4.3779) -- (4.7596, 4.3713) -- (4.7609, 4.3657) -- (4.761, 4.36) -- (4.7451, 
  4.3574) -- (4.7452, 4.3554) -- (4.7453, 4.3531) -- (4.7757, 4.2457) -- 
  (4.7801, 4.2418) -- (4.7827, 4.2394) -- (4.7869, 4.2308) -- (4.7873, 4.2264) 
  -- (4.7873, 4.2252) -- (4.7745, 4.2021) -- (4.7697, 4.1969) -- (4.7106, 
  4.1577) -- (4.6908, 4.155) -- (4.6307, 4.152) -- (4.6302, 4.152) -- (4.5429, 
  4.1503) -- (4.5344, 4.1535) -- (4.5334, 4.1541) -- (4.527, 4.1589) -- (4.5177,
   4.1621) -- (4.5146, 4.1631) -- (4.5097, 4.164) -- (4.5005, 4.164) -- (4.4972,
   4.1619) -- (4.4947, 4.159) -- (4.4935, 4.1569) -- (4.4926, 4.1515) -- 
  (4.4863, 4.1345) -- (4.4842, 4.1329) -- (4.4768, 4.127) -- (4.4692, 4.1214) --
   (4.4548, 4.1144) -- (4.4362, 4.1271) -- (4.4332, 4.1168) -- (4.4149, 4.1078) 
  -- (4.3459, 4.0881) -- (4.3249, 4.086) -- (4.3016, 4.0855) -- (4.2955, 4.0848)
   -- (4.2472, 4.0593) -- (4.2155, 4.031) -- (4.2136, 4.0091) -- (4.1913, 
  3.9604) -- (4.1948, 3.9473) -- (4.2041, 3.9162) -- (4.2054, 3.9156) -- 
  (4.2057, 3.9149) -- (4.2078, 3.9086) -- (4.2167, 3.8584) -- (4.2183, 3.8439) 
  -- (4.2062, 3.8392) -- (4.1915, 3.8313) -- (4.2068, 3.7901) -- (4.2184, 
  3.7733) -- (4.2221, 3.7559) -- (4.2225, 3.7534) -- (4.2259, 3.7312) -- 
  (4.2258, 3.6649) -- (4.2444, 3.6609) -- (4.2473, 3.6595) -- (4.2522, 3.6541) 
  -- (4.2517, 3.6505) -- (4.2456, 3.6454) -- (4.2667, 3.6269) -- (4.2794, 
  3.6179) -- (4.2647, 3.5949) -- (4.267, 3.5795) -- (4.2789, 3.5707) -- (4.2824,
   3.5686) -- (4.2868, 3.5628) -- (4.2864, 3.5535) -- (4.2859, 3.5506) -- 
  (4.2567, 3.4876) -- (4.2501, 3.4792) -- (4.2479, 3.4779) -- (4.2426, 3.4742) 
  -- (4.2422, 3.4745) -- (4.2398, 3.4844) -- (4.2357, 3.4921) -- (4.2228, 
  3.5086) -- (4.2219, 3.5095) -- (4.2156, 3.5045) -- (4.2131, 3.5017) -- 
  (4.2109, 3.4991) -- (4.2115, 3.4925) -- (4.2016, 3.4971) -- (4.1684, 3.5081) 
  -- (4.1661, 3.5079) -- (4.1366, 3.5031) -- (4.1334, 3.5023) -- (4.1246, 
  3.4985) -- (4.1242, 3.5066) -- (4.124, 3.5088) -- (4.1224, 3.5137) -- (4.1009,
   3.5413) -- (4.0985, 3.5434) -- (4.0786, 3.5577) -- (4.047, 3.5779) -- 
  (4.0393, 3.5804) -- (4.0207, 3.5754) -- (3.995, 3.5701) -- (3.9784, 3.5746) --
   (3.9346, 3.5885) -- (3.9333, 3.5937) -- (3.9316, 3.6059) -- (3.9287, 3.617) 
  -- (3.9144, 3.6259) -- (3.8567, 3.635) -- (3.8548, 3.6333) -- (3.8517, 3.6264)
   -- (3.8493, 3.6243) -- (3.8416, 3.6178) -- (3.8061, 3.617) -- (3.8056, 
  3.6195) -- (3.7973, 3.6234) -- (3.7929, 3.6267) -- (3.7852, 3.6582) -- 
  (3.7973, 3.6604) -- (3.7998, 3.665) -- (3.8047, 3.6765) -- (3.8047, 3.679) -- 
  (3.7948, 3.7015) -- (3.7804, 3.7162) -- (3.7725, 3.7195) -- (3.7703, 3.7197) 
  -- (3.7616, 3.7156) -- (3.7471, 3.716) -- (3.7328, 3.7273) -- (3.7338, 3.7347)
   -- (3.7391, 3.7439) -- (3.7422, 3.7488) -- (3.7442, 3.7502) -- (3.7459, 
  3.7503) -- (3.7528, 3.7425) -- (3.7533, 3.7424) -- (3.7576, 3.744) -- (3.7601,
   3.7459) -- (3.7649, 3.7513) -- (3.7724, 3.762) -- (3.7858, 3.7816) -- 
  (3.7923, 3.7917) -- (3.7959, 3.798) -- (3.7965, 3.7994) -- (3.7961, 3.8019) --
   (3.7919, 3.8065) -- (3.7799, 3.8173) -- (3.7763, 3.8192) -- (3.7734, 3.8202) 
  -- (3.7673, 3.8261) -- (3.7627, 3.832) -- (3.7629, 3.8333) -- (3.7635, 3.8347)
   -- (3.7683, 3.8393) -- (3.7518, 3.8448) -- (3.7516, 3.8485) -- (3.7465, 
  3.8578) -- (3.7491, 3.875) -- (3.7487, 3.8781) -- (3.7465, 3.8808) -- (3.7081,
   3.8956) -- (3.6827, 3.8966) -- (3.6763, 3.8953) -- (3.6634, 3.895) -- 
  (3.6044, 3.8963) -- (3.6001, 3.8967) -- (3.541, 3.9154) -- (3.5352, 3.9126) --
   (3.5246, 3.9088) -- (3.5177, 3.9081) -- (3.51, 3.9091) -- (3.5021, 3.9179) --
   (3.495, 3.9221) -- (3.4962, 3.947) -- (3.4976, 3.9689) -- (3.4888, 3.9801) --
   (3.4734, 4.0039) -- (3.4542, 4.0338) -- (3.4431, 4.0603) -- (3.4475, 4.075) 
  -- (3.4628, 4.0955) -- (3.4512, 4.0906) -- (3.4332, 4.1073) -- (3.4292, 
  4.1101) -- (3.4025, 4.1164) -- (3.3849, 4.1179) -- (3.3538, 4.1205) -- 
  (3.3369, 4.1168) -- (3.3335, 4.1179) -- (3.333, 4.1183) -- (3.3303, 4.1221) --
   (3.3187, 4.1407) -- (3.3174, 4.1443) -- (3.3154, 4.156) -- (3.3146, 4.1639) 
  -- (3.3165, 4.1687) -- (3.3079, 4.1892) -- (3.2803, 4.23) -- (3.2708, 4.2352) 
  -- (3.2644, 4.2487) -- (3.262, 4.2746) -- (3.2619, 4.3063) -- (3.2632, 4.311) 
  -- (3.2665, 4.3152) -- (3.2862, 4.3315) -- (3.2897, 4.333) -- (3.2932, 4.3331)
   -- (3.2949, 4.3343) -- (3.3042, 4.349) -- (3.3052, 4.3522) -- (3.3052, 
  4.3589) -- (3.3045, 4.3649) -- (3.3036, 4.3676) -- (3.2988, 4.3735) -- 
  (3.2914, 4.3765) -- (3.2832, 4.3795) -- (3.283, 4.3944) -- (3.2832, 4.3961) --
   (3.2901, 4.4062) -- (3.3018, 4.41) -- (3.2676, 4.4358) -- (3.2521, 4.4576) --
   (3.2907, 4.4616) -- (3.3069, 4.4776) -- (3.3074, 4.4789) -- (3.3078, 4.4815) 
  -- (3.3064, 4.4856) -- (3.3073, 4.4943) -- (3.325, 4.4996) -- (3.3343, 4.5015)
   -- (3.3805, 4.502) -- (3.3828, 4.502) -- (3.3915, 4.5) -- (3.4082, 4.5005) --
   (3.4209, 4.5032) -- (3.4242, 4.5046) -- (3.4328, 4.5107) -- (3.4376, 4.5115) 
  -- (3.469, 4.5125) -- (3.4847, 4.511) -- (3.4857, 4.5101) -- (3.4885, 4.5384) 
  -- (3.492, 4.556) -- (3.5106, 4.5693) -- (3.5279, 4.5834) -- (3.5388, 4.5997) 
  -- (3.5403, 4.6022) -- (3.5407, 4.6156) -- (3.5398, 4.6188) -- (3.5392, 
  4.6195) -- (3.5377, 4.6197) -- (3.5296, 4.6186) -- (3.5253, 4.6189) -- 
  (3.5192, 4.6226) -- (3.5158, 4.6253) -- (3.5141, 4.627) -- (3.5129, 4.6299) --
   (3.5123, 4.6337) -- (3.5134, 4.6406) -- (3.5172, 4.6502) -- (3.5198, 4.6552) 
  -- (3.5228, 4.6571) -- (3.532, 4.6614) -- (3.5384, 4.663) -- (3.5483, 4.6651) 
  -- (3.5537, 4.6722) -- (3.554, 4.6728) -- (3.5538, 4.6736) -- (3.5498, 4.6824)
   -- (3.5366, 4.7008) -- (3.5242, 4.7133) -- (3.511, 4.7277) -- (3.5105, 
  4.7286) -- (3.5067, 4.7662) -- (3.518, 4.7819) -- (3.5231, 4.7807) -- (3.5248,
   4.7812) -- (3.5351, 4.7859) -- (3.5425, 4.7918) -- (3.543, 4.7935) -- 
  (3.5434, 4.8064) -- (3.5429, 4.81) -- (3.5372, 4.8155) -- (3.5302, 4.8216) -- 
  (3.5281, 4.8226) -- (3.5252, 4.8231) -- (3.5133, 4.8304) -- (3.4718, 4.8834) 
  -- (3.4639, 4.9002) -- (3.4697, 4.9145) -- (3.4902, 4.9256) -- (3.5062, 
  4.9249) -- (3.4894, 4.9443) -- (3.4774, 4.9597) -- (3.4704, 4.9715) -- 
  (3.4673, 4.9784) -- (3.4642, 4.9908) -- (3.4659, 5.0109) -- (3.466, 5.0113) --
   (3.4403, 5.061) -- (3.4061, 5.1098) -- (3.3822, 5.1274) -- (3.358, 5.1934) --
   (3.3588, 5.1986) -- (3.3602, 5.2041) -- (3.3581, 5.225) -- (3.3583, 5.2268) 
  -- (3.363, 5.2469) -- (3.3819, 5.2522) -- (3.4064, 5.2573) -- (3.462, 5.2587) 
  -- (3.4882, 5.2871) -- (3.49, 5.291) -- (3.4902, 5.2925) -- (3.4894, 5.2966) 
  -- (3.4898, 5.3035) -- (3.4913, 5.3095) -- (3.4943, 5.313) -- (3.4967, 5.314) 
  -- (3.4998, 5.3143) -- (3.5847, 5.3106) -- (3.6205, 5.3036) -- (3.6273, 
  5.2965) -- (3.6302, 5.2935) -- (3.6314, 5.2906) -- (3.6314, 5.2881) -- 
  (3.6318, 5.2869) -- (3.6327, 5.2863) -- (3.6356, 5.2857) -- (3.6646, 5.2827) 
  -- (3.6749, 5.2856) -- (3.7271, 5.3111) -- (3.7285, 5.3125) -- (3.7367, 
  5.3232) -- (3.7528, 5.3388) -- (3.7631, 5.3446) -- (3.7743, 5.3472) -- 
  (3.7822, 5.3471) -- (3.7749, 5.3643) -- (3.7802, 5.3937) -- (3.7822, 5.403) --
   (3.7853, 5.4095) -- (3.793, 5.4109) -- (3.7937, 5.4105) -- (3.7972, 5.4057) 
  -- (3.8032, 5.4024) -- (3.81, 5.4017) -- (3.8103, 5.4022) -- (3.833, 5.4356) 
  -- (3.8331, 5.4364) -- cycle;


  \node[text=black,line width=0.0092cm,anchor=center] (text17) at (3.9258, 
  4.4537){\resizebox{\ifdim\width>1.5em 1.5em\else\width\fi}{!}{#2}};
}
\newcommand{\drawschleswig}[2]{%
  %Schleswig Holstein
  \path[draw=black,fill=#1,line join=round,line width=0.0046cm] (3.4288, 
  6.2971) -- (3.4371, 6.2695) -- (3.4284, 6.2499) -- (3.4043, 6.2517) -- 
  (3.3564, 6.218) -- (3.3434, 6.1972) -- (3.3436, 6.1806) -- (3.3527, 6.1579) --
   (3.3545, 6.1544) -- (3.3576, 6.1462) -- (3.3582, 6.1415) -- (3.3565, 6.1278) 
  -- (3.3527, 6.0991) -- (3.3876, 6.0652) -- (3.4016, 6.0567) -- (3.4036, 
  6.0558) -- (3.4151, 6.0564) -- (3.4188, 6.0574) -- (3.4191, 6.0591) -- 
  (3.4211, 6.0603) -- (3.4226, 6.0606) -- (3.433, 6.0585) -- (3.4412, 6.0545) --
   (3.4533, 6.0412) -- (3.4598, 6.0065) -- (3.447, 5.9887) -- (3.4433, 5.9664) 
  -- (3.3832, 5.8867) -- (3.3635, 5.8635) -- (3.3543, 5.8534) -- (3.3515, 
  5.8505) -- (3.3163, 5.8239) -- (3.3037, 5.8266) -- (3.2977, 5.8291) -- 
  (3.2931, 5.8294) -- (3.2839, 5.8241) -- (3.2783, 5.8123) -- (3.262, 5.738) -- 
  (3.2625, 5.7377) -- (3.2617, 5.7372) -- (3.2577, 5.7348) -- (3.2548, 5.7335) 
  -- (3.2503, 5.7339) -- (3.2124, 5.745) -- (3.2071, 5.7472) -- (3.1987, 5.7521)
   -- (3.1912, 5.7574) -- (3.169, 5.7689) -- (3.1149, 5.7936) -- (3.1002, 
  5.8016) -- (3.0952, 5.8111) -- (3.061, 5.8479) -- (3.0588, 5.8552) -- (3.0601,
   5.8609) -- (3.0557, 5.8603) -- (3.0522, 5.8611) -- (3.0202, 5.8835) -- 
  (3.0182, 5.8853) -- (3.0104, 5.9075) -- (3.0098, 5.9266) -- (3.0125, 5.9375) 
  -- (3.0185, 5.9433) -- (3.0374, 5.9585) -- (3.0489, 5.983) -- (3.0492, 5.9839)
   -- (3.0513, 5.99) -- (3.0367, 6.0099) -- (3.0298, 6.0181) -- (3.0285, 6.0609)
   -- (3.0341, 6.0739) -- (3.0352, 6.0818) -- (3.0234, 6.0885) -- (3.0216, 
  6.0892) -- (3.019, 6.0896) -- (2.9797, 6.0706) -- (2.9666, 6.062) -- (2.9711, 
  6.0563) -- (2.9638, 6.0384) -- (2.919, 6.0038) -- (2.9121, 6.0038) -- (2.9075,
   6.0043) -- (2.8987, 6.0063) -- (2.8964, 6.0081) -- (2.885, 6.0084) -- 
  (2.8747, 6.0078) -- (2.874, 6.0008) -- (2.8741, 5.9978) -- (2.8711, 5.9927) --
   (2.8677, 5.9875) -- (2.8354, 5.947) -- (2.8284, 5.9436) -- (2.8236, 5.9457) 
  -- (2.8139, 5.9546) -- (2.7983, 5.9736) -- (2.7972, 5.9798) -- (2.798, 5.9845)
   -- (2.7945, 5.9843) -- (2.7923, 5.9814) -- (2.7857, 5.9622) -- (2.7833, 
  5.954) -- (2.7795, 5.9397) -- (2.7754, 5.9189) -- (2.76, 5.9205) -- (2.7449, 
  5.9256) -- (2.6957, 5.9577) -- (2.6749, 5.9743) -- (2.6709, 5.98) -- (2.6672, 
  5.988) -- (2.664, 5.9995) -- (2.6632, 6.0142) -- (2.6623, 6.0185) -- (2.6585, 
  6.0278) -- (2.6513, 6.0451) -- (2.6481, 6.0506) -- (2.6391, 6.0606) -- 
  (2.6177, 6.0765) -- (2.6098, 6.0838) -- (2.6044, 6.0914) -- (2.5976, 6.1031) 
  -- (2.5906, 6.1201) -- (2.5739, 6.1561) -- (2.5691, 6.1649) -- (2.5474, 
  6.1961) -- (2.5309, 6.208) -- (2.4984, 6.2221) -- (2.481, 6.2244) -- (2.462, 
  6.2241) -- (2.4287, 6.2246) -- (2.3994, 6.2266) -- (2.3901, 6.2263) -- 
  (2.3824, 6.2249) -- (2.38, 6.2489) -- (2.3784, 6.2485) -- (2.3649, 6.2417) -- 
  (2.362, 6.2406) -- (2.3577, 6.2407) -- (2.3544, 6.2421) -- (2.3448, 6.2499) --
   (2.3273, 6.2642) -- (2.3187, 6.2763) -- (2.3128, 6.292) -- (2.3098, 6.3021) 
  -- (2.2976, 6.3315) -- (2.2785, 6.3618) -- (2.2751, 6.3614) -- (2.2722, 
  6.3618) -- (2.2717, 6.3638) -- (2.2855, 6.3752) -- (2.2963, 6.3812) -- 
  (2.3051, 6.3831) -- (2.328, 6.3776) -- (2.3324, 6.3746) -- (2.3364, 6.3702) --
   (2.3424, 6.3693) -- (2.3501, 6.3713) -- (2.3548, 6.3733) -- (2.357, 6.375) --
   (2.3594, 6.3776) -- (2.3614, 6.3811) -- (2.3628, 6.3864) -- (2.3643, 6.3925) 
  -- (2.3644, 6.3948) -- (2.3641, 6.3976) -- (2.3632, 6.4009) -- (2.3336, 6.462)
   -- (2.3314, 6.4647) -- (2.3251, 6.4666) -- (2.3227, 6.4671) -- (2.314, 
  6.4654) -- (2.3104, 6.4631) -- (2.2989, 6.4555) -- (2.2941, 6.4585) -- 
  (2.2841, 6.4649) -- (2.2828, 6.4666) -- (2.2779, 6.4759) -- (2.2685, 6.5015) 
  -- (2.2676, 6.5058) -- (2.2678, 6.5074) -- (2.27, 6.5129) -- (2.2797, 6.5335) 
  -- (2.2856, 6.546) -- (2.2865, 6.5499) -- (2.2893, 6.5819) -- (2.2898, 6.5885)
   -- (2.2898, 6.5903) -- (2.2877, 6.5908) -- (2.2893, 6.5921) -- (2.2893, 
  6.5949) -- (2.2888, 6.6002) -- (2.2876, 6.6041) -- (2.2851, 6.6088) -- 
  (2.2817, 6.6132) -- (2.2801, 6.6144) -- (2.2774, 6.6151) -- (2.2425, 6.6152) 
  -- (2.2356, 6.6148) -- (2.2133, 6.6131) -- (2.204, 6.604) -- (2.1982, 6.5971) 
  -- (2.1987, 6.5965) -- (2.2031, 6.5955) -- (2.2044, 6.5954) -- (2.2052, 
  6.5959) -- (2.2055, 6.5957) -- (2.205, 6.5947) -- (2.204, 6.5935) -- (2.2026, 
  6.5929) -- (2.1993, 6.5926) -- (2.1956, 6.593) -- (2.1549, 6.6145) -- (2.1497,
   6.618) -- (2.1447, 6.6275) -- (2.1438, 6.6323) -- (2.1443, 6.6391) -- 
  (2.1466, 6.6523) -- (2.1486, 6.6597) -- (2.1518, 6.6677) -- (2.1536, 6.6712) 
  -- (2.1546, 6.6723) -- (2.163, 6.6766) -- (2.1635, 6.6683) -- (2.1748, 6.6603)
   -- (2.18, 6.6607) -- (2.1815, 6.6613) -- (2.2054, 6.6759) -- (2.2066, 6.6768)
   -- (2.2074, 6.6788) -- (2.2074, 6.6871) -- (2.2072, 6.6887) -- (2.2066, 
  6.6938) -- (2.2115, 6.7159) -- (2.2723, 6.7265) -- (2.2842, 6.7252) -- 
  (2.2969, 6.7192) -- (2.3069, 6.7219) -- (2.3184, 6.7271) -- (2.3284, 6.7324) 
  -- (2.3786, 6.7747) -- (2.3799, 6.7761) -- (2.3858, 6.7859) -- (2.3874, 6.796)
   -- (2.3872, 6.7992) -- (2.3853, 6.8063) -- (2.3737, 6.8321) -- (2.3148, 
  6.9157) -- (2.2875, 6.9545) -- (2.2767, 6.961) -- (2.2601, 6.9688) -- (2.2493,
   6.9795) -- (2.2461, 6.9835) -- (2.2259, 7.036) -- (2.2188, 7.0707) -- 
  (2.2108, 7.0874) -- (2.2097, 7.0889) -- (2.1891, 7.1133) -- (2.1697, 7.1349) 
  -- (2.1694, 7.1368) -- (2.1729, 7.1699) -- (2.1865, 7.2008) -- (2.2033, 
  7.2005) -- (2.2096, 7.1997) -- (2.2134, 7.1979) -- (2.2244, 7.1904) -- 
  (2.2743, 7.193) -- (2.2807, 7.1941) -- (2.2877, 7.1948) -- (2.2926, 7.1945) --
   (2.33, 7.192) -- (2.3548, 7.1899) -- (2.36, 7.1884) -- (2.3623, 7.1874) -- 
  (2.3908, 7.1742) -- (2.3989, 7.1684) -- (2.3986, 7.1665) -- (2.4026, 7.164) --
   (2.409, 7.1603) -- (2.4171, 7.1595) -- (2.4322, 7.1584) -- (2.4442, 7.1616) 
  -- (2.4566, 7.1615) -- (2.4603, 7.1608) -- (2.467, 7.1586) -- (2.494, 7.1473) 
  -- (2.5131, 7.1385) -- (2.5156, 7.1327) -- (2.5122, 7.1283) -- (2.5103, 
  7.1247) -- (2.5106, 7.1209) -- (2.5119, 7.1168) -- (2.5182, 7.1005) -- 
  (2.5613, 7.0962) -- (2.5706, 7.0971) -- (2.584, 7.1089) -- (2.5856, 7.1106) --
   (2.5863, 7.1164) -- (2.6102, 7.1211) -- (2.6203, 7.1167) -- (2.6327, 7.1217) 
  -- (2.645, 7.1251) -- (2.6535, 7.1291) -- (2.7006, 7.1711) -- (2.7049, 7.172) 
  -- (2.7247, 7.1649) -- (2.721, 7.1589) -- (2.7115, 7.1415) -- (2.7508, 7.1254)
   -- (2.7812, 7.113) -- (2.8353, 7.0517) -- (2.8446, 7.0468) -- (2.8528, 
  7.0445) -- (2.8576, 7.044) -- (2.8584, 7.044) -- (2.8594, 7.045) -- (2.8742, 
  7.0769) -- (2.8734, 7.0789) -- (2.875, 7.0834) -- (2.8767, 7.0863) -- (2.8776,
   7.087) -- (2.8873, 7.081) -- (2.9032, 7.0702) -- (2.913, 7.0522) -- (2.944, 
  6.9551) -- (2.9435, 6.9352) -- (2.9419, 6.9074) -- (2.9417, 6.858) -- (2.9413,
   6.8539) -- (2.9259, 6.8275) -- (2.9242, 6.8251) -- (2.9184, 6.8184) -- 
  (2.9089, 6.8104) -- (2.8942, 6.8019) -- (2.8705, 6.7897) -- (2.8544, 6.7818) 
  -- (2.8506, 6.7809) -- (2.8484, 6.7825) -- (2.8467, 6.7847) -- (2.8411, 
  6.7849) -- (2.8381, 6.7842) -- (2.8356, 6.7829) -- (2.8346, 6.7816) -- 
  (2.8411, 6.7687) -- (2.8436, 6.7644) -- (2.8469, 6.7607) -- (2.8514, 6.758) --
   (2.8539, 6.7574) -- (2.8722, 6.7602) -- (2.8785, 6.7615) -- (2.8813, 6.7632) 
  -- (2.8987, 6.7708) -- (2.8987, 6.7715) -- (2.9057, 6.7738) -- (2.9525, 6.785)
   -- (2.9876, 6.7919) -- (2.9909, 6.7925) -- (2.9956, 6.7926) -- (3.0, 6.7918) 
  -- (3.0136, 6.7829) -- (3.0197, 6.7779) -- (3.0328, 6.765) -- (3.0286, 6.7578)
   -- (3.019, 6.7433) -- (3.0206, 6.7378) -- (3.0275, 6.7033) -- (3.0161, 
  6.6937) -- (3.011, 6.6833) -- (3.0098, 6.6777) -- (3.0222, 6.6614) -- (3.0253,
   6.6726) -- (3.0465, 6.7191) -- (3.0507, 6.7237) -- (3.0926, 6.7444) -- 
  (3.102, 6.7462) -- (3.1071, 6.7464) -- (3.1128, 6.7454) -- (3.1237, 6.7424) --
   (3.1393, 6.738) -- (3.2534, 6.6811) -- (3.2601, 6.6773) -- (3.2737, 6.6686) 
  -- (3.2771, 6.6656) -- (3.2815, 6.6589) -- (3.2836, 6.6541) -- (3.283, 6.6536)
   -- (3.2841, 6.651) -- (3.3028, 6.6308) -- (3.3067, 6.6276) -- (3.3089, 
  6.6264) -- (3.3141, 6.6249) -- (3.3198, 6.625) -- (3.3443, 6.6262) -- (3.3479,
   6.6267) -- (3.3528, 6.6282) -- (3.3597, 6.631) -- (3.3645, 6.6339) -- 
  (3.3816, 6.6477) -- (3.3882, 6.6545) -- (3.3987, 6.667) -- (3.4149, 6.6813) --
   (3.438, 6.699) -- (3.4489, 6.7018) -- (3.5237, 6.7052) -- (3.5214, 6.6724) --
   (3.5171, 6.6651) -- (3.5166, 6.6622) -- (3.5179, 6.641) -- (3.5235, 6.5794) 
  -- (3.5303, 6.5274) -- (3.53, 6.5264) -- (3.5293, 6.5258) -- (3.5054, 6.5063) 
  -- (3.4994, 6.502) -- (3.4697, 6.4812) -- (3.4425, 6.4561) -- (3.4185, 6.426) 
  -- (3.4144, 6.4226) -- (3.4098, 6.4198) -- (3.4052, 6.418) -- (3.3848, 6.4217)
   -- (3.3819, 6.4225) -- (3.3775, 6.4256) -- (3.3486, 6.3948) -- (3.3461, 
  6.3922) -- (3.3446, 6.389) -- (3.343, 6.3835) -- (3.3424, 6.3791) -- (3.3425, 
  6.3765) -- (3.3435, 6.3719) -- (3.3498, 6.3565) -- (3.3552, 6.3465) -- 
  (3.3587, 6.3409) -- (3.3626, 6.336) -- (3.3647, 6.3339) -- (3.3695, 6.3309) --
   (3.3737, 6.3303) -- (3.391, 6.3303) -- (3.3932, 6.3305) -- (3.4012, 6.3326) 
  -- (3.4042, 6.3322) -- (3.4089, 6.33) -- (3.4126, 6.3269) -- (3.4155, 6.3231) 
  -- (3.4174, 6.3183) -- (3.4175, 6.3166) -- (3.4162, 6.307) -- (3.4174, 6.2997)
   -- (3.4222, 6.2977) -- (3.425, 6.2973) -- cycle(2.0193, 7.2486) -- (2.0091, 
  7.2223) -- (2.0007, 7.1986) -- (1.998, 7.1856) -- (1.9957, 7.1669) -- (1.9938,
   7.1451) -- (1.988, 7.0612) -- (1.9879, 7.0547) -- (1.9929, 7.043) -- (1.9958,
   7.0392) -- (1.997, 7.0396) -- (1.998, 7.0413) -- (1.9982, 7.0451) -- (2.0004,
   7.1167) -- (1.9991, 7.1188) -- (1.9979, 7.1231) -- (1.9979, 7.1247) -- 
  (1.9993, 7.1384) -- (2.0033, 7.1511) -- (2.0056, 7.1541) -- (2.0251, 7.1668) 
  -- (2.0307, 7.162) -- (2.0318, 7.1605) -- (2.045, 7.1531) -- (2.0578, 7.1438) 
  -- (2.0621, 7.1423) -- (2.0641, 7.1423) -- (2.0892, 7.1551) -- (2.1052, 
  7.1685) -- (2.1045, 7.1692) -- (2.1013, 7.17) -- (2.0964, 7.1705) -- (2.0766, 
  7.1697) -- (2.0717, 7.1705) -- (2.0692, 7.1713) -- (2.0523, 7.1864) -- 
  (2.0519, 7.1868) -- (2.0369, 7.2023) -- (2.0331, 7.2568) -- (2.0436, 7.2776) 
  -- (2.0482, 7.2835) -- (2.057, 7.2905) -- (2.0579, 7.291) -- (2.069, 7.2936) 
  -- (2.0807, 7.3009) -- (2.095, 7.329) -- (2.0948, 7.3298) -- (2.0938, 7.3307) 
  -- (2.0744, 7.3387) -- (2.0691, 7.3403) -- (2.066, 7.3404) -- (2.0643, 7.3399)
   -- (2.0596, 7.3369) -- (2.0584, 7.3356) -- (2.0545, 7.3293) -- (2.037, 
  7.2882) -- (2.0336, 7.2796) -- cycle(2.0967, 7.0518) -- (2.092, 7.0519) -- 
  (2.0742, 7.0485) -- (2.069, 7.0466) -- (2.0651, 7.0436) -- (2.0601, 7.0385) --
   (2.0564, 7.0319) -- (2.0516, 7.0202) -- (2.0509, 7.0127) -- (2.0517, 7.0088) 
  -- (2.0532, 7.006) -- (2.0551, 7.0044) -- (2.0918, 6.9864) -- (2.0994, 6.9842)
   -- (2.1083, 6.9837) -- (2.1382, 6.9825) -- (2.1421, 6.9827) -- (2.144, 
  6.9836) -- (2.161, 7.0191) -- (2.1555, 7.041) -- (2.1542, 7.0435) -- (2.1458, 
  7.0493) -- (2.1356, 7.0529) -- (2.1315, 7.0539) -- (2.1288, 7.0545) -- 
  (2.1222, 7.055) -- (2.1146, 7.0523) -- (2.1117, 7.0512) -- cycle(1.9954, 
  6.9686) -- (1.9955, 6.9666) -- (1.9983, 6.9595) -- (2.0095, 6.9415) -- 
  (2.0154, 6.9324) -- (2.0221, 6.922) -- (2.0237, 6.9212) -- (2.0269, 6.9207) --
   (2.0373, 6.9208) -- (2.038, 6.921) -- (2.0489, 6.9264) -- (2.05, 6.9275) -- 
  (2.0528, 6.9344) -- (2.0524, 6.9354) -- (2.0516, 6.9358) -- (2.0406, 6.9413) 
  -- (2.0311, 6.9559) -- (2.0195, 6.9883) -- (2.0182, 6.9951) -- (2.0198, 
  6.9997) -- (2.0224, 7.0043) -- (2.0317, 7.0135) -- (2.0309, 7.0145) -- 
  (2.0294, 7.015) -- (2.0257, 7.0133) -- (2.0032, 6.9929) -- (2.0003, 6.9891) --
   (1.9969, 6.9806) -- (1.9965, 6.9729) -- cycle(2.1536, 6.9355) -- (2.1871, 
  6.9438) -- (2.1915, 6.9464) -- (2.1947, 6.9505) -- (2.1964, 6.956) -- (2.1981,
   6.9654) -- (2.1965, 6.9661) -- (2.1389, 6.9435) -- (2.1237, 6.9373) -- 
  (2.1224, 6.9335) -- (2.1229, 6.9322) -- (2.1325, 6.9243) -- cycle(2.1145, 
  6.8883) -- (2.112, 6.888) -- (2.1102, 6.8864) -- (2.1107, 6.8848) -- (2.113, 
  6.8812) -- (2.1146, 6.88) -- (2.1293, 6.8691) -- (2.1319, 6.8673) -- (2.1397, 
  6.8672) -- (2.1445, 6.8686) -- (2.1455, 6.8705) -- (2.1454, 6.8746) -- 
  (2.1386, 6.8851) -- (2.1376, 6.8867) -- (2.1366, 6.8874) -- (2.1336, 6.8883) 
  -- (2.1275, 6.8865) -- cycle(2.1651, 6.8081) -- (2.1702, 6.8035) -- (2.173, 
  6.8026) -- (2.1792, 6.8029) -- (2.1885, 6.8049) -- (2.1965, 6.8072) -- 
  (2.1987, 6.8085) -- (2.2019, 6.8121) -- (2.2121, 6.8345) -- (2.2176, 6.8474) 
  -- (2.2183, 6.861) -- (2.2162, 6.8623) -- (2.2105, 6.8632) -- (2.2054, 6.8626)
   -- (2.185, 6.8562) -- (2.1821, 6.8553) -- (2.1584, 6.8454) -- (2.155, 6.8437)
   -- (2.1531, 6.8412) -- (2.1526, 6.8387) -- (2.1528, 6.8282) -- (2.154, 
  6.8238) -- (2.1554, 6.8209) -- (2.1575, 6.818) -- (2.1588, 6.817) -- 
  cycle(2.3047, 6.8327) -- (2.2759, 6.814) -- (2.2721, 6.8093) -- (2.2721, 
  6.8068) -- (2.272, 6.7832) -- (2.2782, 6.7783) -- (2.2836, 6.7756) -- (2.2907,
   6.774) -- (2.3106, 6.772) -- (2.319, 6.7721) -- (2.3307, 6.7818) -- (2.3358, 
  6.7881) -- (2.3366, 6.7892) -- (2.3503, 6.8102) -- (2.3506, 6.8117) -- 
  (2.3484, 6.8236) -- (2.3455, 6.8269) -- (2.3369, 6.8288) -- (2.3163, 6.8296) 
  -- cycle(3.519, 6.8394) -- (3.5147, 6.8408) -- (3.513, 6.8425) -- (3.5121, 
  6.844) -- (3.5106, 6.8434) -- (3.5076, 6.841) -- (3.5004, 6.8328) -- (3.4948, 
  6.8251) -- (3.4916, 6.8193) -- (3.4852, 6.8051) -- (3.482, 6.796) -- (3.4811, 
  6.793) -- (3.4783, 6.7617) -- (3.4795, 6.7566) -- (3.4807, 6.7547) -- (3.4951,
   6.745) -- (3.4974, 6.744) -- (3.501, 6.7434) -- (3.5015, 6.7439) -- (3.501, 
  6.7443) -- (3.5008, 6.7439) -- (3.5002, 6.7439) -- (3.4966, 6.7451) -- 
  (3.4879, 6.7507) -- (3.488, 6.7564) -- (3.4897, 6.7581) -- (3.5109, 6.7655) --
   (3.5206, 6.7624) -- (3.5242, 6.7609) -- (3.528, 6.7593) -- (3.5292, 6.7569) 
  -- (3.5318, 6.7492) -- (3.531, 6.7407) -- (3.5297, 6.729) -- (3.5761, 6.7203) 
  -- (3.6477, 6.7197) -- (3.6483, 6.7198) -- (3.6486, 6.7207) -- (3.6482, 
  6.7262) -- (3.6367, 6.751) -- (3.6272, 6.771) -- (3.5983, 6.8152) -- (3.5687, 
  6.8306) -- cycle(2.2004, 6.3818) -- (2.1987, 6.3829) -- (2.1955, 6.3862) -- 
  (2.1909, 6.3918) -- (2.191, 6.3955) -- (2.1921, 6.3999) -- (2.194, 6.4038) -- 
  (2.1955, 6.4059) -- (2.1975, 6.4079) -- (2.1994, 6.409) -- (2.2017, 6.41) -- 
  (2.204, 6.4102) -- (2.2068, 6.4079) -- (2.207, 6.4073) -- (2.2057, 6.3839) -- 
  (2.203, 6.3821) -- cycle(1.7669, 6.5287) -- (1.7777, 6.5298) -- (1.7784, 
  6.5242) -- (1.7705, 6.5102) -- (1.7637, 6.5108) -- (1.7611, 6.5114) -- (1.753,
   6.5262) -- (1.7534, 6.5284) -- (1.7583, 6.5318) -- (1.7623, 6.5332) -- cycle;

   \node[text=black,line width=0.0092cm,anchor=center] (text8) at (2.7492, 
   6.4649){\resizebox{\ifdim\width>1.5em 1.5em\else\width\fi}{!}{#2}};

}

\newcommand{\drawthuringen}[2]{%
  %Thüringen
  \path[draw=black,fill=#1,line join=round,line width=0.0046cm] (3.3373, 
  4.117) -- (3.3542, 4.1207) -- (3.3853, 4.1182) -- (3.4028, 4.1166) -- (3.4296,
   4.1103) -- (3.4335, 4.1075) -- (3.4515, 4.0908) -- (3.4631, 4.0958) -- 
  (3.4478, 4.0752) -- (3.4434, 4.0605) -- (3.4546, 4.034) -- (3.4737, 4.0041) --
   (3.4891, 3.9804) -- (3.4979, 3.9691) -- (3.4966, 3.9472) -- (3.4954, 3.9224) 
  -- (3.5025, 3.9181) -- (3.5104, 3.9093) -- (3.5181, 3.9083) -- (3.525, 3.909) 
  -- (3.5356, 3.9128) -- (3.5413, 3.9157) -- (3.6005, 3.897) -- (3.6047, 3.8965)
   -- (3.6637, 3.8952) -- (3.6766, 3.8955) -- (3.6831, 3.8968) -- (3.7085, 
  3.8958) -- (3.7469, 3.881) -- (3.749, 3.8783) -- (3.7494, 3.8752) -- (3.7469, 
  3.8581) -- (3.7519, 3.8487) -- (3.7521, 3.8451) -- (3.7686, 3.8395) -- 
  (3.7638, 3.8349) -- (3.7632, 3.8335) -- (3.763, 3.8322) -- (3.7675, 3.8264) --
   (3.7737, 3.8204) -- (3.7766, 3.8195) -- (3.7802, 3.8175) -- (3.7922, 3.8068) 
  -- (3.7964, 3.8021) -- (3.7968, 3.7996) -- (3.7962, 3.7983) -- (3.7926, 3.792)
   -- (3.7861, 3.7818) -- (3.7727, 3.7621) -- (3.7652, 3.7515) -- (3.7604, 
  3.7461) -- (3.7579, 3.7443) -- (3.7536, 3.7426) -- (3.7531, 3.7427) -- 
  (3.7462, 3.7505) -- (3.7445, 3.7504) -- (3.7425, 3.749) -- (3.7393, 3.7442) --
   (3.7341, 3.7349) -- (3.7331, 3.7276) -- (3.7474, 3.7162) -- (3.762, 3.7158) 
  -- (3.7705, 3.7199) -- (3.7728, 3.7197) -- (3.7807, 3.7164) -- (3.7951, 
  3.7017) -- (3.805, 3.6792) -- (3.805, 3.6767) -- (3.8001, 3.6653) -- (3.7976, 
  3.6606) -- (3.7855, 3.6584) -- (3.7931, 3.627) -- (3.7976, 3.6236) -- (3.8059,
   3.6197) -- (3.8064, 3.6173) -- (3.8419, 3.618) -- (3.8496, 3.6245) -- (3.852,
   3.6266) -- (3.8551, 3.6335) -- (3.857, 3.6352) -- (3.9147, 3.6261) -- (3.929,
   3.6173) -- (3.9319, 3.6061) -- (3.9336, 3.5939) -- (3.9349, 3.5887) -- 
  (3.9787, 3.5748) -- (3.9953, 3.5703) -- (4.021, 3.5757) -- (4.0396, 3.5806) --
   (4.0473, 3.5781) -- (4.0789, 3.5579) -- (4.0988, 3.5437) -- (4.1011, 3.5415) 
  -- (4.1227, 3.5139) -- (4.1243, 3.509) -- (4.1245, 3.5068) -- (4.1249, 3.4987)
   -- (4.1337, 3.5026) -- (4.1368, 3.5033) -- (4.1664, 3.5081) -- (4.1686, 
  3.5083) -- (4.2018, 3.4973) -- (4.2118, 3.4928) -- (4.2111, 3.4994) -- 
  (4.2134, 3.5019) -- (4.2159, 3.5047) -- (4.2221, 3.5097) -- (4.2231, 3.5088) 
  -- (4.236, 3.4923) -- (4.24, 3.4846) -- (4.2425, 3.4748) -- (4.2429, 3.4745) 
  -- (4.2481, 3.4782) -- (4.2504, 3.4793) -- (4.2569, 3.4878) -- (4.2861, 
  3.5509) -- (4.2867, 3.5537) -- (4.2871, 3.5631) -- (4.2827, 3.5688) -- 
  (4.2792, 3.5709) -- (4.2672, 3.5798) -- (4.2649, 3.5952) -- (4.2797, 3.6182) 
  -- (4.3071, 3.6316) -- (4.308, 3.6317) -- (4.3785, 3.6096) -- (4.4051, 3.5999)
   -- (4.4085, 3.5977) -- (4.4103, 3.5952) -- (4.4125, 3.587) -- (4.4113, 3.581)
   -- (4.4107, 3.5696) -- (4.4135, 3.5534) -- (4.4362, 3.5363) -- (4.458, 
  3.5281) -- (4.4802, 3.5225) -- (4.4833, 3.5074) -- (4.4887, 3.4865) -- 
  (4.4901, 3.4842) -- (4.4961, 3.4741) -- (4.4985, 3.4718) -- (4.5035, 3.4684) 
  -- (4.496, 3.457) -- (4.4704, 3.4487) -- (4.4574, 3.4464) -- (4.4522, 3.4476) 
  -- (4.4348, 3.4449) -- (4.4054, 3.4375) -- (4.403, 3.4361) -- (4.3896, 3.4248)
   -- (4.3893, 3.4133) -- (4.3809, 3.3893) -- (4.379, 3.3885) -- (4.3674, 
  3.3851) -- (4.367, 3.3857) -- (4.3619, 3.3924) -- (4.3565, 3.3944) -- (4.3421,
   3.3944) -- (4.3393, 3.3935) -- (4.3244, 3.3819) -- (4.3243, 3.3812) -- 
  (4.3244, 3.3686) -- (4.2845, 3.3542) -- (4.2723, 3.3555) -- (4.2671, 3.3623) 
  -- (4.2628, 3.3535) -- (4.2559, 3.3036) -- (4.2551, 3.2939) -- (4.2554, 
  3.2893) -- (4.2555, 3.2867) -- (4.2784, 3.2461) -- (4.2857, 3.2389) -- 
  (4.3128, 3.2316) -- (4.3107, 3.2197) -- (4.3008, 3.2066) -- (4.2999, 3.2056) 
  -- (4.2812, 3.186) -- (4.2634, 3.182) -- (4.2621, 3.1821) -- (4.2287, 3.1678) 
  -- (4.199, 3.1548) -- (4.1943, 3.1528) -- (4.1923, 3.1514) -- (4.1897, 3.1437)
   -- (4.2077, 3.1383) -- (4.2109, 3.1412) -- (4.2172, 3.141) -- (4.2181, 
  3.1407) -- (4.2202, 3.137) -- (4.2188, 3.132) -- (4.2179, 3.1312) -- (4.2159, 
  3.1303) -- (4.2082, 3.1294) -- (4.205, 3.1307) -- (4.1944, 3.1237) -- (4.1717,
   3.1265) -- (4.1353, 3.1247) -- (4.1395, 3.1505) -- (4.1355, 3.1787) -- 
  (4.1286, 3.1859) -- (4.1252, 3.185) -- (4.1239, 3.1842) -- (4.1116, 3.1729) --
   (4.1111, 3.1654) -- (4.0905, 3.1441) -- (4.0886, 3.1434) -- (4.0819, 3.1403) 
  -- (4.0784, 3.1382) -- (4.0497, 3.0979) -- (4.0486, 3.0948) -- (4.0485, 
  3.0939) -- (4.0554, 3.0774) -- (4.0582, 3.0728) -- (4.0666, 3.0726) -- 
  (4.0769, 3.0739) -- (4.0845, 3.0669) -- (4.1027, 3.0498) -- (4.1049, 3.0453) 
  -- (4.1052, 3.0428) -- (4.1022, 3.038) -- (4.0853, 3.0221) -- (4.0703, 3.0142)
   -- (4.0647, 3.0096) -- (4.0606, 3.0032) -- (4.0619, 3.0014) -- (4.0664, 
  2.9969) -- (4.0808, 2.9868) -- (4.0563, 2.9759) -- (4.0187, 2.9625) -- 
  (3.9042, 2.9468) -- (3.9022, 2.9468) -- (3.8995, 2.9483) -- (3.8944, 2.9532) 
  -- (3.8905, 2.9576) -- (3.8879, 2.9596) -- (3.8843, 2.9613) -- (3.8722, 
  2.9587) -- (3.8708, 2.9578) -- (3.8693, 2.9555) -- (3.8679, 2.9484) -- 
  (3.8677, 2.9465) -- (3.8637, 2.9427) -- (3.8463, 2.9372) -- (3.841, 2.9399) --
   (3.8256, 2.9571) -- (3.8195, 2.968) -- (3.8145, 2.9881) -- (3.7984, 2.9861) 
  -- (3.7802, 2.9969) -- (3.7793, 2.9985) -- (3.7785, 3.0037) -- (3.7782, 3.023)
   -- (3.7795, 3.0374) -- (3.7835, 3.0434) -- (3.7876, 3.0433) -- (3.7883, 
  3.0443) -- (3.7907, 3.0505) -- (3.7907, 3.0518) -- (3.7856, 3.0594) -- 
  (3.7826, 3.0638) -- (3.7801, 3.0658) -- (3.7483, 3.0719) -- (3.7338, 3.07) -- 
  (3.7323, 3.0694) -- (3.7223, 3.0588) -- (3.7203, 3.052) -- (3.7001, 3.0357) --
   (3.6862, 3.0293) -- (3.6853, 3.0307) -- (3.6811, 3.032) -- (3.6772, 3.0307) 
  -- (3.6752, 3.0284) -- (3.6738, 3.0227) -- (3.6738, 3.0208) -- (3.6749, 
  3.0013) -- (3.6812, 2.9969) -- (3.6853, 2.9954) -- (3.688, 2.9878) -- (3.699, 
  2.9184) -- (3.699, 2.9178) -- (3.6806, 2.8548) -- (3.6521, 2.8501) -- (3.6435,
   2.8328) -- (3.6426, 2.8315) -- (3.6351, 2.8313) -- (3.6148, 2.8509) -- 
  (3.6084, 2.8602) -- (3.6082, 2.8611) -- (3.6073, 2.8656) -- (3.6082, 2.8709) 
  -- (3.6121, 2.8753) -- (3.6151, 2.8779) -- (3.6202, 2.8817) -- (3.6241, 
  2.8829) -- (3.6245, 2.8886) -- (3.612, 2.9103) -- (3.5972, 2.9228) -- (3.589, 
  2.9229) -- (3.5821, 2.9213) -- (3.581, 2.9204) -- (3.5775, 2.9098) -- (3.5667,
   2.9052) -- (3.5543, 2.901) -- (3.5506, 2.9015) -- (3.5177, 2.9073) -- 
  (3.5054, 2.9208) -- (3.4949, 2.9394) -- (3.4899, 2.9445) -- (3.4882, 2.9449) 
  -- (3.4867, 2.9449) -- (3.4258, 2.9445) -- (3.3569, 2.9164) -- (3.355, 2.9078)
   -- (3.3567, 2.8798) -- (3.3575, 2.8764) -- (3.3608, 2.8735) -- (3.4113, 
  2.8379) -- (3.4129, 2.8374) -- (3.4186, 2.836) -- (3.4303, 2.8321) -- (3.4357,
   2.8293) -- (3.4395, 2.8245) -- (3.44, 2.8119) -- (3.438, 2.8062) -- (3.4367, 
  2.8047) -- (3.4304, 2.8004) -- (3.424, 2.7988) -- (3.4214, 2.7991) -- (3.3955,
   2.8051) -- (3.3733, 2.8102) -- (3.3679, 2.8113) -- (3.3653, 2.8092) -- 
  (3.3646, 2.8073) -- (3.365, 2.7917) -- (3.3663, 2.7911) -- (3.3676, 2.775) -- 
  (3.3635, 2.768) -- (3.3617, 2.7672) -- (3.3592, 2.7668) -- (3.3113, 2.7844) --
   (3.2944, 2.7886) -- (3.2927, 2.7926) -- (3.2889, 2.8033) -- (3.2827, 2.8297) 
  -- (3.2859, 2.8624) -- (3.2883, 2.8882) -- (3.2604, 2.8946) -- (3.2237, 
  2.9084) -- (3.2028, 2.9313) -- (3.1999, 2.9421) -- (3.1976, 2.9515) -- 
  (3.1895, 2.9474) -- (3.181, 2.9435) -- (3.164, 2.9453) -- (3.1624, 2.9455) -- 
  (3.1576, 2.9524) -- (3.1574, 2.9578) -- (3.1538, 2.9799) -- (3.1315, 3.0272) 
  -- (3.1269, 3.032) -- (3.1218, 3.0361) -- (3.079, 3.054) -- (3.0656, 3.0721) 
  -- (3.0487, 3.0941) -- (3.0438, 3.0947) -- (2.9829, 3.0882) -- (2.9663, 
  3.0697) -- (2.9572, 3.0607) -- (2.9538, 3.0591) -- (2.9513, 3.0583) -- 
  (2.9494, 3.1473) -- (2.951, 3.1507) -- (2.9546, 3.1565) -- (2.9565, 3.1586) --
   (2.9731, 3.1725) -- (2.9663, 3.1914) -- (2.9592, 3.2038) -- (2.9565, 3.206) 
  -- (2.9448, 3.2072) -- (2.925, 3.2084) -- (2.9206, 3.2072) -- (2.8983, 3.1985)
   -- (2.8961, 3.1956) -- (2.8964, 3.19) -- (2.9077, 3.1759) -- (2.9028, 3.1646)
   -- (2.8943, 3.1628) -- (2.8844, 3.1645) -- (2.8544, 3.1698) -- (2.8538, 
  3.1784) -- (2.8544, 3.2027) -- (2.8572, 3.2068) -- (2.8634, 3.2157) -- 
  (2.8762, 3.2314) -- (2.8877, 3.2559) -- (2.892, 3.2673) -- (2.892, 3.2703) -- 
  (2.8905, 3.2838) -- (2.882, 3.2941) -- (2.8943, 3.298) -- (2.8968, 3.2989) -- 
  (2.8979, 3.3002) -- (2.8978, 3.3045) -- (2.8969, 3.3134) -- (2.8934, 3.32) -- 
  (2.8934, 3.3251) -- (2.8966, 3.3416) -- (2.8979, 3.3448) -- (2.908, 3.3529) --
   (2.9122, 3.3559) -- (2.927, 3.354) -- (2.9397, 3.3544) -- (2.9463, 3.365) -- 
  (2.9612, 3.4026) -- (2.961, 3.4099) -- (2.9521, 3.464) -- (2.9387, 3.5082) -- 
  (2.9442, 3.5198) -- (2.9456, 3.5225) -- (2.9469, 3.5234) -- (2.9494, 3.5247) 
  -- (2.9573, 3.5256) -- (2.9704, 3.5243) -- (2.9915, 3.5214) -- (3.0032, 3.515)
   -- (3.0059, 3.5117) -- (3.0086, 3.5097) -- (3.0129, 3.5084) -- (3.0167, 
  3.5089) -- (3.0324, 3.511) -- (3.0346, 3.5127) -- (3.0362, 3.5123) -- (3.041, 
  3.5141) -- (3.0435, 3.5156) -- (3.0485, 3.5216) -- (3.0472, 3.5232) -- 
  (3.0526, 3.5339) -- (3.0541, 3.5384) -- (3.0493, 3.5471) -- (3.0447, 3.5519) 
  -- (3.0428, 3.5539) -- (3.0346, 3.5594) -- (3.0181, 3.5654) -- (3.0162, 
  3.6049) -- (3.0106, 3.6162) -- (2.9982, 3.6425) -- (2.9972, 3.6449) -- 
  (3.0014, 3.6501) -- (3.0253, 3.656) -- (3.035, 3.6461) -- (3.0383, 3.6339) -- 
  (3.0409, 3.6299) -- (3.0407, 3.6295) -- (3.0441, 3.6231) -- (3.0494, 3.6231) 
  -- (3.0511, 3.6251) -- (3.0515, 3.6323) -- (3.0511, 3.6487) -- (3.0492, 
  3.6484) -- (3.0459, 3.6532) -- (3.0456, 3.6557) -- (3.0495, 3.6679) -- 
  (3.0526, 3.6707) -- (3.0559, 3.6724) -- (3.0616, 3.6764) -- (3.0645, 3.6837) 
  -- (3.062, 3.6893) -- (3.0506, 3.692) -- (3.0385, 3.7039) -- (3.0331, 3.7082) 
  -- (3.0257, 3.7128) -- (3.0105, 3.718) -- (2.9687, 3.7392) -- (2.9603, 3.7657)
   -- (2.943, 3.7731) -- (2.9117, 3.7843) -- (2.9005, 3.7983) -- (2.8874, 
  3.8299) -- (2.882, 3.8498) -- (2.8807, 3.859) -- (2.8825, 3.8674) -- (2.8843, 
  3.8661) -- (2.8863, 3.8667) -- (2.8908, 3.8684) -- (2.919, 3.8874) -- (2.9801,
   3.9229) -- (3.0289, 3.9328) -- (3.1024, 3.9827) -- (3.1262, 4.0016) -- 
  (3.1275, 4.0027) -- (3.1413, 4.0234) -- (3.1427, 4.0264) -- (3.1448, 4.0338) 
  -- (3.1444, 4.0347) -- (3.1404, 4.0365) -- (3.1488, 4.0543) -- (3.1551, 4.065)
   -- (3.1768, 4.0707) -- (3.1786, 4.0709) -- (3.1863, 4.0684) -- (3.2004, 
  4.0623) -- (3.2042, 4.0606) -- (3.211, 4.0563) -- (3.2146, 4.054) -- (3.2257, 
  4.0375) -- (3.231, 4.0347) -- (3.2533, 4.0375) -- (3.2952, 4.0482) -- (3.3037,
   4.0644) -- (3.3228, 4.1151) -- (3.3253, 4.1147) -- (3.3359, 4.1162) -- cycle;



  \node[text=black,line width=0.0092cm,anchor=center] (text16) at (3.477, 
  3.375){\resizebox{\ifdim\width>1.75em 1.75em\else\width\fi}{!}{#2}};
}

%% Formatierung
\input{config/chapter_no_vertical_space_before}%Eliminiert vertikalen Einzug vor Kapitel
\input{config/draft_showframe}%Fügt Layoutboxen zum Draft-Modus hinzu
\input{config/toc_space} %Mehr Platz für Figure-/Table-Nummerierung und Seitenangabe in Verzeichnissen
\input{config/hyperref_colours} %Bessere Standardfarben für Hyperlinks
\hypersetup{ pdfborderstyle={/W 1 /S /U} } %Unterstrichene Links die nicht gedruckt werden
\input{config/intentionally_empty} %Leere Seiten mit Absichtlich-Text
\renewcommand{\emptypageline}{Diese Seite wurde absichtlich leer gelassen} %Diesen Wert neu definieren, um die gewünschte Meldung anzuzeigen
%\AfterCalculatingTypearea{\setlength{\marginparwidth}{\marginparwidth}}
%\AfterCalculatingTypearea{\setlength{\textwidth}{0.925\textwidth}}
\recalctypearea %Passt den Textbereich an Einstellungen an
\usepackage[pass]{geometry} %Erlaube Textbereichveränderung ohne aktuelle Änderungen


%% Dokumentmetadaten %%
\subject{Publikationspraktiken von Forschungsdaten in Hochschulschriften}
\title{Publikationspraktiken für Forschungsdaten in Hochschulschriften}
\subtitle{Eine Untersuchung der Veröffentlichungsformate und -methoden}
\author{Dr. David Krassnig}
\date{14.06.2024}
\publishers{Verlag}
\dedication{Widmung}
\programme{Bibliotheks- und Informationswissenschaft im Fernstudium}
\degreecontext{im Rahmen des Weiterbildenden Masterstudiengangs}
\degree{Masterarbeit}
\faculty{Philosophische Fakultät}
\institute{Institut für Bibliotheks- und Informationswissenschaft}
\firstsupervisor{Dr. Sarah Dellmann}
\secondsupervisor{Prof. Dr. Robert Jäschke}
\keywords{Forschungsdatenmanagement, Bibliothekswissenschaft, TIB, Repositorium}
\languagemetadata{de-DE}
%
% PDF Metadaten
\input{config/set_pdf_metadata}


%% Bibliografie %%
\addbibresource{matter/backmatter/bibliography.bib} %Hauptbibliographieinformationen hinzufügen
%Anmerkung: Das Bibliografieformat muss oben unter Dokumentpakete eingestellt werden

%% Hauptdokument %%
\includeonly{content/einfuehrung,content/forschungsstand} %Selektive Dokumentgenerierung
\begin{document}
    %% Vorderer Teil
    \frontmatter
        %\makeatletter
\begin{titlepage}
\newgeometry{top=42.5mm,bottom=42.5mm,left=30mm,right=30mm}
\doublespacing\centering%
\textbf{\Huge\sffamily \@title}\vskip 2.5mm

\textbf{\Large\sffamily \@subtitle}\vfill

{\large eingereicht von}\\
\resizebox{%
      \ifdim\width>\textwidth
        \textwidth
      \else
        \width
      \fi
    }{!}{%
    \textbf{\Large\sffamily \@author}}\vfill

{\large als}\\
    \textbf{\Large\sffamily\degreevar}\vfill

{\large\degreecontextvar}\\
\resizebox{%
      \ifdim\width>\textwidth
        \textwidth
      \else
        \width
      \fi
    }{!}{%
    \textbf{\Large\sffamily\programmevar}}\vfill

{\large an der}\\
\textbf{\Large\sffamily Humboldt-Universität zu Berlin}\vfill

\resizebox{%
      \ifdim\width>\textwidth
        \textwidth
      \else
        \width
      \fi
    }{!}{%
    \textbf{\Large\sffamily\facultyvar}}

\resizebox{%
      \ifdim\width>\textwidth
        \textwidth
      \else
        \width
      \fi
    }{!}{%
    \textbf{\Large\sffamily\institutevar}}\vfill

\parbox{0cm}{\large%
    \begin{tabbing}
        \textbf{\sffamily1. Gutachter/in:} \= \firstsupervisorvar\\
        \textbf{\sffamily2. Gutachter/in:} \> \secondsupervisorvar
    \end{tabbing}
}\vfill


\textbf{\large\sffamily Berlin, \@date}
\end{titlepage}
\makeatother
\thispagestyle{empty}
\hbox{}\vfill\enlargethispage{15mm}
\begin{tabbing}
\noindent\textbf{Autor:}~~~~~\=Dr. David Krassnig\\
\>\orcidlink{0000-0002-1626-7987} \url{https://orcid.org/0000-0002-1626-7987}\\[\baselineskip]
\textbf{Lizenz (Dokument):}~~~~~\= \href{https://creativecommons.org/licenses/by/4.0/deed.de}{CC BY 4.0}\\
\textbf{DOI (Dokument):}\> \href{https://www.doi.org/10.5281/zenodo.11506621}{10.5281/zenodo.11506621}\\[.5\baselineskip]
\textbf{Lizenz (Daten):}\> \href{https://creativecommons.org/publicdomain/zero/1.0/deed.de}{CC0 1.0}\\
\textbf{DOI (Daten):}\> \href{https://www.doi.org/10.5281/zenodo.11401021}{10.5281/zenodo.11401021}
\end{tabbing}\newpage
\thispagestyle{empty}
\hbox{}
\vspace{30mm}
\begin{flushright}
{\LARGE\scshape Wasing the always of wanting of knowing}\\[.5\baselineskip]
{\footnotesize --Brandon Sanderson, \textit{Shadows of Self}}
\end{flushright}
\vfill
\hbox{}
\enlargethispage{30mm}
\vfill
\begin{figure}[b]
  \centering
\resizebox{22.5mm}{!}{
\begin{tikzpicture}[y=1mm, x=1mm, yscale=\globalscale,xscale=\globalscale, every node/.append style={scale=\globalscale}, inner sep=0pt, outer sep=0pt]
  \path[fill=c125b85,line width=3.175mm] (17.2619, 60.3734) -- (72.963, 95.5279)
   -- (14.8698, 131.9794) -- (72.963, 168.4314) -- (0.4748, 212.6876).. controls
   (0.3032, 187.8857) and (0.4119, 163.0838) .. (0.7807, 138.2818).. controls 
  (1.2623, 105.754) and (2.8937, 80.3161) .. (17.2619, 60.3734) -- cycle;



  \begin{scope}[fill=black,line width=1.7639mm,shift={(-4.758, 4.9148)}]
    \path[fill=black,line width=1.7639mm,miter limit=10.0] (74.0056, 74.9494).. 
  controls (74.2329, 73.9757) and (73.6718, 73.0178) .. (72.8205, 72.6021).. 
  controls (73.5744, 70.4553) and (75.9458, 71.7565) .. (74.9117, 73.6449).. 
  controls (74.7565, 74.1529) and (74.4414, 74.6695) .. (74.0056, 74.9494) -- 
  cycle(138.4185, 74.9038).. controls (138.3843, 74.9089) and (138.3086, 
  74.8922) .. (138.1743, 74.8452).. controls (137.6007, 74.7266) and (137.0486, 
  74.5335) .. (136.5009, 74.3254).. controls (135.8172, 74.4105) and (135.13, 
  74.4756) .. (134.4406, 74.4572).. controls (130.5626, 73.9996) and (126.4593, 
  72.6845) .. (122.5305, 73.9911).. controls (118.7325, 74.8439) and (113.9174, 
  74.5883) .. (111.4755, 71.1197).. controls (109.4837, 68.242) and (110.6627, 
  63.5553) .. (114.1636, 62.4728).. controls (118.9665, 60.8081) and (124.1583, 
  61.7104) .. (129.0949, 61.8911).. controls (136.5177, 62.36) and (143.9012, 
  64.0507) .. (151.3688, 63.8412).. controls (153.5412, 63.7184) and (156.4181, 
  62.2849) .. (155.8908, 59.6963).. controls (155.2096, 56.6484) and (151.4638, 
  56.0864) .. (148.8332, 56.0619).. controls (141.7311, 55.9693) and (135.0431, 
  59.0436) .. (128.0021, 59.3772).. controls (123.0159, 59.8684) and (117.9668, 
  59.5715) .. (113.1831, 57.9176).. controls (109.8742, 57.1998) and (105.6619, 
  56.2265) .. (102.8823, 58.795).. controls (101.9401, 59.7317) and (101.389, 
  61.033) .. (101.3001, 62.3528).. controls (101.8535, 63.5068) and (103.7288, 
  62.0123) .. (104.1982, 62.6887).. controls (102.5133, 63.0692) and (101.3809, 
  65.0022) .. (101.6962, 66.6313).. controls (101.9198, 66.7757) and (103.8952, 
  67.2426) .. (102.9333, 67.1798).. controls (101.3528, 66.9067) and (100.6688, 
  68.5328) .. (100.4786, 69.8022).. controls (100.2812, 70.2045) and (99.9983, 
  70.5249) .. (99.6658, 70.7974).. controls (99.8891, 71.7156) and (99.6631, 
  73.0456) .. (98.4427, 72.8674).. controls (97.8443, 72.8358) and (97.2591, 
  72.6874) .. (96.6972, 72.4789).. controls (96.8627, 73.2092) and (96.7223, 
  73.8282) .. (95.6028, 73.9635).. controls (95.0811, 73.8833) and (94.4796, 
  73.6944) .. (94.2029, 73.3145).. controls (92.9605, 73.5818) and (91.6884, 
  73.7351) .. (90.4339, 73.7714).. controls (88.9244, 73.9462) and (88.1105, 
  72.24) .. (86.5679, 72.4881).. controls (84.5808, 72.2535) and (82.6912, 
  71.2433) .. (80.7675, 71.0263).. controls (80.7304, 70.0931) and (81.5326, 
  69.4345) .. (81.0431, 68.5076).. controls (80.8135, 66.0916) and (84.0186, 
  64.4543) .. (86.0123, 65.7267).. controls (87.5081, 66.3458) and (89.1125, 
  65.9124) .. (90.2435, 64.9253).. controls (90.1591, 64.8692) and (90.0886, 
  64.8242) .. (89.9749, 64.7468).. controls (90.5772, 63.5638) and (90.7743, 
  62.1817) .. (90.6624, 60.8303).. controls (89.7975, 61.859) and (87.9907, 
  62.5602) .. (86.5788, 62.5721).. controls (84.2263, 62.5931) and (81.876, 
  60.2002) .. (79.6595, 61.985).. controls (79.2217, 62.3436) and (80.1238, 
  61.0665) .. (80.5059, 61.0522).. controls (83.0929, 59.9711) and (85.3784, 
  62.8581) .. (88.028, 61.9345).. controls (88.9719, 61.7392) and (89.8665, 
  61.3358) .. (90.6597, 60.7907).. controls (90.5829, 59.9119) and (90.3766, 
  59.0469) .. (90.0617, 58.2594).. controls (90.1279, 58.1517) and (90.2013, 
  58.0609) .. (90.2798, 57.9833).. controls (89.925, 57.7772) and (89.4985, 
  57.6738) .. (88.9895, 57.725).. controls (88.5118, 57.6285) and (87.2999, 
  57.715) .. (87.2109, 57.617).. controls (87.7928, 55.8173) and (85.4092, 
  54.9567) .. (84.0502, 54.573).. controls (83.9241, 54.5516) and (83.8003, 
  54.5238) .. (83.678, 54.4927).. controls (79.2098, 56.9291) and (74.817, 
  60.3467) .. (73.4119, 65.4001).. controls (73.1895, 66.7038) and (74.1887, 
  67.7711) .. (74.279, 69.0372).. controls (74.6129, 69.4838) and (75.6252, 
  69.8849) .. (75.2069, 70.5727).. controls (73.9268, 70.3314) and (72.7238, 
  71.5249) .. (72.3316, 72.6281).. controls (71.7672, 71.6998) and (72.1217, 
  71.1474) .. (72.1379, 70.1658).. controls (72.7389, 69.016) and (71.9384, 
  66.7239) .. (70.4184, 67.78).. controls (69.2478, 68.3949) and (68.697, 
  69.8618) .. (69.3088, 71.0334).. controls (69.233, 71.613) and (68.7003, 
  72.0389) .. (68.5475, 71.232).. controls (67.8815, 70.281) and (66.437, 
  70.0516) .. (65.5073, 70.3199).. controls (65.7173, 69.4283) and (67.041, 
  69.5708) .. (67.14, 68.5087).. controls (67.8283, 67.4293) and (70.2038, 
  66.8508) .. (69.2914, 65.2281).. controls (68.6504, 63.7411) and (66.4334, 
  64.0436) .. (66.1449, 65.5976).. controls (65.5926, 66.4667) and (65.7495, 
  65.7654) .. (65.7808, 65.1424).. controls (65.7808, 64.7539) and (65.6213, 
  64.4196) .. (65.3852, 64.1385).. controls (65.8082, 64.9784) and (65.1492, 
  66.4714) .. (63.9061, 66.0946).. controls (63.1079, 66.1297) and (61.8943, 
  65.3745) .. (61.8898, 64.8206).. controls (62.945, 65.1268) and (63.8034, 
  64.4045) .. (64.357, 63.6041).. controls (64.5576, 63.595) and (64.7284, 
  63.6237) .. (64.8741, 63.679).. controls (64.5416, 63.441) and (64.1805, 
  63.2658) .. (63.9018, 63.1608).. controls (64.565, 62.6878) and (65.3903, 
  62.9028) .. (65.834, 62.0685).. controls (66.9846, 61.506) and (68.4979, 
  61.9892) .. (69.522, 60.9703).. controls (71.0032, 60.1516) and (71.8134, 
  57.4413) .. (69.8937, 56.6734).. controls (69.8813, 56.6273) and (69.8689, 
  56.5596) .. (69.8568, 56.5036).. controls (69.2476, 56.5988) and (68.5632, 
  56.4872) .. (67.9528, 56.6718) -- (67.9745, 56.5655).. controls (68.085, 
  55.669) and (68.9844, 55.2451) .. (69.6137, 55.4867).. controls (69.5483, 
  55.2917) and (69.4684, 55.1153) .. (69.3614, 54.9962).. controls (71.2478, 
  54.0295) and (74.2586, 54.1003) .. (75.0826, 51.7406).. controls (75.1792, 
  49.9794) and (72.7743, 50.4729) .. (71.9551, 49.548).. controls (74.5725, 
  49.7791) and (77.0316, 48.5921) .. (78.7919, 46.7297).. controls (78.6297, 
  45.8029) and (79.0377, 44.7653) .. (78.7582, 43.8186).. controls (78.952, 
  42.6942) and (79.2938, 41.6152) .. (78.8109, 40.5082).. controls (79.1498, 
  39.3746) and (77.7144, 38.9032) .. (77.2574, 39.9618).. controls (76.7621, 
  40.9221) and (76.1942, 41.8346) .. (75.6648, 42.7503).. controls (75.912, 
  43.3265) and (75.5439, 44.5601) .. (74.9839, 43.5739).. controls (74.2948, 
  43.0379) and (73.1036, 43.1904) .. (72.5226, 43.2771).. controls (72.9522, 
  42.5064) and (73.4178, 42.3717) .. (74.2118, 41.8078).. controls (75.3028, 
  41.4399) and (75.7883, 39.5358) .. (74.317, 39.385).. controls (73.2887, 
  39.0696) and (72.9467, 40.535) .. (72.1227, 40.4116).. controls (72.6767, 
  39.4277) and (72.0353, 38.0857) .. (71.3842, 37.5255).. controls (72.0542, 
  37.4802) and (72.63, 38.1902) .. (73.3295, 37.6118).. controls (74.3819, 
  37.444) and (76.1444, 38.2914) .. (76.3751, 36.652).. controls (76.6819, 
  35.3956) and (74.5883, 35.4239) .. (74.3279, 34.9791).. controls (75.4526, 
  35.0425) and (76.3525, 33.9814) .. (76.2997, 32.8944).. controls (76.7551, 
  33.4485) and (76.9351, 34.0658) .. (77.7501, 34.2092).. controls (78.6288, 
  35.2352) and (80.7696, 35.4451) .. (80.882, 33.7029).. controls (80.7611, 
  33.4092) and (80.8172, 33.1493) .. (80.9292, 32.902).. controls (80.5987, 
  32.4921) and (79.9305, 31.8648) .. (79.7902, 31.6171).. controls (80.468, 
  31.2832) and (81.2426, 31.7642) .. (81.2086, 32.3876).. controls (81.3234, 
  32.1836) and (81.4208, 31.9794) .. (81.4229, 31.762).. controls (82.754, 
  32.5406) and (84.7111, 34.6164) .. (86.2505, 33.1315).. controls (86.7676, 
  32.5599) and (85.8794, 31.5966) .. (86.0519, 31.2731).. controls (88.6368, 
  34.1002) and (92.8614, 33.1787) .. (96.2072, 32.7924).. controls (97.3351, 
  32.5081) and (97.6354, 33.7209) .. (97.0336, 34.4739).. controls (96.7076, 
  35.6775) and (95.8937, 37.5679) .. (97.1296, 38.3872).. controls (99.4923, 
  37.4691) and (102.3162, 37.7796) .. (104.3468, 39.4018).. controls (108.5894, 
  42.3359) and (110.9891, 47.7208) .. (116.0936, 49.3874).. controls (117.0296, 
  49.6407) and (117.9604, 49.6686) .. (118.875, 49.5491).. controls (117.6105, 
  49.0637) and (116.658, 47.9003) .. (115.6899, 46.975).. controls (115.0442, 
  46.1616) and (115.9453, 46.6526) .. (116.2906, 46.899).. controls (117.8985, 
  47.8209) and (119.9299, 48.2341) .. (121.5766, 48.6532).. controls (121.5703, 
  48.7096) and (121.5589, 48.7633) .. (121.544, 48.8149).. controls (121.7604, 
  48.7278) and (121.9748, 48.6353) .. (122.1876, 48.5393).. controls (119.6257, 
  47.6886) and (117.161, 46.6053) .. (115.0399, 44.9082).. controls (113.1551, 
  43.1123) and (111.4327, 40.7568) .. (111.4685, 38.0426).. controls (111.1808, 
  37.0501) and (109.8385, 38.4124) .. (109.8727, 38.124).. controls (110.9981, 
  37.6035) and (110.4952, 36.2686) .. (109.3708, 36.3096).. controls (108.8912, 
  36.2842) and (108.2167, 34.8726) .. (108.877, 35.6856).. controls (109.7032, 
  36.4224) and (111.376, 36.561) .. (111.5222, 35.1902).. controls (112.2334, 
  33.9257) and (114.2083, 33.9914) .. (114.6997, 32.6899).. controls (114.5258, 
  30.7928) and (112.3978, 29.901) .. (111.3068, 28.5455).. controls (109.5652, 
  26.9572) and (106.8181, 25.1308) .. (104.5606, 26.8509).. controls (103.678, 
  27.6579) and (102.4196, 27.4456) .. (101.413, 27.8878).. controls (102.0712, 
  26.9718) and (101.391, 25.6269) .. (101.0049, 25.0007).. controls (101.6727, 
  25.0955) and (102.3512, 25.5096) .. (103.0522, 25.0587).. controls (104.1816, 
  25.237) and (105.7322, 23.4326) .. (104.2274, 22.8552).. controls (103.3527, 
  22.0638) and (102.6146, 20.7651) .. (101.2806, 20.8373).. controls (100.7037, 
  20.4819) and (100.8628, 20.2661) .. (101.4895, 20.2252).. controls (102.4576, 
  19.9281) and (103.0891, 18.8603) .. (103.2182, 18.0184).. controls (103.5438, 
  18.8956) and (103.8313, 19.9448) .. (104.9149, 20.2143).. controls (105.7605, 
  20.9608) and (106.3861, 23.1187) .. (107.852, 22.1851).. controls (108.3595, 
  20.6986) and (107.1809, 18.9579) .. (106.8509, 17.8519).. controls (107.743, 
  18.3388) and (108.9818, 17.828) .. (109.5352, 17.2951).. controls (109.1213, 
  18.1337) and (109.6103, 18.7439) .. (110.0143, 19.3901).. controls (110.3736, 
  20.9712) and (109.6521, 23.3766) .. (111.3475, 24.3164).. controls (112.1213, 
  24.7732) and (112.8505, 24.0479) .. (113.3361, 23.7472).. controls (114.1373, 
  24.0913) and (114.9697, 22.6761) .. (115.1744, 23.8715).. controls (116.065, 
  25.2345) and (115.9155, 27.3467) .. (117.3215, 28.3366).. controls (118.5545, 
  28.8235) and (118.8606, 27.0537) .. (119.2239, 26.5519).. controls (119.5312, 
  29.2178) and (121.6164, 31.2604) .. (124.0661, 32.1174).. controls (125.631, 
  32.7702) and (125.8475, 34.9839) .. (124.3064, 35.8066).. controls (123.5751, 
  36.2621) and (121.7086, 36.3476) .. (122.3395, 37.6514).. controls (124.5447, 
  39.801) and (126.0031, 42.5489) .. (127.287, 45.3043).. controls (127.4764, 
  45.6914) and (127.676, 46.0779) .. (127.8892, 46.4568).. controls (130.4484, 
  46.0439) and (133.1025, 46.2672) .. (135.5882, 47.0802).. controls (137.1431, 
  47.5067) and (138.9095, 46.7277) .. (139.4141, 45.2072).. controls (139.2962, 
  45.2396) and (139.1496, 45.1759) .. (138.9904, 44.9152).. controls (139.2755, 
  44.6021) and (139.46, 44.5136) .. (139.5666, 44.5398).. controls (140.0837, 
  42.5111) and (139.2282, 39.5601) .. (136.7575, 39.5831).. controls (135.3558, 
  39.555) and (134.1255, 40.8601) .. (132.7037, 40.0796).. controls (131.9323, 
  40.3547) and (131.8489, 40.1208) .. (132.3755, 39.5907).. controls (132.8763, 
  38.867) and (132.7887, 37.8143) .. (132.3223, 37.1604).. controls (132.9538, 
  37.3226) and (133.472, 38.0107) .. (134.2437, 37.5863).. controls (135.2, 
  37.3062) and (135.4766, 36.1703) .. (134.5936, 35.5407).. controls (134.1334, 
  34.6401) and (133.2272, 34.3481) .. (132.3538, 34.1158).. controls (133.6573, 
  34.0536) and (134.6748, 32.8375) .. (134.5698, 31.6144).. controls (135.3636, 
  32.1693) and (134.9443, 33.5241) .. (135.9268, 34.0523).. controls (136.7079, 
  34.5089) and (137.6611, 35.1945) .. (138.1846, 34.1413).. controls (138.0666, 
  35.1748) and (138.7021, 33.5672) .. (138.0712, 33.5602).. controls (138.8896, 
  33.5713) and (139.675, 32.9868) .. (139.8938, 32.2547).. controls (142.3734, 
  32.4057) and (143.1067, 35.4928) .. (142.5276, 37.5103).. controls (142.3974, 
  38.5992) and (143.3595, 39.3909) .. (144.1619, 39.7779).. controls (144.2489, 
  40.5173) and (145.1052, 40.8986) .. (145.5672, 41.0709).. controls (144.543, 
  43.1405) and (144.2853, 45.6431) .. (145.5233, 47.6907).. controls (146.7109, 
  49.0505) and (145.6984, 51.6863) .. (143.698, 50.7998).. controls (141.8579, 
  49.8009) and (138.7648, 49.747) .. (138.0283, 52.1487).. controls (137.5931, 
  53.1951) and (136.1941, 53.9307) .. (136.0283, 54.853).. controls (136.452, 
  56.301) and (138.2401, 54.8087) .. (139.3072, 54.9219).. controls (144.433, 
  54.0942) and (149.9989, 53.1166) .. (154.8197, 55.6566).. controls (157.4227, 
  56.9065) and (158.5548, 60.4142) .. (156.9011, 62.8455).. controls (155.375, 
  65.2792) and (152.3527, 65.5903) .. (149.7686, 65.7712).. controls (148.2728, 
  66.1689) and (146.8723, 66.9253) .. (145.6454, 67.8499).. controls (145.8813, 
  67.011) and (146.4084, 65.2333) .. (144.915, 65.3762).. controls (143.1974, 
  65.3927) and (141.5988, 66.7267) .. (140.4266, 67.4983).. controls (140.2197, 
  66.6467) and (141.2775, 65.0462) .. (139.7929, 64.9649).. controls (138.0214, 
  64.9568) and (136.7284, 66.4754) .. (135.3316, 67.2189).. controls (135.2099, 
  66.3437) and (136.2233, 64.6453) .. (134.7244, 64.5401).. controls (132.9134, 
  64.6067) and (131.2049, 66.0433) .. (129.8638, 66.7072).. controls (130.1611, 
  65.745) and (130.5444, 63.762) .. (128.9202, 63.8938).. controls (127.0907, 
  63.9975) and (125.8248, 65.9156) .. (124.396, 66.4967).. controls (124.4531, 
  65.5576) and (125.2388, 63.5304) .. (123.5962, 63.5965).. controls (121.9143, 
  63.62) and (121.0497, 65.8) .. (119.9287, 66.291).. controls (119.7233, 
  64.4449) and (117.7417, 63.367) .. (116.0339, 63.9931).. controls (113.2404, 
  64.4618) and (111.007, 67.8989) .. (112.8841, 70.4132).. controls (115.3436, 
  73.6637) and (120.2369, 73.4121) .. (123.5755, 71.7887).. controls (125.7789, 
  70.3815) and (128.5143, 68.703) .. (131.1845, 70.0166).. controls (132.2171, 
  70.4477) and (133.2796, 71.047) .. (134.3674, 71.5619).. controls (134.762, 
  71.5478) and (135.4088, 71.6282) .. (135.7684, 71.5695).. controls (137.4107, 
  71.622) and (139.052, 70.8464) .. (140.6708, 71.2569).. controls (139.4596, 
  72.5904) and (137.853, 72.4762) .. (136.2307, 72.3042).. controls (137.537, 
  72.6988) and (138.8759, 72.7813) .. (140.2416, 72.1365).. controls (141.8216, 
  71.36) and (144.4393, 71.5895) .. (145.4853, 72.487).. controls (143.7679, 
  71.898) and (142.2303, 72.5825) .. (140.6724, 73.2619).. controls (141.9807, 
  73.2138) and (143.292, 73.3539) .. (144.4175, 74.187).. controls (143.1274, 
  73.6293) and (141.6759, 73.6018) .. (140.3219, 73.4132).. controls (139.521, 
  73.7576) and (138.7098, 74.0697) .. (137.8628, 74.1892).. controls (137.88, 
  74.2299) and (137.8985, 74.2665) .. (137.9149, 74.3102).. controls (137.935, 
  74.4497) and (138.5665, 74.8814) .. (138.4184, 74.9038) -- cycle(65.7569, 
  73.4415).. controls (65.6539, 73.4463) and (65.5507, 73.4447) .. (65.4476, 
  73.4371).. controls (64.8503, 73.3987) and (64.2333, 73.145) .. (65.0488, 
  72.901).. controls (65.8818, 72.5711) and (65.8665, 71.496) .. (66.1118, 
  70.9553).. controls (67.0187, 70.8036) and (68.7166, 71.5291) .. (67.6794, 
  72.5565).. controls (67.1825, 73.0783) and (66.4778, 73.408) .. (65.7569, 
  73.4415) -- cycle(82.8489, 64.6187).. controls (82.8215, 64.6148) and 
  (82.7814, 64.5868) .. (82.7257, 64.5249).. controls (82.619, 63.492) and 
  (83.0404, 64.6461) .. (82.8489, 64.6187) -- cycle(82.8738, 58.7836).. controls
   (82.8282, 58.7803) and (82.8121, 58.7516) .. (82.8466, 58.6805).. controls 
  (82.9263, 58.033) and (82.1731, 57.4099) .. (83.2813, 57.7353).. controls 
  (83.6946, 57.1975) and (84.213, 58.1525) .. (84.8429, 58.0999).. controls 
  (84.3788, 58.2396) and (83.6721, 58.4572) .. (83.4771, 58.5845).. controls 
  (83.4088, 58.5738) and (83.0103, 58.7933) .. (82.8738, 58.7836) -- 
  cycle(84.8429, 58.0999).. controls (84.9308, 58.0735) and (85.012, 58.0486) ..
   (85.074, 58.0299).. controls (84.9954, 58.0695) and (84.9183, 58.0936) .. 
  (84.8429, 58.0999) -- cycle(85.074, 58.0299).. controls (85.0965, 58.0187) and
   (85.1187, 58.0112) .. (85.1413, 57.9968).. controls (85.3325, 57.951) and 
  (85.2459, 57.9782) .. (85.074, 58.0299) -- cycle(72.382, 45.9153).. controls 
  (72.2981, 45.92) and (72.2134, 45.919) .. (72.1286, 45.9121).. controls 
  (72.574, 45.5418) and (72.9826, 44.6419) .. (72.7309, 43.9164).. controls 
  (73.5565, 43.4545) and (74.9938, 44.3908) .. (73.936, 45.1622).. controls 
  (73.5275, 45.5794) and (72.9696, 45.8828) .. (72.382, 45.9153) -- 
  cycle(145.2579, 40.594).. controls (145.0875, 40.6016) and (144.8859, 40.5533)
   .. (144.6507, 40.4328).. controls (144.8163, 39.776) and (145.6369, 39.1161) 
  .. (145.8629, 38.7051).. controls (146.144, 39.4797) and (145.9961, 40.5606) 
  .. (145.2579, 40.594) -- cycle(70.8681, 40.13).. controls (69.9789, 40.1081) 
  and (68.9882, 39.3814) .. (68.8616, 38.5684).. controls (69.5351, 39.0149) and
   (70.3918, 38.7618) .. (70.962, 38.3774).. controls (71.3334, 38.8821) and 
  (72.0276, 40.1331) .. (70.8681, 40.13) -- cycle(131.3211, 39.5739).. controls 
  (130.2048, 39.2536) and (129.5362, 38.0869) .. (129.5598, 36.9623).. controls 
  (130.0011, 37.6921) and (130.9092, 37.7305) .. (131.5805, 37.6357).. controls 
  (132.1251, 38.0424) and (131.9422, 39.4215) .. (131.3211, 39.5739) -- 
  cycle(74.9572, 34.4631).. controls (74.8752, 34.4629) and (74.7923, 34.4559) 
  .. (74.7108, 34.4424).. controls (73.4688, 34.3234) and (73.6805, 32.03) .. 
  (74.3815, 31.6394).. controls (74.3523, 32.6168) and (75.3198, 33.0318) .. 
  (75.9339, 33.456).. controls (76.0726, 34.1404) and (75.5312, 34.4646) .. 
  (74.9572, 34.4631) -- cycle(132.4991, 33.4772).. controls (131.4508, 33.0383) 
  and (131.3899, 31.3819) .. (131.7731, 30.4223).. controls (132.1036, 31.2702) 
  and (133.1015, 31.7081) .. (133.7438, 32.0084).. controls (133.6472, 32.6398) 
  and (133.2599, 33.5145) .. (132.4991, 33.4772) -- cycle(138.0081, 32.8168).. 
  controls (137.2717, 32.5706) and (137.4882, 30.8696) .. (137.7754, 30.7734).. 
  controls (137.8911, 31.4635) and (138.6204, 31.8955) .. (139.0608, 32.1831).. 
  controls (138.9113, 32.6039) and (138.4385, 32.8452) .. (138.0081, 32.8168) --
   cycle(100.6304, 27.3583).. controls (99.7634, 27.324) and (98.6909, 26.804) 
  .. (98.4557, 26.0321).. controls (99.1152, 26.3901) and (100.0033, 26.1827) ..
   (100.4248, 25.6241).. controls (100.8975, 25.8845) and (101.2016, 27.1137) ..
   (100.6304, 27.3583) -- cycle(113.8059, 23.4434).. controls (113.2861, 
  23.1096) and (113.3104, 21.9849) .. (113.4456, 21.7423).. controls (114.5091, 
  22.1248) and (114.1572, 22.9059) .. (113.8059, 23.4434) -- cycle(101.2116, 
  20.0809).. controls (100.3749, 19.3186) and (100.1258, 17.7701) .. (100.619, 
  16.7004).. controls (100.8483, 17.5769) and (101.604, 18.4652) .. (102.5279, 
  18.4791).. controls (102.2617, 19.0862) and (101.9149, 19.9175) .. (101.2116, 
  20.0809) -- cycle(107.6544, 17.1448).. controls (106.6715, 17.052) and 
  (107.5212, 15.5369) .. (107.9192, 15.0933).. controls (108.2537, 14.746) and 
  (109.0055, 14.3685) .. (108.5627, 15.1052).. controls (108.2747, 15.8757) and 
  (109.3892, 16.7754) .. (108.2073, 17.0846).. controls (108.0289, 17.1386) and 
  (107.8401, 17.1567) .. (107.6544, 17.1448) -- cycle;



    \path[fill=black,line width=1.7639mm,miter limit=10.0] (133.9103, 
  227.7543).. controls (133.896, 227.7582) and (133.8673, 227.754) .. (133.8209,
   227.7393).. controls (133.1262, 227.6136) and (132.4655, 227.3664) .. 
  (131.8076, 227.1099).. controls (131.1918, 227.1839) and (130.5759, 227.237) 
  .. (129.9411, 227.2246).. controls (126.1714, 226.774) and (122.3305, 
  225.5747) .. (118.548, 226.6835).. controls (113.7528, 227.4472) and 
  (108.4247, 227.5979) .. (104.0429, 225.203).. controls (101.173, 223.7623) and
   (100.657, 219.4237) .. (103.1406, 217.3849).. controls (107.0503, 214.3737) 
  and (112.2106, 215.2806) .. (116.8313, 215.2119).. controls (126.4887, 
  215.2667) and (136.0111, 217.2186) .. (145.6662, 217.6464).. controls 
  (149.3454, 217.629) and (153.4501, 218.053) .. (156.7198, 216.031).. controls 
  (158.5323, 214.8459) and (157.8154, 211.928) .. (155.834, 211.3698).. controls
   (151.7653, 210.0862) and (147.3671, 210.9647) .. (143.2834, 211.7558).. 
  controls (135.7135, 213.8114) and (127.6936, 214.4066) .. (119.9298, 
  212.9888).. controls (112.7464, 212.1877) and (105.7056, 209.63) .. (98.4019, 
  210.1383).. controls (95.6838, 210.214) and (92.6525, 211.664) .. (92.1542, 
  214.5897).. controls (92.7709, 215.8417) and (94.0859, 214.2983) .. (95.0801, 
  214.9638).. controls (93.3865, 215.2621) and (92.2069, 217.175) .. (92.5335, 
  218.7951).. controls (92.8483, 218.9557) and (94.7721, 219.4172) .. (93.5427, 
  219.3083).. controls (92.0831, 219.1564) and (91.5745, 220.6975) .. (91.3729, 
  221.842).. controls (91.1895, 222.233) and (90.9217, 222.5437) .. (90.6044, 
  222.8063).. controls (90.8302, 223.7069) and (90.6624, 225.026) .. (89.4334, 
  224.8614).. controls (88.8633, 224.8325) and (88.3037, 224.6972) .. (87.7679, 
  224.4992).. controls (87.8858, 225.092) and (87.7732, 225.6161) .. (87.0377, 
  225.9131).. controls (86.9274, 225.9458) and (86.8103, 225.9465) .. (86.6967, 
  225.9363).. controls (86.1454, 225.8649) and (85.538, 225.6555) .. (85.2973, 
  225.2413).. controls (84.1148, 225.5006) and (82.9041, 225.6516) .. (81.7073, 
  225.6728).. controls (80.2529, 225.8332) and (79.4594, 224.2072) .. (77.9763, 
  224.4263).. controls (76.0629, 224.2126) and (74.3543, 223.1965) .. (72.4123, 
  223.0858).. controls (72.306, 222.1577) and (73.1763, 221.5037) .. (72.6841, 
  220.5899).. controls (72.443, 218.3116) and (75.4229, 216.7048) .. (77.338, 
  217.8443).. controls (78.7608, 218.4801) and (80.3546, 218.109) .. (81.4815, 
  217.1787).. controls (81.4512, 217.1516) and (81.4216, 217.1256) .. (81.3895, 
  217.094).. controls (82.389, 215.1788) and (82.3039, 212.707) .. (81.4773, 
  210.7486).. controls (81.5329, 210.6558) and (81.5938, 210.5752) .. (81.6587, 
  210.5047).. controls (81.304, 210.2753) and (80.8692, 210.1545) .. (80.3405, 
  210.2003).. controls (79.8879, 210.056) and (78.3462, 210.2718) .. (78.615, 
  210.0763).. controls (79.1575, 208.3217) and (76.8271, 207.4265) .. (75.4565, 
  207.1571).. controls (74.9816, 207.1374) and (74.5855, 206.9894) .. (74.2323, 
  206.77).. controls (69.8444, 209.0888) and (65.3764, 212.4088) .. (64.0918, 
  217.403).. controls (63.9973, 218.5812) and (64.8356, 219.5597) .. (64.9357, 
  220.7129).. controls (65.2389, 221.0282) and (66.3043, 221.6709) .. (65.7527, 
  222.0993).. controls (64.5707, 221.9078) and (63.4532, 223.1127) .. (63.0991, 
  224.0315).. controls (62.5659, 223.035) and (62.961, 222.2692) .. (63.0676, 
  221.1387).. controls (63.4027, 220.0845) and (62.4071, 218.6786) .. (61.3189, 
  219.5258).. controls (60.2219, 220.1064) and (59.7051, 221.4798) .. (60.2781, 
  222.5804).. controls (60.2103, 223.1044) and (59.7251, 223.5006) .. (59.5748, 
  222.7773).. controls (58.9899, 221.8186) and (57.4683, 221.7738) .. (56.7538, 
  221.8616).. controls (57.0308, 221.1038) and (58.1698, 221.193) .. (58.2746, 
  220.2245).. controls (58.9208, 219.2118) and (61.16, 218.6469) .. (60.2885, 
  217.1296).. controls (59.5608, 215.5616) and (57.4586, 216.227) .. (57.1703, 
  217.7275).. controls (56.6128, 218.291) and (57.2727, 216.8579) .. (56.8577, 
  216.5203).. controls (56.8029, 216.3244) and (56.692, 216.1573) .. (56.5528, 
  216.0113).. controls (57.0715, 216.7712) and (56.4442, 218.3203) .. (55.2293, 
  217.9554).. controls (54.4869, 217.997) and (53.3274, 217.2587) .. (53.3535, 
  216.7498).. controls (54.3428, 217.0382) and (55.1529, 216.3671) .. (55.6701, 
  215.6129).. controls (55.8669, 215.6077) and (56.0328, 215.6409) .. (56.1724, 
  215.7008).. controls (55.8323, 215.4753) and (55.4536, 215.3185) .. (55.2257, 
  215.1917).. controls (55.8406, 214.7442) and (56.6164, 214.9606) .. (57.0328, 
  214.183).. controls (58.1169, 213.6496) and (59.5428, 214.11) .. (60.5081, 
  213.1521).. controls (61.8894, 212.3833) and (62.6835, 209.8432) .. (60.8677, 
  209.126).. controls (60.8558, 209.0808) and (60.8452, 209.0171) .. (60.8342, 
  208.9632).. controls (60.2599, 209.0538) and (59.572, 208.9591) .. (59.0622, 
  209.0402).. controls (59.1533, 208.232) and (60.0126, 207.7867) .. (60.6047, 
  208.0015).. controls (60.5435, 207.8356) and (60.4668, 207.6846) .. (60.3598, 
  207.5762).. controls (62.1393, 206.6716) and (65.107, 206.7161) .. (65.7388, 
  204.4182).. controls (65.6903, 202.8616) and (63.6078, 203.3186) .. (62.8004, 
  202.4664).. controls (66.0873, 202.7681) and (69.1109, 200.6644) .. (70.6573, 
  197.8667).. controls (70.6976, 197.9184) and (70.74, 197.969) .. (70.7814, 
  198.0202).. controls (70.7859, 197.3544) and (70.8326, 196.6917) .. (71.0619, 
  196.0332).. controls (70.6116, 195.0974) and (71.2059, 193.4103) .. (69.7003, 
  193.4975).. controls (66.3343, 193.049) and (62.758, 192.7292) .. (59.4668, 
  193.7104).. controls (58.4581, 194.2942) and (58.2846, 195.5895) .. (57.3904, 
  196.3045).. controls (57.1639, 196.8361) and (57.5835, 197.818) .. (56.7734, 
  197.9814).. controls (56.2293, 196.8159) and (54.5123, 196.7818) .. (53.5886, 
  196.9743).. controls (54.1492, 195.477) and (56.3738, 195.6322) .. (56.9155, 
  193.9734).. controls (56.8863, 192.4447) and (54.2976, 192.1975) .. (53.4532, 
  193.2815).. controls (53.4463, 193.9445) and (51.8371, 194.3604) .. (52.4812, 
  193.4546).. controls (52.8867, 192.3797) and (51.9758, 191.1178) .. (51.4327, 
  190.6113).. controls (52.2239, 190.2909) and (52.837, 191.3149) .. (53.6796, 
  190.7927).. controls (54.8801, 190.6838) and (56.6803, 192.1338) .. (57.3739, 
  190.5286).. controls (58.1097, 189.1994) and (56.616, 187.7367) .. (55.2944, 
  188.3934).. controls (54.348, 188.4797) and (54.9234, 188.1885) .. (55.4055, 
  187.8523).. controls (55.6951, 187.6383) and (55.845, 187.3326) .. (55.9151, 
  187.0007).. controls (55.6939, 187.513) and (55.0018, 187.8559) .. (54.402, 
  187.8575).. controls (53.2873, 187.401) and (52.7552, 185.4916) .. (53.3788, 
  184.8117).. controls (53.7471, 185.8114) and (54.8127, 186.0624) .. (55.7559, 
  186.0271).. controls (55.8684, 186.1807) and (55.932, 186.3293) .. (55.9637, 
  186.472).. controls (55.9637, 186.0246) and (55.8799, 185.5879) .. (55.8272, 
  185.3067).. controls (56.5631, 185.538) and (56.8901, 186.263) .. (57.7692, 
  186.1237).. controls (58.8474, 186.6485) and (59.3263, 188.0679) .. (60.6709, 
  188.243).. controls (62.1798, 188.8638) and (64.3656, 187.9783) .. (64.2676, 
  186.2209).. controls (64.2765, 186.2163) and (64.2845, 186.2103) .. (64.2932, 
  186.2054).. controls (63.8847, 185.7509) and (63.6429, 185.0166) .. (63.0907, 
  184.729).. controls (63.8378, 184.3116) and (64.6954, 184.7303) .. (64.857, 
  185.3605).. controls (64.9299, 185.1276) and (64.9785, 184.9024) .. (65.0348, 
  184.7533).. controls (66.854, 185.5784) and (68.6072, 187.9645) .. (70.8231, 
  187.0849).. controls (72.1315, 186.2902) and (70.5527, 184.9911) .. (70.9016, 
  183.9399).. controls (74.4396, 187.7919) and (80.3389, 187.4734) .. (84.9943, 
  186.3113).. controls (85.8181, 185.8328) and (87.8064, 186.0829) .. (87.0515, 
  187.4446).. controls (86.7458, 188.4763) and (86.0741, 189.8227) .. (86.543, 
  190.7937).. controls (87.465, 191.4282) and (88.6444, 190.2445) .. (89.7227, 
  190.4423).. controls (96.791, 189.8911) and (102.6004, 194.6594) .. (107.8337,
   198.755).. controls (111.6071, 201.8461) and (116.1865, 204.7564) .. 
  (121.1931, 204.7061).. controls (121.1519, 204.6588) and (121.1114, 204.6141) 
  .. (121.0696, 204.5645).. controls (120.8971, 204.4133) and (120.7248, 
  204.2619) .. (120.5523, 204.1108).. controls (119.939, 203.5683) and 
  (118.7353, 202.7961) .. (118.769, 202.0101).. controls (119.9547, 202.7436) 
  and (121.2312, 203.3395) .. (122.56, 203.7718).. controls (121.7436, 202.6143)
   and (121.6897, 201.1183) .. (121.3301, 199.8159).. controls (120.8129, 
  199.142) and (120.1179, 200.3938) .. (119.8258, 200.2448).. controls 
  (120.6554, 199.8186) and (120.6315, 198.0957) .. (119.4093, 198.5566).. 
  controls (118.9422, 198.641) and (118.112, 198.1112) .. (118.122, 197.9571).. 
  controls (119.4308, 198.9467) and (121.4708, 198.1993) .. (121.9031, 
  196.6161).. controls (123.914, 194.1342) and (127.4177, 194.2215) .. 
  (130.2174, 193.4453).. controls (130.9335, 192.1921) and (128.8909, 191.8202) 
  .. (128.0284, 191.7709).. controls (124.7865, 191.6666) and (121.2971, 
  191.176) .. (118.2434, 192.5228).. controls (116.2903, 193.3613) and 
  (117.0543, 195.8731) .. (115.6384, 197.1127).. controls (115.5068, 197.5152) 
  and (115.281, 198.5098) .. (115.0421, 197.5825).. controls (114.6649, 
  196.7748) and (113.411, 196.5946) .. (112.9368, 196.3981).. controls 
  (113.4338, 195.9579) and (114.2036, 195.7678) .. (114.3429, 194.9568).. 
  controls (115.1444, 194.3233) and (115.0355, 192.1248) .. (113.6902, 
  192.6396).. controls (112.5206, 192.8282) and (111.0632, 192.3714) .. 
  (110.1959, 193.4163).. controls (109.5419, 193.5928) and (109.4695, 193.3565) 
  .. (109.8713, 192.8639).. controls (110.3258, 191.9776) and (109.9206, 190.82)
   .. (109.4579, 190.1705).. controls (110.2688, 190.4974) and (111.1923, 
  191.0463) .. (112.0789, 190.4341).. controls (113.1805, 190.2451) and 
  (115.1449, 191.3302) .. (115.4813, 189.6574).. controls (115.0371, 188.621) 
  and (113.9136, 188.5041) .. (112.9688, 188.4068).. controls (112.6489, 
  188.1058) and (111.2894, 188.3909) .. (112.234, 187.949).. controls (112.9893,
   187.4977) and (113.1238, 186.2427) .. (113.1202, 185.7191).. controls 
  (113.5387, 186.516) and (114.2808, 186.5189) .. (115.0255, 186.5816).. 
  controls (116.4713, 187.167) and (117.754, 189.3279) .. (119.4816, 188.52).. 
  controls (120.2389, 188.2309) and (120.0981, 187.2896) .. (120.2536, 
  186.8095).. controls (120.7036, 186.4705) and (120.9251, 185.8859) .. 
  (120.8768, 185.3321).. controls (122.9769, 185.5381) and (124.7555, 186.947) 
  .. (126.7917, 187.2586).. controls (128.3038, 187.0187) and (127.4854, 
  185.1939) .. (126.9819, 184.3848).. controls (126.907, 183.8429) and 
  (127.8661, 185.0891) .. (128.2309, 185.2096).. controls (131.0627, 187.5684) 
  and (135.233, 188.2403) .. (138.5848, 186.639).. controls (140.0176, 185.8801)
   and (142.3063, 187.3153) .. (141.4337, 188.9411).. controls (139.7764, 
  191.2405) and (136.9207, 192.5968) .. (135.8806, 195.3857).. controls 
  (135.5657, 196.1426) and (135.3177, 196.9539) .. (135.214, 197.7768).. 
  controls (136.1924, 197.2915) and (137.2192, 196.9044) .. (138.3244, 
  196.6776).. controls (141.911, 195.6758) and (145.7811, 196.1322) .. 
  (149.2632, 197.4037).. controls (152.4799, 198.2755) and (156.1659, 199.7972) 
  .. (159.3975, 198.229).. controls (161.215, 197.1917) and (161.1373, 193.5814)
   .. (158.9148, 193.1657).. controls (157.3619, 193.3596) and (155.7989, 
  194.2615) .. (154.2319, 193.5192).. controls (153.0225, 193.7871) and 
  (154.5895, 192.6445) .. (154.2764, 191.9482).. controls (154.3131, 191.4706) 
  and (153.5506, 190.2978) .. (154.3332, 190.9762).. controls (155.2575, 
  191.6826) and (157.4228, 190.9524) .. (156.9506, 189.7044).. controls 
  (156.2878, 188.7868) and (155.4852, 187.911) .. (154.3156, 187.796).. controls
   (155.6132, 187.7534) and (156.5691, 186.5337) .. (156.5119, 185.3605).. 
  controls (157.2132, 185.8024) and (156.9508, 186.9341) .. (157.839, 
  187.3066).. controls (158.6518, 188.1356) and (160.2394, 188.6805) .. 
  (160.8548, 187.9216).. controls (160.7427, 188.7257) and (161.3163, 187.2629) 
  .. (160.7447, 187.2374).. controls (161.5435, 187.2502) and (162.2771, 
  186.6442) .. (162.4934, 185.9599).. controls (164.9687, 186.1231) and 
  (165.6336, 189.2221) .. (165.0127, 191.2066).. controls (164.6647, 192.3024) 
  and (165.7896, 192.9296) .. (166.4725, 193.3445).. controls (166.621, 
  193.9408) and (167.4611, 194.1514) .. (167.7407, 194.4214).. controls 
  (166.6181, 196.8925) and (163.2626, 197.7281) .. (162.9446, 200.6862).. 
  controls (162.8296, 202.0183) and (163.8568, 204.1877) .. (162.2201, 
  204.9329).. controls (160.0583, 204.8417) and (158.3145, 203.1276) .. 
  (156.084, 203.2178).. controls (151.1383, 202.7054) and (145.9422, 204.0368) 
  .. (142.2585, 207.4914).. controls (141.6869, 208.3295) and (139.7481, 
  208.9496) .. (140.2312, 210.0107).. controls (140.8306, 210.962) and 
  (142.4308, 209.7062) .. (143.4501, 209.805).. controls (148.1431, 209.1308) 
  and (153.1483, 208.0609) .. (157.625, 210.0463).. controls (160.2368, 
  211.1402) and (161.1396, 215.0053) .. (158.9148, 216.9193).. controls 
  (155.6975, 219.828) and (151.0528, 219.7644) .. (147.0117, 220.0126).. 
  controls (145.9642, 220.0958) and (144.8532, 221.0191) .. (144.3106, 
  221.5175).. controls (144.5991, 220.3318) and (143.5026, 219.2602) .. 
  (142.337, 219.7801).. controls (141.0937, 219.951) and (139.7272, 221.2861) ..
   (139.0556, 221.6255).. controls (139.3612, 220.8755) and (139.6516, 219.2875)
   .. (138.3435, 219.4519).. controls (136.7201, 219.4749) and (135.2162, 
  220.7026) .. (134.1065, 221.4668).. controls (133.9102, 220.6519) and 
  (134.9133, 219.1424) .. (133.5009, 219.0649).. controls (131.8306, 219.0562) 
  and (130.6097, 220.497) .. (129.2934, 221.1857).. controls (129.1873, 
  220.3566) and (130.1352, 218.7382) .. (128.7084, 218.6484).. controls 
  (127.137, 218.6632) and (125.8839, 220.0223) .. (124.7831, 220.7681).. 
  controls (124.6138, 219.9529) and (125.4836, 218.3579) .. (124.0855, 
  218.3011).. controls (122.3795, 218.3519) and (120.9528, 220.017) .. 
  (119.7979, 220.6813).. controls (120.1029, 219.7526) and (120.4023, 217.87) ..
   (118.848, 218.0458).. controls (117.1352, 218.1656) and (115.9356, 219.9772) 
  .. (114.6064, 220.5087).. controls (114.7622, 219.5974) and (115.326, 
  217.5043) .. (113.6644, 217.8097).. controls (112.1801, 217.9826) and 
  (111.4516, 219.7761) .. (110.4041, 220.3377).. controls (110.2641, 218.693) 
  and (108.5699, 217.5835) .. (107.0012, 218.1239).. controls (105.0696, 
  218.2872) and (102.6678, 219.784) .. (103.3508, 222.0228).. controls 
  (104.3211, 224.8772) and (107.8596, 225.415) .. (110.4635, 225.6836).. 
  controls (113.7462, 226.0468) and (117.0699, 225.4022) .. (119.9363, 
  223.754).. controls (122.4668, 222.4505) and (125.4959, 222.4608) .. 
  (128.0309, 223.7514).. controls (128.5768, 224.0136) and (129.1462, 224.2781) 
  .. (129.7295, 224.5177).. controls (130.0409, 224.4393) and (130.8722, 
  224.5631) .. (131.3139, 224.4971).. controls (132.9161, 224.5408) and 
  (134.5227, 223.7928) .. (136.1007, 224.2015).. controls (135.0152, 225.3743) 
  and (133.6211, 225.4149) .. (132.178, 225.2753).. controls (133.5064, 
  225.5225) and (134.8529, 225.456) .. (136.1343, 224.7611).. controls 
  (137.6783, 224.2994) and (139.5646, 224.3697) .. (140.8808, 225.3802).. 
  controls (139.0805, 224.6342) and (137.4888, 225.4416) .. (135.862, 
  226.1455).. controls (137.2965, 226.0689) and (138.7468, 226.263) .. (139.901,
   227.1677).. controls (138.54, 226.5196) and (136.9683, 226.5352) .. (135.54, 
  226.2835).. controls (134.8112, 226.5884) and (134.0719, 226.8499) .. 
  (133.3009, 226.9222).. controls (133.3408, 227.0041) and (133.3789, 227.0923) 
  .. (133.4146, 227.1884).. controls (133.4175, 227.296) and (134.0107, 
  227.7267) .. (133.9102, 227.7542) -- cycle(64.7104, 226.235).. controls 
  (64.9432, 225.3025) and (64.2905, 224.5104) .. (63.6133, 224.0542).. controls 
  (63.8633, 222.9738) and (65.8121, 222.412) .. (65.6809, 223.9504).. controls 
  (65.7453, 224.7854) and (65.3993, 225.8242) .. (64.7104, 226.235) -- 
  cycle(56.9646, 224.8521).. controls (56.8683, 224.8565) and (56.7716, 
  224.8556) .. (56.6752, 224.8485).. controls (56.1462, 224.816) and (55.5402, 
  224.5959) .. (56.3001, 224.3731).. controls (57.0545, 224.0527) and (57.0957, 
  223.1121) .. (57.2643, 222.5494).. controls (58.1083, 222.3918) and (59.742, 
  223.061) .. (58.7645, 224.0305).. controls (58.2975, 224.5158) and (57.6385, 
  224.8212) .. (56.9646, 224.8521) -- cycle(74.4421, 216.8986).. controls 
  (74.4117, 216.8983) and (74.3682, 216.8752) .. (74.3088, 216.8195).. controls 
  (74.2374, 215.7837) and (74.6552, 216.9011) .. (74.4421, 216.8986) -- 
  cycle(78.03, 214.9819).. controls (75.7991, 215.0178) and (73.55, 212.7142) ..
   (71.4015, 214.3778).. controls (70.8048, 214.8957) and (71.8955, 213.3387) ..
   (72.3818, 213.4063).. controls (74.8303, 212.6497) and (76.9189, 215.2063) ..
   (79.4082, 214.3546).. controls (80.3562, 214.1601) and (81.253, 213.7469) .. 
  (82.0411, 213.1888).. controls (81.3038, 214.2236) and (79.3901, 214.9755) .. 
  (78.03, 214.9819) -- cycle(74.5181, 211.1568).. controls (74.4826, 211.1545) 
  and (74.4494, 211.1492) .. (74.4183, 211.1403).. controls (74.5879, 210.6876) 
  and (73.7436, 209.844) .. (74.5682, 210.2256).. controls (75.2259, 209.6434) 
  and (75.5037, 210.5568) .. (76.2766, 210.4995).. controls (76.8855, 210.387) 
  and (77.9515, 210.1863) .. (76.7474, 210.4509).. controls (76.046, 210.5276) 
  and (75.0496, 211.191) .. (74.5181, 211.1568) -- cycle(123.8076, 204.3774).. 
  controls (123.8309, 204.3714) and (123.8546, 204.366) .. (123.8779, 
  204.3598).. controls (123.8564, 204.3524) and (123.8351, 204.345) .. 
  (123.8138, 204.3376).. controls (123.8111, 204.3509) and (123.8098, 204.3643) 
  .. (123.8077, 204.3774) -- cycle(53.2491, 199.7854).. controls (52.1243, 
  199.8019) and (53.602, 199.296) .. (53.6315, 198.7602).. controls (53.9678, 
  198.2052) and (53.3167, 197.0851) .. (54.4625, 197.4419).. controls (55.7364, 
  197.3153) and (56.0014, 198.9297) .. (54.7942, 199.3209).. controls (54.339, 
  199.6193) and (53.7974, 199.8055) .. (53.2491, 199.7854) -- cycle(112.6439, 
  199.0134).. controls (112.5787, 199.0155) and (112.5134, 199.015) .. 
  (112.4486, 199.0123).. controls (111.2761, 198.9675) and (112.9724, 198.5738) 
  .. (112.9281, 197.9473).. controls (113.0162, 197.4642) and (112.898, 
  196.8903) .. (113.6345, 197.2527).. controls (114.3069, 197.4383) and 
  (114.7668, 198.2978) .. (113.9131, 198.5648).. controls (113.5602, 198.8619) 
  and (113.1006, 198.9982) .. (112.6439, 199.0134) -- cycle(167.4747, 
  193.6706).. controls (167.434, 193.6695) and (167.3905, 193.6643) .. 
  (167.3429, 193.6545) -- (167.0809, 193.2478).. controls (167.4055, 193.0323) 
  and (167.6819, 192.3881) .. (167.8385, 192.2975).. controls (167.9364, 
  192.7522) and (168.0841, 193.6847) .. (167.4747, 193.6705) -- cycle(50.9903, 
  193.6215).. controls (49.9272, 193.6353) and (48.67, 192.6047) .. (48.6835, 
  191.6976).. controls (49.4493, 192.2405) and (50.4198, 192.0294) .. (51.1014, 
  191.4831).. controls (51.6085, 192.0875) and (52.3705, 193.6164) .. (50.9903, 
  193.6215) -- cycle(152.8942, 193.1507).. controls (151.7934, 192.8322) and 
  (151.1414, 191.6754) .. (151.177, 190.5674).. controls (151.6034, 191.2858) 
  and (152.4953, 191.3606) .. (153.1588, 191.2459).. controls (153.6926, 
  191.6534) and (153.521, 193.0099) .. (152.8942, 193.1507) -- cycle(109.3842, 
  193.0091).. controls (109.3316, 193.0149) and (109.2727, 193.0143) .. 
  (109.2069, 193.0065).. controls (108.2137, 192.7444) and (107.1102, 192.1052) 
  .. (106.7761, 191.1218).. controls (107.5563, 191.5737) and (108.727, 
  191.5717) .. (109.3299, 190.9399).. controls (109.5376, 191.3983) and 
  (110.1727, 192.9225) .. (109.3842, 193.009) -- cycle(111.4212, 187.3133).. 
  controls (110.6535, 186.7928) and (110.0725, 185.4113) .. (110.5732, 
  184.806).. controls (110.8727, 185.4787) and (111.6198, 185.7658) .. 
  (112.3142, 185.7796).. controls (112.2782, 186.3022) and (112.0476, 187.3594) 
  .. (111.4212, 187.3133) -- cycle(154.4517, 187.2317).. controls (153.4156, 
  186.7979) and (153.3523, 185.1589) .. (153.752, 184.2174).. controls (154.072,
   185.0552) and (155.0594, 185.4849) .. (155.6806, 185.793).. controls 
  (155.581, 186.4206) and (155.1936, 187.2568) .. (154.4517, 187.2317) -- 
  cycle(120.1266, 186.6017).. controls (119.7596, 186.2674) and (119.16, 
  185.5882) .. (119.0151, 185.3631).. controls (120.0445, 184.9191) and 
  (120.3326, 185.8632) .. (120.1266, 186.6017) -- cycle(160.6208, 186.5836).. 
  controls (159.8647, 186.3384) and (160.1101, 184.5523) .. (160.4177, 
  184.5848).. controls (160.5097, 185.2712) and (161.2934, 185.6484) .. 
  (161.6544, 185.9594).. controls (161.5085, 186.3724) and (161.0434, 186.6121) 
  .. (160.6208, 186.5836) -- cycle(127.0957, 183.774).. controls (127.0838, 
  183.7907) and (127.0846, 183.7797) .. (127.1335, 183.6913).. controls 
  (127.1308, 183.713) and (127.1075, 183.7573) .. (127.0957, 183.774) -- cycle;



    \path[fill=black,line width=1.7639mm,miter limit=10.0] (79.9638, 150.6516)..
   controls (80.248, 150.4723) and (80.45, 150.166) .. (80.6366, 149.8583).. 
  controls (79.8587, 149.7011) and (79.0941, 149.4819) .. (78.3412, 149.2212).. 
  controls (76.952, 149.426) and (75.4542, 149.4597) .. (74.1367, 150.0635) -- 
  (74.2923, 149.9482).. controls (75.3802, 149.1954) and (76.7048, 149.0345) .. 
  (78.0001, 149.0997).. controls (77.2969, 148.8426) and (76.6035, 148.5529) .. 
  (75.9232, 148.2331).. controls (75.1432, 147.8608) and (73.6153, 148.2026) .. 
  (73.2934, 148.23).. controls (75.1709, 146.9838) and (77.5511, 147.538) .. 
  (79.5002, 148.2998).. controls (78.9556, 147.9265) and (78.4109, 147.5532) .. 
  (77.8662, 147.1799).. controls (79.9084, 146.8532) and (81.9739, 147.7069) .. 
  (84.0524, 147.4988).. controls (86.398, 146.5427) and (88.6867, 145.233) .. 
  (91.2789, 145.9299).. controls (94.3104, 146.6819) and (96.9411, 148.7957) .. 
  (100.1889, 148.6842).. controls (103.3464, 148.6997) and (107.1942, 148.6636) 
  .. (109.4426, 146.0916).. controls (110.7662, 144.4968) and (109.7358, 
  142.0293) .. (107.7745, 141.5818).. controls (106.2656, 140.9528) and 
  (103.8786, 140.7163) .. (103.2409, 142.649).. controls (103.1025, 143.1481) 
  and (103.0514, 143.7037) .. (102.6043, 143.0805).. controls (101.6773, 
  142.1186) and (100.4558, 140.2924) .. (98.9254, 141.0976).. controls (98.2744,
   141.7699) and (99.1639, 143.1026) .. (98.804, 143.559).. controls (97.3061, 
  142.7239) and (95.8064, 140.5064) .. (93.8916, 141.3669).. controls (93.1179, 
  141.9512) and (94.0244, 143.4187) .. (93.7071, 143.7052).. controls (92.3873, 
  142.7744) and (90.8666, 140.9855) .. (89.0883, 141.555).. controls (88.4205, 
  142.1651) and (89.2087, 143.1383) .. (88.926, 143.8422).. controls (87.7094, 
  142.9211) and (86.1713, 141.2586) .. (84.516, 141.941).. controls (83.851, 
  142.5228) and (84.9026, 143.8562) .. (84.3961, 144.1781).. controls (83.0374, 
  143.3833) and (81.569, 141.6883) .. (79.8532, 142.3311).. controls (79.2379, 
  142.9693) and (80.0767, 143.8192) .. (79.7964, 144.5703).. controls (78.4887, 
  143.5087) and (76.7183, 142.1489) .. (74.9605, 142.7136).. controls (74.2555, 
  143.1976) and (75.1901, 144.5876) .. (74.854, 144.6804).. controls (73.655, 
  143.6815) and (71.8284, 142.187) .. (70.2254, 143.0789).. controls (69.7762, 
  143.4391) and (69.8796, 144.2728) .. (69.7985, 144.6085).. controls (67.8481, 
  142.2226) and (64.5338, 143.3099) .. (61.9096, 142.7084).. controls (59.0956, 
  142.2343) and (55.5941, 141.307) .. (54.4501, 138.3464).. controls (53.6708, 
  135.816) and (55.7116, 133.3489) .. (58.1336, 132.8165).. controls (63.1649, 
  131.5401) and (68.4426, 132.6624) .. (73.4386, 133.6981).. controls (74.5844, 
  133.2794) and (73.733, 132.3708) .. (73.1213, 131.8145).. controls (71.5053, 
  129.5004) and (69.4892, 127.062) .. (66.467, 126.7662).. controls (63.3178, 
  126.338) and (59.9426, 126.4424) .. (57.128, 128.0814).. controls (56.194, 
  128.6451) and (54.878, 128.3538) .. (54.8175, 127.1476).. controls (54.6366, 
  125.583) and (55.4802, 123.8734) .. (54.3282, 122.476).. controls (53.3458, 
  120.6661) and (50.9132, 119.8738) .. (50.2659, 117.9321).. controls (51.2617, 
  117.7055) and (51.471, 116.6063) .. (52.4244, 116.3493).. controls (53.6728, 
  115.2016) and (52.2049, 113.3905) .. (52.8533, 111.9836).. controls (53.1426, 
  110.8475) and (54.2326, 109.5224) .. (55.4351, 109.7678).. controls (55.6092, 
  110.4875) and (56.3493, 110.9081) .. (57.05, 110.9444).. controls (56.4998, 
  111.0526) and (57.1381, 112.5454) .. (56.9957, 111.4664).. controls (57.8282, 
  112.6533) and (59.3502, 111.5019) .. (60.2115, 110.7687).. controls (60.9082, 
  110.4522) and (60.5886, 109.2289) .. (61.2957, 109.1528).. controls (61.0626, 
  110.3963) and (62.2475, 111.4035) .. (63.3379, 111.4917).. controls (62.1356, 
  111.5906) and (61.4092, 112.5897) .. (60.6766, 113.4368).. controls (60.5415, 
  114.8095) and (62.7474, 115.3506) .. (63.572, 114.4119).. controls (63.936, 
  114.4482) and (63.0749, 115.4819) .. (63.4299, 115.9922).. controls (63.3771, 
  116.521) and (64.515, 117.3223) .. (63.4521, 117.0469).. controls (61.7584, 
  117.9471) and (60.0711, 116.4855) .. (58.3904, 116.8133).. controls (56.1821, 
  118.1205) and (57.1402, 121.899) .. (59.6818, 122.1412).. controls (63.2209, 
  123.0835) and (66.4598, 121.0744) .. (69.673, 119.9651).. controls (72.7467, 
  119.1764) and (75.9344, 119.7787) .. (78.7406, 121.1557).. controls (78.3943, 
  118.2379) and (76.1667, 115.7274) .. (73.866, 113.9995).. controls (72.5736, 
  113.1958) and (71.7235, 111.0781) .. (73.5693, 110.3233).. controls (75.5644, 
  109.7918) and (77.3194, 111.5018) .. (79.3525, 111.1837).. controls (82.1413, 
  111.2672) and (84.8166, 109.7401) .. (86.8037, 108.094).. controls (86.233, 
  108.9071) and (85.4912, 110.8639) .. (87.0962, 110.9672).. controls (89.0112, 
  110.5746) and (90.8488, 109.2039) .. (92.7656, 109.1166).. controls (92.6735, 
  109.5407) and (92.8618, 109.8417) .. (93.0721, 110.1238).. controls (93.3938, 
  109.3848) and (93.8769, 108.6085) .. (94.7918, 109.1409).. controls (94.2555, 
  109.5315) and (93.7994, 110.1252) .. (93.1914, 110.2876).. controls (93.4101, 
  110.5773) and (93.6028, 110.8662) .. (93.4782, 111.2695).. controls (93.803, 
  112.7636) and (95.7282, 112.6149) .. (96.5747, 111.6514).. controls (97.5507, 
  110.8822) and (98.6367, 110.0307) .. (99.9393, 110.2142).. controls (100.3551,
   110.0306) and (100.95, 108.89) .. (100.731, 109.9605).. controls (100.6796, 
  110.7502) and (101.383, 111.591) .. (101.8607, 111.8777).. controls (100.9771,
   112.0746) and (100.1072, 112.1979) .. (99.2014, 112.2896).. controls 
  (98.2213, 112.5143) and (97.5983, 113.9281) .. (98.8272, 114.2316).. controls 
  (99.9653, 114.3301) and (101.1527, 113.7859) .. (102.24, 114.3571).. controls 
  (102.8504, 114.6192) and (103.5841, 113.683) .. (103.982, 113.8828).. controls
   (103.1418, 114.6967) and (103.2233, 116.1855) .. (103.8672, 117.0138).. 
  controls (102.9249, 117.2899) and (102.5643, 116.0487) .. (101.5645, 
  116.2826).. controls (100.7377, 116.4463) and (99.9197, 116.1763) .. (99.1332,
   116.202).. controls (98.0748, 117.167) and (99.1189, 118.643) .. (99.8168, 
  119.4809).. controls (100.2591, 119.5526) and (101.0033, 120.1364) .. 
  (100.1078, 120.0581).. controls (99.2655, 120.1013) and (98.5251, 120.9857) ..
   (98.2371, 121.4621).. controls (97.9225, 120.3875) and (96.8611, 119.6423) ..
   (96.9297, 118.4065).. controls (96.5091, 115.7587) and (93.4082, 115.3061) ..
   (91.2055, 115.2863).. controls (88.7382, 115.3969) and (86.0014, 114.8613) ..
   (83.7274, 116.0175).. controls (82.8273, 116.8194) and (84.1513, 117.5252) ..
   (84.9356, 117.3843).. controls (87.7613, 117.7709) and (91.1637, 118.4268) ..
   (92.4866, 121.2978).. controls (93.1727, 122.1845) and (94.9672, 121.9021) ..
   (95.4213, 121.5412).. controls (94.8681, 122.4687) and (93.0719, 121.4148) ..
   (93.2074, 122.7763).. controls (93.1156, 123.1989) and (94.2399, 124.0499) ..
   (93.4084, 123.5313).. controls (92.8849, 122.6207) and (91.8947, 123.3719) ..
   (92.1072, 124.2465).. controls (91.9347, 125.2576) and (91.6839, 126.4592) ..
   (90.9848, 127.1858).. controls (92.3583, 126.7454) and (93.6831, 126.0767) ..
   (94.8911, 125.4495).. controls (94.3512, 126.5374) and (93.3874, 127.2981) ..
   (92.4199, 128.054).. controls (92.9123, 128.0608) and (93.4109, 128.0319) .. 
  (93.9164, 127.9584).. controls (99.5602, 127.4544) and (103.9965, 123.4617) ..
   (107.9399, 119.7666).. controls (111.6658, 116.3409) and (116.547, 113.2571) 
  .. (121.8723, 114.1597).. controls (122.6661, 114.2029) and (124.2944, 
  115.2985) .. (124.4014, 113.9122).. controls (124.4801, 112.5944) and 
  (123.4438, 111.3657) .. (123.697, 110.1171).. controls (125.1248, 109.2368) 
  and (126.8472, 110.6032) .. (128.3877, 110.5677).. controls (131.875, 
  111.2552) and (136.1916, 110.7093) .. (138.3411, 107.6433).. controls 
  (138.5454, 108.7467) and (137.0288, 110.1972) .. (138.5576, 110.8804).. 
  controls (140.7532, 111.5541) and (142.4624, 109.0491) .. (144.1242, 
  108.5916).. controls (144.1036, 108.8372) and (144.1693, 109.0384) .. 
  (144.2684, 109.221).. controls (144.3716, 108.6178) and (145.0682, 108.1167) 
  .. (145.7582, 108.4505).. controls (146.0145, 108.6323) and (145.1489, 
  109.1864) .. (145.0079, 109.621).. controls (144.9243, 109.745) and (144.8982,
   109.8989) .. (144.8089, 109.9988).. controls (144.9272, 110.2008) and 
  (144.9973, 110.4215) .. (144.9438, 110.6969).. controls (146.0104, 113.0192) 
  and (149.5682, 112.1734) .. (150.576, 110.206).. controls (151.2841, 109.6154)
   and (152.247, 109.9186) .. (152.7966, 109.1554).. controls (153.5852, 
  108.7649) and (152.6496, 110.025) .. (152.9526, 110.4447).. controls 
  (153.0398, 111.1543) and (153.5724, 111.7597) .. (154.2383, 111.9955).. 
  controls (153.3063, 112.078) and (151.9089, 111.5867) .. (151.4891, 
  112.8523).. controls (151.0388, 114.0908) and (152.2398, 115.5) .. (153.5102, 
  114.7742).. controls (154.6046, 114.1285) and (155.912, 114.853) .. (156.926, 
  114.1468).. controls (158.0864, 114.1143) and (156.5442, 114.7705) .. 
  (156.5338, 115.3524).. controls (156.1026, 115.9846) and (156.4436, 116.996) 
  .. (156.448, 117.4593).. controls (155.4225, 117.5576) and (155.2865, 
  115.9645) .. (154.1252, 116.1617).. controls (152.8925, 115.866) and 
  (151.1877, 117.2207) .. (152.5165, 118.3037).. controls (153.3885, 119.0954) 
  and (154.8909, 119.309) .. (155.2884, 120.4984).. controls (154.1767, 
  120.0827) and (152.7008, 120.4802) .. (152.0917, 121.4456).. controls 
  (151.2726, 120.8816) and (152.0774, 119.6422) .. (151.0277, 119.1651).. 
  controls (150.4341, 118.1785) and (149.8184, 116.9681) .. (148.4738, 
  116.9094).. controls (145.7958, 116.4907) and (142.9246, 116.5601) .. 
  (140.3513, 117.343).. controls (140.1357, 118.0767) and (140.0089, 119.0324) 
  .. (139.9059, 119.8638).. controls (140.1344, 120.4595) and (140.1819, 
  121.0642) .. (140.1813, 121.6725).. controls (140.3248, 121.6549) and 
  (140.4628, 121.7019) .. (140.5937, 121.8363).. controls (142.0694, 124.4165) 
  and (145.1408, 126.0775) .. (147.9524, 125.9161).. controls (147.0634, 
  126.6119) and (144.8434, 126.3467) .. (145.1443, 128.0157).. controls 
  (145.992, 130.0895) and (148.8561, 129.9843) .. (150.3538, 130.951).. controls
   (150.2404, 131.0565) and (150.1677, 131.1776) .. (150.1176, 131.306).. 
  controls (150.6851, 131.0396) and (151.4993, 131.5275) .. (151.6126, 
  132.3188).. controls (151.0809, 132.1925) and (150.4272, 132.3391) .. 
  (149.9084, 132.232).. controls (149.8322, 132.4407) and (149.6923, 132.6052) 
  .. (149.3756, 132.6702).. controls (147.8761, 134.2283) and (149.7133, 
  136.5644) .. (151.4307, 136.9738).. controls (152.4704, 137.1579) and 
  (153.688, 136.9459) .. (154.3024, 138.0399).. controls (154.645, 137.9897) and
   (155.9949, 138.3712) .. (154.9256, 138.5546).. controls (153.9551, 138.8993) 
  and (153.3776, 140.0421) .. (153.7438, 141.018).. controls (153.1484, 
  140.3012) and (152.5943, 138.8966) .. (151.3522, 139.4063).. controls 
  (150.1504, 139.8458) and (149.791, 141.5834) .. (151.0887, 142.1518).. 
  controls (152.2767, 142.6565) and (152.4919, 144.1531) .. (153.6895, 
  144.5496).. controls (154.3225, 145.4008) and (152.8848, 144.5617) .. (152.47,
   144.9263).. controls (151.7267, 145.0369) and (151.1046, 145.7251) .. 
  (150.7807, 146.2229).. controls (149.884, 145.5078) and (151.2817, 144.3503) 
  .. (150.3492, 143.5197).. controls (149.8553, 142.4364) and (147.8283, 
  141.7531) .. (147.6594, 143.4086).. controls (147.5177, 144.8065) and 
  (148.2796, 145.8637) .. (147.6248, 147.0456) -- (147.5933, 146.9495).. 
  controls (147.2155, 145.8361) and (145.9781, 145.0614) .. (144.9335, 
  145.0664).. controls (144.8789, 144.1196) and (146.2996, 144.0291) .. 
  (146.0776, 142.96).. controls (146.4397, 141.8562) and (147.0828, 140.6806) ..
   (146.4295, 139.5132).. controls (144.6689, 135.1643) and (140.7614, 132.1861)
   .. (136.673, 130.0885).. controls (136.3786, 130.2416) and (136.0505, 
  130.3383) .. (135.6648, 130.3381).. controls (134.3273, 130.632) and 
  (132.4269, 131.2937) .. (132.3818, 132.9141).. controls (133.1268, 133.7423) 
  and (131.5372, 133.135) .. (131.0449, 133.357).. controls (130.4413, 133.2999)
   and (129.8748, 133.4435) .. (129.4156, 133.7389).. controls (129.6045, 
  134.0939) and (129.6117, 134.5697) .. (129.2698, 134.96).. controls (129.1831,
   135.4652) and (129.1145, 135.9902) .. (129.0766, 136.5191).. controls 
  (129.3173, 136.3009) and (130.0256, 137.117) .. (130.4754, 137.1485).. 
  controls (132.4945, 138.1695) and (134.7065, 137.464) .. (136.6213, 
  136.5992).. controls (137.782, 136.0086) and (139.8223, 136.84) .. (139.8666, 
  137.6885).. controls (138.0331, 135.9818) and (135.7348, 137.7505) .. 
  (133.8091, 138.0942).. controls (133.4548, 138.1495) and (133.0953, 138.1763) 
  .. (132.7368, 138.1609).. controls (131.3801, 138.0496) and (130.1577, 
  137.347) .. (129.0735, 136.5677).. controls (128.9867, 137.8513) and 
  (129.1062, 139.1576) .. (129.7504, 140.2744).. controls (129.7103, 140.3142) 
  and (129.6701, 140.3489) .. (129.6305, 140.3798).. controls (130.282, 
  140.9602) and (131.1182, 141.3205) .. (132.0893, 141.3054).. controls 
  (133.8018, 141.3349) and (135.6476, 139.8613) .. (137.1836, 141.2976).. 
  controls (138.259, 141.9948) and (138.8046, 143.3418) .. (138.2553, 
  144.5434).. controls (138.3812, 145.1218) and (139.3223, 146.5098) .. 
  (138.0729, 146.3965).. controls (136.0038, 146.6116) and (134.1542, 147.7167) 
  .. (132.0485, 147.7639).. controls (131.0902, 148.2228) and (130.2986, 
  149.0651) .. (129.1163, 148.8703).. controls (128.0195, 148.7733) and 
  (126.9028, 148.6491) .. (125.8121, 148.4238).. controls (125.7137, 148.6796) 
  and (125.5003, 148.8699) .. (125.1011, 148.9266).. controls (124.8354, 
  149.0352) and (124.5526, 149.1303) .. (124.2619, 149.1219).. controls 
  (123.3958, 148.964) and (123.2604, 148.3479) .. (123.3761, 147.6678).. 
  controls (122.7668, 147.8908) and (122.13, 148.0618) .. (121.4796, 148.0393)..
   controls (120.3959, 147.9947) and (120.3345, 146.8715) .. (120.5179, 
  145.963).. controls (119.9385, 145.4939) and (119.5184, 144.8996) .. 
  (119.4746, 144.0261).. controls (119.359, 142.8643) and (118.0441, 142.2569) 
  .. (117.0757, 142.5285).. controls (117.8601, 142.1605) and (119.162, 
  142.0628) .. (118.5909, 140.862).. controls (118.434, 139.5451) and (117.1454,
   138.3482) .. (116.1115, 138.1216).. controls (117.0477, 137.5367) and 
  (118.6145, 139.0488) .. (118.9413, 137.6431).. controls (118.3179, 134.5172) 
  and (114.9059, 133.2424) .. (112.0445, 133.2609).. controls (105.1162, 
  133.1223) and (98.5631, 135.8744) .. (91.7119, 136.4602).. controls (84.4617, 
  137.6107) and (77.0514, 136.6534) .. (70.0104, 134.8076).. controls (66.1977, 
  134.1438) and (62.0696, 133.3367) .. (58.3134, 134.6582).. controls (56.3066, 
  135.3228) and (55.9044, 138.3674) .. (57.8509, 139.3628).. controls (61.7159, 
  141.368) and (66.3001, 140.7509) .. (70.4936, 140.6821).. controls (79.6918, 
  139.9844) and (88.7981, 138.3586) .. (98.0242, 138.3603).. controls (102.0683,
   138.3833) and (106.7157, 137.8106) .. (110.1129, 140.5194).. controls 
  (112.5459, 142.5132) and (112.048, 146.768) .. (109.2349, 148.1794).. controls
   (104.9734, 150.5181) and (99.7484, 150.3678) .. (95.168, 149.6609).. controls
   (91.4127, 148.5487) and (87.5961, 149.7041) .. (83.8592, 150.1498).. controls
   (83.6807, 150.1567) and (83.5022, 150.1503) .. (83.3238, 150.1467).. controls
   (82.8656, 150.1436) and (82.4112, 150.1153) .. (81.9606, 150.0661).. controls
   (81.475, 150.2191) and (80.9656, 150.3715) .. (80.6681, 150.5151).. controls 
  (80.4182, 150.5825) and (80.1963, 150.6317) .. (79.9638, 150.6516) -- 
  cycle(146.0962, 149.1902) -- (146.0161, 149.1287).. controls (145.2291, 
  148.4633) and (144.7165, 146.7176) .. (145.6089, 146.0467).. controls 
  (146.2946, 145.7916) and (147.1204, 146.6413) .. (147.1256, 147.1634).. 
  controls (146.394, 147.5459) and (145.9281, 148.3515) .. (146.0962, 149.1902) 
  -- cycle(80.2739, 148.3711).. controls (80.7937, 148.3838) and (81.3027, 
  148.327) .. (81.8045, 148.2238).. controls (81.0608, 148.2217) and (80.2971, 
  148.1727) .. (79.6139, 148.3148).. controls (79.8357, 148.3457) and (80.056, 
  148.3658) .. (80.2739, 148.3711) -- cycle(153.5231, 147.814).. controls 
  (152.5704, 147.7679) and (150.9031, 146.5731) .. (152.0607, 145.8255).. 
  controls (152.7172, 145.5509) and (153.7393, 145.3121) .. (153.5366, 
  146.3195).. controls (153.6512, 146.9365) and (154.2316, 147.3692) .. 
  (154.7432, 147.5903).. controls (154.423, 147.7517) and (154.0604, 147.823) ..
   (153.703, 147.8073).. controls (153.6468, 147.8149) and (153.5867, 147.8171) 
  .. (153.5231, 147.814) -- cycle(154.9954, 141.0728).. controls (153.7078, 
  141.1473) and (153.4919, 138.6546) .. (154.9659, 138.7923).. controls 
  (155.4787, 139.5669) and (156.3286, 140.2146) .. (157.3172, 139.9029).. 
  controls (156.8981, 140.6808) and (155.9878, 141.1029) .. (155.1235, 
  141.0573).. controls (155.0798, 141.0652) and (155.0369, 141.0704) .. 
  (154.9954, 141.0728) -- cycle(136.8172, 140.1034) -- (136.5629, 140.03).. 
  controls (136.7398, 139.4897) and (136.8112, 139.4662) .. (136.8172, 140.1034)
   -- cycle(136.6813, 134.4329).. controls (136.6559, 134.439) and (136.6164, 
  134.425) .. (136.5578, 134.3833).. controls (135.8411, 134.1082) and 
  (134.7949, 133.7268) .. (134.1057, 133.558).. controls (134.7849, 133.7015) 
  and (135.5329, 133.7917) .. (135.9666, 133.1121).. controls (136.5181, 
  133.7584) and (137.2684, 132.8905) .. (136.7474, 133.8908).. controls 
  (136.6484, 134.0074) and (136.7919, 134.4066) .. (136.6813, 134.4329) -- 
  cycle(134.1057, 133.558).. controls (133.9908, 133.5338) and (133.8729, 
  133.5108) .. (133.7641, 133.4867).. controls (133.8617, 133.5012) and 
  (133.9798, 133.5272) .. (134.1057, 133.558) -- cycle(155.7018, 123.2362).. 
  controls (155.5937, 123.2439) and (155.4852, 123.2427) .. (155.3773, 
  123.2335).. controls (154.4303, 123.262) and (152.4623, 121.9234) .. 
  (153.8192, 121.0977).. controls (154.5019, 120.7915) and (155.4405, 120.8451) 
  .. (155.1287, 121.7546).. controls (155.068, 122.3861) and (155.8792, 
  122.9425) .. (156.0227, 123.186).. controls (155.9174, 123.2119) and 
  (155.8099, 123.2284) .. (155.7018, 123.2361) -- cycle(100.7744, 122.4807).. 
  controls (100.2222, 122.4199) and (99.1737, 122.1114) .. (99.1171, 121.4828)..
   controls (99.2784, 120.8377) and (100.747, 120.2572) .. (100.4866, 
  121.2823).. controls (100.4356, 121.8426) and (101.3861, 122.27) .. (101.3625,
   122.43).. controls (101.1706, 122.4771) and (100.9715, 122.4936) .. 
  (100.7744, 122.4807) -- cycle(50.487, 117.2562).. controls (49.8984, 117.2794)
   and (50.045, 116.3775) .. (50.1382, 115.9369).. controls (50.2966, 116.1305) 
  and (50.5652, 116.6175) .. (50.8586, 116.8567).. controls (50.7773, 116.9842) 
  and (50.6959, 117.1116) .. (50.6147, 117.2391).. controls (50.5687, 117.2493) 
  and (50.5263, 117.2546) .. (50.487, 117.2562) -- cycle(157.821, 117.1838).. 
  controls (157.7502, 117.1865) and (157.6793, 117.1862) .. (157.6087, 
  117.1823).. controls (156.5664, 116.8971) and (157.3195, 115.474) .. 
  (157.7968, 115.1162).. controls (158.436, 115.7236) and (159.4813, 115.7469) 
  .. (160.1481, 115.3064).. controls (159.9505, 116.3824) and (158.8841, 
  117.1434) .. (157.821, 117.1838) -- cycle(64.6944, 116.7678).. controls 
  (63.9419, 116.3133) and (64.0736, 114.3832) .. (65.1874, 114.969).. controls 
  (65.7031, 115.1129) and (66.3082, 114.3482) .. (66.4576, 114.3923).. controls 
  (66.4389, 115.4605) and (65.7623, 116.5232) .. (64.6944, 116.7678) -- 
  cycle(103.9794, 116.6278).. controls (103.3496, 116.267) and (103.8689, 
  114.8121) .. (104.1773, 114.6517).. controls (104.7922, 115.3402) and (106.01,
   115.0238) .. (106.5715, 114.8341).. controls (106.1368, 115.8173) and 
  (105.014, 116.4923) .. (103.9794, 116.6278) -- cycle(154.3489, 111.5594).. 
  controls (153.4225, 111.5717) and (152.4099, 110.3627) .. (153.3526, 
  109.7316).. controls (154.2637, 109.7989) and (155.2319, 109.4313) .. 
  (155.5664, 108.4996).. controls (156.1039, 109.5786) and (155.5102, 111.2263) 
  .. (154.3489, 111.5594) -- cycle(102.2823, 111.084).. controls (101.6657, 
  110.7954) and (101.0743, 109.2609) .. (102.1531, 109.4262).. controls 
  (102.6528, 109.365) and (103.0412, 108.8916) .. (103.2461, 108.542).. controls
   (103.6508, 109.4811) and (103.0384, 110.5448) .. (102.2823, 111.084) -- 
  cycle(63.1674, 110.9424) -- (63.0961, 110.9367).. controls (62.4164, 110.8433)
   and (61.5141, 109.3154) .. (62.6067, 109.2804).. controls (63.1892, 109.0833)
   and (63.6793, 108.4698) .. (63.9519, 108.0759).. controls (64.2833, 108.9673)
   and (64.2413, 110.5475) .. (63.2496, 110.9413) -- cycle(57.0784, 110.3042).. 
  controls (56.4554, 110.314) and (55.8211, 109.5561) .. (56.6898, 109.3988).. 
  controls (57.268, 109.2461) and (57.4408, 107.6705) .. (57.6251, 108.7988).. 
  controls (57.6707, 109.2224) and (57.743, 110.3562) .. (57.0784, 110.3042) -- 
  cycle;



  \end{scope}
  \path[draw=black,fill=cd5a544,line width=2.6458mm] (1.5869, 240.4162).. 
  controls (1.441, 231.1735) and (1.345, 221.9304) .. (1.281, 212.6876) -- 
  (73.7205, 168.4314) -- (15.676, 131.9794) -- (73.7205, 95.5279) -- (18.0681, 
  60.3734).. controls (18.1852, 60.2109) and (18.2862, 60.0412) .. (18.405, 
  59.8794).. controls (36.1035, 35.7694) and (65.5886, 18.3724) .. (103.5239, 
  1.2773).. controls (141.4521, 18.3724) and (170.9405, 35.7729) .. (188.6391, 
  59.8794).. controls (188.7584, 60.0419) and (188.8599, 60.2122) .. (188.9776, 
  60.3755) -- (133.4156, 95.5279) -- (191.3738, 131.9794) -- (133.4156, 
  168.4314) -- (205.7626, 212.684).. controls (205.6986, 221.9281) and 
  (205.6026, 231.1721) .. (205.4567, 240.4162) -- cycle(129.1652, 232.7567).. 
  controls (129.2657, 232.7292) and (128.6725, 232.2985) .. (128.6696, 
  232.1908).. controls (128.6339, 232.0948) and (128.5958, 232.0065) .. 
  (128.5559, 231.9247).. controls (129.3269, 231.8523) and (130.0663, 231.5909) 
  .. (130.795, 231.286).. controls (132.2233, 231.5376) and (133.795, 231.5221) 
  .. (135.156, 232.1702).. controls (134.0018, 231.2655) and (132.5515, 
  231.0713) .. (131.117, 231.148).. controls (132.7438, 230.4441) and (134.3355,
   229.6367) .. (136.1358, 230.3827).. controls (134.8196, 229.3722) and 
  (132.9333, 229.3018) .. (131.3893, 229.7636).. controls (130.1079, 230.4584) 
  and (128.7614, 230.5249) .. (127.433, 230.2778).. controls (128.8761, 
  230.4173) and (130.2702, 230.3767) .. (131.3558, 229.2039).. controls 
  (129.7777, 228.7953) and (128.1711, 229.5433) .. (126.569, 229.4995).. 
  controls (126.1272, 229.5656) and (125.2959, 229.4418) .. (124.9846, 
  229.5202).. controls (124.4013, 229.2806) and (123.8318, 229.0161) .. 
  (123.286, 228.7538).. controls (120.751, 227.4633) and (117.7219, 227.4529) ..
   (115.1914, 228.7565).. controls (112.3249, 230.4047) and (109.0012, 231.0493)
   .. (105.7186, 230.6861).. controls (103.1146, 230.4175) and (99.5761, 
  229.8797) .. (98.6058, 227.0253).. controls (97.9228, 224.7865) and (100.3246,
   223.2897) .. (102.2562, 223.1263).. controls (103.8249, 222.586) and 
  (105.5192, 223.6955) .. (105.6591, 225.3402).. controls (106.7066, 224.7786) 
  and (107.4351, 222.9851) .. (108.9194, 222.8121).. controls (110.581, 
  222.5068) and (110.0172, 224.5998) .. (109.8615, 225.5112).. controls 
  (111.1906, 224.9797) and (112.3902, 223.1681) .. (114.103, 223.0483).. 
  controls (115.6573, 222.8725) and (115.3579, 224.7551) .. (115.0529, 
  225.6838).. controls (116.2079, 225.0195) and (117.6345, 223.3543) .. 
  (119.3405, 223.3036).. controls (120.7387, 223.3604) and (119.8688, 224.9553) 
  .. (120.0381, 225.7706).. controls (121.1389, 225.0248) and (122.392, 
  223.6657) .. (123.9634, 223.6509).. controls (125.3902, 223.7407) and 
  (124.4423, 225.3591) .. (124.5484, 226.1882).. controls (125.8647, 225.4994) 
  and (127.0856, 224.0587) .. (128.7559, 224.0674).. controls (130.1683, 
  224.1449) and (129.1652, 225.6543) .. (129.3616, 226.4693).. controls 
  (130.4713, 225.7051) and (131.9751, 224.4774) .. (133.5985, 224.4544).. 
  controls (134.9066, 224.29) and (134.6162, 225.8779) .. (134.3106, 226.6279)..
   controls (134.9822, 226.2886) and (136.3487, 224.9535) .. (137.5921, 
  224.7826).. controls (138.7576, 224.2626) and (139.8541, 225.3343) .. 
  (139.5656, 226.5199).. controls (140.1082, 226.0215) and (141.2192, 225.0983) 
  .. (142.2667, 225.0151).. controls (146.3078, 224.7669) and (150.9525, 
  224.8305) .. (154.1698, 221.9218).. controls (156.3947, 220.0078) and 
  (155.4919, 216.1427) .. (152.88, 215.0488).. controls (148.4034, 213.0633) and
   (143.3982, 214.1333) .. (138.7052, 214.8075).. controls (137.6858, 214.7087) 
  and (136.0857, 215.9644) .. (135.4862, 215.0131).. controls (135.0031, 
  213.9521) and (136.9419, 213.3319) .. (137.5135, 212.4939).. controls 
  (141.1972, 209.0393) and (146.3933, 207.7079) .. (151.339, 208.2203).. 
  controls (153.5696, 208.13) and (155.3133, 209.8442) .. (157.4751, 209.9354)..
   controls (159.1118, 209.1901) and (158.0846, 207.0207) .. (158.1996, 
  205.6886).. controls (158.5176, 202.7306) and (161.8731, 201.895) .. 
  (162.9957, 199.4239).. controls (162.7161, 199.1539) and (161.876, 198.9433) 
  .. (161.7275, 198.347).. controls (161.0446, 197.9321) and (159.9197, 
  197.3049) .. (160.2677, 196.2091).. controls (160.8886, 194.2245) and 
  (160.2237, 191.1256) .. (157.7484, 190.9624).. controls (157.5321, 191.6467) 
  and (156.7985, 192.2527) .. (155.9997, 192.2398).. controls (156.5713, 
  192.2654) and (155.9971, 193.7282) .. (156.1098, 192.924).. controls 
  (155.4945, 193.683) and (153.9068, 193.1381) .. (153.0939, 192.3091).. 
  controls (152.2058, 191.9366) and (152.4682, 190.8049) .. (151.7669, 
  190.363).. controls (151.8241, 191.5362) and (150.8682, 192.7559) .. 
  (149.5706, 192.7985).. controls (150.7403, 192.9134) and (151.5428, 193.7893) 
  .. (152.2056, 194.7069).. controls (152.6779, 195.9549) and (150.5125, 
  196.6851) .. (149.5882, 195.9786).. controls (148.8056, 195.3003) and 
  (149.5681, 196.4731) .. (149.5314, 196.9507).. controls (149.8445, 197.647) 
  and (148.2775, 198.7896) .. (149.4869, 198.5216).. controls (151.0539, 
  199.264) and (152.6169, 198.3621) .. (154.1698, 198.1682).. controls 
  (156.3923, 198.5839) and (156.47, 202.1942) .. (154.6525, 203.2314).. controls
   (151.4209, 204.7997) and (147.7349, 203.278) .. (144.5182, 202.4062).. 
  controls (141.0361, 201.1346) and (137.166, 200.6782) .. (133.5794, 
  201.6801).. controls (132.4742, 201.9069) and (131.4474, 202.294) .. (130.469,
   202.7793).. controls (130.5727, 201.9564) and (130.8207, 201.145) .. 
  (131.1356, 200.3882).. controls (132.1757, 197.5993) and (135.0314, 196.243) 
  .. (136.6887, 193.9436).. controls (137.5613, 192.3178) and (135.2726, 
  190.8826) .. (133.8398, 191.6414).. controls (130.488, 193.2428) and 
  (126.3177, 192.5709) .. (123.4859, 190.2121).. controls (123.1211, 190.0916) 
  and (122.162, 188.8454) .. (122.2369, 189.3873).. controls (122.7404, 
  190.1964) and (123.5588, 192.0212) .. (122.0468, 192.261).. controls 
  (120.0106, 191.9495) and (118.2319, 190.5405) .. (116.1319, 190.3345).. 
  controls (116.1801, 190.8883) and (115.9586, 191.473) .. (115.5086, 191.812)..
   controls (115.3531, 192.292) and (115.4938, 193.2333) .. (114.7366, 
  193.5225).. controls (113.009, 194.3303) and (111.7263, 192.1695) .. 
  (110.2806, 191.5841).. controls (109.5358, 191.5214) and (108.7938, 191.5185) 
  .. (108.3752, 190.7216).. controls (108.3789, 191.2451) and (108.2443, 
  192.5002) .. (107.489, 192.9514).. controls (106.5444, 193.3933) and 
  (107.9039, 193.1082) .. (108.2238, 193.4093).. controls (109.1686, 193.5066) 
  and (110.2921, 193.6235) .. (110.7363, 194.6599).. controls (110.3999, 
  196.3327) and (108.4355, 195.2475) .. (107.334, 195.4366).. controls 
  (106.4473, 196.0488) and (105.5238, 195.4999) .. (104.7129, 195.173).. 
  controls (105.1756, 195.8225) and (105.5808, 196.9801) .. (105.1263, 
  197.8664).. controls (104.7245, 198.359) and (104.7969, 198.5953) .. 
  (105.4509, 198.4188).. controls (106.3182, 197.3738) and (107.7756, 197.8307) 
  .. (108.9452, 197.6421).. controls (110.2905, 197.1272) and (110.3995, 
  199.3258) .. (109.5979, 199.9593).. controls (109.4586, 200.7703) and 
  (108.6888, 200.9604) .. (108.1918, 201.4005).. controls (108.666, 201.5971) 
  and (109.9199, 201.7773) .. (110.2971, 202.585).. controls (110.5361, 
  203.5122) and (110.7618, 202.5177) .. (110.8934, 202.1152).. controls 
  (112.3093, 200.8756) and (111.5453, 198.3638) .. (113.4984, 197.5253).. 
  controls (116.5522, 196.1785) and (120.0416, 196.6691) .. (123.2834, 
  196.7734).. controls (124.1459, 196.8226) and (126.1885, 197.1945) .. 
  (125.4724, 198.4477).. controls (122.6727, 199.224) and (119.169, 199.1366) ..
   (117.1582, 201.6186).. controls (116.7258, 203.2018) and (114.6859, 203.9492)
   .. (113.377, 202.9596).. controls (113.3669, 203.1137) and (114.1972, 
  203.6435) .. (114.6642, 203.5591).. controls (115.8865, 203.0982) and 
  (115.9105, 204.8211) .. (115.0808, 205.2473).. controls (115.373, 205.3963) 
  and (116.0679, 204.1445) .. (116.5851, 204.8184).. controls (116.9447, 
  206.1208) and (116.9986, 207.6168) .. (117.815, 208.7742).. controls 
  (116.4862, 208.342) and (115.2097, 207.746) .. (114.024, 207.0126).. controls 
  (113.9903, 207.7986) and (115.194, 208.5708) .. (115.8073, 209.1132).. 
  controls (115.9798, 209.2644) and (116.1521, 209.4158) .. (116.3246, 
  209.567).. controls (116.3664, 209.6165) and (116.4069, 209.6612) .. 
  (116.4481, 209.7085).. controls (111.4415, 209.7589) and (106.8621, 206.8486) 
  .. (103.0887, 203.7575).. controls (97.8555, 199.6619) and (92.046, 194.8935) 
  .. (84.9777, 195.4448).. controls (83.8994, 195.247) and (82.72, 196.4307) .. 
  (81.7981, 195.7962).. controls (81.3291, 194.8252) and (82.0009, 193.4787) .. 
  (82.3066, 192.4471).. controls (83.0614, 191.0854) and (81.0731, 190.8353) .. 
  (80.2493, 191.3138).. controls (75.5939, 192.4758) and (69.6947, 192.7943) .. 
  (66.1566, 188.9424).. controls (65.8077, 189.9935) and (67.3865, 191.2927) .. 
  (66.0781, 192.0874).. controls (63.8622, 192.967) and (62.109, 190.5809) .. 
  (60.2898, 189.7558).. controls (60.2335, 189.9049) and (60.185, 190.1301) .. 
  (60.112, 190.363).. controls (59.9504, 189.7328) and (59.0928, 189.3141) .. 
  (58.3457, 189.7315).. controls (58.8979, 190.0191) and (59.1397, 190.7534) .. 
  (59.5482, 191.2079).. controls (59.5395, 191.2128) and (59.5313, 191.2188) .. 
  (59.5223, 191.2234).. controls (59.6203, 192.9808) and (57.4345, 193.8663) .. 
  (55.9256, 193.2455).. controls (54.581, 193.0703) and (54.1021, 191.6509) .. 
  (53.024, 191.1262).. controls (52.1448, 191.2655) and (51.8178, 190.5405) .. 
  (51.082, 190.3092).. controls (51.1346, 190.5903) and (51.2184, 191.0271) .. 
  (51.2184, 191.4745).. controls (51.1868, 191.3318) and (51.1232, 191.1832) .. 
  (51.0107, 191.0296).. controls (50.0675, 191.0648) and (49.0019, 190.8138) .. 
  (48.6336, 189.8142).. controls (48.01, 190.494) and (48.542, 192.4035) .. 
  (49.6568, 192.86).. controls (50.2566, 192.8584) and (50.9487, 192.5155) .. 
  (51.1698, 192.0032).. controls (51.0997, 192.3351) and (50.9498, 192.6408) .. 
  (50.6603, 192.8548).. controls (50.1782, 193.191) and (49.6028, 193.4822) .. 
  (50.5492, 193.3958).. controls (51.8707, 192.7391) and (53.3644, 194.2019) .. 
  (52.6287, 195.5311).. controls (51.9351, 197.1363) and (50.1349, 195.6862) .. 
  (48.9343, 195.7952).. controls (48.0917, 196.3174) and (47.4786, 195.2934) .. 
  (46.6874, 195.6138).. controls (47.2306, 196.1202) and (48.1414, 197.3822) .. 
  (47.7359, 198.457).. controls (47.0919, 199.3629) and (48.7011, 198.947) .. 
  (48.708, 198.2839).. controls (49.5523, 197.2) and (52.1411, 197.4472) .. 
  (52.1703, 198.9759).. controls (51.6286, 200.6346) and (49.404, 200.4794) .. 
  (48.8434, 201.9767).. controls (49.7671, 201.7843) and (51.4841, 201.8183) .. 
  (52.0282, 202.9839).. controls (52.8383, 202.8204) and (52.4187, 201.8386) .. 
  (52.6452, 201.307).. controls (53.5394, 200.592) and (53.7129, 199.2966) .. 
  (54.7215, 198.7128).. controls (58.0128, 197.7316) and (61.5891, 198.0515) .. 
  (64.955, 198.4999).. controls (66.4606, 198.4128) and (65.8664, 200.0999) .. 
  (66.3167, 201.0357).. controls (66.0873, 201.6942) and (66.0406, 202.3568) .. 
  (66.0361, 203.0227).. controls (65.9948, 202.9715) and (65.9523, 202.9209) .. 
  (65.9121, 202.8692).. controls (64.3657, 205.6669) and (61.3421, 207.7706) .. 
  (58.0552, 207.4689).. controls (58.8626, 208.3211) and (60.945, 207.8641) .. 
  (60.9935, 209.4207).. controls (60.3617, 211.7186) and (57.3941, 211.6741) .. 
  (55.6145, 212.5787).. controls (55.7216, 212.6871) and (55.7982, 212.8381) .. 
  (55.8595, 213.004).. controls (55.2673, 212.7892) and (54.4081, 213.2345) .. 
  (54.3169, 214.0427).. controls (54.8267, 213.9615) and (55.5146, 214.0563) .. 
  (56.0889, 213.9657).. controls (56.1, 214.0196) and (56.1106, 214.0833) .. 
  (56.1225, 214.1284).. controls (57.9382, 214.8457) and (57.1442, 217.3858) .. 
  (55.7628, 218.1545).. controls (54.7975, 219.1125) and (53.3717, 218.652) .. 
  (52.2876, 219.1855).. controls (51.8712, 219.9631) and (51.0954, 219.7466) .. 
  (50.4805, 220.1942).. controls (50.7083, 220.321) and (51.0871, 220.4778) .. 
  (51.4272, 220.7032).. controls (51.2876, 220.6434) and (51.1217, 220.6101) .. 
  (50.9249, 220.6154).. controls (50.4076, 221.3695) and (49.5975, 222.0407) .. 
  (48.6082, 221.7523).. controls (48.582, 222.2612) and (49.7417, 222.9995) .. 
  (50.4841, 222.9579).. controls (51.699, 223.3228) and (52.3263, 221.7736) .. 
  (51.8075, 221.0138).. controls (51.9468, 221.1598) and (52.0577, 221.3269) .. 
  (52.1124, 221.5228).. controls (52.5274, 221.8604) and (51.8676, 223.2935) .. 
  (52.4251, 222.73).. controls (52.7134, 221.2295) and (54.8156, 220.5641) .. 
  (55.5432, 222.1321).. controls (56.4147, 223.6494) and (54.1755, 224.2142) .. 
  (53.5294, 225.227).. controls (53.4246, 226.1955) and (52.2856, 226.1062) .. 
  (52.0085, 226.8641).. controls (52.7231, 226.7763) and (54.2447, 226.8211) .. 
  (54.8296, 227.7798).. controls (54.9798, 228.5031) and (55.4651, 228.1069) .. 
  (55.5329, 227.5829).. controls (54.9599, 226.4822) and (55.4767, 225.1088) .. 
  (56.5736, 224.5283).. controls (57.6619, 223.681) and (58.6574, 225.0869) .. 
  (58.3224, 226.1411).. controls (58.2157, 227.2717) and (57.8206, 228.0375) .. 
  (58.3539, 229.034).. controls (58.708, 228.1151) and (59.8254, 226.9103) .. 
  (61.0075, 227.1018).. controls (61.5591, 226.6734) and (60.4937, 226.0307) .. 
  (60.1905, 225.7153).. controls (60.0904, 224.5621) and (59.2521, 223.5837) .. 
  (59.3466, 222.4055).. controls (60.6311, 217.4113) and (65.0991, 214.0913) .. 
  (69.4871, 211.7725).. controls (69.8403, 211.9919) and (70.2364, 212.1398) .. 
  (70.7113, 212.1596).. controls (72.0818, 212.429) and (74.4122, 213.3241) .. 
  (73.8697, 215.0788).. controls (73.601, 215.2743) and (75.1426, 215.0585) .. 
  (75.5952, 215.2028).. controls (76.1239, 215.157) and (76.5588, 215.2778) .. 
  (76.9135, 215.5072).. controls (76.8486, 215.5777) and (76.7876, 215.6582) .. 
  (76.7321, 215.7511).. controls (77.5587, 217.7095) and (77.6438, 220.1813) .. 
  (76.6443, 222.0964).. controls (76.6764, 222.128) and (76.7059, 222.1541) .. 
  (76.7362, 222.1812).. controls (75.6094, 223.1115) and (74.0156, 223.4826) .. 
  (72.5928, 222.8468).. controls (70.6777, 221.7073) and (67.6977, 223.3141) .. 
  (67.9388, 225.5923).. controls (68.4311, 226.5061) and (67.5608, 227.1602) .. 
  (67.667, 228.0883).. controls (69.6091, 228.199) and (71.3176, 229.2151) .. 
  (73.231, 229.4288).. controls (74.7142, 229.2097) and (75.5077, 230.8357) .. 
  (76.9621, 230.6752).. controls (78.1588, 230.6541) and (79.3696, 230.503) .. 
  (80.552, 230.2437).. controls (80.7928, 230.6579) and (81.4002, 230.8673) .. 
  (81.9514, 230.9388).. controls (82.0651, 230.949) and (82.1822, 230.9483) .. 
  (82.2925, 230.9155).. controls (83.028, 230.6186) and (83.1406, 230.0944) .. 
  (83.0227, 229.5017).. controls (83.5584, 229.6997) and (84.1181, 229.835) .. 
  (84.6882, 229.8639).. controls (85.9171, 230.0285) and (86.0849, 228.7093) .. 
  (85.8592, 227.8087).. controls (86.1765, 227.5461) and (86.4442, 227.2355) .. 
  (86.6276, 226.8445).. controls (86.8293, 225.6999) and (87.3379, 224.1589) .. 
  (88.7975, 224.3108).. controls (90.0269, 224.4197) and (88.1031, 223.9582) .. 
  (87.7883, 223.7976).. controls (87.4617, 222.1775) and (88.6412, 220.2645) .. 
  (90.3349, 219.9663).. controls (89.3407, 219.3008) and (88.0257, 220.8441) .. 
  (87.409, 219.5922).. controls (87.9073, 216.6665) and (90.9385, 215.2165) .. 
  (93.6567, 215.1408).. controls (100.9603, 214.6325) and (108.0011, 217.1902) 
  .. (115.1845, 217.9912).. controls (122.9484, 219.409) and (130.9682, 
  218.8139) .. (138.5382, 216.7582).. controls (142.6219, 215.9672) and 
  (147.0201, 215.0887) .. (151.0888, 216.3722).. controls (153.0702, 216.9305) 
  and (153.787, 219.8484) .. (151.9745, 221.0334).. controls (148.7049, 
  223.0555) and (144.6002, 222.6314) .. (140.9209, 222.6488).. controls 
  (131.2658, 222.221) and (121.7434, 220.2692) .. (112.086, 220.2144).. controls
   (107.4654, 220.2831) and (102.305, 219.3761) .. (98.3954, 222.3874).. 
  controls (95.9117, 224.4262) and (96.4278, 228.7648) .. (99.2976, 230.2055).. 
  controls (103.6794, 232.6004) and (109.0075, 232.4497) .. (113.8027, 
  231.686).. controls (117.5852, 230.5772) and (121.4261, 231.7765) .. 
  (125.1958, 232.2271).. controls (125.8307, 232.2395) and (126.4466, 232.1864) 
  .. (127.0624, 232.1123).. controls (127.7203, 232.3689) and (128.381, 
  232.6161) .. (129.0757, 232.7418).. controls (129.1221, 232.7565) and 
  (129.1508, 232.7607) .. (129.1651, 232.7567) -- cycle(59.9653, 231.2374).. 
  controls (60.6541, 230.8266) and (61.0002, 229.7879) .. (60.9358, 228.9528).. 
  controls (61.067, 227.4145) and (59.1182, 227.9762) .. (58.8682, 229.0567).. 
  controls (59.5454, 229.5129) and (60.1981, 230.3049) .. (59.9653, 231.2374) --
   cycle(52.2195, 229.8546).. controls (52.8934, 229.8236) and (53.5524, 
  229.5182) .. (54.0194, 229.0329).. controls (54.9968, 228.0635) and (53.3631, 
  227.3942) .. (52.5192, 227.5519).. controls (52.3506, 228.1145) and (52.3093, 
  229.0551) .. (51.5549, 229.3755).. controls (50.7951, 229.5984) and (51.4011, 
  229.8184) .. (51.9301, 229.8509).. controls (52.0265, 229.858) and (52.1232, 
  229.859) .. (52.2195, 229.8546) -- cycle(69.697, 221.901).. controls (69.9101,
   221.9035) and (69.4923, 220.7861) .. (69.5637, 221.822).. controls (69.6231, 
  221.8776) and (69.6666, 221.9008) .. (69.697, 221.901) -- cycle(73.2849, 
  219.9844).. controls (74.645, 219.9779) and (76.5586, 219.226) .. (77.296, 
  218.1912).. controls (76.5079, 218.7493) and (75.6111, 219.1625) .. (74.6631, 
  219.357).. controls (72.1738, 220.2087) and (70.0852, 217.6521) .. (67.6367, 
  218.4087).. controls (67.1504, 218.3411) and (66.0597, 219.8981) .. (66.6563, 
  219.3803).. controls (68.8048, 217.7167) and (71.054, 220.0203) .. (73.2849, 
  219.9844) -- cycle(69.773, 216.1593).. controls (70.3045, 216.1934) and 
  (71.3009, 215.53) .. (72.0023, 215.4534).. controls (73.2063, 215.1887) and 
  (72.1404, 215.3895) .. (71.5315, 215.5019).. controls (70.7586, 215.5592) and 
  (70.4808, 214.6459) .. (69.8231, 215.2281).. controls (68.9984, 214.8465) and 
  (69.8429, 215.6901) .. (69.6732, 216.1427).. controls (69.7043, 216.1516) and 
  (69.7375, 216.1569) .. (69.773, 216.1593) -- cycle(119.0624, 209.3798).. 
  controls (119.0651, 209.3667) and (119.0659, 209.3534) .. (119.0685, 209.34)..
   controls (119.09, 209.3475) and (119.1112, 209.3549) .. (119.1326, 
  209.3623).. controls (119.1093, 209.3684) and (119.0856, 209.3738) .. 
  (119.0623, 209.3798) -- cycle(48.5039, 204.7879).. controls (49.0522, 204.808)
   and (49.5939, 204.6217) .. (50.0491, 204.3233).. controls (51.2563, 203.9321)
   and (50.9912, 202.3177) .. (49.7173, 202.4443).. controls (48.5715, 202.0876)
   and (49.2227, 203.2076) .. (48.8864, 203.7626).. controls (48.8568, 204.2985)
   and (47.3791, 204.8043) .. (48.5039, 204.7879) -- cycle(107.8988, 204.0158)..
   controls (108.3554, 204.0006) and (108.8151, 203.8643) .. (109.168, 
  203.5673).. controls (110.0217, 203.3003) and (109.5618, 202.4407) .. 
  (108.8894, 202.2552).. controls (108.1529, 201.8927) and (108.2711, 202.4666) 
  .. (108.183, 202.9497).. controls (108.2273, 203.5762) and (106.531, 203.9699)
   .. (107.7034, 204.0148).. controls (107.7682, 204.0174) and (107.8335, 
  204.0179) .. (107.8988, 204.0158) -- cycle(162.7295, 198.673).. controls 
  (163.3389, 198.6871) and (163.1913, 197.7546) .. (163.0933, 197.2999).. 
  controls (162.9368, 197.3906) and (162.6604, 198.0347) .. (162.3358, 198.2503)
   -- (162.5978, 198.657).. controls (162.6454, 198.6668) and (162.6889, 
  198.6719) .. (162.7295, 198.673) -- cycle(46.2452, 198.6239).. controls 
  (47.6253, 198.6188) and (46.8633, 197.0899) .. (46.3563, 196.4855).. controls 
  (45.6747, 197.0318) and (44.7042, 197.2429) .. (43.9383, 196.7).. controls 
  (43.9249, 197.6072) and (45.182, 198.6377) .. (46.2452, 198.6239) -- 
  cycle(148.149, 198.1531).. controls (148.7758, 198.0123) and (148.9474, 
  196.6558) .. (148.4136, 196.2483).. controls (147.7501, 196.3631) and 
  (146.8582, 196.2882) .. (146.4318, 195.5698).. controls (146.3963, 196.6778) 
  and (147.0483, 197.8346) .. (148.149, 198.1531) -- cycle(104.639, 198.0115).. 
  controls (105.4275, 197.925) and (104.7925, 196.4008) .. (104.5848, 
  195.9424).. controls (103.9819, 196.5742) and (102.8112, 196.5762) .. 
  (102.0309, 196.1243).. controls (102.3651, 197.1077) and (103.4685, 197.7469) 
  .. (104.4618, 198.0089).. controls (104.5275, 198.0168) and (104.5865, 
  198.0173) .. (104.639, 198.0116) -- cycle(106.6761, 192.3158).. controls 
  (107.3025, 192.3618) and (107.5331, 191.3046) .. (107.5691, 190.782).. 
  controls (106.8747, 190.7683) and (106.1276, 190.4812) .. (105.8281, 
  189.8084).. controls (105.3274, 190.4137) and (105.9083, 191.7953) .. 
  (106.6761, 192.3158) -- cycle(149.7066, 192.2341).. controls (150.4484, 
  192.2592) and (150.8358, 191.423) .. (150.9354, 190.7954).. controls 
  (150.3143, 190.4873) and (149.3269, 190.0577) .. (149.0069, 189.2198).. 
  controls (148.6071, 190.1613) and (148.6705, 191.8003) .. (149.7066, 192.2341)
   -- cycle(115.3815, 191.6042).. controls (115.5875, 190.8656) and (115.2994, 
  189.9215) .. (114.27, 190.3655).. controls (114.4149, 190.5907) and (115.0145,
   191.2698) .. (115.3815, 191.6042) -- cycle(155.8757, 191.5861).. controls 
  (156.2983, 191.6146) and (156.7633, 191.3748) .. (156.9092, 190.9618).. 
  controls (156.5483, 190.6509) and (155.7645, 190.2736) .. (155.6726, 
  189.5872).. controls (155.365, 189.5548) and (155.1195, 191.3408) .. 
  (155.8757, 191.5861) -- cycle(205.8411, 191.1949).. controls (205.8411, 
  189.3533) and (205.835, 187.5117) .. (205.8329, 185.6702).. controls 
  (205.8356, 187.5117) and (205.8419, 189.3533) .. (205.8411, 191.1949) -- 
  cycle(205.7558, 166.4295).. controls (205.7461, 164.9903) and (205.7381, 
  163.5511) .. (205.7264, 162.1119).. controls (205.7383, 163.5511) and 
  (205.7457, 164.9903) .. (205.7558, 166.4295) -- cycle(75.2186, 155.6539).. 
  controls (75.4511, 155.634) and (75.673, 155.5848) .. (75.923, 155.5175).. 
  controls (76.2204, 155.3738) and (76.7298, 155.2214) .. (77.2154, 155.0684).. 
  controls (77.666, 155.1176) and (78.1204, 155.1459) .. (78.5786, 155.149).. 
  controls (78.757, 155.1526) and (78.9355, 155.1591) .. (79.114, 155.1521).. 
  controls (82.8509, 154.7064) and (86.6675, 153.5511) .. (90.4228, 154.6633).. 
  controls (95.0032, 155.3701) and (100.2282, 155.5204) .. (104.4897, 
  153.1817).. controls (107.3027, 151.7704) and (107.8007, 147.5155) .. 
  (105.3677, 145.5217).. controls (101.9705, 142.8129) and (97.3231, 143.3856) 
  .. (93.279, 143.3627).. controls (84.0529, 143.3608) and (74.9466, 144.9868) 
  .. (65.7484, 145.6845).. controls (61.5549, 145.7533) and (56.9707, 146.3703) 
  .. (53.1057, 144.3652).. controls (51.1592, 143.3697) and (51.5614, 140.3252) 
  .. (53.5683, 139.6606).. controls (57.3244, 138.3391) and (61.4525, 139.1462) 
  .. (65.2652, 139.8099).. controls (72.3062, 141.6557) and (79.7165, 142.613) 
  .. (86.9667, 141.4625).. controls (93.8179, 140.8767) and (100.371, 138.1246) 
  .. (107.2993, 138.2632).. controls (110.1607, 138.2448) and (113.5727, 
  139.5195) .. (114.1961, 142.6454).. controls (113.8693, 144.0512) and 
  (112.3025, 142.539) .. (111.3663, 143.1239).. controls (112.4002, 143.3506) 
  and (113.6888, 144.5475) .. (113.8457, 145.8643).. controls (114.4168, 
  147.0651) and (113.115, 147.1629) .. (112.3305, 147.5309).. controls 
  (113.2989, 147.2593) and (114.6138, 147.8666) .. (114.7294, 149.0285).. 
  controls (114.7732, 149.9019) and (115.1933, 150.4962) .. (115.7727, 
  150.9653).. controls (115.5893, 151.8738) and (115.6507, 152.997) .. 
  (116.7344, 153.0416).. controls (117.3848, 153.0641) and (118.0216, 152.8931) 
  .. (118.6309, 152.6701).. controls (118.5152, 153.3502) and (118.6505, 
  153.9664) .. (119.5167, 154.1243).. controls (119.8074, 154.1327) and 
  (120.0902, 154.0375) .. (120.3559, 153.9289).. controls (120.7552, 153.8723) 
  and (120.9685, 153.6819) .. (121.067, 153.4261).. controls (122.1576, 
  153.6515) and (123.2743, 153.7756) .. (124.3712, 153.8726).. controls 
  (125.5534, 154.0674) and (126.345, 153.2251) .. (127.3033, 152.7662).. 
  controls (129.409, 152.719) and (131.2586, 151.6139) .. (133.3277, 151.3989)..
   controls (134.5771, 151.5121) and (133.636, 150.1242) .. (133.5101, 
  149.5457).. controls (134.0594, 148.3442) and (133.5139, 146.9972) .. 
  (132.4384, 146.2999).. controls (130.9024, 144.8636) and (129.0566, 146.3373) 
  .. (127.3441, 146.3077).. controls (126.373, 146.3228) and (125.5368, 
  145.9626) .. (124.8853, 145.3822).. controls (124.9249, 145.3512) and 
  (124.9651, 145.3165) .. (125.0052, 145.2768).. controls (124.361, 144.16) and 
  (124.2415, 142.8536) .. (124.3283, 141.57).. controls (125.4125, 142.3494) and
   (126.635, 143.0519) .. (127.9916, 143.1632).. controls (128.3501, 143.1787) 
  and (128.7097, 143.1519) .. (129.0639, 143.0965).. controls (130.9896, 
  142.7528) and (133.288, 140.9842) .. (135.1214, 142.6909).. controls 
  (135.0771, 141.8424) and (133.0368, 141.0109) .. (131.8761, 141.6015).. 
  controls (129.9613, 142.4663) and (127.7493, 143.1718) .. (125.7303, 
  142.1508).. controls (125.2804, 142.1193) and (124.5721, 141.3032) .. 
  (124.3314, 141.5214).. controls (124.3693, 140.9925) and (124.4379, 140.4675) 
  .. (124.5246, 139.9623).. controls (124.8665, 139.572) and (124.8593, 
  139.0962) .. (124.6704, 138.7412).. controls (125.1296, 138.4459) and 
  (125.6961, 138.3022) .. (126.2997, 138.3593).. controls (126.792, 138.1373) 
  and (128.3816, 138.7446) .. (127.6366, 137.9165).. controls (127.6817, 
  136.296) and (129.5821, 135.6343) .. (130.9196, 135.3404).. controls 
  (131.3053, 135.3407) and (131.6334, 135.2439) .. (131.9278, 135.0908).. 
  controls (136.0162, 137.1885) and (139.9238, 140.1667) .. (141.6843, 
  144.5156).. controls (142.3376, 145.6829) and (141.6944, 146.8585) .. 
  (141.3324, 147.9624).. controls (141.5544, 149.0314) and (140.1338, 149.1219) 
  .. (140.1883, 150.0687).. controls (141.2329, 150.0637) and (142.4703, 
  150.8384) .. (142.8481, 151.9518) -- (142.8796, 152.0479).. controls 
  (143.5344, 150.866) and (142.7725, 149.8088) .. (142.9142, 148.4109).. 
  controls (143.0831, 146.7554) and (145.1101, 147.4387) .. (145.604, 148.522)..
   controls (146.5365, 149.3526) and (145.1389, 150.5101) .. (146.0355, 
  151.2252).. controls (146.3595, 150.7275) and (146.9816, 150.0393) .. 
  (147.7248, 149.9287).. controls (148.1396, 149.564) and (149.5773, 150.4031) 
  .. (148.9443, 149.5519).. controls (147.7467, 149.1555) and (147.5315, 
  147.6589) .. (146.3435, 147.1542).. controls (145.0458, 146.5858) and 
  (145.4053, 144.8481) .. (146.607, 144.4086).. controls (147.8491, 143.8989) 
  and (148.4032, 145.3036) .. (148.9986, 146.0204).. controls (148.6324, 
  145.0445) and (149.2099, 143.9016) .. (150.1804, 143.557).. controls 
  (151.2497, 143.3735) and (149.8998, 142.9921) .. (149.5572, 143.0423).. 
  controls (148.9428, 141.9482) and (147.7252, 142.1602) .. (146.6856, 
  141.9762).. controls (144.9681, 141.5667) and (143.131, 139.2307) .. 
  (144.6304, 137.6726).. controls (144.9471, 137.6075) and (145.087, 137.4431) 
  .. (145.1632, 137.2343).. controls (145.682, 137.3414) and (146.3357, 
  137.1948) .. (146.8675, 137.3212).. controls (146.7541, 136.5298) and 
  (145.9399, 136.0419) .. (145.3725, 136.3083).. controls (145.4225, 136.18) and
   (145.4952, 136.0588) .. (145.6086, 135.9533).. controls (144.1109, 134.9866) 
  and (141.2468, 135.0919) .. (140.3991, 133.0181).. controls (140.0982, 
  131.3491) and (142.3182, 131.6143) .. (143.2072, 130.9185).. controls 
  (140.3957, 131.0799) and (137.3242, 129.4189) .. (135.8485, 126.8386).. 
  controls (135.7176, 126.7043) and (135.5796, 126.6572) .. (135.4361, 
  126.6748).. controls (135.4361, 126.0665) and (135.3892, 125.4619) .. 
  (135.1607, 124.8661).. controls (135.2637, 124.0348) and (135.3905, 123.079) 
  .. (135.6061, 122.3453).. controls (138.1794, 121.5625) and (141.0506, 
  121.4931) .. (143.7286, 121.9118).. controls (145.0732, 121.9704) and 
  (145.6889, 123.1809) .. (146.2825, 124.1674).. controls (147.3322, 124.6446) 
  and (146.5274, 125.8839) .. (147.3465, 126.4479).. controls (147.9556, 
  125.4825) and (149.4315, 125.085) .. (150.5432, 125.5007).. controls 
  (150.1457, 124.3114) and (148.6433, 124.0977) .. (147.7713, 123.306).. 
  controls (146.4425, 122.2231) and (148.1473, 120.8683) .. (149.38, 121.164).. 
  controls (150.5414, 120.9668) and (150.6773, 122.5599) .. (151.7028, 
  122.4616).. controls (151.6983, 121.9984) and (151.3574, 120.9869) .. 
  (151.7886, 120.3548).. controls (151.7989, 119.7728) and (153.3412, 119.1167) 
  .. (152.1808, 119.1491).. controls (151.1668, 119.8554) and (149.8594, 
  119.1309) .. (148.765, 119.7765).. controls (147.4946, 120.5023) and 
  (146.2936, 119.0931) .. (146.7439, 117.8547).. controls (147.1637, 116.5891) 
  and (148.5611, 117.0804) .. (149.4932, 116.9979).. controls (148.8272, 
  116.762) and (148.2946, 116.1566) .. (148.2074, 115.447).. controls (147.9044,
   115.0273) and (148.84, 113.7672) .. (148.0514, 114.1577).. controls 
  (147.5018, 114.921) and (146.5389, 114.6177) .. (145.8308, 115.2083).. 
  controls (144.823, 117.1757) and (141.2652, 118.0215) .. (140.1986, 
  115.6992).. controls (140.2521, 115.4238) and (140.182, 115.2031) .. 
  (140.0637, 115.0011).. controls (140.153, 114.9013) and (140.1791, 114.7474) 
  .. (140.2627, 114.6233).. controls (140.4037, 114.1888) and (141.2693, 
  113.6346) .. (141.013, 113.4529).. controls (140.323, 113.119) and (139.6264, 
  113.6201) .. (139.5232, 114.2234).. controls (139.4241, 114.0407) and 
  (139.3584, 113.8395) .. (139.379, 113.5939).. controls (137.7172, 114.0514) 
  and (136.008, 116.5565) .. (133.8124, 115.8827).. controls (132.2836, 
  115.1995) and (133.8003, 113.749) .. (133.5959, 112.6457).. controls 
  (131.4464, 115.7116) and (127.1298, 116.2576) .. (123.6425, 115.57).. controls
   (122.102, 115.6055) and (120.3796, 114.2391) .. (118.9518, 115.1194).. 
  controls (118.6986, 116.3681) and (119.7349, 117.5967) .. (119.6562, 
  118.9145).. controls (119.5492, 120.3008) and (117.9209, 119.2052) .. 
  (117.1271, 119.1621).. controls (111.8018, 118.2594) and (106.9206, 121.3433) 
  .. (103.1947, 124.769).. controls (99.2513, 128.464) and (94.815, 132.4567) ..
   (89.1712, 132.9607).. controls (88.6657, 133.0342) and (88.1671, 133.0631) ..
   (87.6747, 133.0563).. controls (88.6422, 132.3004) and (89.606, 131.5397) .. 
  (90.1459, 130.4518).. controls (88.9379, 131.079) and (87.6131, 131.7477) .. 
  (86.2396, 132.1881).. controls (86.9387, 131.4615) and (87.1895, 130.2599) .. 
  (87.3621, 129.2488).. controls (87.1495, 128.3743) and (88.1397, 127.623) .. 
  (88.6633, 128.5336).. controls (89.4947, 129.0522) and (88.3704, 128.2012) .. 
  (88.4622, 127.7786).. controls (88.3267, 126.4171) and (90.1229, 127.471) .. 
  (90.6761, 126.5435).. controls (90.222, 126.9045) and (88.4275, 127.1869) .. 
  (87.7414, 126.3001).. controls (86.4185, 123.4292) and (83.0161, 122.7732) .. 
  (80.1904, 122.3867).. controls (79.4061, 122.5276) and (78.0821, 121.8217) .. 
  (78.9822, 121.0198).. controls (81.2562, 119.8637) and (83.993, 120.3992) .. 
  (86.4603, 120.2886).. controls (88.663, 120.3084) and (91.7639, 120.7611) .. 
  (92.1845, 123.4088).. controls (92.1159, 124.6447) and (93.1774, 125.3898) .. 
  (93.4919, 126.4645).. controls (93.78, 125.9881) and (94.5204, 125.1036) .. 
  (95.3626, 125.0604).. controls (96.2581, 125.1388) and (95.5139, 124.5549) .. 
  (95.0717, 124.4832).. controls (94.3737, 123.6453) and (93.3296, 122.1693) .. 
  (94.388, 121.2043).. controls (95.1745, 121.1786) and (95.9925, 121.4487) .. 
  (96.8193, 121.2849).. controls (97.8191, 121.051) and (98.1797, 122.2922) .. 
  (99.1221, 122.0162).. controls (98.4781, 121.1878) and (98.3966, 119.699) .. 
  (99.2368, 118.8851).. controls (98.8389, 118.6853) and (98.1052, 119.6215) .. 
  (97.4948, 119.3595).. controls (96.4075, 118.7882) and (95.2201, 119.3324) .. 
  (94.082, 119.2339).. controls (92.8531, 118.9304) and (93.4761, 117.5166) .. 
  (94.4562, 117.2919).. controls (95.362, 117.2002) and (96.2319, 117.0769) .. 
  (97.1155, 116.88).. controls (96.6378, 116.5933) and (95.9344, 115.7526) .. 
  (95.9858, 114.9628).. controls (96.2049, 113.8923) and (95.6099, 115.0329) .. 
  (95.1941, 115.2166).. controls (93.8915, 115.0331) and (92.8055, 115.8845) .. 
  (91.8295, 116.6537).. controls (90.9831, 117.6172) and (89.0578, 117.7659) .. 
  (88.733, 116.2718).. controls (88.8576, 115.8685) and (88.6649, 115.5796) .. 
  (88.4462, 115.29).. controls (89.0542, 115.1276) and (89.5103, 114.5339) .. 
  (90.0466, 114.1433).. controls (89.1317, 113.6108) and (88.6486, 114.3871) .. 
  (88.3269, 115.1261).. controls (88.1166, 114.844) and (87.9283, 114.5431) .. 
  (88.0204, 114.119).. controls (86.1036, 114.2063) and (84.266, 115.5769) .. 
  (82.351, 115.9695).. controls (80.746, 115.8663) and (81.4878, 113.9095) .. 
  (82.0585, 113.0963).. controls (80.0714, 114.7424) and (77.3961, 116.2695) .. 
  (74.6073, 116.186).. controls (72.5742, 116.5041) and (70.8192, 114.7941) .. 
  (68.8242, 115.3256).. controls (66.9783, 116.0805) and (67.8284, 118.1981) .. 
  (69.1208, 119.0019).. controls (71.4215, 120.7297) and (73.6491, 123.2403) .. 
  (73.9954, 126.158).. controls (71.1892, 124.781) and (68.0015, 124.1788) .. 
  (64.9278, 124.9674).. controls (61.7146, 126.0767) and (58.4756, 128.0858) .. 
  (54.9367, 127.1435).. controls (52.3951, 126.9013) and (51.4369, 123.1228) .. 
  (53.6452, 121.8157).. controls (55.3259, 121.4878) and (57.0132, 122.9495) .. 
  (58.707, 122.0492).. controls (59.7697, 122.3247) and (58.632, 121.5233) .. 
  (58.6847, 120.9945).. controls (58.3297, 120.4842) and (59.1908, 119.4505) .. 
  (58.8268, 119.4142).. controls (58.0022, 120.3529) and (55.7963, 119.8118) .. 
  (55.9314, 118.4391).. controls (56.664, 117.5921) and (57.3904, 116.593) .. 
  (58.5927, 116.494).. controls (57.5023, 116.4058) and (56.3174, 115.3986) .. 
  (56.5505, 114.1551).. controls (55.8434, 114.2312) and (56.163, 115.4545) .. 
  (55.4663, 115.7711).. controls (54.605, 116.5042) and (53.083, 117.6556) .. 
  (52.2505, 116.4687).. controls (52.3929, 117.5477) and (51.7546, 116.055) .. 
  (52.3048, 115.9468).. controls (51.6041, 115.9104) and (50.864, 115.4898) .. 
  (50.6899, 114.7701).. controls (49.4874, 114.5248) and (48.3974, 115.8498) .. 
  (48.1081, 116.986).. controls (47.4597, 118.3928) and (48.9277, 120.204) .. 
  (47.6792, 121.3516).. controls (46.7258, 121.6087) and (46.5165, 122.7078) .. 
  (45.5207, 122.9344).. controls (46.168, 124.8762) and (48.6006, 125.6685) .. 
  (49.583, 127.4784).. controls (50.735, 128.8758) and (49.8914, 130.5853) .. 
  (50.0723, 132.1499).. controls (50.1328, 133.3562) and (51.4488, 133.6474) .. 
  (52.3828, 133.0837).. controls (55.1974, 131.4448) and (58.5726, 131.3404) .. 
  (61.7218, 131.7685).. controls (64.744, 132.0643) and (66.7601, 134.5027) .. 
  (68.3761, 136.8168).. controls (68.9879, 137.3731) and (69.8392, 138.2817) .. 
  (68.6934, 138.7004).. controls (63.6974, 137.6647) and (58.4197, 136.5424) .. 
  (53.3884, 137.8188).. controls (50.9664, 138.3513) and (48.9256, 140.8184) .. 
  (49.7049, 143.3487).. controls (50.8489, 146.3093) and (54.3504, 147.2366) .. 
  (57.1644, 147.7107).. controls (59.7886, 148.3122) and (63.1029, 147.2249) .. 
  (65.0533, 149.6109).. controls (65.1344, 149.2751) and (65.0311, 148.4414) .. 
  (65.4802, 148.0812).. controls (67.0833, 147.1893) and (68.9098, 148.6838) .. 
  (70.1088, 149.6827).. controls (70.4449, 149.5899) and (69.5104, 148.1999) .. 
  (70.2153, 147.7159).. controls (71.9731, 147.1512) and (73.7436, 148.511) .. 
  (75.0512, 149.5726).. controls (75.3315, 148.8215) and (74.4927, 147.9717) .. 
  (75.108, 147.3335).. controls (76.8238, 146.6906) and (78.2922, 148.3856) .. 
  (79.6509, 149.1804).. controls (80.1574, 148.8585) and (79.1058, 147.5251) .. 
  (79.7708, 146.9433).. controls (81.4261, 146.2609) and (82.9643, 147.9234) .. 
  (84.1808, 148.8445).. controls (84.4635, 148.1407) and (83.6753, 147.1674) .. 
  (84.3431, 146.5573).. controls (86.1214, 145.9878) and (87.6421, 147.7767) .. 
  (88.962, 148.7076).. controls (89.2793, 148.421) and (88.3727, 146.9535) .. 
  (89.1464, 146.3692).. controls (91.0612, 145.5087) and (92.5609, 147.7262) .. 
  (94.0588, 148.5613).. controls (94.4188, 148.105) and (93.5293, 146.7723) .. 
  (94.1802, 146.1).. controls (95.7106, 145.2947) and (96.9321, 147.1209) .. 
  (97.8591, 148.0828).. controls (98.3062, 148.706) and (98.3573, 148.1504) .. 
  (98.4957, 147.6513).. controls (99.1334, 145.7186) and (101.5204, 145.9551) ..
   (103.0293, 146.5842).. controls (104.9906, 147.0316) and (106.0211, 149.4992)
   .. (104.6974, 151.094).. controls (102.449, 153.6659) and (98.6012, 153.702) 
  .. (95.4437, 153.6866).. controls (92.1959, 153.798) and (89.5652, 151.6842) 
  .. (86.5337, 150.9322).. controls (83.9415, 150.2354) and (81.6528, 151.545) 
  .. (79.3072, 152.5011).. controls (77.2288, 152.7093) and (75.1632, 151.8555) 
  .. (73.1211, 152.1823).. controls (73.6657, 152.5556) and (74.2104, 152.9288) 
  .. (74.7551, 153.3021).. controls (72.8059, 152.5404) and (70.4258, 151.9861) 
  .. (68.5482, 153.2323).. controls (68.8702, 153.205) and (70.398, 152.8631) ..
   (71.178, 153.2354).. controls (71.8583, 153.5553) and (72.5517, 153.845) .. 
  (73.2549, 154.102).. controls (71.9596, 154.0368) and (70.635, 154.1977) .. 
  (69.5471, 154.9506) -- (69.3916, 155.0658).. controls (70.709, 154.4621) and 
  (72.2069, 154.4283) .. (73.596, 154.2235).. controls (74.3489, 154.4842) and 
  (75.1135, 154.7034) .. (75.8914, 154.8607).. controls (75.7048, 155.1683) and 
  (75.5028, 155.4747) .. (75.2186, 155.6539) -- cycle(141.351, 154.1925).. 
  controls (141.1829, 153.3539) and (141.6488, 152.5482) .. (142.3804, 
  152.1657).. controls (142.3751, 151.6437) and (141.5494, 150.794) .. 
  (140.8637, 151.049).. controls (139.9713, 151.7199) and (140.4839, 153.4656) 
  .. (141.2709, 154.131) -- cycle(75.5287, 153.3734).. controls (75.3108, 
  153.3681) and (75.0905, 153.348) .. (74.8687, 153.3171).. controls (75.5519, 
  153.175) and (76.3156, 153.224) .. (77.0593, 153.2261).. controls (76.5575, 
  153.3294) and (76.0485, 153.3861) .. (75.5287, 153.3734) -- cycle(148.7779, 
  152.8163).. controls (148.8415, 152.8194) and (148.9017, 152.8171) .. 
  (148.9578, 152.8096).. controls (149.3152, 152.8254) and (149.6779, 152.754) 
  .. (149.998, 152.5926).. controls (149.4865, 152.3716) and (148.906, 151.9388)
   .. (148.7914, 151.3219).. controls (148.9941, 150.3144) and (147.972, 
  150.5532) .. (147.3155, 150.8278).. controls (146.1579, 151.5754) and 
  (147.8252, 152.7702) .. (148.7779, 152.8163) -- cycle(150.2502, 146.0752).. 
  controls (150.2917, 146.0728) and (150.3346, 146.0676) .. (150.3784, 
  146.0597).. controls (151.2426, 146.1052) and (152.1529, 145.6831) .. 
  (152.572, 144.9052).. controls (151.5834, 145.2169) and (150.7335, 144.5693) 
  .. (150.2208, 143.7947).. controls (148.7468, 143.6569) and (148.9626, 
  146.1496) .. (150.2502, 146.0752) -- cycle(132.072, 145.1057).. controls 
  (132.0659, 144.4686) and (131.9946, 144.492) .. (131.8177, 145.0323) -- 
  cycle(131.9361, 139.4352).. controls (132.0467, 139.409) and (131.9032, 
  139.0097) .. (132.0022, 138.8932).. controls (132.5232, 137.8928) and 
  (131.773, 138.7607) .. (131.2214, 138.1144).. controls (130.7877, 138.7941) 
  and (130.0397, 138.7038) .. (129.3605, 138.5604).. controls (130.0498, 
  138.7292) and (131.0959, 139.1105) .. (131.8126, 139.3856).. controls 
  (131.8712, 139.4273) and (131.9107, 139.4413) .. (131.9361, 139.4352) -- 
  cycle(150.9566, 128.2385).. controls (151.0647, 128.2308) and (151.1723, 
  128.2144) .. (151.2775, 128.1884).. controls (151.134, 127.9449) and 
  (150.3228, 127.3885) .. (150.3835, 126.757).. controls (150.6953, 125.8475) 
  and (149.7568, 125.7939) .. (149.0741, 126.1001).. controls (147.7171, 
  126.9258) and (149.6851, 128.2644) .. (150.6321, 128.2359).. controls (150.74,
   128.2451) and (150.8485, 128.2462) .. (150.9566, 128.2386) -- cycle(96.0292, 
  127.483).. controls (96.2263, 127.496) and (96.4254, 127.4795) .. (96.6173, 
  127.4324).. controls (96.6409, 127.2724) and (95.6904, 126.8449) .. (95.7414, 
  126.2846).. controls (96.0018, 125.2595) and (94.5332, 125.84) .. (94.372, 
  126.4851).. controls (94.4285, 127.1137) and (95.477, 127.4223) .. (96.0292, 
  127.483) -- cycle(45.7419, 122.2585).. controls (45.7811, 122.2569) and 
  (45.8235, 122.2517) .. (45.8695, 122.2415).. controls (45.9507, 122.114) and 
  (46.0322, 121.9866) .. (46.1134, 121.8591).. controls (45.82, 121.6198) and 
  (45.5514, 121.1329) .. (45.393, 120.9392).. controls (45.2998, 121.3799) and 
  (45.1532, 122.2818) .. (45.7419, 122.2585) -- cycle(153.0759, 122.1862).. 
  controls (154.1389, 122.1458) and (155.2053, 121.3847) .. (155.4029, 
  120.3088).. controls (154.7361, 120.7493) and (153.6908, 120.726) .. 
  (153.0516, 120.1186).. controls (152.5743, 120.4764) and (151.8212, 121.8995) 
  .. (152.8635, 122.1846).. controls (152.9341, 122.1886) and (153.005, 
  122.1889) .. (153.0759, 122.1862) -- cycle(59.9493, 121.7702).. controls 
  (61.0171, 121.5255) and (61.6937, 120.4628) .. (61.7124, 119.3946).. controls 
  (61.563, 119.3505) and (60.9579, 120.1152) .. (60.4422, 119.9713).. controls 
  (59.3284, 119.3856) and (59.1967, 121.3157) .. (59.9493, 121.7702) -- 
  cycle(99.2342, 121.6301).. controls (100.2688, 121.4946) and (101.3916, 
  120.8197) .. (101.8263, 119.8364).. controls (101.2649, 120.0261) and 
  (100.047, 120.3426) .. (99.4321, 119.654).. controls (99.1237, 119.8145) and 
  (98.6044, 121.2693) .. (99.2342, 121.6301) -- cycle(149.6037, 116.5617).. 
  controls (150.765, 116.2287) and (151.3587, 114.581) .. (150.8212, 113.502).. 
  controls (150.4867, 114.4336) and (149.5185, 114.8013) .. (148.6074, 
  114.7339).. controls (147.6647, 115.365) and (148.6773, 116.5741) .. 
  (149.6037, 116.5617) -- cycle(97.5371, 116.0863).. controls (98.2932, 
  115.5471) and (98.9056, 114.4834) .. (98.5009, 113.5443).. controls (98.296, 
  113.8939) and (97.9076, 114.3674) .. (97.4079, 114.4285).. controls (96.3291, 
  114.2632) and (96.9205, 115.7978) .. (97.5371, 116.0863) -- cycle(58.4222, 
  115.9447) -- (58.5044, 115.9436).. controls (59.4961, 115.5498) and (59.5381, 
  113.9696) .. (59.2067, 113.0782).. controls (58.9341, 113.4722) and (58.444, 
  114.0856) .. (57.8615, 114.2828).. controls (56.7689, 114.3178) and (57.6713, 
  115.8457) .. (58.3509, 115.939) -- cycle(52.3332, 115.3065).. controls 
  (52.9978, 115.3585) and (52.9255, 114.2248) .. (52.8799, 113.8012).. controls 
  (52.6956, 112.6728) and (52.5228, 114.2485) .. (51.9446, 114.4011).. controls 
  (51.0759, 114.5585) and (51.7102, 115.3164) .. (52.3332, 115.3065) -- 
  cycle(203.1679, 99.3214).. controls (203.0917, 98.7791) and (202.9977, 
  98.2539) .. (202.9162, 97.7169).. controls (202.9975, 98.254) and (203.0919, 
  98.7789) .. (203.1679, 99.3214) -- cycle(202.3995, 94.4556).. controls 
  (202.2342, 93.519) and (202.0421, 92.6058) .. (201.8579, 91.6857).. controls 
  (202.0417, 92.606) and (202.2346, 93.5188) .. (202.3995, 94.4556) -- 
  cycle(201.4708, 89.7308).. controls (201.2244, 88.6013) and (200.9466, 
  87.4964) .. (200.6678, 86.392).. controls (200.9461, 87.4966) and (201.2249, 
  88.6011) .. (201.4708, 89.7308) -- cycle(200.3593, 85.1285).. controls 
  (200.0273, 83.8817) and (199.6608, 82.6593) .. (199.2829, 81.4445).. controls 
  (199.6602, 82.6596) and (200.0278, 83.8814) .. (200.3593, 85.1285) -- 
  cycle(199.054, 80.6724).. controls (198.6429, 79.3926) and (198.1953, 78.1369)
   .. (197.7284, 76.8923).. controls (198.1947, 78.137) and (198.6436, 79.3925) 
  .. (199.054, 80.6724) -- cycle(69.2603, 79.9521).. controls (69.6961, 79.6722)
   and (70.0116, 79.1557) .. (70.1667, 78.6477).. controls (71.2008, 76.7593) 
  and (68.8293, 75.4576) .. (68.0754, 77.6044).. controls (68.9266, 78.0201) and
   (69.4876, 78.9784) .. (69.2603, 79.9521) -- cycle(133.6734, 79.9066).. 
  controls (133.8215, 79.8842) and (133.1897, 79.4523) .. (133.1696, 79.3128).. 
  controls (133.1532, 79.2692) and (133.1351, 79.2326) .. (133.1179, 79.1919).. 
  controls (133.9649, 79.0724) and (134.7757, 78.76) .. (135.5767, 78.4157).. 
  controls (136.9307, 78.6043) and (138.3825, 78.6316) .. (139.6726, 79.1893).. 
  controls (138.547, 78.3562) and (137.2359, 78.2163) .. (135.9276, 78.2643).. 
  controls (137.4854, 77.5849) and (139.0228, 76.9006) .. (140.7402, 77.4897).. 
  controls (139.6942, 76.5921) and (137.0765, 76.3628) .. (135.4966, 77.1393).. 
  controls (134.1309, 77.7841) and (132.7918, 77.7014) .. (131.4855, 77.3067).. 
  controls (133.1077, 77.4788) and (134.7143, 77.5927) .. (135.9255, 76.2593).. 
  controls (134.3067, 75.8487) and (132.6658, 76.6245) .. (131.0235, 76.5719).. 
  controls (130.6639, 76.6306) and (130.0171, 76.5506) .. (129.6225, 76.5647).. 
  controls (128.5347, 76.0498) and (127.4719, 75.4502) .. (126.4392, 75.019).. 
  controls (123.7692, 73.7055) and (121.0337, 75.3838) .. (118.8304, 76.791).. 
  controls (115.4918, 78.4144) and (110.5986, 78.6664) .. (108.1391, 75.4159).. 
  controls (106.262, 72.9016) and (108.4953, 69.4644) .. (111.2887, 68.9956).. 
  controls (112.9966, 68.3695) and (114.9782, 69.4475) .. (115.1836, 71.2937).. 
  controls (116.3046, 70.8026) and (117.1692, 68.6228) .. (118.8511, 68.5993).. 
  controls (120.4937, 68.5332) and (119.7082, 70.5602) .. (119.651, 71.4994).. 
  controls (121.0799, 70.9183) and (122.3458, 69.0001) .. (124.1753, 68.8964).. 
  controls (125.7995, 68.7646) and (125.4162, 70.7475) .. (125.1189, 71.7097).. 
  controls (126.46, 71.0458) and (128.1681, 69.6096) .. (129.9791, 69.5429).. 
  controls (131.4779, 69.6481) and (130.4646, 71.3461) .. (130.5863, 72.2213).. 
  controls (131.9831, 71.4778) and (133.2765, 69.9595) .. (135.048, 69.9677).. 
  controls (136.5326, 70.0489) and (135.4747, 71.6492) .. (135.6816, 72.5008).. 
  controls (136.8537, 71.7292) and (138.4525, 70.3955) .. (140.1702, 70.379).. 
  controls (141.6636, 70.2361) and (141.1363, 72.0138) .. (140.9004, 72.8528).. 
  controls (142.1273, 71.9282) and (143.5279, 71.1715) .. (145.0237, 70.7738).. 
  controls (147.6078, 70.5929) and (150.6299, 70.2816) .. (152.156, 67.8479).. 
  controls (153.8097, 65.4166) and (152.6775, 61.9091) .. (150.0745, 60.6592).. 
  controls (145.2537, 58.1193) and (139.6881, 59.0967) .. (134.5623, 59.9244).. 
  controls (133.4951, 59.8112) and (131.7071, 61.3037) .. (131.2834, 59.8556).. 
  controls (131.4492, 58.9334) and (132.848, 58.1979) .. (133.2833, 57.1514).. 
  controls (134.0198, 54.7498) and (137.1126, 54.8033) .. (138.9527, 55.8021).. 
  controls (140.9531, 56.6887) and (141.966, 54.0531) .. (140.7784, 52.6933).. 
  controls (139.5405, 50.6457) and (139.7981, 48.1431) .. (140.8224, 46.0735).. 
  controls (140.3603, 45.9012) and (139.5038, 45.52) .. (139.4168, 44.7806).. 
  controls (138.6143, 44.3936) and (137.6526, 43.6019) .. (137.7827, 42.513).. 
  controls (138.3619, 40.4955) and (137.6284, 37.408) .. (135.1488, 37.257).. 
  controls (134.93, 37.9891) and (134.1446, 38.5739) .. (133.3262, 38.5629).. 
  controls (133.9571, 38.5699) and (133.3214, 40.1772) .. (133.4393, 39.1437).. 
  controls (132.9159, 40.1968) and (131.9627, 39.5114) .. (131.1816, 39.0548).. 
  controls (130.199, 38.5266) and (130.6184, 37.1716) .. (129.8246, 36.6167).. 
  controls (129.9296, 37.8398) and (128.9122, 39.0562) .. (127.6087, 39.1184).. 
  controls (128.4821, 39.3506) and (129.3881, 39.6425) .. (129.8483, 40.5431).. 
  controls (130.7313, 41.1727) and (130.4548, 42.3089) .. (129.4985, 42.589).. 
  controls (128.7269, 43.0133) and (128.2086, 42.3254) .. (127.5772, 42.1632).. 
  controls (128.0436, 42.8171) and (128.1312, 43.8693) .. (127.6304, 44.593).. 
  controls (127.1038, 45.1231) and (127.1871, 45.3575) .. (127.9585, 45.0824).. 
  controls (129.3803, 45.8629) and (130.6108, 44.5577) .. (132.0126, 44.5858).. 
  controls (134.4832, 44.5628) and (135.3388, 47.5139) .. (134.8217, 49.5426).. 
  controls (134.7151, 49.5164) and (134.5305, 49.6046) .. (134.2455, 49.9177).. 
  controls (134.4046, 50.1784) and (134.5513, 50.2421) .. (134.6692, 50.2097).. 
  controls (134.1646, 51.7302) and (132.398, 52.5094) .. (130.8431, 52.083).. 
  controls (128.3574, 51.27) and (125.7035, 51.0464) .. (123.1444, 51.4592).. 
  controls (122.9311, 51.0804) and (122.7312, 50.694) .. (122.5418, 50.3069).. 
  controls (121.258, 47.5514) and (119.7994, 44.8036) .. (117.5943, 42.6541).. 
  controls (116.9635, 41.3503) and (118.8298, 41.2647) .. (119.5611, 40.8092).. 
  controls (121.1021, 39.9866) and (120.8857, 37.7728) .. (119.3208, 37.1201).. 
  controls (116.8713, 36.263) and (114.7861, 34.2203) .. (114.4787, 31.5545).. 
  controls (114.1155, 32.0562) and (113.8095, 33.8258) .. (112.5765, 33.3389).. 
  controls (111.1706, 32.3491) and (111.32, 30.2371) .. (110.4294, 28.8741).. 
  controls (110.2247, 27.6787) and (109.3925, 29.0941) .. (108.5912, 28.75).. 
  controls (108.1056, 29.0507) and (107.3761, 29.7757) .. (106.6022, 29.319).. 
  controls (104.9069, 28.3792) and (105.6287, 25.9737) .. (105.2695, 24.3927).. 
  controls (104.8655, 23.7465) and (104.3761, 23.1362) .. (104.7899, 22.2977).. 
  controls (104.2366, 22.8306) and (102.9979, 23.3413) .. (102.1059, 22.8543).. 
  controls (102.4359, 23.9603) and (103.6143, 25.7013) .. (103.1068, 27.1879).. 
  controls (101.6409, 28.1214) and (101.0156, 25.9634) .. (100.1701, 25.2169).. 
  controls (99.0864, 24.9474) and (98.7986, 23.8983) .. (98.473, 23.0212).. 
  controls (98.344, 23.8631) and (97.7125, 24.9306) .. (96.7444, 25.2278).. 
  controls (96.1178, 25.2687) and (95.9588, 25.4842) .. (96.5357, 25.8396).. 
  controls (97.8697, 25.7675) and (98.6075, 27.0662) .. (99.4822, 27.8576).. 
  controls (100.987, 28.4349) and (99.4366, 30.2394) .. (98.3071, 30.0611).. 
  controls (97.6062, 30.512) and (96.9275, 30.0981) .. (96.2597, 30.0032).. 
  controls (96.6458, 30.6294) and (97.3262, 31.9743) .. (96.668, 32.8903).. 
  controls (97.6746, 32.4481) and (98.933, 32.6607) .. (99.8155, 31.8537).. 
  controls (102.073, 30.1336) and (104.8203, 31.9599) .. (106.5619, 33.5482).. 
  controls (107.6529, 34.9037) and (109.7806, 35.7956) .. (109.9545, 37.6926).. 
  controls (109.4631, 38.9941) and (107.4881, 38.9283) .. (106.7769, 40.1927).. 
  controls (106.6307, 41.5635) and (104.9583, 41.4251) .. (104.1321, 40.6883).. 
  controls (103.4718, 39.8754) and (104.1461, 41.2866) .. (104.6256, 41.3121).. 
  controls (105.75, 41.2711) and (106.2528, 42.6059) .. (105.1274, 43.1264).. 
  controls (105.0932, 43.4148) and (106.4355, 42.0528) .. (106.7232, 43.0453).. 
  controls (106.6874, 45.7594) and (108.4103, 48.1146) .. (110.295, 49.9105).. 
  controls (112.4161, 51.6076) and (114.8805, 52.6911) .. (117.4424, 53.5418).. 
  controls (117.2296, 53.6379) and (117.0154, 53.7301) .. (116.799, 53.8172).. 
  controls (116.8138, 53.7656) and (116.8252, 53.7124) .. (116.8316, 53.656).. 
  controls (115.1848, 53.2369) and (113.1535, 52.8235) .. (111.5456, 51.9016).. 
  controls (111.2003, 51.6552) and (110.2988, 51.1642) .. (110.9446, 51.9776).. 
  controls (111.9127, 52.9029) and (112.8654, 54.0662) .. (114.1299, 54.5516).. 
  controls (113.2153, 54.6712) and (112.2847, 54.6431) .. (111.3487, 54.3898).. 
  controls (106.2441, 52.7232) and (103.8442, 47.3384) .. (99.6016, 44.4044).. 
  controls (97.5711, 42.7821) and (94.7471, 42.4719) .. (92.3845, 43.39).. 
  controls (91.1485, 42.5707) and (91.9623, 40.6801) .. (92.2884, 39.4765).. 
  controls (92.8902, 38.7235) and (92.5899, 37.5107) .. (91.4621, 37.795).. 
  controls (88.1162, 38.1812) and (83.8915, 39.1027) .. (81.3066, 36.2757).. 
  controls (81.1341, 36.5992) and (82.0227, 37.5623) .. (81.5056, 38.134).. 
  controls (79.9662, 39.6188) and (78.009, 37.5431) .. (76.6779, 36.7645).. 
  controls (76.6753, 36.9819) and (76.5782, 37.1863) .. (76.4635, 37.3903).. 
  controls (76.4975, 36.7669) and (75.7227, 36.2859) .. (75.045, 36.6198).. 
  controls (75.1851, 36.8675) and (75.8534, 37.4946) .. (76.1839, 37.9045).. 
  controls (76.072, 38.1518) and (76.016, 38.4118) .. (76.1369, 38.7055).. 
  controls (76.0245, 40.4477) and (73.8835, 40.238) .. (73.0048, 39.2119).. 
  controls (72.1898, 39.0686) and (72.0101, 38.4508) .. (71.5547, 37.8968).. 
  controls (71.6075, 38.9837) and (70.7075, 40.0453) .. (69.5828, 39.9819).. 
  controls (69.8432, 40.4267) and (71.937, 40.3984) .. (71.6302, 41.6547).. 
  controls (71.3995, 43.2941) and (69.6368, 42.4465) .. (68.5844, 42.6143).. 
  controls (67.8849, 43.1927) and (67.3093, 42.4827) .. (66.6393, 42.528).. 
  controls (67.2904, 43.0882) and (67.9317, 44.4301) .. (67.3777, 45.4141).. 
  controls (68.2017, 45.5375) and (68.5436, 44.0724) .. (69.5719, 44.3878).. 
  controls (71.0432, 44.5386) and (70.5576, 46.4426) .. (69.4665, 46.8104).. 
  controls (68.6726, 47.3743) and (68.2073, 47.5088) .. (67.7777, 48.2796).. 
  controls (68.3587, 48.1928) and (69.5495, 48.0407) .. (70.2385, 48.5767).. 
  controls (70.7986, 49.5629) and (71.1668, 48.3293) .. (70.9197, 47.753).. 
  controls (71.4491, 46.8374) and (72.017, 45.9249) .. (72.5123, 44.9645).. 
  controls (72.9693, 43.9059) and (74.4047, 44.3771) .. (74.0657, 45.5108).. 
  controls (74.5487, 46.6177) and (74.2068, 47.6967) .. (74.013, 48.8212).. 
  controls (74.2925, 49.7678) and (73.8844, 50.8053) .. (74.0466, 51.7321).. 
  controls (72.2863, 53.5945) and (69.8272, 54.7817) .. (67.2098, 54.5505).. 
  controls (68.029, 55.4755) and (70.4343, 54.9819) .. (70.3378, 56.7432).. 
  controls (69.5137, 59.1028) and (66.5025, 59.032) .. (64.6161, 59.9988).. 
  controls (64.7231, 60.1179) and (64.8034, 60.2941) .. (64.8688, 60.4892).. 
  controls (64.2395, 60.2476) and (63.3401, 60.6718) .. (63.2297, 61.5682) -- 
  (63.208, 61.6741).. controls (63.8184, 61.4895) and (64.5025, 61.6013) .. 
  (65.1117, 61.5062).. controls (65.1236, 61.5622) and (65.1361, 61.6301) .. 
  (65.1484, 61.6762).. controls (67.0681, 62.4441) and (66.258, 65.1543) .. 
  (64.7769, 65.9731).. controls (63.7527, 66.992) and (62.2394, 66.5087) .. 
  (61.0887, 67.0712).. controls (60.645, 67.9055) and (59.8197, 67.6901) .. 
  (59.1565, 68.1631).. controls (59.4353, 68.2682) and (59.7965, 68.4434) .. 
  (60.1291, 68.6814).. controls (59.9834, 68.6262) and (59.8124, 68.5974) .. 
  (59.6118, 68.6065).. controls (59.0582, 69.4069) and (58.1999, 70.1292) .. 
  (57.1448, 69.823).. controls (57.1493, 70.3769) and (58.363, 71.1324) .. 
  (59.1612, 71.0973).. controls (60.4042, 71.4741) and (61.0632, 69.9812) .. 
  (60.6402, 69.1414).. controls (60.8763, 69.4224) and (61.0355, 69.7564) .. 
  (61.0355, 70.1449).. controls (61.0042, 70.7679) and (60.8475, 71.4693) .. 
  (61.3998, 70.6002).. controls (61.6883, 69.0462) and (63.9054, 68.7437) .. 
  (64.5464, 70.2307).. controls (65.4587, 71.8534) and (63.0834, 72.4317) .. 
  (62.3951, 73.5111).. controls (62.2961, 74.5733) and (60.972, 74.4308) .. 
  (60.7621, 75.3224).. controls (61.6918, 75.0541) and (63.1362, 75.2835) .. 
  (63.8023, 76.2345).. controls (63.955, 77.0414) and (64.4882, 76.6157) .. 
  (64.564, 76.036).. controls (63.9521, 74.8644) and (64.5028, 73.3975) .. 
  (65.6734, 72.7825).. controls (67.1934, 71.7264) and (67.9937, 74.0186) .. 
  (67.3927, 75.1684).. controls (67.3763, 76.15) and (67.0221, 76.7024) .. 
  (67.5865, 77.6308).. controls (67.9787, 76.5276) and (69.1817, 75.3337) .. 
  (70.4618, 75.5751).. controls (70.8802, 74.8873) and (69.868, 74.4864) .. 
  (69.5342, 74.0398).. controls (69.4439, 72.7737) and (68.4447, 71.7065) .. 
  (68.6671, 70.4028).. controls (70.0722, 65.3494) and (74.4649, 61.9319) .. 
  (78.9331, 59.4954).. controls (79.0555, 59.5266) and (79.1791, 59.5542) .. 
  (79.3052, 59.5755).. controls (80.6642, 59.9592) and (83.0477, 60.8201) .. 
  (82.4657, 62.6198).. controls (82.5547, 62.7178) and (83.7667, 62.6313) .. 
  (84.2444, 62.7278).. controls (84.7534, 62.6766) and (85.1799, 62.7796) .. 
  (85.5348, 62.9857).. controls (85.4563, 63.0633) and (85.3828, 63.1544) .. 
  (85.3167, 63.2621).. controls (85.6316, 64.0496) and (85.8378, 64.9144) .. 
  (85.9146, 65.7932).. controls (85.1215, 66.3384) and (84.2271, 66.7415) .. 
  (83.2832, 66.9368).. controls (80.6336, 67.8604) and (78.3477, 64.9736) .. 
  (75.7607, 66.0547).. controls (75.3786, 66.069) and (74.4765, 67.3461) .. 
  (74.9142, 66.9875).. controls (77.1307, 65.2027) and (79.4813, 67.5956) .. 
  (81.8337, 67.5745).. controls (83.2456, 67.5627) and (85.0523, 66.8617) .. 
  (85.9172, 65.833).. controls (86.0291, 67.1845) and (85.8322, 68.5666) .. 
  (85.2299, 69.7496).. controls (85.3436, 69.827) and (85.4142, 69.8718) .. 
  (85.4986, 69.9279).. controls (84.3676, 70.915) and (82.7631, 71.3485) .. 
  (81.2673, 70.7294).. controls (79.2736, 69.4569) and (76.0685, 71.0941) .. 
  (76.2981, 73.5101).. controls (76.7876, 74.437) and (75.9851, 75.0956) .. 
  (76.0222, 76.0288).. controls (77.9459, 76.2457) and (79.8358, 77.2562) .. 
  (81.8228, 77.4907).. controls (83.3654, 77.2426) and (84.1793, 78.9487) .. 
  (85.6888, 78.7738).. controls (86.9432, 78.7376) and (88.2157, 78.5844) .. 
  (89.4581, 78.317).. controls (89.7348, 78.697) and (90.3363, 78.8859) .. 
  (90.858, 78.9661).. controls (91.9776, 78.8308) and (92.1174, 78.2118) .. 
  (91.9519, 77.4814).. controls (92.5139, 77.6899) and (93.0991, 77.8384) .. 
  (93.6976, 77.87).. controls (94.918, 78.0483) and (95.144, 76.7181) .. 
  (94.9207, 75.7999).. controls (95.2532, 75.5274) and (95.5362, 75.2068) .. 
  (95.7336, 74.8046).. controls (95.9238, 73.5351) and (96.6078, 71.9094) .. 
  (98.1883, 72.1825).. controls (99.15, 72.2453) and (97.1748, 71.7782) .. 
  (96.9511, 71.6337).. controls (96.6359, 70.0047) and (97.7684, 68.0718) .. 
  (99.4533, 67.6913).. controls (98.984, 67.0149) and (97.1082, 68.5094) .. 
  (96.5548, 67.3554).. controls (96.6436, 66.0356) and (97.1949, 64.7342) .. 
  (98.1371, 63.7975).. controls (100.9166, 61.229) and (105.1293, 62.2023) .. 
  (108.4383, 62.92).. controls (113.222, 64.5739) and (118.2708, 64.8711) .. 
  (123.257, 64.3799).. controls (130.2981, 64.0463) and (136.9862, 60.9718) .. 
  (144.0883, 61.0643).. controls (146.719, 61.0888) and (150.4646, 61.6509) .. 
  (151.1458, 64.6987).. controls (151.673, 67.2873) and (148.7959, 68.7209) .. 
  (146.6236, 68.8437).. controls (139.156, 69.0532) and (131.7727, 67.3623) .. 
  (124.35, 66.8934).. controls (119.4134, 66.7127) and (114.2214, 65.8106) .. 
  (109.4186, 67.4753).. controls (105.9177, 68.5579) and (104.7386, 73.2447) .. 
  (106.7304, 76.1223).. controls (109.1722, 79.5909) and (113.9875, 79.8463) .. 
  (117.7855, 78.9935).. controls (121.7143, 77.6868) and (125.8173, 79.002) .. 
  (129.6954, 79.4596).. controls (130.3847, 79.478) and (131.072, 79.4129) .. 
  (131.7557, 79.3278).. controls (132.3034, 79.536) and (132.8554, 79.7292) .. 
  (133.429, 79.8477).. controls (133.5633, 79.8947) and (133.6393, 79.9118) .. 
  (133.6734, 79.9066) -- cycle(61.0117, 78.4441).. controls (61.7326, 78.4107) 
  and (62.4373, 78.0807) .. (62.9341, 77.5589).. controls (63.9714, 76.5315) and
   (62.2737, 75.8064) .. (61.3667, 75.958).. controls (61.1215, 76.4988) and 
  (61.1368, 77.5737) .. (60.3038, 77.9036).. controls (59.4883, 78.1476) and 
  (60.1053, 78.4011) .. (60.7027, 78.4395).. controls (60.8058, 78.4471) and 
  (60.9087, 78.4489) .. (61.0117, 78.4441) -- cycle(197.5316, 76.3399).. 
  controls (197.0391, 75.0558) and (196.5069, 73.7953) .. (195.9497, 72.5484).. 
  controls (196.5063, 73.7954) and (197.0397, 75.0556) .. (197.5316, 76.3399) --
   cycle(195.7787, 72.1448).. controls (195.2115, 70.8953) and (194.6015, 
  69.6687) .. (193.9643, 68.4561).. controls (194.6011, 69.6688) and (195.2119, 
  70.8951) .. (195.7787, 72.1448) -- cycle(78.1037, 69.6214).. controls 
  (78.2952, 69.6488) and (77.874, 68.4945) .. (77.9807, 69.5274).. controls 
  (78.0364, 69.5893) and (78.0763, 69.6175) .. (78.1037, 69.6214) -- 
  cycle(193.7731, 68.0737).. controls (193.1304, 66.8687) and (192.4416, 
  65.6864) .. (191.7247, 64.5173).. controls (192.4413, 65.6864) and (193.1307, 
  66.8685) .. (193.7731, 68.0737) -- cycle(191.5009, 64.1355).. controls 
  (190.7998, 63.0089) and (190.0499, 61.9051) .. (189.2752, 60.8116).. controls 
  (190.0498, 61.9049) and (190.8001, 63.009) .. (191.5009, 64.1355) -- 
  cycle(78.1285, 63.7861).. controls (78.2651, 63.7958) and (78.6638, 63.5765) 
  .. (78.7321, 63.5872).. controls (78.927, 63.4599) and (79.6338, 63.2421) .. 
  (80.0979, 63.1024).. controls (79.468, 63.155) and (78.9496, 62.1999) .. 
  (78.5362, 62.7376).. controls (77.4281, 62.4122) and (78.1813, 63.0358) .. 
  (78.1016, 63.6833).. controls (78.0671, 63.7544) and (78.0829, 63.7829) .. 
  (78.1285, 63.7861) -- cycle(67.6372, 50.9177).. controls (68.2247, 50.8851) 
  and (68.7826, 50.5819) .. (69.1911, 50.1647).. controls (70.2488, 49.3933) and
   (68.8115, 48.4569) .. (67.986, 48.9188).. controls (68.2377, 49.6443) and 
  (67.8288, 50.5442) .. (67.3834, 50.9146).. controls (67.4682, 50.9214) and 
  (67.5532, 50.9223) .. (67.6372, 50.9177) -- cycle(140.5128, 45.5965).. 
  controls (141.251, 45.5632) and (141.3991, 44.4823) .. (141.118, 43.7078).. 
  controls (140.892, 44.1187) and (140.0713, 44.7785) .. (139.9056, 45.4353).. 
  controls (140.1408, 45.5558) and (140.3425, 45.6042) .. (140.5128, 45.5965) --
   cycle(66.123, 45.1325).. controls (67.2825, 45.1355) and (66.5885, 43.8848) 
  .. (66.2171, 43.3801).. controls (65.6468, 43.7645) and (64.7899, 44.0173) .. 
  (64.1165, 43.5708).. controls (64.2431, 44.3838) and (65.2339, 45.1106) .. 
  (66.123, 45.1325) -- cycle(126.5762, 44.5764).. controls (127.1972, 44.4241) 
  and (127.3803, 43.0448) .. (126.8356, 42.6381).. controls (126.1643, 42.7329) 
  and (125.2558, 42.6945) .. (124.8145, 41.9647).. controls (124.791, 43.0893) 
  and (125.4598, 44.2562) .. (126.5762, 44.5764) -- cycle(70.2122, 39.4656).. 
  controls (70.7862, 39.4672) and (71.3276, 39.1429) .. (71.1889, 38.4585).. 
  controls (70.5747, 38.0343) and (69.6073, 37.6194) .. (69.6365, 36.642).. 
  controls (68.9355, 37.0327) and (68.7237, 39.3259) .. (69.9657, 39.445).. 
  controls (70.0471, 39.4584) and (70.1302, 39.4654) .. (70.2122, 39.4656) -- 
  cycle(127.7539, 38.4797).. controls (128.5147, 38.517) and (128.9022, 37.6425)
   .. (128.9988, 37.011).. controls (128.3564, 36.7107) and (127.3583, 36.273) 
  .. (127.0279, 35.4251).. controls (126.6446, 36.3846) and (126.7056, 38.0407) 
  .. (127.7539, 38.4797) -- cycle(133.2631, 37.8192).. controls (133.6935, 
  37.8476) and (134.1663, 37.6065) .. (134.3158, 37.1857).. controls (133.8754, 
  36.8981) and (133.1458, 36.466) .. (133.0301, 35.7759).. controls (132.7429, 
  35.8722) and (132.5267, 37.573) .. (133.2631, 37.8192) -- cycle(95.8856, 
  32.3607).. controls (96.4567, 32.1161) and (96.1526, 30.8868) .. (95.6799, 
  30.6264).. controls (95.2584, 31.185) and (94.37, 31.3926) .. (93.7105, 
  31.0346).. controls (93.9458, 31.8065) and (95.0185, 32.3264) .. (95.8856, 
  32.3607) -- cycle(109.061, 28.4462).. controls (109.4123, 27.9087) and 
  (109.7643, 27.1274) .. (108.7008, 26.745).. controls (108.5656, 26.9875) and 
  (108.5412, 28.1124) .. (109.061, 28.4462) -- cycle(96.4664, 25.0836).. 
  controls (97.1697, 24.9202) and (97.5164, 24.0887) .. (97.7826, 23.4816).. 
  controls (96.8587, 23.4677) and (96.1035, 22.5794) .. (95.8742, 21.7029).. 
  controls (95.3809, 22.7726) and (95.6297, 24.3213) .. (96.4664, 25.0836) -- 
  cycle(102.9094, 22.1473).. controls (103.0952, 22.1592) and (103.2839, 
  22.1414) .. (103.4623, 22.0874).. controls (104.6443, 21.7782) and (103.5299, 
  20.8782) .. (103.8179, 20.1077).. controls (104.2607, 19.371) and (103.5085, 
  19.7485) .. (103.174, 20.0958).. controls (102.776, 20.5394) and (101.9265, 
  22.0544) .. (102.9094, 22.1473) -- cycle;



  \path[fill=c125b85,line width=3.175mm] (190.2557, 60.3734) -- (134.5546, 
  95.5279) -- (192.6478, 131.9794) -- (134.5546, 168.4314) -- (207.0428, 
  212.6876).. controls (207.2144, 187.8857) and (207.1057, 163.0838) .. 
  (206.7369, 138.2818).. controls (206.2553, 105.754) and (204.6239, 80.3161) ..
   (190.2557, 60.3734) -- cycle;



  \node[shift={(-333.9473, 69.9584)},anchor=south west] (text423) at (603.9368, 
  -61.4862){};



  \path[draw=black,line width=5.2917mm] (205.496, 240.4929).. controls 
  (206.0334, 206.4482) and (206.0023, 172.4034) .. (205.496, 138.3586).. 
  controls (205.0036, 105.5622) and (203.3703, 79.9645) .. (188.6782, 59.9565)..
   controls (170.9796, 35.8499) and (141.4913, 18.4495) .. (103.5628, 1.3543).. 
  controls (65.6274, 18.4495) and (36.1426, 35.8464) .. (18.444, 59.9565).. 
  controls (3.7484, 79.9714) and (2.1117, 105.5657) .. (1.6262, 138.3586).. 
  controls (1.12, 172.4034) and (1.0887, 206.4482) .. (1.6262, 240.4929) -- 
  cycle;




\end{tikzpicture}
}
\end{figure}
\restoregeometry    %Titelseite ohne Logo
        %\include{matter/titlepage/titlepage_logo_bw}%Titelseite mit Logo (schwarz-weiß)
        \include{matter/titlepage/titlepage_logo_colour}%Titelseite mit Logo (farbig)
        \newgeometry{left=35mm,right=35mm,marginparwidth=20mm}%Erhöhe Seitenränder für Randnotizen
        \include{matter/frontmatter/abstract}      %Deutscher und Englischer Abstract
        \tableofcontents
\listoftables
\listoffigures
\printglossary[type=\acronymtype]  %Inhalts-, Tabellen- und Abbildungsverzeichnis
    %
    %% Hauptteil
    \mainmatter
        \chapter{Einführung}\label{ch:einfuehrung}
\glsresetall
        \chapter{Aktueller Stand der Forschung}\label{ch:forschungsstand}
\gls{fair}\glsresetall
        \chapter{Richtlinien zu Forschungsdaten aus deutschen Promotionsvorhaben}\label{ch:richtlinien}

\parsum{Thema des Kapitels}
Dieses Kapitel behandelt die verschiedenen verwaltungsrechtlichen Dokumente wissenschaftlicher Institutionen, die ein Promotionsvorhaben in Bezug auf \gls{fdm} entweder spezifisch oder auch nur allgemein betreffen.
Es wird überprüft, inwiefern promotionsberechtigte Institutionen in Deutschland bereits derlei Richtlinien erlassen haben, in welcher Form diese existieren und welche Anforderungen diese stellen.

\parsum{Aufbau des Kapitels}
Hierfür wird in \cref{sec:policy-material-methods} aufgeführt, wie die zu untersuchenden Institutionen ausgewählt wurden, wie welche Materialien der Institutionen ausgesucht wurden und mit welchen Methoden das gesammelte Material daraufhin ausgewertet wurde.
In \cref{sec:policy-results} werden die entsprechenden Ergebnisse der Materialauswertung dargestellt.
Abschließend werden in \cref{sec:policy-discussion} die dargestellten Ergebnisse evaluiert und diskutiert.

\section{Material \&\ Methoden}\label{sec:policy-material-methods}
\parsum{Aufbau des Abschnitts}
In diesem Abschnitt wird das zu untersuchende Material in \cref{sec:policy-material} und die Methoden der Untersuchung in \cref{sec:policy-methods} dargestellt.
\subsection{Material}\label{sec:policy-material}
\parsum{Datengrundlage}
Da es in Deutschland hierzu keine offizielle und öffentlich zugängliche Liste aller Universitäten mit Promotionsrecht seitens des Bundesministeriums für Bildung gibt, wird als Datengrundlage für dieses Kapitel die von der \citeauthor{Hochschulkompass-Liste} geführte Liste aller wissenschaftlichen Institutionen aus dem tertiären Bildungsbereich in Deutschland \autocite{Hochschulkompass-Liste} genutzt.
Diese Liste wird tagesaktuell geführt, basiert auf der Selbstauskunft aller involvierten Institutionen ($n=428$), kodifiziert unter anderem welche Institutionen das Promotions- und Habilationsrecht führen, umfasst auch Institutionen, die nicht Mitglied der \citeauthor{Hochschulkompass-Liste} sind und besitzt einen \textit{de facto} wenn auch nicht \textit{de jure} Status als Datengrundlage für allgemeine Informationen zu wissenschaftlichen Institutionen aus Deutschland. Für die tagesspezifische Version der Liste, die für diese Arbeit genutzt wurde, siehe \fxfatal*{ADD LINK TO DIGITAL APPENDIX!}{FIXME!}.

\parsum{Grundmengenbeschreibung}
Um die zentrale Forschungsfrage dieses Kapitels zu beantworten, wurde diese Liste auf nur jene Institutionen gefiltert, welche das Promotionsrecht besitzen ($n=163$).
Die resultierende Liste promotionsberechtigter Institutionen besteht aus Forschungsinstutionen verschiedener Hochschultypen sowie unterschiedlicher Trägerschaften.
Von den Hochschultypen her umfasst die Liste Universitäten, \glspl{fh}, \glspl{haw}, \glspl{kh}, eine \gls{vh} sowie eine \gls{hset}.
Von den Trägerschaften her umfasst die Liste öffentlich-rechtliche, private sowie kirchliche Institutionen. Alle Institutionen sind staatlich anerkannt.
Die relative sowie die absolute Distribution aller promotionsberechtigter Institutionen in Deutschland nach Hochschultyp und Trägerschaft ist in \cref{tab:grundmenge-beschreibung-art} gegeben.
\begin{table}[!htbp]
	\caption{Die Verteilung aller promotionsberechtigter Institutionen in Deutschland nach $\text{\textit{Hochschultyp}}\times\text{\textit{Trägerschaft}}$ aufgegliedert. Absolute Werte in Klammern angegeben.}
    \resizebox{\ifdim\width>\textwidth\textwidth\else\width\fi}{!}{%
        \begin{tabular}{lS[table-format=3.2]@{\,}S[table-text-alignment = left]lS[table-format=3.2]@{\,}S[table-text-alignment = left]lS[table-format=3.2]@{\,}S[table-text-alignment = left]lS[table-format=3.2]@{\,}S[table-text-alignment = left]l}
            \toprule
            & \multicolumn{3}{c}{\textbf{Öffentlich-Rechtlich}} & \multicolumn{3}{c}{\textbf{Privat}} & \multicolumn{3}{c}{\textbf{Kirchlich}} & \multicolumn{3}{c}{\textbf{Summe}}    \\
            \midrule
            \textbf{Universität}  & 53,37 & \si{\percent} & (87)  & 7,98 & \si{\percent} & (13) & 6,13 & \si{\percent} & (10) & 67,48  & \si{\percent} & (110) \\
            \textbf{\gls{fh} / \gls{haw}}     & 6,75  & \si{\percent} & (11)  & 0,00 & \si{\percent} & (0)  & 0,00 & \si{\percent} & (0)  & 6,75   & \si{\percent} & (11)  \\
            \textbf{\gls{kh}}          & 23,93 & \si{\percent} & (39)  & 0,61 & \si{\percent} & (1)  & 0,00 & \si{\percent} & (0)  & 24,54  & \si{\percent} & (40)  \\
            \textbf{\gls{hset}}         & 0,61  & \si{\percent} & (1)   & 0,00 & \si{\percent} & (0)  & 0,00 & \si{\percent} & (0)  & 0,61   & \si{\percent} & (1)   \\
            \textbf{\gls{vh}}          & 0,61  & \si{\percent} & (1)   & 0,00 & \si{\percent} & (0)  & 0,00 & \si{\percent} & (0)  & 0,61   & \si{\percent} & (1)   \\\midrule
            \textbf{Summe}        & 85,28 & \si{\percent} & (139) & 8,59 & \si{\percent} & (14) & 6,13 & \si{\percent} & (10) & 100,00 & \si{\percent} & (163) \\
            \bottomrule
        \end{tabular}
    }
	\label{tab:grundmenge-beschreibung-art}
\end{table}

\noindent Geografisch gesehen sind, in der gefilterten Liste, zu unterschiedlich hohen Anteilen, Institutionen aus allen deutschen Bundesländern vertreten.
Die genaue Verteilung ist in \cref{fig:DE-grundmenge-beschreibung} wiedergegeben.
\begin{figure}[!htbp]
    \centering
    \begin{tikzpicture}[y=1cm, x=1cm, yscale=.8,xscale=.8, every node/.append style={scale=1}, inner sep=0pt, outer sep=0pt]
  \footnotesize
  \drawgermany
  \drawbw{colorblindC1!93}{25}
  \drawbav{colorblindC1!78}{21}
  \drawbrandenburg{colorblindC1!15}{4}
  \drawhessen{colorblindC1!59}{16}
  \drawmecklenburg{colorblindC1!11}{3}
  \drawniedersachsen{colorblindC1!48}{13}
  \drawnrw{colorblindC1!100}{27}
  \drawrheinland{colorblindC1!30}{8}
  \drawsaarland{colorblindC1!11}{3}
  \drawsachsen{colorblindC1!33}{9}
  \drawsachsenanhalt{colorblindC1!26}{7}
  \drawschleswig{colorblindC1!19}{5}
  \drawthuringen{colorblindC1!19}{5}
  \drawbremen{colorblindC1!7}{2}
  \drawhamburg{colorblindC1!30}{8}
  \drawberlin{colorblindC1!26}{7}
\end{tikzpicture}
    \caption{Die absolute Anzahl promotionsberechtigter Institutionen nach Bundesland.}
    \label{fig:DE-grundmenge-beschreibung}
\end{figure}

\noindent Für die gesamte Liste promotionsberechtigter Institutionen, siehe \fxfatal*{ADD LINK TO DIGITAL APPENDIX!}{FIXME!}

\parsum{Stichprobenziehung}
Diese Liste promotionsberechtigter Institutionen bildete die Grundmenge für die Ziehung einer einfachen Zufallsstichprobe.
Bei der Auswahl der Stichprobe wurde ein Konfidenzintervall von \SI{95}{\percent} und eine Fehlerspanne von \SI{5}{\percent} zugrunde gelegt.
Diese Parameter gewährleisten, dass die Ergebnisse der Stichprobe mit hoher Wahrscheinlichkeit repräsentativ für die gesamte Population sind und die Unsicherheit der Schätzungen innerhalb akzeptabler Grenzen bleibt.
Um den Prozess der Stichprobenziehung zu automatisieren und eine zufällige Auswahl zu gewährleisten, wurde eine auf Python basierende Software \autocite{Krassnig2024-csv} genutzt, welche im Rahmen dieser Arbeit geschrieben wurde.%
\footnote{%
Die Software von \citeauthor{Krassnig2024-csv} \autocite{Krassnig2024-csv} nutzt standardmäßig die Anzahl an Nanosekunden seit dem Beginn der System-Epoche (1970-01-01T00:00:00Z) als Startwert für die Zufallsfunktion.
Der genutzte Startwert wird als begleitendes Metadatum der Stichprobe abgespeichert.
Die Ziehung ist somit wiederholbar und das Datum der Ziehung verifizierbar.} 

\parsum{Stichprobenbeschreibung}
Die so gezogene Stichprobe ($n=115$) besteht aus ca. \SI{71}{\percent} aller promotionsberechtigter Institutionen.
Die Stichprobe umfasst Institutionen aller Trägerschaften aus der Grundmenge: öffentlich-rechtliche, private sowie kirchliche Institutionen.
Darüber hinaus umfasst die Stichprobe von den Hochschultypen her Universitäten, \glspl{fh}, \glspl{haw} und eine \gls{hset}.
In der Stichprobe befindet sich nicht die \gls{vh}, die sich in der Grundmenge befindet.
Mit dieser Ausnahme sind somit alle anderen Hochschultypen vertreten.
Die relative sowie die absolute Distribution aller Institutionen in der Stichprobe nach Hochschultyp und Trägerschaft ist in \cref{tab:stichprobe-beschreibung-art} gegeben.
\begin{table}[!htbp]
	\caption{Die Verteilung der Institutionen in der Stichprobe nach $\text{\textit{Hochschultyp}}\times\text{\textit{Trägerschaft}}$ aufgegliedert. Absolute Werte in Klammern angegeben.}
    \resizebox{\ifdim\width>\textwidth\textwidth\else\width\fi}{!}{%
        \begin{tabular}{lS[table-format=3.2]@{\,}S[table-text-alignment = left]lS[table-format=3.2]@{\,}S[table-text-alignment = left]lS[table-format=3.2]@{\,}S[table-text-alignment = left]lS[table-format=3.2]@{\,}S[table-text-alignment = left]l}
            \toprule
            & \multicolumn{3}{c}{\textbf{Öffentlich-Rechtlich}} & \multicolumn{3}{c}{\textbf{Privat}} & \multicolumn{3}{c}{\textbf{Kirchlich}} & \multicolumn{3}{c}{\textbf{Summe}}    \\
            \midrule
            \textbf{Universität}  & 55,65 & \si{\percent} & (64)  & 8,70 & \si{\percent} & (10) & 4,35 & \si{\percent} & (5) & 68,70  & \si{\percent} & (79) \\
            \textbf{\gls{fh} / \gls{haw}}     & 6,96  & \si{\percent} & (8)  & 0,00 & \si{\percent} & (0)  & 0,00 & \si{\percent} & (0)  & 6,96   & \si{\percent} & (8)  \\
            \textbf{\gls{kh}}          & 22,61 & \si{\percent} & (26)  & 0,87 & \si{\percent} & (1)  & 0,00 & \si{\percent} & (0)  & 23,48  & \si{\percent} & (27)  \\
            \textbf{\gls{hset}}         & 0,87  & \si{\percent} & (1)   & 0,00 & \si{\percent} & (0)  & 0,00 & \si{\percent} & (0)  & 0,87   & \si{\percent} & (1)   \\
            \textbf{\gls{vh}}          & 0,00  & \si{\percent} & (0)   & 0,00 & \si{\percent} & (0)  & 0,00 & \si{\percent} & (0)  & 0,00   & \si{\percent} & (0)   \\\midrule
            \textbf{Summe}        & 86,09 & \si{\percent} & (99) & 9,57 & \si{\percent} & (11) & 4,35 & \si{\percent} & (5) & 100,00 & \si{\percent} & (115) \\
            \bottomrule
        \end{tabular}
    }
	\label{tab:stichprobe-beschreibung-art}
\end{table}

\noindent Geografisch gesehen sind, in der Stichprobe, zu unterschiedlich hohen Anteilen, Institutionen aus allen deutschen Bundesländern vertreten.
Die genaue Verteilung ist in \cref{fig:DE-stichprobe-beschreibung} wiedergegeben.
\begin{figure}[!htbp]
    \centering
    \begin{tikzpicture}[y=1cm, x=1cm, yscale=.8,xscale=.8, every node/.append style={scale=1}, inner sep=0pt, outer sep=0pt]
  \node[text=black,line width=0.0092cm,anchor=west] (title1) at (0,7){\textbf{(A)}};
  \footnotesize
  \drawgermany
  \drawbw{colorblindC1!77}{17}
  \drawbav{colorblindC1!55}{12}
  \drawbrandenburg{colorblindC1!18}{4}
  \drawhessen{colorblindC1!45}{10}
  \drawmecklenburg{colorblindC1!9}{2}
  \drawniedersachsen{colorblindC1!50}{11}
  \drawnrw{colorblindC1!100}{22}
  \drawrheinland{colorblindC1!23}{5}
  \drawsaarland{colorblindC1!9}{2}
  \drawsachsen{colorblindC1!23}{5}
  \drawsachsenanhalt{colorblindC1!27}{6}
  \drawschleswig{colorblindC1!9}{2}
  \drawthuringen{colorblindC1!9}{2}
  \drawbremen{colorblindC1!9}{2}
  \drawhamburg{colorblindC1!36}{8}
  \drawberlin{colorblindC1!23}{5}
\end{tikzpicture}\hfill%
\begin{tikzpicture}[y=1cm, x=1cm, yscale=.8,xscale=.8, every node/.append style={scale=1}, inner sep=0pt, outer sep=0pt]
  \node[text=black,line width=0.0092cm,anchor=west] (title1) at (-0.4,7){\textbf{(B)}};
  \footnotesize
  \drawgermany
  \drawbw{colorblindC1!68}{\SI{68}{\percent}}
  \drawbav{colorblindC1!57}{\SI{57}{\percent}}
  \drawbrandenburg{colorblindC1!100}{\SI{100}{\percent}}
  \drawhessen{colorblindC1!63}{\SI{63}{\percent}}
  \drawmecklenburg{colorblindC1!67}{\SI{67}{\percent}}
  \drawniedersachsen{colorblindC1!85}{\SI{85}{\percent}}
  \drawnrw{colorblindC1!81}{\SI{81}{\percent}}
  \drawrheinland{colorblindC1!63}{\SI{63}{\percent}}
  \drawsaarland{colorblindC1!67}{\SI{67}{\percent}}
  \drawsachsen{colorblindC1!56}{\SI{56}{\percent}}
  \drawsachsenanhalt{colorblindC1!86}{\SI{86}{\percent}}
  \drawschleswig{colorblindC1!40}{\SI{40}{\percent}}
  \drawthuringen{colorblindC1!40}{\SI{40}{\percent}}
  \drawbremen{colorblindC1!100}{\SI{100}{\percent}}
  \drawhamburg{colorblindC1!100}{\SI{100}{\percent}}
  \drawberlin{colorblindC1!71}{\SI{71}{\percent}}
\end{tikzpicture}

    \caption{Verteilung der Institutionen in der gezogenen Stichprobe nach Bundesland. \textbf{(A)}~Die absolute Anzahl der Institutionen nach Bundesland. \fxfatal*{Deutlicher formulieren!}{\textbf{(B)}~Der relative Anteil der aus der Grundmenge gezogenen Institutionen nach Bundesland.}}
    \label{fig:DE-stichprobe-beschreibung}
\end{figure}

\parsum{Dokumentesammlung}
Für die Evaluation, inwiefern die Institutionen der Stichprobe verwaltungsrechtliche Dokumente besitzen, die entweder allgemeine \gls{fdm}-Richtlinien für alle Forschenden der Institution oder spezifische Regelungen für Promovierende beinhalten, wurde deren gesamte öffentlich zugängliche Online-Präsenz nach relevanten Dateien und Webseiten durchsucht.
Diese Suche fand im Allgemeinen auf zwei Wege statt: über die interne Suchfunktion der Institution und die externe Durchsuchung der Institutions-Domäne via der Suchmaschine \href{https://www.duckduckgo.com/}{DuckDuckGo}.

\parsum{Allgemeine Dokumente}
Für allgemeingültige Richtlinien wurden in erster Linie eigenständige \gls{fdm}-Richtlinien gesucht.
Hierfür wurde, jenseits der allgemeinen Suchmethode auch die dedizierte Forschungsdatenpolicies-Liste der Informationsplattform \citeauthor{Forschungsdaten2024} genutzt \autocite{Forschungsdaten2024}.%
\footnote{%
    Die \gls{fdm}-Richtlinien, welche sich nicht auf diesem Portal haben finden lassen, werden vom Autor nach Beendigung dieser Arbeit dort nach Möglichkeit eingetragen.%
}
Jenseits von spezifischen \gls{fdm}-Richtlinien wurden auch Richtlinien zur Sicherung der \gls{gwp} sowie andere Richtlinien gesucht, die anderweitige allgemeine wissenschaftliche Empfehlungen bzw. Auflagen für Forschende aussprechen.
Wenn es von einem Dokument sowohl eine HTML- wie auch eine PDF-Datei gab, so wurde nur die PDF-Datei zur Evaluation weitergenutzt.
Hierbei wurden insgesamt 142 Dokumente zur Weiterverarbeitung aufgenommen.

\parsum{Promotionsspezifische Dokumente}
Für promotionsspezifische Richtlinien wurden Promotions- und Prüfungsordnungen gesucht.
Dabei wurden sowohl fachspezifische Ordnungen wie auch verbindliche Rahmenbedingungen und anderweitige übergreifende Ordnungen aufgenommen.
Die heterogene Handhabung dieser Dokumente seitens der Institutionen führte zu folgenden Selektionsregeln bei der Auswahl:
Forschungsdaten.org
(i)~Wenn es eine aktuelle Lesefassung der Promotionsordnung gibt, so wird diese bevorzugt.
(ii)~Sollte es keine aktuelle Lesefassung der Promotionsordnung geben, so wird die aktuellste Gesamtversion der Promotionsordnung bevorzugt.
(iii)~Sollte es keine aktuelle Gesamtversion geben, so werden zusätzlich zu der letzten Version der Promotionsordnung auch alle seither erschienen relevanten verwaltungsrechtlichen Addenda aufgenommen.
(iv)~Wenn es von einem Dokument sowohl eine HTML- wie auch eine PDF-Datei gab, so wurde nur die PDF-Datei zur Evaluation weitergenutzt.
Hierbei wurden insgesamt 754 Dokumente zur Weiterverarbeitung aufgenommen.

\subsection{Methoden}\label{sec:policy-methods}
\parsum{Allgemeine Dokumente}
Die in \cref{sec:policy-material} gesammelten allgemeingültigen Dokumente und deren Institutionen wurden dann wie folgt klassifiziert:
(i)~Jedes Dokument wurde durch manuelle Überprüfung einem Typ zugeordnet.
Diese Typen waren, hierarchisch geordnet: 
\begin{enumerate}
    \item Nicht relevant
    \item Richtlinie zu \gls{gwp}
    \item Anderweitige Richtlinie die \gls{fdm} beinhaltet
    \item Richtlinie zu \gls{forschungsdaten}~/~\gls{fdm}
\end{enumerate}    
(ii)~Bei Dokumenten, welche als \enquote{Richtlinie zu \gls{forschungsdaten}~/~\gls{fdm}} klassifiziert wurden, wird zusätzlich notiert, ob es sich dabei um eine Leitlinie, einen Grundsatz, eine Policy, eine Empfehlung oder eine Richtlinie handelt (nach dokumenteigener Angabe).
(iii)~Jede Institution erhielt dann die Klassifikation des Dokumentes, welche die höchste hierarchische Klassifikationsstufe besitzt, es sei denn, die Institution hatte kein öffentlich zugängliches Dokument dieser Art.
In diesem Fall wurde dies stattdessen als Klassifikationsstufe vermerkt.

\parsum{Promotionsspezifische Dokumente}
Die in \cref{sec:policy-material} gesammelten promotionsspezifischen Dokumente und deren Institutionen wurden dann wie folgt klassifiziert:
(i)~Es wurde überprüft, ob alle PDF-Dateien lesbaren eingebetteten Text besitzen.
Hierfür wurde mit \cref{lst:pdfheadtail} der Anfang und das Ende des eingebetteten Textes via \emph{pdftotext} aus der \emph{Poppler}-Softwaresammlung \autocite{Poppler} angezeigt.
\lstinputlisting[language=Bash,label={lst:pdfheadtail},caption={Ein Bash-Skript, welches die ersten und letzten 20 Zeilen aller PDF-Dokumente in allen Unterordnern des jetzigen Ordners anzeigt.}]{content/code/pdfheadtail.sh}
(ii)~Texte, die sich als maschinell nicht lesbar erwiesen haben, wurden notiert und später manuell untersucht und klassifiziert.
(iii)~Die restlichen Texte wurden zuerst mit \cref{lst:fdmchecker} daraufhin überprüft, ob sie Text beinhalten, der \glspl{forschungsdaten}, \gls{fdm} oder \gls{gwp} betrifft.
\lstinputlisting[language=Bash,label={lst:fdmchecker},caption={Ein Bash-Skript, welches Erwähnung von \glspl{forschungsdaten} und \gls{gwp} in PDF-Dateien überprüft und den dazugehörigen Kontext anzeigt.}]{content/code/fdmchecker.sh}
(iv)~Die durch das Skript angezeigten Treffer wurden dann auf Kontext überprüft und entsprechend manuell klassifiziert.
Hierbei gab es fünf Klassifkationsstufen:
\begin{enumerate}
    \item Keinerlei Richtlinien zu \glspl{forschungsdaten}~/~\gls{fdm} enhalten
    \item Richtlinien zu \gls{gwp} als Empfehlung enthalten
    \item Richtlinien zu \gls{gwp} als Verpflichtung enthalten
    \item Richtlinien zu \glspl{forschungsdaten}~/~\gls{fdm} als Empfehlung enthalten
    \item Richtlinien zu \glspl{forschungsdaten}~/~\gls{fdm} als Verpflichtung enthalten
\end{enumerate}
(v)~Jede Institution erhielt dann die Klassifikation des Dokumentes, welche die höchste hierarchische Klassifikationsstufe besitzt, es sei denn, die Institution hatte kein öffentlich zugängliches Dokument dieser Art.
In diesem Fall wurde dies stattdessen als Klassifikationsstufe vermerkt.
Es wurde zusätzlich notiert, ob die promotionsspezifischen Richtlinien für alle Promovierenden gelten oder ob dies nur auf eine Teilmenge zutrifft.

\section{Resultate}\label{sec:policy-results}

\subsection{Allgemeingültige Dokumente}

\subsection{Promotionsspezifische Dokumente}

\section{Diskussion}\label{sec:policy-discussion}\glsresetall
        \chapter{Forschungsdaten im Repositorium der Leibniz Universität Hannover}\label{ch:luh-repo}
\parsum{Thema des Kapitels}
Dieses Kapitel behandelt die Auswertung von eingebetteten, begleitenden sowie referenzierten \glspl{forschungsdaten} von Dissertationen der \gls{luh}, die im institutionellen \gls{luh-repo} veröffentlicht worden sind.
Die Arbeit beschränkt sich hierbei exklusiv auf \glspl{forschungsdaten}, die originelle \glspl{pd} darstellen, die im Rahmen des Promotionsvorhabens entstanden sind.
Es wird hierbei überprüft, welcher Anteil an Dissertationen originelle \glspl{forschungsdaten} beinhaltet, auf welche Art und Weise die \glspl{forschungsdaten} inkludiert wurden, wie sich diese über die einzelnen Fakultäten verteilen, wie sich diese über die letzten zwölf Jahre entwickelt haben und wie die Existenz von \glspl{forschungsdaten} in den Metadaten kenntlich gemacht wurden (sowohl im \gls{luh-repo} wie auch in etwaigen externen \gls{forschungsdaten}-Repositorien).

\parsum{Aufbau des Kapitels}
Hierfür wird in \cref{sec:luh-repo-material-methods} aufgeführt, wie die zu untersuchenden Dissertationen ausgewählt wurden, wie das entsprechende Material gesammelt wurde und mit welchen Methoden dieses daraufhin ausgewertet wurde.
In \cref{sec:luh-repo-results} werden die entsprechenden Ergebnisse der Materialauswertung dargestellt.
Abschließend werden in \cref{sec:luh-repo-discussion} die dargestellten Ergebnisse evaluiert und diskutiert.

\section{Material \&\ Methoden}\label{sec:luh-repo-material-methods}
\parsum{Aufbau des Abschnitts}
In diesem Abschnitt wird das zu untersuchende Material in \cref{sec:luh-repo-material} und die Methoden der Untersuchung in \cref{sec:luh-repo-methods} dargestellt.

\subsection{Material}\label{sec:luh-repo-material}
\parsum{Datengrundlage}
Als Datengrundlage für dieses Kapitel gilt die Metadaten-Datenbank aller Dissertationen des \gls{luh-repo}, welche via \gls{oai-pmh} der Öffentlichkeit frei zugänglich sind \autocite{luh-repo}.
Da sich das Thema dieses Kapitels explizit auf Dissertationen beschränkt, wurden von der Metadaten-Datenbank des \gls{luh-repo} alle Einträge der Sammlung \textit{Dissertationen} die am 21.03.2024 um 10:41 Uhr (UTC+01:00) existierten ($n=5095$) über die Administrationsübersicht des \gls{luh-repo} heruntergeladen \autocite{my-dataset}.

\parsum{Grundmengenbeschreibung}
Da die zentrale Forschungsfrage dieses Kapitels sich auf den Zeitraum von 2012--2023 beschränkt, wurde diese Liste durch \cref{lst:python-luh-repo-stratification} auf nur jene Metadateneinträge gefiltert, deren Publikationsjahr in diese Zeitspanne fällt und deren Sperrfrist auch, insofern vorhanden, spätestens 2023 endete ($n=1898$).
Die daraus resultierende Dissertationsliste enthält Einträge zu jeder Fakultät der \gls{luh}.
Aus Gründen der Übersichtlichkeit und des Platzes nutzen wir folgende Abkürzungen für die Fakultäten der \gls{luh}:
\begin{itemize}
    \item \gls{fakultät2}
    \item \gls{fakultät3}
    \item \gls{fakultät4}
    \item \gls{fakultät5}
    \item \gls{fakultät6}
    \item \gls{fakultät7}
    \item \gls{fakultät8}
    \item \gls{fakultät9}
    \item \gls{fakultät10}
\end{itemize}
Der Name der Fakultät wird fortan nur noch separat erwähnt, wenn dieser relevant zur Diskussion und Verständlichkeit der Daten erscheint.
Der Zeitraum von 2012--2023 wurde für die weitere Bearbeitung wiederum in drei kontinuierliche Zeitintervalle von jeweils vier Jahren aufgeteilt.
Die relative sowie die absolute Distribution aller Metadateneinträge nach Zeitraum und Fakultät ist in \cref{tab:luh-repo-grundmenge-beschreibung} gegeben.
\begin{table}[!htbp]
	\caption{Die Verteilung der Grundmengen-Metadateneinträge nach $\text{\textit{Fakultät}}\times\text{\textit{Zeitraum}}$ aufgegliedert.
    Absolute Werte in Klammern angegeben.}
    \resizebox{\ifdim\width>\textwidth\textwidth\else\width\fi}{!}{%
	\begin{tabular}{lS[table-format=3.2]@{\,}S[table-text-alignment = left]lS[table-format=3.2]@{\,}S[table-text-alignment = left]lS[table-format=3.2]@{\,}S[table-text-alignment = left]lS[table-format=3.2]@{\,}S[table-text-alignment = left]l}
		\toprule
		& \multicolumn{3}{c}{\textbf{2012--2015}} & \multicolumn{3}{c}{\textbf{2016--2019}} & \multicolumn{3}{c}{\textbf{2020--2023}} & \multicolumn{3}{c}{\textbf{Summe}}    \\
		\midrule
		\textbf{\gls{fakultät2}}  & 0,68  & \si{\percent} & (13)  & 1,21  & \si{\percent} & (23)  & 1,37  & \si{\percent} & (26) & 3,27   & \si{\percent} & (62)         \\
		\textbf{\gls{fakultät3}}  & 1,00  & \si{\percent} & (19)  & 1,79  & \si{\percent} & (34)  & 3,64  & \si{\percent} & (69)  & 6,43    & \si{\percent} & (122)          \\
		\textbf{\gls{fakultät4}}  & 1,90  & \si{\percent} & (36)  & 2,63  & \si{\percent} & (50)  & 4,21  & \si{\percent} & (80)  & 8,75   & \si{\percent} & (166)          \\
		\textbf{\gls{fakultät5}}  & 0,00  & \si{\percent} & (0)   & 0,16  & \si{\percent} & (3)   & 0,05  & \si{\percent} & (1)  & 0,21    & \si{\percent} & (4)           \\
		\textbf{\gls{fakultät6}}  & 1,69  & \si{\percent} & (32)  & 1,95  & \si{\percent} & (37)  & 3,48  & \si{\percent} & (66)  & 7,11    & \si{\percent} & (135)           \\
		\textbf{\gls{fakultät7}}  & 5,32  & \si{\percent} & (101)  & 4,48  & \si{\percent} & (85)  & 6,38  & \si{\percent} & (121)  & 16,17    & \si{\percent} & (307)           \\
		\textbf{\gls{fakultät8}}  & 13,22 & \si{\percent} & (251) & 11,80  & \si{\percent} & (224) & 15,12 & \si{\percent} & (287)  & 40,15    & \si{\percent} & (762)           \\
		\textbf{\gls{fakultät9}}  & 2,00  & \si{\percent} & (38)  & 1,95  & \si{\percent} & (37)  & 3,37  & \si{\percent} & (64)  & 7,32    & \si{\percent} & (139)           \\
		\textbf{\gls{fakultät10}} & 3,16  & \si{\percent} & (60)  & 3,37  & \si{\percent} & (64)  & 4,06  & \si{\percent} & (7)  & 10,59    & \si{\percent} & (201)           \\
		\midrule
		\textbf{Summe}            & 28,98 & \si{\percent} & (550) & 29,35 & \si{\percent} & (557) & 41,68 & \si{\percent} & (791) & 100,00  & \si{\percent} & (1898)         \\
		\bottomrule
	\end{tabular}
}
	\label{tab:luh-repo-grundmenge-beschreibung}
\end{table}

\noindent Für eine Liste aller inkludierter Dissertationsmetadaten, siehe \fxfatal*{Fix bibliographic data}{\autocite{my-dataset}}.

\parsum{Stichprobenziehung}
Diese Liste an Dissertationsmetadaten bildete die Grundmenge für die Ziehung einer mehrschichtigen Zufallsstichprobe.
Die Schichten der Zufallsstichprobe entsprachen dabei $\text{\textit{Fakultät}}\times\text{\textit{Jahresspanne}}$ und ergaben daher insgesamt $\num{9}\times\num{3}=\num{27}$ Stichprobengruppierungen.
Für jede Stichprobengruppierung wurde durch \cref{lst:python-luh-repo-stratification} eine eigene CSV-Tabellendatei erstellt.
Auf die einzelnen Stichprobengruppierungen wurde dann jeweils das Verfahren einer einfachen Stichprobenziehung angewandt.
Bei der Auswahl der Stichproben wurde jeweils ein Konfidenzintervall von \SI[round-mode=places,round-precision=2]{95}{\percent} und eine Fehlerspanne von \SI[round-mode=places,round-precision=2]{5}{\percent} zugrunde gelegt.
Diese Parameter gewährleisten, dass die Ergebnisse der Stichprobe mit hoher Wahrscheinlichkeit repräsentativ für die gesamte Population sowie der einzelnen Stichprobengruppierungen sind und die Unsicherheit der Schätzungen innerhalb akzeptabler Grenzen bleibt.
Um den Prozess der Stichprobenziehung zu automatisieren und eine zufällige Auswahl zu gewährleisten, wurde eine auf Python basierende Software \autocite{Krassnig2024-csv} genutzt, welche im Rahmen dieser Arbeit geschrieben wurde.%
\footnote{%
Die Software von \citeauthor{Krassnig2024-csv} \autocite{Krassnig2024-csv} nutzt standardmäßig die Anzahl an Nanosekunden seit dem Beginn der System-Epoche (1970-01-01T00:00:00Z) als Startwert für die Zufallsfunktion.
Der genutzte Startwert wird als begleitendes Metadatum der Stichprobe abgespeichert.
Die Ziehung ist somit wiederholbar und das Datum der Ziehung verifizierbar.} 

\parsum{Stichprobenbeschreibung}
Die so gezogene Stichprobe ($n=1441$) besteht aus ca.~\SI[round-mode=places,round-precision=2]{76}{\percent} aller Metadateneinträge der Grundmenge.
Die relative sowie die absolute Distribution aller Institutionen in der Stichprobe nach \textit{Fakultät} und \textit{Zeitraum} ist in \cref{tab:luh-repo-stichprobe-beschreibung} gegeben.
\begin{table}[!htbp]
	\caption{Die Verteilung der Stichproben-Metadateneinträge nach $\text{\textit{Fakultät}}\times\text{\textit{Zeitraum}}$ aufgegliedert.
    Absolute Werte in Klammern angegeben.}
    \resizebox{\ifdim\width>\textwidth\textwidth\else\width\fi}{!}{%
	\begin{tabular}{lS[table-format=3.2]@{\,}S[table-text-alignment = left]lS[table-format=3.2]@{\,}S[table-text-alignment = left]lS[table-format=3.2]@{\,}S[table-text-alignment = left]lS[table-format=3.2]@{\,}S[table-text-alignment = left]l}
		\toprule
		& \multicolumn{3}{c}{\textbf{2012--2015}} & \multicolumn{3}{c}{\textbf{2016--2019}} & \multicolumn{3}{c}{\textbf{2020--2023}} & \multicolumn{3}{c}{\textbf{Summe}}    \\
		\midrule
		\textbf{\gls{fakultät2}}  & 0,90  & \si{\percent} & (13)  & 1,53  & \si{\percent} & (22)  & 1,73  & \si{\percent} & (25) & 4,16   & \si{\percent} & (60)         \\
		\textbf{\gls{fakultät3}}  & 1,32  & \si{\percent} & (19)  & 2,22  & \si{\percent} & (32)  & 4,09  & \si{\percent} & (59)  & 7,63    & \si{\percent} & (110)          \\
		\textbf{\gls{fakultät4}}  & 2,29  & \si{\percent} & (33)  & 3,12  & \si{\percent} & (45)  & 4,65  & \si{\percent} & (67)  & 10,06   & \si{\percent} & (145)          \\
		\textbf{\gls{fakultät5}}  & 0,00  & \si{\percent} & (0)   & 0,21  & \si{\percent} & (3)   & 0,07  & \si{\percent} & (1)  & 0,28    & \si{\percent} & (4)           \\
		\textbf{\gls{fakultät6}}  & 2,08  & \si{\percent} & (30)  & 2,36  & \si{\percent} & (34)  & 3,96  & \si{\percent} & (57)  & 8,40    & \si{\percent} & (121)           \\
		\textbf{\gls{fakultät7}}  & 5,62  & \si{\percent} & (81)  & 4,86  & \si{\percent} & (70)  & 6,45  & \si{\percent} & (93)  & 16,93    & \si{\percent} & (244)           \\
		\textbf{\gls{fakultät8}}  & 10,62 & \si{\percent} & (153) & 9,85  & \si{\percent} & (142) & 11,45 & \si{\percent} & (165)  & 31,92    & \si{\percent} & (460)           \\
		\textbf{\gls{fakultät9}}  & 2,43  & \si{\percent} & (35)  & 2,36  & \si{\percent} & (34)  & 3,82  & \si{\percent} & (55)  & 8,61    & \si{\percent} & (124)           \\
		\textbf{\gls{fakultät10}} & 3,68  & \si{\percent} & (53)  & 3,82  & \si{\percent} & (55)  & 4,51  & \si{\percent} & (65)  & 12,01    & \si{\percent} & (173)           \\
		\midrule
		\textbf{Summe}            & 28,94 & \si{\percent} & (417) & 30,33 & \si{\percent} & (437) & 40,74 & \si{\percent} & (587) & 100,00  & \si{\percent} & (1441)         \\
		\bottomrule
	\end{tabular}
}
	\label{tab:luh-repo-stichprobe-beschreibung}
\end{table}
Der jeweils relative Anteil der Stichprobengruppierungen zu dem entsprechenden Datensatz aus der Grundmenge sowie die Differenz zwischen der respektiven Anzahl ist, auch nach \textit{Fakultät} und \textit{Zeitraum} aufgegliedert, in \cref{tab:luh-repo-stichprobe-beschreibung-relativ} gegeben.
\begin{table}[!htbp]
	\caption{Die Stichproben-Metadateneinträge nach $\text{\textit{Fakultät}}\times\text{\textit{Zeitraum}}$ aufgegliedert relativ zu der Anzahl an Metadateneinträgen aus der Grundmenge.
    Absolute Differenzwerte in Klammern angegeben.}
    \resizebox{\ifdim\width>\textwidth\textwidth\else\width\fi}{!}{%
	\begin{tabular}{lS[table-format=3.2]@{\,}S[table-text-alignment = left]lS[table-format=3.2]@{\,}S[table-text-alignment = left]lS[table-format=3.2]@{\,}S[table-text-alignment = left]lS[table-format=3.2]@{\,}S[table-text-alignment = left]l}
		\toprule
		& \multicolumn{3}{c}{\textbf{2012--2015}} & \multicolumn{3}{c}{\textbf{2016--2019}} & \multicolumn{3}{c}{\textbf{2020--2023}} & \multicolumn{3}{c}{\textbf{Alle}}    \\
		\midrule
		\textbf{\gls{fakultät2}}  & 100,00  & \si{\percent} & (0)  & 95,65  & \si{\percent} & (-1)  & 96,15  & \si{\percent} & (-1) & 96,77   & \si{\percent} & (-2)         \\
		\textbf{\gls{fakultät3}}  & 100,00  & \si{\percent} & (0)  & 94,12  & \si{\percent} & (-2)  & 85,51  & \si{\percent} & (-10)  & 90,16    & \si{\percent} & (-12)          \\
		\textbf{\gls{fakultät4}}  & 91,67  & \si{\percent} & (-3)  & 90,00  & \si{\percent} & (-5)  & 83,75  & \si{\percent} & (-13)  & 87,35   & \si{\percent} & (-21)          \\
		\textbf{\gls{fakultät5}}  & \multicolumn{1}{r}{---}  &  & (0)   & 100,00  & \si{\percent} & (0)   & 100,00  & \si{\percent} & (0)  & 100,00    & \si{\percent} & (0)           \\
		\textbf{\gls{fakultät6}}  & 93,75  & \si{\percent} & (-2)  & 91,89  & \si{\percent} & (-3)  & 86,36  & \si{\percent} & (-9)  & 89,63    & \si{\percent} & (-14)           \\
		\textbf{\gls{fakultät7}}  & 80,20  & \si{\percent} & (-20)  & 82,35  & \si{\percent} & (-15)  & 76,86  & \si{\percent} & (-28)  & 79,48    & \si{\percent} & (-63)           \\
		\textbf{\gls{fakultät8}}  & 60,96 & \si{\percent} & (-98) & 63,39  & \si{\percent} & (-82) & 57,49 & \si{\percent} & (-122)  & 60,37    & \si{\percent} & (-302)           \\
		\textbf{\gls{fakultät9}}  & 92,11  & \si{\percent} & (-3)  & 91,89  & \si{\percent} & (3)  & 85,94  & \si{\percent} & (-9)  & 89,21    & \si{\percent} & (-15)           \\
		\textbf{\gls{fakultät10}} & 88,33  & \si{\percent} & (-7)  & 85,94  & \si{\percent} & (-9)  & 84,42  & \si{\percent} & (-12)  & 86,07    & \si{\percent} & (-28)           \\
		\midrule
		\textbf{Alle}            & 75,82 & \si{\percent} & (-133) & 78,46 & \si{\percent} & (-120) & 74,21 & \si{\percent} & (587) & 75,92  & \si{\percent} & (-457)         \\
		\bottomrule
	\end{tabular}
}
	\label{tab:luh-repo-stichprobe-beschreibung-relativ}
\end{table}

\parsum{Dateisammlung}
Für die Evaluation, inwiefern die Dissertationen der Stichprobe \glspl{forschungsdaten} beinhalten oder auf solche verweisen wurden alle Dateien, die mit Metadateneinträgen assoziiert werden, heruntergeladen.
Dieser Prozess wurde dadurch verkompliziert, dass DSpace~5, auf welches das \gls{luh-repo} zum Zeitpunkt dieser Arbeit noch basierte, keine eingebaute Möglichkeit anbietet, alle Dateien einer Sammlung (jenseits einer schnell erreichten Grenze) oder einer bestimmten Metadatenliste herunterzuladen:
weder intern mit administrativen Rechten noch extern durch die Nutzung einer Schnittstelle.
Daher wurde im Rahmen dieser Arbeit \cref{lst:simple-dspace5-downloader} entwickelt, welches alle Links zu den entsprechenden Dateien aus dem öffentlichen Quellcode der Webseiten extrahiert und automatisch herunterlädt und nach dem Metadaten-Handle sortiert \autocite{Krassnig2024-dspace}.

Hierbei wurden insgesamt \num{1480} Dateien zur Weiterverarbeitung gefunden, heruntergeladen und sortiert.

\parsum{Zahlenspiegel der \gls{luh}}
Zusätzlich zu den oben aufgeführten Dissertationen wurden auch die von der \gls{luh} veröffentlichten Zahlenspiegel für den zu untersuchenden Zeitraum gesammelt, da diese die jährliche Gesamtzahl veröffentlichter Dissertationen enthalten.
Hierbei wurden die Zahlenspiegel von 2013--2023 ausgesucht, da diese jeweils das vorherige Jahr betreffen \autocite{Zahlenspiegel2013,Zahlenspiegel2014,Zahlenspiegel2015,Zahlenspiegel2016,Zahlenspiegel2017,Zahlenspiegel2018,Zahlenspiegel2019,Zahlenspiegel2020,Zahlenspiegel2021,Zahlenspiegel2022,Zahlenspiegel2023}.
Die aktuellen Daten für 2023, bzw.~der Zahlenspiegel aus 2024 wurde zum Zeitpunkt dieser Arbeit noch nicht veröffentlicht.

\subsection{Methoden}\label{sec:luh-repo-methods}
\parsum{Klassifikation}
Die in \cref{sec:luh-repo-material} gesammelten Dissertationsdateien wurden dann, um die zentrale Forschungsfrage dieses Kapitels zu beantworten, nach ihrem Inhalt klassifiziert, ob und auf welche Arte und Weise sie primäre \glspl{forschungsdaten} beinhalten:
\glspl{forschungsdaten} konnten entweder in die PDF-Datei integriert, als Begleitdaten im \gls{luh-repo} eingereicht worden oder auf ein externes \gls{forschungsdaten}-Repositorium hochgeladen worden sein.
Damit diese Klassifikation stattfinden konnte, musste jedoch zuerst bestimmt werden, welcher Inhalt als \gls{forschungsdaten} gewertet wird (hierbei orientierte sich diese Arbeit an \autocite{dfg-richtlinie,Simukovic2014InterviewFD}) und wo sich dieser typischerweise im Dokument befindet.
Hierfür wurden von jeder Fakultät, gleichmäßig auf die drei Zeiträume aufgeteilt, zwölf zufällige Dissertationen ausgewählt und vorläufig evaluiert.
Bei Stichprobengruppierungen von weniger als vier Dissertationen wurden stattdessen alle Dokumente vorläufig ausgewertet.

\parsum{Klassifikationshierarchie}
Bei dieser Auswertung wurde ein provisorisches Klassifikationssystem aufgebaut, welches für den Rest der Arbeit beibehalten wurde.
Hierbei wurden \glspl{forschungsdaten} in drei hierarchische Stufen eingeteilt.
Diese reichen von \textit{Stufe 1}, welche eindeutige und zweifelsfreie \glspl{pd} beinhalten, zu \textit{Stufe 3}, welche Daten beinhalten, die entweder kompromittiert worden sind (z.B.~durch starke Kompression), keine besondere Leistung darstellen (z.B.~einfache Fragebögen) oder durch den Autor kaum auf Originalität überprüfbar waren.
Unter \textit{Stufe 2} befinden sich jene \glspl{forschungsdaten}, welche zwar originell sind, jedoch weniger direkte Wiederverwendbarkeit oder Qualität im Vergleich zu \glspl{forschungsdaten} aus \textit{Stufe 1} haben.
Es folgt eine Auflistung der verschiedenen \gls{forschungsdaten}-Klassifikationen der entsprechenden Klassifikationsstufen.\\
\textbf{Stufe 1:} rohe Beobachtungs-~/~Messdaten, unkomprimierte Rohbilder, Videos, Skripte~/~Software, Transkriptionen von Interviews, (anonymisierte) Beantwortungen von Fragebögen\\
\textbf{Stufe 2:} Pseudocode, Algorithmen, komprimierte Bilder von Gelfärbungen\footnote{Für komprimierte Bilder von Gelfärbungen wurde auf Anraten einer Wissenschaftlerin aus dem Bereich \textit{Life Science} eine Ausnahme gemacht und als \gls{forschungsdaten} der zweiten Stufe kategorisiert \autocite{SarahPC}.}\\
\textbf{Stufe 3:} komprimierte Bilder, Spektraldiagramme, Gensequenzen, Fragebögen, Leitfäden, Montagezeichnungen

\parsum{Ort der \glspl{forschungsdaten}}
Bei der vorlläufigen Testklassifikation konnten keine bestimmten Teile eines Dokumentes vollständig ausgeschlossen werden.
Während die meisten hochstufigen \glspl{forschungsdaten} in dem jeweiligen Appendix zu finden waren, waren z.B.~\glspl{forschungsdaten} der zweiten und dritten Stufe zu großen Teilen im gesamten Dokument verteilt.
Auch \glspl{forschungsdaten} aus \textit{Stufe 1} waren teilweise in anderen Bereichen der Dissertationen zu finden.
Dies galt auch für externe Forschungsdaten.
Hier wurde teilweise in der Präambel darauf hingewiesen, dass einige oder alle \glspl{forschungsdaten} auf ein externes Repositorium hochgeladen wurde.
Teilweise wurden diese externen Datensätze aber auch erst an der jeweils relevanten Stelle im Hauptteil des Dokumentes zitiert.
Für den restlichen Verlauf der Klassifikation wurde daher beschlossen, dass sämtliche Seiten der PDF-Dateien zumindest kurzzeitig begutachtet werden müssen.

\parsum{Klassifikationsstrategie}
Nach Abschluss der vorläufigen Testklassifikation und Aufbau des hierarchischen Klassifikationssystems wurde dann wie folgt vorgegangen.
Es wurden alle PDF-Dateien einzeln evaluiert.
Insofern ein Typ an \gls{forschungsdaten} im Dokument gefunden wurde, so wurde die \gls{forschungsdaten}-Art und Publikationsart des \gls{forschungsdaten} in der zur Stichprobengruppierungen zugehörigen CSV-Tabellendatei vermerkt.
Hierbei wurde für interne und beigefügte \glspl{forschungsdaten} nur der Typ und nicht die Seite innerhalb des Dokumentes vermerkt.
Für extern publizierte \glspl{forschungsdaten} wurden zusätzlich noch die jeweilig dazugehörigen Seiten und Art des externen Repositoriums vermerkt (z.B.~Git-Repositorium oder dediziertes \gls{forschungsdaten}-Repositorium).
Bei externen \glspl{forschungsdaten} wurden zusätzlich die \gls{doi}, das relevante Stichwort oder die dazugehörige Domäne notiert und in einer separaten Datei eingetragen.
Diese Liste wurde dann am Ende der Evaluation genutzt, um \cref{lst:luh-repo-document-search} zu erstellen.
Dieses Skript durchsuchte automatisch den Text aller Textdatei mit den notierten Wörtern, um etwaige übersehene externe \glspl{forschungsdaten} im Nachhinein noch erfassen zu können.
Zusätzlich wurde für jede Dissertation auch eingetragen, ob für sie überhaupt \glspl{pd} produziert worden sind; diese Information wurden wiederum genutzt, um relative Werte zu der jeweiligen Gesamtsumme aller Dissertationen minus jenen ohne produzierte \glspl{pd} zu erstellen.

\parsum{Gesamtklassifikation}
Nach Beendigung der Klassifikationsarbeit wurden jeder Dissertation jeweils vier Werte zugeordnet: 
\begin{itemize}
    \item die höchste Klassifikationstufe aller gefundenen \glspl{forschungsdaten} der Dissertation
    \item die höchste Klassifikationsstufe aller gefunden \glspl{forschungsdaten} die in der PDF-Datei der Dissertation integriert waren
    \item die höchste Klassifikationsstufe aller gefunden \glspl{forschungsdaten} die der Dissertation als separate Datei beigefügt wurden
    \item die höchste Klassifikationsstufe aller gefunden \glspl{forschungsdaten} die auf einem externen Repositorium hochgeladen wurden.
\end{itemize}

\parsum{Auswertung der Ergebnisse}
Nach der Klassifizierung aller Dateien wurden die vollständig evaluierten CSV-Dateien der Stichprobengruppierungen durch \cref{lst:bash-luh-repo-csv-combiner} einerseits in kombinierte Fakultätstabellen und andererseits in eine Gesamttabelle zusammengeführt.
Die vorhandenen Metadaten wurden dann auf plausible Faktoren untersucht, die einen etwaigen Einfluss auf die Präsenz, Art und Publikationsform von \glspl{forschungsdaten}.
Für die Kreuzprodukte aller vermuteter Faktoren sowie für die Ergebnisse der Klassifikationsarbeit aller \gls{forschungsdaten}-Publikationsarten wurden Chi-Quadrat-Tests für Unabhängigkeit durchgeführt, um zu überprüfen, ob statistisch signifikante Relationen zwischen den jeweiligen Faktoren bzw.~Ergebnissen besteht.
Hierbei wurden für alle zu überprüfenden Relationen die Nullhypothese angenommen:
I.e.~für die Kombination \textit{Faktor A}$\times$\textit{Faktor B} wird angenommen, dass \textit{Faktor A} keinen Einfluss auf \textit{Faktor B} hat und dass, bedingt durch die symmetrische Natur des Chi-Quadrat-Tests, auch andersherum kein Einfluss stattfindet.
Die Nullhypothese gilt als widerlegt wenn der respektive Chi-Quadrat-Test für Unabhängigkeit einen Signifikanzwert von $p<\num{0,05}$ erzeugt.
Bei Signifikanzwerten von $p\geqslant0,05$ gilt die Nullhypothese als bestätigt.
Da p-Werte nichts über die Stärke einer Abhängigkeit aussagen, wurden für alle Testergebnisse mit $p<\num{0,05}$ zusätzlich noch der respektive Cramérs V-Wert ($\phi_C$) berechnet, um zu überprüfen, wie stark die statistisch signifikante Abhängigkeit ist.
Bei einem Cramérs V-Wert von $\phi_C>\num{0,1}$ ist von einem schwachem, bei $\phi_C>\num{0,3}$ von einem moderaten und bei $\phi_C>\num{0,5}$ von einem starken Zusammenhang bzw.~Einfluss auszugehen.

\parsum{Zahlenspiegel der \gls{luh}}
Die in \cref{sec:luh-repo-material} gesammelten Zahlenspiegel der \gls{luh} wurden dann wie folgt ausgewertet.
Es wurden alle Gesamtanzahlen der Dissertationen pro Fakultät für jeden Jahrgang extrahiert und in eine gemeinsame Tabelle eingetragen, wo sie in die Jahresgruppen 2012--2015, 2016--2019 und 2020--2023 geklumpt wurden.
Da die offiziellen Zahlen des Jahres 2023 zum Zeitpunkt dieser Arbeit noch nicht veröffentlicht waren, wurden diese durch das arithmetische Mittel der Fakultätswerte aus den Jahren 2012--2022 simuliert, um vergleichbare Jahreswerte für jede Fakultät zu derivieren.
Die Ausnahme zu dieser Regel sind jene Fakultätswerte für das Jahr 2023, die durch das arithmetische Mittel kleiner gewesen wären, als der entsprechende Wert aus dem \gls{luh-repo}.
In diesem Fall wurde der entsprechende Wert aus dem \gls{luh-repo} übernommen.
Die resultierende Verteilung ist in \cref{tab:luh-repo-zahlenspiegel-summary} zu sehen.
\begin{table}[!htbp]
	\caption{Die Verteilung der Dissertationen laut den Zahlenspiegeln der \gls{luh} nach $\text{\textit{Fakultät}}\times\text{\textit{Zeitraum}}$ aufgegliedert.
    Absolute Werte in Klammern angegeben.
    Spalten, die zumindest teilweise auf simulierten Werten basieren, sind mit einem Asterisk (*) markiert.}
    \resizebox{\ifdim\width>\textwidth\textwidth\else\width\fi}{!}{%
	\begin{tabular}{lS[table-format=3.2]@{\,}S[table-text-alignment = left]lS[table-format=3.2]@{\,}S[table-text-alignment = left]lS[table-format=3.2]@{\,}S[table-text-alignment = left]lS[table-format=3.2]@{\,}S[table-text-alignment = left]l}
		\toprule
		& \multicolumn{3}{c}{\textbf{2012--2015}} & \multicolumn{3}{c}{\textbf{2016--2019}} & \multicolumn{3}{c}{\textbf{2020--2023}} & \multicolumn{3}{c}{\textbf{Summe}}    \\
		\midrule
		\textbf{\gls{fakultät2}}  & 1,04  & \si{\percent} & (40)  & 0,88  & \si{\percent} & (22)  & 1,73  & \si{\percent} & (25) & 4,16   & \si{\percent} & (60)         \\
		\textbf{\gls{fakultät3}}  & 2,18  & \si{\percent} & (84)  & 3,13  & \si{\percent} & (32)  & 4,09  & \si{\percent} & (59)  & 7,63    & \si{\percent} & (110)          \\
		\textbf{\gls{fakultät4}}  & 3,42  & \si{\percent} & (132)  & 3,60  & \si{\percent} & (45)  & 4,65  & \si{\percent} & (67)  & 10,06   & \si{\percent} & (145)          \\
		\textbf{\gls{fakultät5}}  & 2,62  & \si{\percent} & (101)   & 1,66  & \si{\percent} & (3)   & 0,07  & \si{\percent} & (1)  & 0,28    & \si{\percent} & (4)           \\
		\textbf{\gls{fakultät6}}  & 5,70  & \si{\percent} & (220)  & 6,84  & \si{\percent} & (34)  & 3,96  & \si{\percent} & (57)  & 8,40    & \si{\percent} & (121)           \\
		\textbf{\gls{fakultät7}}  & 5,02  & \si{\percent} & (194)  & 4,04  & \si{\percent} & (70)  & 6,45  & \si{\percent} & (93)  & 16,93    & \si{\percent} & (244)           \\
		\textbf{\gls{fakultät8}}  & 9,94 & \si{\percent} & (384) & 9,53  & \si{\percent} & (142) & 11,45 & \si{\percent} & (165)  & 31,92    & \si{\percent} & (460)           \\
		\textbf{\gls{fakultät9}}  & 3,68  & \si{\percent} & (142)  & 4,09  & \si{\percent} & (34)  & 3,82  & \si{\percent} & (55)  & 8,61    & \si{\percent} & (124)           \\
		\textbf{\gls{fakultät10}} & 2,77  & \si{\percent} & (107)  & 3,00  & \si{\percent} & (55)  & 4,51  & \si{\percent} & (65)  & 12,01    & \si{\percent} & (173)           \\
		\midrule
		\textbf{Summe}            & 36,35 & \si{\percent} & (1404) & 36,77 & \si{\percent} & (437) & 40,74 & \si{\percent} & (587) & 100,00  & \si{\percent} & (1441)         \\
		\bottomrule
	\end{tabular}
}
	\label{tab:luh-repo-zahlenspiegel-summary}
\end{table}
Die $\text{\textit{Fakultät}}\times\text{\textit{Jahresgruppe}}$-Zahlen wurden dann mit den Daten aus dem \gls{luh-repo} verglichen.
Aus diesem Vergleich wurde dann jeweils abgeleitet, zu welchem Anteil die Promovierenden der verschiedenen Fakultäten das \gls{luh-repo} nutzen.

\section{Resultate}\label{sec:luh-repo-results}
\parsum{Aufbau des Abschnitts}
In diesem Abschnitt werden die Resultate der Zeitspiegelauswertung und der \gls{luh-repo} Datenklassifizierung sowie deren statistische Auswertung dargestellt.
In \cref{sec:luh-repo-results-zahlenspiegel} werden die Zahlen der \gls{luh}-Zahlenspiegel ausgewertet und in Relation zu den Zahlen des \gls{luh-repo}s gesetzt.
Hiermit wird eine Nutzungsrate für die Fakultäten über die einzelnen Zeitgruppen hergestellt.
Fakultäten, deren Verhalten sich ähneln, werden in entsprechende Gruppen zusammengefasst.
In \cref{sec:luh-repo-results-factors} werden mögliche Faktoren für unterschiedliche Produktionsraten und -arten für \glspl{forschungsdaten} identifiziert und auf Unabhängigkeit voneinander überprüft.
In \cref{sec:luh-repo-results-pd} wird die manuelle Klassifikation der Dissertationen genutzt, um zu beurteilen, wie viele der Dissertationen überhaupt \glspl{pd} produziert haben.
Dieser Wert wird für die Auswertung des \gls{forschungsdaten}-Verhaltens in den darauffolgenden Abschnitten genutzt.
In \cref{sec:luh-repo-results-time} werden die \gls{forschungsdaten}-Klassifikationsresultate in einen zeitlichen Kontext dargestellt und gezeigt, inwiefern sich diese über die verschiedenen Zeitgruppen verhalten.
In \cref{sec:luh-repo-results-language} wird der etwaige Faktor \textit{Sprache} untersucht und in Relation zu den Fakultäten sowie den einzelnen Zeitgruppen gebracht.
In \cref{sec:luh-repo-results-faculties} werden die Resultate der Klassifikationsarbeit im Kontext der Fakultäten dargestellt.
Hier wird untersucht, ob und, falls ja, inwiefern sich  die Fakultäten in Bezeug auf \gls{forschungsdaten}-Verhalten unterscheiden.
In \cref{sec:luh-repo-results-external-metadata} werden die Ergebnisse zu den Metadaten gezeigt, die bei den gefundenen externen \glspl{forschungsdaten} betrachtet wurden.

\subsection{Nutzungsrate des LUH-Repos}\label{sec:luh-repo-results-zahlenspiegel}
\parsum{Zahlenspiegel}
Die Auswertung des Zahlenspiegels und der Grundmenge dieser Arbeit ergab, dass nur \SI[round-mode=places,round-precision=2]{45,23}{\percent} ($n=\num{1898},\Delta=\num{-2298}$) aller Dissertationen im Zeitraum von 2012--2023 im \gls{luh-repo} Erst- oder Zweitveröffentlich worden sind.
Der relative Anteil und die absolute Differenz zwischen der Grundmenge und den Werten aus den Zahlenspiegel der \gls{luh} nach \textit{Zeitgruppe} und \textit{Fakultät} aufgegliedert sind in \cref{tab:luh-repo-classification-realrd} gegeben.
\begin{table}[!htbp]
	\caption{Der Anteil der Grundmenge nach $\text{\textit{Fakultät}}\times\text{\textit{Zeitraum}}$ aufgegliedert relativ zu der respektiven $\text{\textit{Fakultät}}\times\text{\textit{Zeitgruppe}}$-Gesamtanzahl aller publizierten Dissertationen.
    Absolute Differenzwerte in Klammern angegeben.
    Spalten, die zumindest teilweise auf simulierten Werten basieren, sind mit einem Asterisk (*) markiert.}
    \resizebox{\ifdim\width>\textwidth\textwidth\else\width\fi}{!}{%
	\begin{tabular}{lS[table-format=3.2]@{\,}S[table-text-alignment = left]lS[table-format=3.2]@{\,}S[table-text-alignment = left]lS[table-format=3.2]@{\,}S[table-text-alignment = left]lS[table-format=3.2]@{\,}S[table-text-alignment = left]l}
		\toprule
		& \multicolumn{3}{c}{\textbf{2012--2015}} & \multicolumn{3}{c}{\textbf{2016--2019}} & \multicolumn{3}{c}{\textbf{2020--2023*}} & \multicolumn{3}{c}{\textbf{Summe*}}    \\
		\midrule
		\textbf{\gls{fakultät2}}  & 32,50  & \si{\percent} & (-27)  & 67,65  & \si{\percent} & (-11)  & 100,00  & \si{\percent} & (0) & 62,00   & \si{\percent} & (-38)         \\
		\textbf{\gls{fakultät3}}  & 22,62  & \si{\percent} & (-65)  & 28,10  & \si{\percent} & (-87)  & 85,19  & \si{\percent} & (-12)  & 42,66    & \si{\percent} & (-164)          \\
		\textbf{\gls{fakultät4}}  & 27,27  & \si{\percent} & (-96) & 35,97  & \si{\percent} & (-89)  & 76,19  & \si{\percent} & (-25)  & 44,15   & \si{\percent} & (-210)          \\
		\textbf{\gls{fakultät5}}  & 0,00  & \si{\percent} & (-101) & 4,69  & \si{\percent} & (-61)   & 1,56  & \si{\percent} & (-63)  & 1,75    & \si{\percent} & (-225)           \\
		\textbf{\gls{fakultät6}}  & 14,55  & \si{\percent} & (-188) & 14,02  & \si{\percent} & (-227)  & 30,56  & \si{\percent} & (-150)  & 19,29    & \si{\percent} & (-565)           \\
		\textbf{\gls{fakultät7}}  & 52,06  & \si{\percent} & (-93) & 54,49  & \si{\percent} & (-71)  & 100,00  & \si{\percent} & (0)  & 65,18    & \si{\percent} & (-164)           \\
		\textbf{\gls{fakultät8}}  & 65,36  & \si{\percent} & (-133) & 60,87  & \si{\percent} & (-144) & 100,00 & \si{\percent} & (0)  & 73,34   & \si{\percent} & (-277)           \\
		\textbf{\gls{fakultät9}}  & 26,76  & \si{\percent} & (-104) & 23,42  & \si{\percent} & (-121)  & 63,37  & \si{\percent} & (-37)  & 34,66    & \si{\percent} & (-262)           \\
		\textbf{\gls{fakultät10}} & 56,07  & \si{\percent} & (-47) & 55,17  & \si{\percent} & (-52)  & 100,00  & \si{\percent} & (0)  & 67,00    & \si{\percent} & (-99)           \\
		\midrule
		\textbf{Summe}            & 39,17 & \si{\percent} & (-854) & 39,23 & \si{\percent} & (-863) & 73,38 & \si{\percent} & (1078) & 48,64  & \si{\percent} & (-2004)         \\
		\bottomrule
	\end{tabular}
}
    \label{tab:luh-repo-zahlenspiegel-relative-grundmenge}
\end{table}
Mit einem arithmetische Mittel aller $\text{\textit{Fakultät}}\times\text{\textit{Zeitgruppe}}$-Kombinationen von $\bar{x}=\SI[round-mode=places,round-precision=2]{42.9875078330178}{\percent}$ und einer Standardabweichung von $s=\SI[round-mode=places,round-precision=2]{26.1457705653948}{\percent}$ ist die relative Nutzung des \gls{luh-repo} für die Veröffentlichung von Dissertationen unter den Fakultäten und Jahresgruppen sehr ungleich.

Die Fakultäten lassen sich hierbei durch durchschnittlicher Nutzungsrate in drei relativ klar abgegrenzte Gruppen einteilen: Geringnutzer ($\bar{x}<\SI[round-mode=places,round-precision=2]{20}{\percent}$), Intermediärnutzer ($\bar{x}\approx\SI[round-mode=places,round-precision=2]{33}{\percent}$) und Intensivnutzer ($\bar{x}\approx\SI[round-mode=places,round-precision=2]{66}{\percent}$).
Die Geringnutzer bestehen aus \gls{fakultät5} ($\bar{x}=\SI[round-mode=places,round-precision=2]{1.9546568627451}{\percent},s=\SI[round-mode=places,round-precision=2]{2.43871779238119}{\percent}$) und \gls{fakultät6} ($\bar{x}=\SI[round-mode=places,round-precision=2]{17.239500265816}{\percent},s=\SI[round-mode=places,round-precision=2]{5.13233379292414}{\percent}$).
Die Intermediärnutzer bestehen aus \gls{fakultät3} ($\bar{x}=\SI[round-mode=places,round-precision=2]{36.5641933823752}{\percent},s=\SI[round-mode=places,round-precision=2]{19.600244551027}{\percent}$), \gls{fakultät4} ($\bar{x}=\SI[round-mode=places,round-precision=2]{39.4721213624712}{\percent},s=\SI[round-mode=places,round-precision=2]{14.2755155381073}{\percent}$) und \gls{fakultät9} ($\bar{x}=\SI[round-mode=places,round-precision=2]{32.709756719589}{\percent},s=\SI[round-mode=places,round-precision=2]{13.1540563313131}{\percent}$).
Die Intensivnutzer bestehen aus \gls{fakultät2} ($\bar{x}=\SI[round-mode=places,round-precision=2]{60.4656862745098}{\percent},s=\SI[round-mode=places,round-precision=2]{25.1559080290468}{\percent}$), \gls{fakultät7} ($\bar{x}=\SI[round-mode=places,round-precision=2]{62.2322031039136}{\percent},s=\SI[round-mode=places,round-precision=2]{15.0749409965671}{\percent}$), \gls{fakultät8} ($\bar{x}=\SI[round-mode=places,round-precision=2]{70.8348384042295}{\percent},s=\SI[round-mode=places,round-precision=2]{13.2930667691554}{\percent}$) und \gls{fakultät10} ($\bar{x}=\SI[round-mode=places,round-precision=2]{65.4146141215107}{\percent},s=\SI[round-mode=places,round-precision=2]{19.1432492140529}{\percent}$).

Die durchschnittliche Nutzungsrate durch die Fakultäten hat sich 2012--2023* um \SI[round-mode=places,round-precision=2]{86.8618178445906}{\percent} ($s=\SI[round-mode=places,round-precision=2]{47.0570062551138}{\percent}$) erhöht.
So lag 2012--2015 für \SI[round-mode=places,round-precision=2]{66,6666666666667}{\percent} ($n=\num{6}$) der Fakultäten die Nutzungsrate noch unter \SI[round-mode=places,round-precision=2]{50}{\percent}, während in 2020--2023 dies nur noch für \SI[round-mode=places,round-precision=2]{33,333333333}{\percent} ($n=\num{3}$) der Fall.
Hierbei ist zusätzlich noch anzumerken, dass eine dieser Fakultäten---\gls{fakultät9}---nur \SI[round-mode=places,round-precision=2]{2.2388059701493}{\percent P} von \SI[round-mode=places,round-precision=2]{50}{\percent} entfernt war.

\subsection{Mögliche relevante Faktoren}\label{sec:luh-repo-results-factors}
\parsum{Mögliche Faktoren}
Die Untersuchung der Metadaten-Datenbank auf mögliche Faktoren, die Einfluss auf die Existenz von \glspl{forschungsdaten} haben könnten ergab, zusätzlich zu \textit{Zeitgruppe} und \textit{Fakultät}, nur noch \textit{Sprache}.

\parsum{Unabhängigkeit der Faktoren}
Zur Überprüfung, ob die zu untersuchenden Faktoren von einander abhängig sind, wurden für die Kreuzprodukte aller Faktorenkombinationen Chi-Quadrat Tests der Unabhängigkeit durchgeführt.
Hierbei zeigte sich, dass $\text{\textit{Zeitgruppe}}\times\text{\textit{Fakultät}}$ ($\chi^2 (\num{16}, n=\num{1441}) = \num[round-mode=places,round-precision=2]{30.11595}$, $p = \num[round-mode=places,round-precision=2]{0.01741020},\phi_C=\num[round-mode=places,round-precision=2]{0.10222362}$), $\text{\textit{Zeitgruppe}}\times\text{\textit{Sprache}}$ ($\chi^2 (\num{6}, n=\num{1441}) = \num[round-mode=places,round-precision=2]{81.2042334}$, $p = \num[round-mode=places,round-precision=2]{2.014543e-15}, \phi_C=\num[round-mode=places,round-precision=2]{0.16785812}$) und $\text{\textit{Fakultät}}\times\text{\textit{Sprache}}$ ($\chi^2 (\num{24}, n=\num{1441}) = \num[round-mode=places,round-precision=2]{239.3091384}$, $p = \num[round-mode=places,round-precision=2]{2.148979e-37},\phi_C=\num[round-mode=places,round-precision=2]{0.23528109}$) alle statistisch signifikant voneinander abhängig sind.
Die Effektstärken sind dabei leicht unterschiedlich ausgeprägt aber konsistent schwacher Natur.

\subsection{Rate an erzeugten Primärdaten}\label{sec:luh-repo-results-pd}
\parsum{Produktion von \glspl{pd}}
Die Evaluation aller Stichproben-Einträge ergab, dass nur \SI[round-mode=places,round-precision=2]{86,81}{\percent} ($n=\num{1252}$) \glspl{pdd} sind.
Die relative und absolute Verteilung nach \textit{Zeitgruppe} und \textit{Fakultät} ist hierfür in \cref{tab:luh-repo-classification-realrd} gegeben.
\begin{table}[!htbp]
	\caption{Anteil an Dissertationen aus der Stichprobe, die \glspl{pd} produziert haben müssten, relativ zu der respektiven $\text{\textit{Fakultät}}\times\text{\textit{Zeitgruppe}}$-Gesamtanzahl.
    Absolute Werte in Klammern angegeben.}
    \resizebox{\ifdim\width>\textwidth\textwidth\else\width\fi}{!}{%
	\begin{tabular}{lS[table-format=3.2]@{\,}S[table-text-alignment = left]lS[table-format=3.2]@{\,}S[table-text-alignment = left]lS[table-format=3.2]@{\,}S[table-text-alignment = left]lS[table-format=3.2]@{\,}S[table-text-alignment = left]l}
		\toprule
		& \multicolumn{3}{c}{\textbf{2012--2015}} & \multicolumn{3}{c}{\textbf{2016--2019}} & \multicolumn{3}{c}{\textbf{2020--2023}} & \multicolumn{3}{c}{\textbf{Alle}}    \\
		\midrule
		\textbf{\gls{fakultät2}}  & 92,31                   & \si{\percent} & (12)  & 81,82  & \si{\percent} & (18)  & 80,00   & \si{\percent} & (20) & 83,33    & \si{\percent} & (50)           \\
		\textbf{\gls{fakultät3}}  & 94,74                   & \si{\percent} & (18)  & 90,63  & \si{\percent} & (29)  & 98,31   & \si{\percent} & (58) & 95,45    & \si{\percent} & (105)          \\
		\textbf{\gls{fakultät4}}  & 96,97                   & \si{\percent} & (32)  & 91,11  & \si{\percent} & (41)  & 100,00  & \si{\percent} & (67) & 96,55    & \si{\percent} & (140)          \\
		\textbf{\gls{fakultät5}}  & \multicolumn{2}{c}{---}                 & (0)   & 0,00   & \si{\percent} & (0)   & 0,00    & \si{\percent} & (0)  & 0,00     & \si{\percent} & (0)            \\
		\textbf{\gls{fakultät6}}  & 96,67                   & \si{\percent} & (29)  & 100,00 & \si{\percent} & (34)  & 100,00  & \si{\percent} & (57) & 99,17    & \si{\percent} & (120)          \\
		\textbf{\gls{fakultät7}}  & 79,01                   & \si{\percent} & (64)  & 81,43  & \si{\percent} & (57)  & 100,00  & \si{\percent} & (93) & 87,70    & \si{\percent} & (214)          \\
		\textbf{\gls{fakultät8}}  & 100,00                  & \si{\percent} & (153) & 98,59  & \si{\percent} & (140) & 98,18   & \si{\percent} & (162)& 98,91    & \si{\percent} & (455)          \\
		\textbf{\gls{fakultät9}}  & 68,57                   & \si{\percent} & (24)  & 58,82  & \si{\percent} & (20)  & 49,09   & \si{\percent} & (27) & 57,26    & \si{\percent} & (71)           \\
		\textbf{\gls{fakultät10}} & 52,83                   & \si{\percent} & (28)  & 63,64  & \si{\percent} & (35)  & 50,77   & \si{\percent} & (33) & 55,49    & \si{\percent} & (96)           \\
		\midrule
		\textbf{Alle}            & 86,33                   & \si{\percent} & (360) & 85,58  & \si{\percent} & (374) & 88,07   & \si{\percent} & (517) & 86,81   & \si{\percent} & (1251)         \\
		\bottomrule
	\end{tabular}
}
    \label{tab:luh-repo-classification-realrd}
\end{table}
Mit $\bar{x}=\SI[round-mode=places,round-precision=2]{77.9798542487882}{\percent},s=\SI[round-mode=places,round-precision=2]{28.490769364902}{\percent}$ für die gesamte $\text{\textit{Fakultät}}\times\text{\textit{Zeitgruppe}}$-Verteilung ist die Distribution von \glspl{pdd} im \gls{luh-repo} unter den Fakultäten und Jahresgruppen relativ ungleich.

Die Fakultäten lassen sich auch hier durch ihre durchschnittliche Anzahl an \glspl{pdd} in drei relativ klar abgegrenzte Gruppen einteilen: Nicht-Erzeuger ($\bar{x}\approx\SI[round-mode=places,round-precision=2]{0}{\percent}$), Intermediärerzeuger ($\bar{x}\approx\SI[round-mode=places,round-precision=2]{50}{\percent}$) und Intensiverzeuger ($\bar{x}\approx\SI[round-mode=places,round-precision=2]{90}{\percent}$).
Die Nicht-Erzeuger bestehen nur aus \gls{fakultät5} ($\bar{x}=\SI[round-mode=places,round-precision=2]{0.00}{\percent},s=\SI[round-mode=places,round-precision=2]{0.00}{\percent}$).
Die Intermediärerzeuger bestehen aus \gls{fakultät9} ($\bar{x}=\SI[round-mode=places,round-precision=2]{58.8286223580341}{\percent},s=\SI[round-mode=places,round-precision=2]{9.74026073887661}{\percent}$) und \gls{fakultät10} ($\bar{x}=\SI[round-mode=places,round-precision=2]{55.7452610282799}{\percent},s=\SI[round-mode=places,round-precision=2]{6.91115129013917}{\percent}$).
Die Intensiverzeuger bestehen aus \gls{fakultät2} ($\bar{x}=\SI[round-mode=places,round-precision=2]{86.0419580419581}{\percent},s=\SI[round-mode=places,round-precision=2]{5.53485790795261}{\percent}$), \gls{fakultät3} ($\bar{x}=\SI[round-mode=places,round-precision=2]{94.5556422836753}{\percent},s=\SI[round-mode=places,round-precision=2]{3.84324738431341}{\percent}$), \gls{fakultät4} ($\bar{x}=\SI[round-mode=places,round-precision=2]{96.026936026936}{\percent},s=\SI[round-mode=places,round-precision=2]{4.51881456425795}{\percent}$), \gls{fakultät6} ($\bar{x}=\SI[round-mode=places,round-precision=2]{98.8888888888889}{\percent},s=\SI[round-mode=places,round-precision=2]{1.92450089729875}{\percent}$), \gls{fakultät7} ($\bar{x}=\SI[round-mode=places,round-precision=2]{86.8136390358613}{\percent},s=\SI[round-mode=places,round-precision=2]{11.4834499748825}{\percent}$) und \gls{fakultät8} ($\bar{x}=\SI[round-mode=places,round-precision=2]{98.9244558258643}{\percent},s=\SI[round-mode=places,round-precision=2]{0.953711879617725}{\percent}$).

\subsection{Zeitliche Entwicklung der Forschungsdaten}\label{sec:luh-repo-results-time}
\parsum{Klassifikation Zeitgruppen}
Von den \num{1252} \glspl{pdd} haben \SI[round-mode=places,round-precision=2]{61.3418530351}{\percent} ($n=\num{768}$) zumindest Teile dieser \glspl{pd} veröffentlicht.
Bei einer Beschränkung auf \glspl{forschungsdaten} von \textit{Stufe~1} und \textit{Stufe~2}, haben nur \SI[round-mode=places,round-precision=2]{31.5495207668}{\percent} ($n=\num{395}$) zumindest Teile ihrer \glspl{pd} veröffentlicht.
Von allen \glspl{pdd} waren insgesamt nur als \textit{Stufe~1} \SI[round-mode=places,round-precision=2]{20.1277955271566}{\percent} ($n=252$) klassifiziert.
Die relative und absolute Verteilung der Klassifikationsstufen ist für $\text{\textit{\gls{forschungsdaten}-Publikationsart}}\times\text{\textit{Zeitgruppe}}\times\text{\textit{Klassifikationsstufe}}$ von \glspl{pdd} in \cref{tab:luh-repo-classification-general-publication-adjusted} gegeben.
\begin{table}[!htbp]
	\caption{\gls{forschungsdaten}-Klassifizierung der \glspl{pdd} aus der Stichprobe nach $\text{\textit{Publikationsart}}\times\text{\textit{Klassifikationsstufe}}\times\text{\textit{Jahresgruppe}}$ aufgegliedert.
    Angaben relativ zu der Gesamtanzahl der Jahresgruppe.
    Absolute Werte in Klammern angegeben.}
    \resizebox{\ifdim\width>\textwidth\textwidth\else\width\fi}{!}{%
	\begin{tabular}{clS[table-format=3.2]@{\,}S[table-text-alignment = left]lS[table-format=3.2]@{\,}S[table-text-alignment = left]lS[table-format=3.2]@{\,}S[table-text-alignment = left]lS[table-format=3.2]@{\,}S[table-text-alignment = left]lS[table-format=3.2]@{\,}S[table-text-alignment = left]l}
		\toprule
		& & \multicolumn{3}{c}{\textbf{2012-2015}} & \multicolumn{3}{c}{\textbf{2016-2019}} & \multicolumn{3}{c}{\textbf{2020-2023}} & \multicolumn{3}{c}{\textbf{Alle}}  \\
		\midrule
		\parbox[t]{2mm}{\multirow{4}{*}{\rotatebox[origin=c]{90}{\textbf{Intern}}}}  & \textbf{Stufe 1} & 13,89 & \si{\percent} & (50)  & 14,44  & \si{\percent} & (54)  & 10,62  & \si{\percent} & (55)  & 12,70            & \si{\percent} & (159)\\
		                                                                             & \textbf{Stufe 2} & 13,33 & \si{\percent} & (48)  & 14,44  & \si{\percent} & (54)  & 13,13 & \si{\percent} & (68)  & 13,58            & \si{\percent} & (170)\\
		                                                                             & \textbf{Stufe 3} & 36,94 & \si{\percent} & (133) & 33,96  & \si{\percent} & (127) & 24,13 & \si{\percent} & (125) & 30,75            & \si{\percent} & (385)\\
		                                                                             & \textbf{Keine}   & 35,83 & \si{\percent} & (129) & 37,17  & \si{\percent} & (139) & 52,12 & \si{\percent} & (270) & 42,97            & \si{\percent} & (538)\\
        \midrule
		\parbox[t]{2mm}{\multirow{4}{*}{\rotatebox[origin=c]{90}{\textbf{Beilage}}}} & \textbf{Stufe 1} & 1,39  & \si{\percent} & (5)   & 2,67   & \si{\percent} & (10)  & 1,35  & \si{\percent} & (7)   & 1,76           & \si{\percent} & (22)\\
		                                                                             & \textbf{Stufe 2} & 0,00  & \si{\percent} & (0)   & 0,00   & \si{\percent} & (0)   & 0,00  & \si{\percent} & (0)   & 0,00            & \si{\percent} & (0)\\
		                                                                             & \textbf{Stufe 3} & 0,56  & \si{\percent} & (2)   & 0,27   & \si{\percent} & (1)   & 0,58  & \si{\percent} & (3)   & 0,48            & \si{\percent} & (6)\\
                                                                                     & \textbf{Keine}   & 98,06 & \si{\percent} & (353) & 97,06  & \si{\percent} & (363) & 98,07 & \si{\percent} & (508) & 97,76            & \si{\percent} & (1224)\\
        \midrule
		\parbox[t]{2mm}{\multirow{4}{*}{\rotatebox[origin=c]{90}{\textbf{Extern}}}}  & \textbf{Stufe 1} & 1,39  & \si{\percent} & (4)   & 3,21   & \si{\percent} & (12)  & 14,29 & \si{\percent} & (74)  & 7,19            & \si{\percent} & (90)\\
		                                                                             & \textbf{Stufe 2} & 0,00  & \si{\percent} & (0)   & 0,00   & \si{\percent} & (0)   & 0,19  & \si{\percent} & (1)   & 0,08            & \si{\percent} & (1)\\
		                                                                             & \textbf{Stufe 3} & 0,56  & \si{\percent} & (0)   & 0,27   & \si{\percent} & (1)   & 0,19  & \si{\percent} & (1)   & 0,16            & \si{\percent} & (2)\\
                                                                                     & \textbf{Keine}   & 98,06 & \si{\percent} & (356) & 96,52  & \si{\percent} & (361) & 85,33  & \si{\percent} & (442) & 92,57            & \si{\percent} & (1159)\\
        \midrule
        \parbox[t]{2mm}{\multirow{4}{*}{\rotatebox[origin=c]{90}{\textbf{Alle}}}}    & \textbf{Stufe 1} & 15,28 & \si{\percent} & (55)  & 19,79  & \si{\percent} & (74)  & 23,75 & \si{\percent} & (123) & 20,13            & \si{\percent} & (252)\\
                                                                                     & \textbf{Stufe 2} & 13,33 & \si{\percent} & (48)  & 12,30  & \si{\percent} & (46)  & 9,46  & \si{\percent} & (49)  & 11,42            & \si{\percent} & (143)\\
                                                                                     & \textbf{Stufe 3} & 36,67 & \si{\percent} & (132) & 33,42  & \si{\percent} & (125) & 22,39 & \si{\percent} & (116) & 29,79            & \si{\percent} & (373)\\
                                                                                     & \textbf{Keine}   & 34,72 & \si{\percent} & (125) & 34,49  & \si{\percent} & (129) & 44,40 & \si{\percent} & (230) & 38,66            & \si{\percent} & (484)\\
		\bottomrule
	\end{tabular}
}
    \label{tab:luh-repo-classification-general-publication-adjusted}
\end{table}
Dieselbe Verteilung für alle Dissertationen ist in \cref{tab:luh-repo-classification-general-publication} gegeben.

Für \glspl{pdd} sind die respektiven Anteile pro Zeitgruppe für \textit{Stufe~1} ($\bar{x}=\SI[round-mode=places,round-precision=2]{19.6030159265453}{\percent},s=\SI[round-mode=places,round-precision=2]{4.23666583797109}{\percent}$), \textit{Stufe~2} ($\bar{x}=\SI[round-mode=places,round-precision=2]{11.6974193444782}{\percent},s=\SI[round-mode=places,round-precision=2]{2.00588363159137}{\percent}$), \textit{Stufe~3} ($\bar{x}=\SI[round-mode=places,round-precision=2]{30.8276496511791}{\percent},s=\SI[round-mode=places,round-precision=2]{7.48186481117491}{\percent}$) und \textit{Keine} ($\bar{x}=\SI[round-mode=places,round-precision=2]{37.8719150777974}{\percent},s=\SI[round-mode=places,round-precision=2]{5.65599658410767}{\percent}$) alle sehr konsistent in ihren Ausmaßen und haben sich über die Jahre nur wenig verändert:
Der Anteil von \textit{Keine \glspl{forschungsdaten}} und \textit{Stufe~1} stieg über die Zeitgruppen hinweg fast stetig an, während der Anteil an \textit{Stufe~2} und \textit{Stufe 3} abnahm.
Diese Interaktion ist mit $\chi^2 (\num{6}, n=\num{1252}) = \num[round-mode=places,round-precision=2]{35.1706024082481}$, $p = \num[round-mode=places,round-precision=2]{3.99360804013405E-06}<\num{0.001},\phi_C=\num[round-mode=places,round-precision=2]{0.118514841833938}$ hochsignifikant aber schwach im Effekt.

Für integriert publizierte \glspl{forschungsdaten} sind die respektiven Anteile pro Zeitgruppe für \textit{Stufe~1} ($\bar{x}=\SI[round-mode=places,round-precision=2]{12.9817173934821}{\percent},s=\SI[round-mode=places,round-precision=2]{2.06560827754195}{\percent}$), \textit{Stufe~2} ($\bar{x}=\SI[round-mode=places,round-precision=2]{13.6330830448477}{\percent},s=\SI[round-mode=places,round-precision=2]{0.705071887699306}{\percent}$), \textit{Stufe~3} ($\bar{x}=\SI[round-mode=places,round-precision=2]{31.6776459423518}{\percent},s=\SI[round-mode=places,round-precision=2]{6.70385517608004}{\percent}$) und \textit{Keine} ($\bar{x}=\SI[round-mode=places,round-precision=2]{41.7075536193183}{\percent},s=\SI[round-mode=places,round-precision=2]{9.04508811986477}{\percent}$):
An den entsprechenden Standardabweichungen lässt sich erkennen, dass sich die Anteile über die Zeitgruppen hinweg geändert haben.
Die Anteile von \textit{Keine \glspl{forschungsdaten}}, \textit{Stufe~1} und \textit{Stufe~3} waren über die ersten zwei Zeitgruppen stabil, stiegen dann aber für \textit{Keine \glspl{forschungsdaten}} stark an, während \textit{Stufe~1} und \textit{Stufe~3} stark abnahmen.
\textit{Stufe~2} verblieb über alle Jahresgruppen hinweg stabil.
Diese Interaktion ist mit $\chi^2 (\num{6}, n=\num{1252}) = \num[round-mode=places,round-precision=2]{33.7783606969378}$, $p = \num[round-mode=places,round-precision=2]{7.42382998991651E-06}<\num{0.001},\phi_C=\num[round-mode=places,round-precision=2]{0.116145428931647}$ hochsignifikant aber schwach im Effekt.

Für begleitende \glspl{forschungsdaten} verändern sich die respektiven Anteile pro Zeitgruppe für \textit{Stufe~1} ($\bar{x}=\SI[round-mode=places,round-precision=2]{1.80467901056136}{\percent},s=\SI[round-mode=places,round-precision=2]{0.75291204962178}{\percent}$), \textit{Stufe~2} ($\bar{x}=\SI[round-mode=places,round-precision=2]{0,00}{\percent},s=\SI[round-mode=places,round-precision=2]{0.00}{\percent}$), \textit{Stufe~3} ($\bar{x}=\SI[round-mode=places,round-precision=2]{0.467361937950173}{\percent},s=\SI[round-mode=places,round-precision=2]{0.173591068786419}{\percent}$) und \textit{Keine} ($\bar{x}=\SI[round-mode=places,round-precision=2]{97.7279590514885}{\percent},s=\SI[round-mode=places,round-precision=2]{0.579530291358788}{\percent}$) fast überhaupt nicht.
Entsprechend ist diese Interaktion mit $\chi^2 (\num{4}, n=\num{1252}) = \num[round-mode=places,round-precision=2]{3.08067913427959}$, $p = \num[round-mode=places,round-precision=2]{0.54441589726215}>\num{0.05}$ auch nicht signifikant.

Für externe \glspl{forschungsdaten} sind die respektiven Anteile pro Zeitgruppe für \textit{Stufe~2} und \textit{Stufe~3} mit Maximalwerten von unter $\SI{0,60}{\percent}$ und $s<\SI{0,15}{\percent}$ vernachlässigbar.
Für \textit{Stufe~1} ($\bar{x}=\SI[round-mode=places,round-precision=2]{6.20179384885267}{\percent},s=\SI[round-mode=places,round-precision=2]{7.07899330022576}{\percent}$) und \textit{Keine} ($\bar{x}=\SI[round-mode=places,round-precision=2]{0.935803794627324}{\percent},s=\SI[round-mode=places,round-precision=2]{7.24376490591247}{\percent}$) zeichnet sich jedoch auch eine eindeutige Veränderung über die Zeitgruppen hinweg ab:
Während \textit{Keine \glspl{forschungsdaten}} über die Zeitgruppen hinweg um $\SI[round-mode=places,round-precision=2]{13.56}{\percent P}$ abnimmt, steigt \textit{Stufe~1} um $\SI[round-mode=places,round-precision=2]{13.18}{\percent P}$ auf einen Maximalwert von $\SI[round-mode=places,round-precision=2]{14.29}{\percent}$ an.
Die verbleibenden $\SI[round-mode=places,round-precision=2]{0.38}{\percent P}$ teilten sich dabei gleichmäßig auf \textit{Stufe~2} und \textit{Stufe~3} mit jeweils einer \glspl{pdd} auf.
Diese Interaktion ist mit $\chi^2 (\num{6}, n=\num{1252}) = \num[round-mode=places,round-precision=2]{70.4530111884674}$, $p = \num[round-mode=places,round-precision=2]{3.3012369430121E-13}<\num{0.001},\phi_C=\num[round-mode=places,round-precision=2]{0.167738446925158}$ hochsignifikant aber schwach im Effekt.
Hier ist allerdings anzumerken, dass sowohl der Signifikanzwert wie auch die Einflusstärke dieser Interaktion höher bzw.~stärker ist als die Werte der anderen bisher untersuchten Interaktionen.

Zusammenfassend lassen sich hier folgende Punkte festhalten:
Der Anteil publizierter \glspl{forschungsdaten} der \textit{Stufe~1} insgesamt über die Zeitgruppen zugenommen hat, dafür weniger integriert und dafür vermehrt extern publiziert werden.
Die Option \glspl{forschungsdaten} begleitend zu publizieren, wird von einem sehr kleinen Nutzerkreis konsistent genutzt.

\subsection{Sprache und Forschungsdaten}\label{sec:luh-repo-results-language}
\parsum{Sprache}
Von allen Dissertationen wurden \SI[round-mode=places,round-precision=2]{54.55}{\percent} ($n=\num{786}$) auf Deutsch und \SI[round-mode=places,round-precision=2]{45,11}{\percent} ($n=\num{650}$) auf Englisch verfasst.
Die restlichen \SI[round-mode=places,round-precision=2]{0.35}{\percent} ($n=\num{5}$) wurden entweder auf Spanisch oder in mehreren Sprachen verfasst.
Für \glspl{pdd} ändern sich diese Anteile auf jeweils \SI[round-mode=places,round-precision=2]{55.9904153355}{\percent} ($n=\num{701}$), \SI[round-mode=places,round-precision=2]{43.6900958466}{\percent} ($n=\num{547}$) und \SI[round-mode=places,round-precision=2]{0.319488817891}{\percent} ($n=\num{4}$).
Für die relative und absolute Verteilung nach $\text{\textit{Sprache}}\times\text{\textit{Zeitgruppe}}$, siehe \cref{tab:luh-repo-sprache-zeitgruppe}.

Für \glspl{pdd} sind die respektiven Anteile pro Zeitgruppe für \textit{Deutsch} ($\bar{x}=\SI[round-mode=places,round-precision=2]{58.0207461089814}{\percent},s=\SI[round-mode=places,round-precision=2]{14.7111835680779}{\percent}$), \textit{Englisch} ($\bar{x}=\SI[round-mode=places,round-precision=2]{41.7218536336183}{\percent},s=\SI[round-mode=places,round-precision=2]{14.2781240348223}{\percent}$) und \textit{Andere} ($\bar{x}=\SI[round-mode=places,round-precision=2]{0.193050193050193}{\percent},s=\SI[round-mode=places,round-precision=2]{0.334372742773914}{\percent}$).
Die hohen Standardabweichungen für \textit{Englisch} und \textit{Deutsch} sind dem verschuldet, dass für die ersten zwei Zeitgruppen \textit{Deutsch} vergleichsweise konsistent mit $\bar{x}=\SI[round-mode=places,round-precision=2]{66.2782234105763}{\percent}$ und \textit{Englisch} mit $\bar{x}=\SI[round-mode=places,round-precision=2]{33.7217765894236}{\percent}$  vertreten war (jeweils $s=\SI[round-mode=places,round-precision=2]{4.87054982822638}{\percent}$), in der dritten Zeitgruppe dann aber ein Wechsel der Mehrheit von \textit{Deutsch} ($\SI[round-mode=places,round-precision=2]{41.5057915057915}{\percent}$) auf \textit{Englisch} ($\SI[round-mode=places,round-precision=2]{57.7220077220077}{\percent}$) stattgefunden hat.
Die Interaktion zwischen \textit{Sprache} und \textit{Zeitgruppe} ist mit $\chi^2 (\num{6}, n=\num{1252}) = \num[round-mode=places,round-precision=2]{81.2042333529611}$, $p = \num[round-mode=places,round-precision=2]{2.01454276715493E-15}<\num{0.01},\phi_C=\num[round-mode=places,round-precision=2]{0.16785811702362}$ statistisch hochsignifikant aber schwach im Effekt.

Auch die Interaktion zwischen \textit{Sprache} und \textit{Fakultät} ist mit $\chi^2 (\num{21}, n=\num{1252}) = \num[round-mode=places,round-precision=2]{201.42368756814}$, $p = \num[round-mode=places,round-precision=2]{1.90053471088499E-31}<\num{0.01},\phi_C=\num[round-mode=places,round-precision=2]{0.231575430181447}$ statistisch hochsignifikant aber schwach im Effekt.
Diese Interaktion ist durch die stark unterschiedliche Verteilung von Sprachwahl unter den Fakultäten bedingt.
So hat \textit{Deutsch} innerhalb der Fakultäten einen Anteil von $\bar{x}=\SI[round-mode=places,round-precision=2]{59.022318726589}{\percent},s=\SI[round-mode=places,round-precision=2]{23.4782123625139}{\percent}$ und \textit{Englisch} von $\bar{x}=\SI[round-mode=places,round-precision=2]{40.283956329479}{\percent},s=\SI[round-mode=places,round-precision=2]{24.3404020527007}{\percent}$ mit respektiven Maximalwerten von $\SI[round-mode=places,round-precision=2]{91.5492957746479}{\percent}$ und $\SI[round-mode=places,round-precision=2]{71.875}{\percent}$.
Zusätzlich dazu ändert sich die Sprachverteilung innerhalb der Fakultäten je nach \textit{Zeitgruppe} unterschiedlich stark, wie in \cref{fig:luh-repo_sprache_x_fakultät_x_zeitgruppe} visualisiert wird.
\begin{figure}[!htbp]
    \resizebox{\ifdim\width>\textwidth\textwidth\else\width\fi}{!}{\begin{tikzpicture}[y=1mm, x=1mm, yscale=\globalscale,xscale=\globalscale, every node/.append style={scale=\globalscale}, inner sep=0pt, outer sep=0pt]
  \path[fill=white,line cap=round,line join=round,miter limit=10.0] ;
  \path[draw=white,fill=white,line cap=round,line join=round,line   width=0.38mm,miter limit=10.0] (0.0, 177.8) rectangle (241.3, 0.0);
  \path[fill=cebebeb,line cap=round,line join=round,line width=0.38mm,miter   limit=10.0] (16.25, 154.91) rectangle (43.7, 27.45);
  \path[draw=white,line cap=butt,line join=round,line width=0.19mm,miter   limit=10.0] (16.25, 47.73) -- (43.7, 47.73);
  \path[draw=white,line cap=butt,line join=round,line width=0.19mm,miter   limit=10.0] (16.25, 76.69) -- (43.7, 76.69);
  \path[draw=white,line cap=butt,line join=round,line width=0.19mm,miter   limit=10.0] (16.25, 105.66) -- (43.7, 105.66);
  \path[draw=white,line cap=butt,line join=round,line width=0.19mm,miter   limit=10.0] (16.25, 134.63) -- (43.7, 134.63);
  \path[draw=white,line cap=butt,line join=round,line width=0.38mm,miter   limit=10.0] (16.25, 33.24) -- (43.7, 33.24);
  \path[draw=white,line cap=butt,line join=round,line width=0.38mm,miter   limit=10.0] (16.25, 62.21) -- (43.7, 62.21);
  \path[draw=white,line cap=butt,line join=round,line width=0.38mm,miter   limit=10.0] (16.25, 91.18) -- (43.7, 91.18);
  \path[draw=white,line cap=butt,line join=round,line width=0.38mm,miter   limit=10.0] (16.25, 120.15) -- (43.7, 120.15);
  \path[draw=white,line cap=butt,line join=round,line width=0.38mm,miter   limit=10.0] (16.25, 149.11) -- (43.7, 149.11);
  \path[draw=white,line cap=butt,line join=round,line width=0.38mm,miter   limit=10.0] (21.4, 27.45) -- (21.4, 154.91);
  \path[draw=white,line cap=butt,line join=round,line width=0.38mm,miter   limit=10.0] (29.97, 27.45) -- (29.97, 154.91);
  \path[draw=white,line cap=butt,line join=round,line width=0.38mm,miter   limit=10.0] (38.55, 27.45) -- (38.55, 154.91);
  \path[draw=black,fill=c77aadd,line cap=butt,line join=miter,line   width=0.38mm,miter limit=10.0] (17.53, 149.11) rectangle (25.26, 33.24);
  \path[fill=ceedd88,line cap=butt,line join=miter,line width=0.38mm,miter   limit=10.0] ;
  \path[fill=cee8866,line cap=butt,line join=miter,line width=0.38mm,miter   limit=10.0] ;
  \path[draw=black,fill=c77aadd,line cap=butt,line join=miter,line   width=0.38mm,miter limit=10.0] (26.11, 149.11) rectangle (33.83, 71.86);
  \path[draw=black,fill=ceedd88,line cap=butt,line join=miter,line   width=0.38mm,miter limit=10.0] (26.11, 71.87) rectangle (33.83, 33.25);
  \path[fill=cee8866,line cap=butt,line join=miter,line width=0.38mm,miter   limit=10.0] ;
  \path[draw=black,fill=c77aadd,line cap=butt,line join=miter,line   width=0.38mm,miter limit=10.0] (34.69, 149.11) rectangle (42.41, 49.79);
  \path[draw=black,fill=ceedd88,line cap=butt,line join=miter,line   width=0.38mm,miter limit=10.0] (34.69, 49.79) rectangle (42.41, 44.28);
  \path[draw=black,fill=cee8866,line cap=butt,line join=miter,line   width=0.38mm,miter limit=10.0] (34.69, 44.28) rectangle (42.41, 33.24);
  \node[anchor=south] (text24) at (21.4, 92.45){100};
  \node[anchor=south] (text25) at (21.4, 87.39){(12)};
  \node[anchor=south] (text26) at (29.97, 111.76){67};
  \node[anchor=south] (text27) at (29.97, 106.7){(12)};
  \node[anchor=south] (text28) at (29.97, 53.83){33};
  \node[anchor=south] (text29) at (29.97, 48.77){(6)};
  \node[anchor=south] (text30) at (38.55, 100.73){86};
  \node[anchor=south] (text31) at (38.55, 95.67){(18)};
  \node[anchor=south,shift={(0.0, 2.12)}] (text32) at (38.55, 48.31){5};
  \node[anchor=south,shift={(0.0, 2.65)}] (text33) at (38.55, 43.25){(1)};
  \node[anchor=south,shift={(0.0, -1.06)}] (text34) at (38.55, 40.03){10};
  \node[anchor=south] (text35) at (38.55, 34.97){(2)};
  \path[fill=cebebeb,line cap=round,line join=round,line width=0.38mm,miter   limit=10.0] (44.2, 154.91) rectangle (71.65, 27.45);
  \path[draw=white,line cap=butt,line join=round,line width=0.19mm,miter   limit=10.0] (44.2, 47.73) -- (71.65, 47.73);
  \path[draw=white,line cap=butt,line join=round,line width=0.19mm,miter   limit=10.0] (44.2, 76.69) -- (71.65, 76.69);
  \path[draw=white,line cap=butt,line join=round,line width=0.19mm,miter   limit=10.0] (44.2, 105.66) -- (71.65, 105.66);
  \path[draw=white,line cap=butt,line join=round,line width=0.19mm,miter   limit=10.0] (44.2, 134.63) -- (71.65, 134.63);
  \path[draw=white,line cap=butt,line join=round,line width=0.38mm,miter   limit=10.0] (44.2, 33.24) -- (71.65, 33.24);
  \path[draw=white,line cap=butt,line join=round,line width=0.38mm,miter   limit=10.0] (44.2, 62.21) -- (71.65, 62.21);
  \path[draw=white,line cap=butt,line join=round,line width=0.38mm,miter   limit=10.0] (44.2, 91.18) -- (71.65, 91.18);
  \path[draw=white,line cap=butt,line join=round,line width=0.38mm,miter   limit=10.0] (44.2, 120.15) -- (71.65, 120.15);
  \path[draw=white,line cap=butt,line join=round,line width=0.38mm,miter   limit=10.0] (44.2, 149.11) -- (71.65, 149.11);
  \path[draw=white,line cap=butt,line join=round,line width=0.38mm,miter   limit=10.0] (49.35, 27.45) -- (49.35, 154.91);
  \path[draw=white,line cap=butt,line join=round,line width=0.38mm,miter   limit=10.0] (57.93, 27.45) -- (57.93, 154.91);
  \path[draw=white,line cap=butt,line join=round,line width=0.38mm,miter   limit=10.0] (66.51, 27.45) -- (66.51, 154.91);
  \path[draw=black,fill=c77aadd,line cap=butt,line join=miter,line   width=0.38mm,miter limit=10.0] (45.49, 149.11) rectangle (53.21, 78.3);
  \path[draw=black,fill=ceedd88,line cap=butt,line join=miter,line   width=0.38mm,miter limit=10.0] (45.49, 78.3) rectangle (53.21, 33.24);
  \path[draw=black,fill=c77aadd,line cap=butt,line join=miter,line   width=0.38mm,miter limit=10.0] (54.07, 149.11) rectangle (61.79, 85.18);
  \path[draw=black,fill=ceedd88,line cap=butt,line join=miter,line   width=0.38mm,miter limit=10.0] (54.07, 85.19) rectangle (61.79, 33.24);
  \path[draw=black,fill=c77aadd,line cap=butt,line join=miter,line   width=0.38mm,miter limit=10.0] (62.64, 149.11) rectangle (70.37, 113.15);
  \path[draw=black,fill=ceedd88,line cap=butt,line join=miter,line   width=0.38mm,miter limit=10.0] (62.64, 113.15) rectangle (70.37, 33.24);
  \node[anchor=south] (text53) at (49.35, 114.98){61};
  \node[anchor=south] (text54) at (49.35, 109.93){(11)};
  \node[anchor=south] (text55) at (49.35, 57.05){39};
  \node[anchor=south] (text56) at (49.35, 51.99){(7)};
  \node[anchor=south] (text57) at (57.93, 118.42){55};
  \node[anchor=south] (text58) at (57.93, 113.37){(16)};
  \node[anchor=south] (text59) at (57.93, 60.49){45};
  \node[anchor=south] (text60) at (57.93, 55.43){(13)};
  \node[anchor=south] (text61) at (66.51, 132.41){31};
  \node[anchor=south] (text62) at (66.51, 127.35){(18)};
  \node[anchor=south] (text63) at (66.51, 74.47){69};
  \node[anchor=south] (text64) at (66.51, 69.41){(40)};
  \path[fill=cebebeb,line cap=round,line join=round,line width=0.38mm,miter   limit=10.0] (72.15, 154.91) rectangle (99.61, 27.45);
  \path[draw=white,line cap=butt,line join=round,line width=0.19mm,miter   limit=10.0] (72.15, 47.73) -- (99.6, 47.73);
  \path[draw=white,line cap=butt,line join=round,line width=0.19mm,miter   limit=10.0] (72.15, 76.69) -- (99.6, 76.69);
  \path[draw=white,line cap=butt,line join=round,line width=0.19mm,miter   limit=10.0] (72.15, 105.66) -- (99.6, 105.66);
  \path[draw=white,line cap=butt,line join=round,line width=0.19mm,miter   limit=10.0] (72.15, 134.63) -- (99.6, 134.63);
  \path[draw=white,line cap=butt,line join=round,line width=0.38mm,miter   limit=10.0] (72.15, 33.24) -- (99.6, 33.24);
  \path[draw=white,line cap=butt,line join=round,line width=0.38mm,miter   limit=10.0] (72.15, 62.21) -- (99.6, 62.21);
  \path[draw=white,line cap=butt,line join=round,line width=0.38mm,miter   limit=10.0] (72.15, 91.18) -- (99.6, 91.18);
  \path[draw=white,line cap=butt,line join=round,line width=0.38mm,miter   limit=10.0] (72.15, 120.15) -- (99.6, 120.15);
  \path[draw=white,line cap=butt,line join=round,line width=0.38mm,miter   limit=10.0] (72.15, 149.11) -- (99.6, 149.11);
  \path[draw=white,line cap=butt,line join=round,line width=0.38mm,miter   limit=10.0] (77.3, 27.45) -- (77.3, 154.91);
  \path[draw=white,line cap=butt,line join=round,line width=0.38mm,miter   limit=10.0] (85.88, 27.45) -- (85.88, 154.91);
  \path[draw=white,line cap=butt,line join=round,line width=0.38mm,miter   limit=10.0] (94.46, 27.45) -- (94.46, 154.91);
  \path[draw=black,fill=c77aadd,line cap=butt,line join=miter,line   width=0.38mm,miter limit=10.0] (73.44, 149.11) rectangle (81.16, 87.56);
  \path[draw=black,fill=ceedd88,line cap=butt,line join=miter,line   width=0.38mm,miter limit=10.0] (73.44, 87.56) rectangle (81.16, 33.24);
  \path[draw=black,fill=c77aadd,line cap=butt,line join=miter,line   width=0.38mm,miter limit=10.0] (82.02, 149.11) rectangle (89.74, 86.94);
  \path[draw=black,fill=ceedd88,line cap=butt,line join=miter,line   width=0.38mm,miter limit=10.0] (82.02, 86.94) rectangle (89.74, 33.24);
  \path[draw=black,fill=c77aadd,line cap=butt,line join=miter,line   width=0.38mm,miter limit=10.0] (90.6, 149.11) rectangle (98.32, 111.07);
  \path[draw=black,fill=ceedd88,line cap=butt,line join=miter,line   width=0.38mm,miter limit=10.0] (90.6, 111.07) rectangle (98.32, 33.24);
  \node[anchor=south] (text82) at (77.3, 119.61){53};
  \node[anchor=south] (text83) at (77.3, 114.55){(17)};
  \node[anchor=south] (text84) at (77.3, 61.68){47};
  \node[anchor=south] (text85) at (77.3, 56.62){(15)};
  \node[anchor=south] (text86) at (85.88, 119.3){54};
  \node[anchor=south] (text87) at (85.88, 114.24){(22)};
  \node[anchor=south] (text88) at (85.88, 61.37){46};
  \node[anchor=south] (text89) at (85.88, 56.31){(19)};
  \node[anchor=south] (text90) at (94.46, 131.36){33};
  \node[anchor=south] (text91) at (94.46, 126.31){(22)};
  \node[anchor=south] (text92) at (94.46, 73.43){67};
  \node[anchor=south] (text93) at (94.46, 68.37){(45)};
  \path[fill=cebebeb,line cap=round,line join=round,line width=0.38mm,miter   limit=10.0] (100.1, 154.91) rectangle (127.56, 27.45);
  \path[draw=white,line cap=butt,line join=round,line width=0.19mm,miter   limit=10.0] (100.1, 47.73) -- (127.56, 47.73);
  \path[draw=white,line cap=butt,line join=round,line width=0.19mm,miter   limit=10.0] (100.1, 76.69) -- (127.56, 76.69);
  \path[draw=white,line cap=butt,line join=round,line width=0.19mm,miter   limit=10.0] (100.1, 105.66) -- (127.56, 105.66);
  \path[draw=white,line cap=butt,line join=round,line width=0.19mm,miter   limit=10.0] (100.1, 134.63) -- (127.56, 134.63);
  \path[draw=white,line cap=butt,line join=round,line width=0.38mm,miter   limit=10.0] (100.1, 33.24) -- (127.56, 33.24);
  \path[draw=white,line cap=butt,line join=round,line width=0.38mm,miter   limit=10.0] (100.1, 62.21) -- (127.56, 62.21);
  \path[draw=white,line cap=butt,line join=round,line width=0.38mm,miter   limit=10.0] (100.1, 91.18) -- (127.56, 91.18);
  \path[draw=white,line cap=butt,line join=round,line width=0.38mm,miter   limit=10.0] (100.1, 120.15) -- (127.56, 120.15);
  \path[draw=white,line cap=butt,line join=round,line width=0.38mm,miter   limit=10.0] (100.1, 149.11) -- (127.56, 149.11);
  \path[draw=white,line cap=butt,line join=round,line width=0.38mm,miter   limit=10.0] (105.25, 27.45) -- (105.25, 154.91);
  \path[draw=white,line cap=butt,line join=round,line width=0.38mm,miter   limit=10.0] (113.83, 27.45) -- (113.83, 154.91);
  \path[draw=white,line cap=butt,line join=round,line width=0.38mm,miter   limit=10.0] (122.41, 27.45) -- (122.41, 154.91);
  \path[draw=black,fill=c77aadd,line cap=butt,line join=miter,line   width=0.38mm,miter limit=10.0] (101.39, 149.11) rectangle (109.11, 41.23);
  \path[draw=black,fill=ceedd88,line cap=butt,line join=miter,line   width=0.38mm,miter limit=10.0] (101.39, 41.23) rectangle (109.11, 33.24);
  \path[draw=black,fill=c77aadd,line cap=butt,line join=miter,line   width=0.38mm,miter limit=10.0] (109.97, 149.11) rectangle (117.69, 46.87);
  \path[draw=black,fill=ceedd88,line cap=butt,line join=miter,line   width=0.38mm,miter limit=10.0] (109.97, 46.87) rectangle (117.69, 33.24);
  \path[draw=black,fill=c77aadd,line cap=butt,line join=miter,line   width=0.38mm,miter limit=10.0] (118.55, 149.11) rectangle (126.27, 65.76);
  \path[draw=black,fill=ceedd88,line cap=butt,line join=miter,line   width=0.38mm,miter limit=10.0] (118.55, 65.77) rectangle (126.27, 33.24);
  \node[anchor=south] (text111) at (105.25, 96.45){93};
  \node[anchor=south] (text112) at (105.25, 91.39){(27)};
  \node[anchor=south] (text113) at (105.25, 38.51){7};
  \node[anchor=south,shift={(0.0, 1.06)}] (text114) at (105.25, 33.45){(2)};
  \node[anchor=south] (text115) at (113.83, 99.27){88};
  \node[anchor=south] (text116) at (113.83, 94.21){(30)};
  \node[anchor=south] (text117) at (113.83, 41.33){12};
  \node[anchor=south] (text118) at (113.83, 36.27){(4)};
  \node[anchor=south] (text119) at (122.41, 108.72){72};
  \node[anchor=south] (text120) at (122.41, 103.66){(41)};
  \node[anchor=south] (text121) at (122.41, 50.78){28};
  \node[anchor=south] (text122) at (122.41, 45.72){(16)};
  \path[fill=cebebeb,line cap=round,line join=round,line width=0.38mm,miter   limit=10.0] (128.06, 154.91) rectangle (155.51, 27.45);
  \path[draw=white,line cap=butt,line join=round,line width=0.19mm,miter   limit=10.0] (128.06, 47.73) -- (155.51, 47.73);
  \path[draw=white,line cap=butt,line join=round,line width=0.19mm,miter   limit=10.0] (128.06, 76.69) -- (155.51, 76.69);
  \path[draw=white,line cap=butt,line join=round,line width=0.19mm,miter   limit=10.0] (128.06, 105.66) -- (155.51, 105.66);
  \path[draw=white,line cap=butt,line join=round,line width=0.19mm,miter   limit=10.0] (128.06, 134.63) -- (155.51, 134.63);
  \path[draw=white,line cap=butt,line join=round,line width=0.38mm,miter   limit=10.0] (128.06, 33.24) -- (155.51, 33.24);
  \path[draw=white,line cap=butt,line join=round,line width=0.38mm,miter   limit=10.0] (128.06, 62.21) -- (155.51, 62.21);
  \path[draw=white,line cap=butt,line join=round,line width=0.38mm,miter   limit=10.0] (128.06, 91.18) -- (155.51, 91.18);
  \path[draw=white,line cap=butt,line join=round,line width=0.38mm,miter   limit=10.0] (128.06, 120.15) -- (155.51, 120.15);
  \path[draw=white,line cap=butt,line join=round,line width=0.38mm,miter   limit=10.0] (128.06, 149.11) -- (155.51, 149.11);
  \path[draw=white,line cap=butt,line join=round,line width=0.38mm,miter   limit=10.0] (133.21, 27.45) -- (133.21, 154.91);
  \path[draw=white,line cap=butt,line join=round,line width=0.38mm,miter   limit=10.0] (141.78, 27.45) -- (141.78, 154.91);
  \path[draw=white,line cap=butt,line join=round,line width=0.38mm,miter   limit=10.0] (150.36, 27.45) -- (150.36, 154.91);
  \path[draw=black,fill=c77aadd,line cap=butt,line join=miter,line   width=0.38mm,miter limit=10.0] (129.34, 149.11) rectangle (137.06, 80.31);
  \path[draw=black,fill=ceedd88,line cap=butt,line join=miter,line   width=0.38mm,miter limit=10.0] (129.34, 80.32) rectangle (137.06, 33.25);
  \path[draw=black,fill=c77aadd,line cap=butt,line join=miter,line   width=0.38mm,miter limit=10.0] (137.92, 149.11) rectangle (145.64, 90.16);
  \path[draw=black,fill=ceedd88,line cap=butt,line join=miter,line   width=0.38mm,miter limit=10.0] (137.92, 90.16) rectangle (145.64, 33.24);
  \path[draw=black,fill=c77aadd,line cap=butt,line join=miter,line   width=0.38mm,miter limit=10.0] (146.5, 149.11) rectangle (154.22, 120.46);
  \path[draw=black,fill=ceedd88,line cap=butt,line join=miter,line   width=0.38mm,miter limit=10.0] (146.5, 120.46) rectangle (154.22, 33.24);
  \node[anchor=south] (text140) at (133.21, 115.99){59};
  \node[anchor=south] (text141) at (133.21, 110.93){(38)};
  \node[anchor=south] (text142) at (133.21, 58.05){41};
  \node[anchor=south] (text143) at (133.21, 52.99){(26)};
  \node[anchor=south] (text144) at (141.78, 120.91){51};
  \node[anchor=south] (text145) at (141.78, 115.85){(29)};
  \node[anchor=south] (text146) at (141.78, 62.98){49};
  \node[anchor=south] (text147) at (141.78, 57.92){(28)};
  \node[anchor=south] (text148) at (150.36, 136.06){25};
  \node[anchor=south] (text149) at (150.36, 131.0){(23)};
  \node[anchor=south] (text150) at (150.36, 78.13){75};
  \node[anchor=south] (text151) at (150.36, 73.07){(70)};
  \path[fill=cebebeb,line cap=round,line join=round,line width=0.38mm,miter   limit=10.0] (156.01, 154.91) rectangle (183.46, 27.45);
  \path[draw=white,line cap=butt,line join=round,line width=0.19mm,miter   limit=10.0] (156.01, 47.73) -- (183.46, 47.73);
  \path[draw=white,line cap=butt,line join=round,line width=0.19mm,miter   limit=10.0] (156.01, 76.69) -- (183.46, 76.69);
  \path[draw=white,line cap=butt,line join=round,line width=0.19mm,miter   limit=10.0] (156.01, 105.66) -- (183.46, 105.66);
  \path[draw=white,line cap=butt,line join=round,line width=0.19mm,miter   limit=10.0] (156.01, 134.63) -- (183.46, 134.63);
  \path[draw=white,line cap=butt,line join=round,line width=0.38mm,miter   limit=10.0] (156.01, 33.24) -- (183.46, 33.24);
  \path[draw=white,line cap=butt,line join=round,line width=0.38mm,miter   limit=10.0] (156.01, 62.21) -- (183.46, 62.21);
  \path[draw=white,line cap=butt,line join=round,line width=0.38mm,miter   limit=10.0] (156.01, 91.18) -- (183.46, 91.18);
  \path[draw=white,line cap=butt,line join=round,line width=0.38mm,miter   limit=10.0] (156.01, 120.15) -- (183.46, 120.15);
  \path[draw=white,line cap=butt,line join=round,line width=0.38mm,miter   limit=10.0] (156.01, 149.11) -- (183.46, 149.11);
  \path[draw=white,line cap=butt,line join=round,line width=0.38mm,miter   limit=10.0] (161.16, 27.45) -- (161.16, 154.91);
  \path[draw=white,line cap=butt,line join=round,line width=0.38mm,miter   limit=10.0] (169.74, 27.45) -- (169.74, 154.91);
  \path[draw=white,line cap=butt,line join=round,line width=0.38mm,miter   limit=10.0] (178.32, 27.45) -- (178.32, 154.91);
  \path[draw=black,fill=c77aadd,line cap=butt,line join=miter,line   width=0.38mm,miter limit=10.0] (157.3, 149.11) rectangle (165.02, 62.78);
  \path[draw=black,fill=ceedd88,line cap=butt,line join=miter,line   width=0.38mm,miter limit=10.0] (157.3, 62.78) rectangle (165.02, 33.24);
  \path[fill=cee8866,line cap=butt,line join=miter,line width=0.38mm,miter   limit=10.0] ;
  \path[draw=black,fill=c77aadd,line cap=butt,line join=miter,line   width=0.38mm,miter limit=10.0] (165.88, 149.11) rectangle (173.6, 71.31);
  \path[draw=black,fill=ceedd88,line cap=butt,line join=miter,line   width=0.38mm,miter limit=10.0] (165.88, 71.31) rectangle (173.6, 33.24);
  \path[fill=cee8866,line cap=butt,line join=miter,line width=0.38mm,miter   limit=10.0] ;
  \path[draw=black,fill=c77aadd,line cap=butt,line join=miter,line   width=0.38mm,miter limit=10.0] (174.46, 149.11) rectangle (182.18, 102.62);
  \path[draw=black,fill=ceedd88,line cap=butt,line join=miter,line   width=0.38mm,miter limit=10.0] (174.46, 102.62) rectangle (182.18, 33.96);
  \path[draw=black,fill=cee8866,line cap=butt,line join=miter,line   width=0.38mm,miter limit=10.0] (174.46, 33.96) rectangle (182.18, 33.24);
  \node[anchor=south] (text172) at (161.16, 107.22){75};
  \node[anchor=south] (text173) at (161.16, 102.16){(114)};
  \node[anchor=south] (text174) at (161.16, 49.29){25};
  \node[anchor=south] (text175) at (161.16, 44.23){(39)};
  \node[anchor=south] (text176) at (169.74, 111.49){67};
  \node[anchor=south] (text177) at (169.74, 106.43){(94)};
  \node[anchor=south] (text178) at (169.74, 53.55){33};
  \node[anchor=south] (text179) at (169.74, 48.49){(46)};
  \node[anchor=south] (text180) at (178.32, 127.14){40};
  \node[anchor=south] (text181) at (178.32, 122.08){(65)};
  \node[anchor=south] (text182) at (178.32, 69.56){59};
  \node[anchor=south] (text183) at (178.32, 64.51){(96)};
  \node[anchor=south,shift={(0.0, -0.33)}] (text184) at (178.32, 34.88){1};
  \node[anchor=south,shift={(0.0, 0.2)}] (text185) at (178.32, 29.82){(1)};
  \path[fill=cebebeb,line cap=round,line join=round,line width=0.38mm,miter   limit=10.0] (183.96, 154.91) rectangle (211.42, 27.45);
  \path[draw=white,line cap=butt,line join=round,line width=0.19mm,miter   limit=10.0] (183.96, 47.73) -- (211.42, 47.73);
  \path[draw=white,line cap=butt,line join=round,line width=0.19mm,miter   limit=10.0] (183.96, 76.69) -- (211.42, 76.69);
  \path[draw=white,line cap=butt,line join=round,line width=0.19mm,miter   limit=10.0] (183.96, 105.66) -- (211.42, 105.66);
  \path[draw=white,line cap=butt,line join=round,line width=0.19mm,miter   limit=10.0] (183.96, 134.63) -- (211.42, 134.63);
  \path[draw=white,line cap=butt,line join=round,line width=0.38mm,miter   limit=10.0] (183.96, 33.24) -- (211.42, 33.24);
  \path[draw=white,line cap=butt,line join=round,line width=0.38mm,miter   limit=10.0] (183.96, 62.21) -- (211.42, 62.21);
  \path[draw=white,line cap=butt,line join=round,line width=0.38mm,miter   limit=10.0] (183.96, 91.18) -- (211.42, 91.18);
  \path[draw=white,line cap=butt,line join=round,line width=0.38mm,miter   limit=10.0] (183.96, 120.15) -- (211.42, 120.15);
  \path[draw=white,line cap=butt,line join=round,line width=0.38mm,miter   limit=10.0] (183.96, 149.11) -- (211.42, 149.11);
  \path[draw=white,line cap=butt,line join=round,line width=0.38mm,miter   limit=10.0] (189.11, 27.45) -- (189.11, 154.91);
  \path[draw=white,line cap=butt,line join=round,line width=0.38mm,miter   limit=10.0] (197.69, 27.45) -- (197.69, 154.91);
  \path[draw=white,line cap=butt,line join=round,line width=0.38mm,miter   limit=10.0] (206.27, 27.45) -- (206.27, 154.91);
  \path[draw=black,fill=c77aadd,line cap=butt,line join=miter,line   width=0.38mm,miter limit=10.0] (185.25, 149.11) rectangle (192.97, 33.24);
  \path[fill=ceedd88,line cap=butt,line join=miter,line width=0.38mm,miter   limit=10.0] ;
  \path[fill=c77aadd,line cap=butt,line join=miter,line width=0.38mm,miter   limit=10.0] ;
  \path[draw=black,fill=c77aadd,line cap=butt,line join=miter,line   width=0.38mm,miter limit=10.0] (193.83, 149.11) rectangle (201.55, 39.03);
  \path[draw=black,fill=ceedd88,line cap=butt,line join=miter,line   width=0.38mm,miter limit=10.0] (193.83, 39.03) rectangle (201.55, 33.24);
  \path[fill=c77aadd,line cap=butt,line join=miter,line width=0.38mm,miter   limit=10.0] ;
  \path[draw=black,fill=c77aadd,line cap=butt,line join=miter,line   width=0.38mm,miter limit=10.0] (202.41, 149.11) rectangle (210.13, 54.7);
  \path[draw=black,fill=ceedd88,line cap=butt,line join=miter,line   width=0.38mm,miter limit=10.0] (202.41, 54.7) rectangle (210.13, 37.54);
  \path[draw=black,fill=cee8866,line cap=butt,line join=miter,line   width=0.38mm,miter limit=10.0] (202.41, 37.54) rectangle (210.13, 33.25);
  \node[anchor=south] (text206) at (189.11, 92.45){100};
  \node[anchor=south] (text207) at (189.11, 87.39){(24)};
  \node[anchor=south] (text208) at (197.69, 95.35){95};
  \node[anchor=south] (text209) at (197.69, 90.29){(19)};
  \node[anchor=south,shift={(0.0, -2.83)}] (text210) at (197.69, 37.42){5};
  \node[anchor=south,shift={(0.0, -2.34)}] (text211) at (197.69, 32.36){(1)};
  \node[anchor=south] (text212) at (206.27, 103.18){81};
  \node[anchor=south] (text213) at (206.27, 98.12){(22)};
  \node[anchor=south] (text214) at (206.27, 47.39){15};
  \node[anchor=south] (text215) at (206.27, 42.33){(4)};
  \node[anchor=south,shift={(0.0, -2.12)}] (text216) at (206.27, 36.66){4};
  \node[anchor=south,shift={(0.0, -1.59)}] (text217) at (206.27, 31.61){(1)};
  \path[fill=cebebeb,line cap=round,line join=round,line width=0.38mm,miter   limit=10.0] (211.91, 154.91) rectangle (239.37, 27.45);
  \path[draw=white,line cap=butt,line join=round,line width=0.19mm,miter   limit=10.0] (211.91, 47.73) -- (239.37, 47.73);
  \path[draw=white,line cap=butt,line join=round,line width=0.19mm,miter   limit=10.0] (211.91, 76.69) -- (239.37, 76.69);
  \path[draw=white,line cap=butt,line join=round,line width=0.19mm,miter   limit=10.0] (211.91, 105.66) -- (239.37, 105.66);
  \path[draw=white,line cap=butt,line join=round,line width=0.19mm,miter   limit=10.0] (211.91, 134.63) -- (239.37, 134.63);
  \path[draw=white,line cap=butt,line join=round,line width=0.38mm,miter   limit=10.0] (211.91, 33.24) -- (239.37, 33.24);
  \path[draw=white,line cap=butt,line join=round,line width=0.38mm,miter   limit=10.0] (211.91, 62.21) -- (239.37, 62.21);
  \path[draw=white,line cap=butt,line join=round,line width=0.38mm,miter   limit=10.0] (211.91, 91.18) -- (239.37, 91.18);
  \path[draw=white,line cap=butt,line join=round,line width=0.38mm,miter   limit=10.0] (211.91, 120.15) -- (239.37, 120.15);
  \path[draw=white,line cap=butt,line join=round,line width=0.38mm,miter   limit=10.0] (211.91, 149.11) -- (239.37, 149.11);
  \path[draw=white,line cap=butt,line join=round,line width=0.38mm,miter   limit=10.0] (217.06, 27.45) -- (217.06, 154.91);
  \path[draw=white,line cap=butt,line join=round,line width=0.38mm,miter   limit=10.0] (225.64, 27.45) -- (225.64, 154.91);
  \path[draw=white,line cap=butt,line join=round,line width=0.38mm,miter   limit=10.0] (234.22, 27.45) -- (234.22, 154.91);
  \path[draw=black,fill=c77aadd,line cap=butt,line join=miter,line   width=0.38mm,miter limit=10.0] (213.2, 149.11) rectangle (220.92, 116.01);
  \path[draw=black,fill=ceedd88,line cap=butt,line join=miter,line   width=0.38mm,miter limit=10.0] (213.2, 116.01) rectangle (220.92, 33.24);
  \path[draw=black,fill=c77aadd,line cap=butt,line join=miter,line   width=0.38mm,miter limit=10.0] (221.78, 149.11) rectangle (229.5, 106.07);
  \path[draw=black,fill=ceedd88,line cap=butt,line join=miter,line   width=0.38mm,miter limit=10.0] (221.78, 106.08) rectangle (229.5, 33.24);
  \path[draw=black,fill=c77aadd,line cap=butt,line join=miter,line   width=0.38mm,miter limit=10.0] (230.36, 149.11) rectangle (238.08, 128.04);
  \path[draw=black,fill=ceedd88,line cap=butt,line join=miter,line   width=0.38mm,miter limit=10.0] (230.36, 128.05) rectangle (238.08, 33.24);
  \node[anchor=south] (text235) at (217.06, 133.84){29};
  \node[anchor=south] (text236) at (217.06, 128.78){(8)};
  \node[anchor=south] (text237) at (217.06, 75.9){71};
  \node[anchor=south] (text238) at (217.06, 70.84){(20)};
  \node[anchor=south] (text239) at (225.64, 128.87){37};
  \node[anchor=south] (text240) at (225.64, 123.81){(13)};
  \node[anchor=south] (text241) at (225.64, 70.93){63};
  \node[anchor=south] (text242) at (225.64, 65.87){(22)};
  \node[anchor=south] (text243) at (234.22, 139.86){18};
  \node[anchor=south] (text244) at (234.22, 134.8){(6)};
  \node[anchor=south] (text245) at (234.22, 81.92){82};
  \node[anchor=south] (text246) at (234.22, 76.86){(27)};
  \node[text=c1a1a1a,anchor=south] (text247) at (29.97, 156.96){\gls{fakultät2}};
  \node[text=c1a1a1a,anchor=south] (text249) at (57.93, 156.96){\gls{fakultät3}};
  \node[text=c1a1a1a,anchor=south] (text251) at (85.88, 156.96){\gls{fakultät4}};
  \node[text=c1a1a1a,anchor=south] (text253) at (113.83, 156.96){\gls{fakultät6}};
  \node[text=c1a1a1a,anchor=south] (text255) at (141.78, 156.96){\gls{fakultät7}};
  \node[text=c1a1a1a,anchor=south] (text257) at (169.74, 156.96){\gls{fakultät8}};
  \node[text=c1a1a1a,anchor=south] (text259) at (197.69, 156.96){\gls{fakultät9}};
  \node[text=c1a1a1a,anchor=south] (text261) at (225.64, 156.96){\gls{fakultät10}};
  \path[draw=c333333,line cap=butt,line join=round,line width=0.38mm,miter   limit=10.0] (21.4, 26.48) -- (21.4, 27.45);
  \path[draw=c333333,line cap=butt,line join=round,line width=0.38mm,miter   limit=10.0] (29.97, 26.48) -- (29.97, 27.45);
  \path[draw=c333333,line cap=butt,line join=round,line width=0.38mm,miter   limit=10.0] (38.55, 26.48) -- (38.55, 27.45);
  \node[text=c4d4d4d,anchor=south east,cm={ 0.71,0.71,-0.71,0.71,(23.53,   -154.23)}] (text264) at (0.0, 177.8){2012-2015};
  \node[text=c4d4d4d,anchor=south east,cm={ 0.71,0.71,-0.71,0.71,(32.11,   -154.23)}] (text265) at (0.0, 177.8){2016-2019};
  \node[text=c4d4d4d,anchor=south east,cm={ 0.71,0.71,-0.71,0.71,(40.69,   -154.23)}] (text266) at (0.0, 177.8){2020-2023};
  \path[draw=c333333,line cap=butt,line join=round,line width=0.38mm,miter   limit=10.0] (49.35, 26.48) -- (49.35, 27.45);
  \path[draw=c333333,line cap=butt,line join=round,line width=0.38mm,miter   limit=10.0] (57.93, 26.48) -- (57.93, 27.45);
  \path[draw=c333333,line cap=butt,line join=round,line width=0.38mm,miter   limit=10.0] (66.51, 26.48) -- (66.51, 27.45);
  \node[text=c4d4d4d,anchor=south east,cm={ 0.71,0.71,-0.71,0.71,(51.48,   -154.23)}] (text268) at (0.0, 177.8){2012-2015};
  \node[text=c4d4d4d,anchor=south east,cm={ 0.71,0.71,-0.71,0.71,(60.06,   -154.23)}] (text269) at (0.0, 177.8){2016-2019};
  \node[text=c4d4d4d,anchor=south east,cm={ 0.71,0.71,-0.71,0.71,(68.64,   -154.23)}] (text270) at (0.0, 177.8){2020-2023};
  \path[draw=c333333,line cap=butt,line join=round,line width=0.38mm,miter   limit=10.0] (77.3, 26.48) -- (77.3, 27.45);
  \path[draw=c333333,line cap=butt,line join=round,line width=0.38mm,miter   limit=10.0] (85.88, 26.48) -- (85.88, 27.45);
  \path[draw=c333333,line cap=butt,line join=round,line width=0.38mm,miter   limit=10.0] (94.46, 26.48) -- (94.46, 27.45);
  \node[text=c4d4d4d,anchor=south east,cm={ 0.71,0.71,-0.71,0.71,(79.43,   -154.23)}] (text272) at (0.0, 177.8){2012-2015};
  \node[text=c4d4d4d,anchor=south east,cm={ 0.71,0.71,-0.71,0.71,(88.01,   -154.23)}] (text273) at (0.0, 177.8){2016-2019};
  \node[text=c4d4d4d,anchor=south east,cm={ 0.71,0.71,-0.71,0.71,(96.59,   -154.23)}] (text274) at (0.0, 177.8){2020-2023};
  \path[draw=c333333,line cap=butt,line join=round,line width=0.38mm,miter   limit=10.0] (105.25, 26.48) -- (105.25, 27.45);
  \path[draw=c333333,line cap=butt,line join=round,line width=0.38mm,miter   limit=10.0] (113.83, 26.48) -- (113.83, 27.45);
  \path[draw=c333333,line cap=butt,line join=round,line width=0.38mm,miter   limit=10.0] (122.41, 26.48) -- (122.41, 27.45);
  \node[text=c4d4d4d,anchor=south east,cm={ 0.71,0.71,-0.71,0.71,(107.39,   -154.23)}] (text276) at (0.0, 177.8){2012-2015};
  \node[text=c4d4d4d,anchor=south east,cm={ 0.71,0.71,-0.71,0.71,(115.97,   -154.23)}] (text277) at (0.0, 177.8){2016-2019};
  \node[text=c4d4d4d,anchor=south east,cm={ 0.71,0.71,-0.71,0.71,(124.55,   -154.23)}] (text278) at (0.0, 177.8){2020-2023};
  \path[draw=c333333,line cap=butt,line join=round,line width=0.38mm,miter   limit=10.0] (133.21, 26.48) -- (133.21, 27.45);
  \path[draw=c333333,line cap=butt,line join=round,line width=0.38mm,miter   limit=10.0] (141.78, 26.48) -- (141.78, 27.45);
  \path[draw=c333333,line cap=butt,line join=round,line width=0.38mm,miter   limit=10.0] (150.36, 26.48) -- (150.36, 27.45);
  \node[text=c4d4d4d,anchor=south east,cm={ 0.71,0.71,-0.71,0.71,(135.34,   -154.23)}] (text280) at (0.0, 177.8){2012-2015};
  \node[text=c4d4d4d,anchor=south east,cm={ 0.71,0.71,-0.71,0.71,(143.92,   -154.23)}] (text281) at (0.0, 177.8){2016-2019};
  \node[text=c4d4d4d,anchor=south east,cm={ 0.71,0.71,-0.71,0.71,(152.5,   -154.23)}] (text282) at (0.0, 177.8){2020-2023};
  \path[draw=c333333,line cap=butt,line join=round,line width=0.38mm,miter   limit=10.0] (161.16, 26.48) -- (161.16, 27.45);
  \path[draw=c333333,line cap=butt,line join=round,line width=0.38mm,miter   limit=10.0] (169.74, 26.48) -- (169.74, 27.45);
  \path[draw=c333333,line cap=butt,line join=round,line width=0.38mm,miter   limit=10.0] (178.32, 26.48) -- (178.32, 27.45);
  \node[text=c4d4d4d,anchor=south east,cm={ 0.71,0.71,-0.71,0.71,(163.29,   -154.23)}] (text284) at (0.0, 177.8){2012-2015};
  \node[text=c4d4d4d,anchor=south east,cm={ 0.71,0.71,-0.71,0.71,(171.87,   -154.23)}] (text285) at (0.0, 177.8){2016-2019};
  \node[text=c4d4d4d,anchor=south east,cm={ 0.71,0.71,-0.71,0.71,(180.45,   -154.23)}] (text286) at (0.0, 177.8){2020-2023};
  \path[draw=c333333,line cap=butt,line join=round,line width=0.38mm,miter   limit=10.0] (189.11, 26.48) -- (189.11, 27.45);
  \path[draw=c333333,line cap=butt,line join=round,line width=0.38mm,miter   limit=10.0] (197.69, 26.48) -- (197.69, 27.45);
  \path[draw=c333333,line cap=butt,line join=round,line width=0.38mm,miter   limit=10.0] (206.27, 26.48) -- (206.27, 27.45);
  \node[text=c4d4d4d,anchor=south east,cm={ 0.71,0.71,-0.71,0.71,(191.25,   -154.23)}] (text288) at (0.0, 177.8){2012-2015};
  \node[text=c4d4d4d,anchor=south east,cm={ 0.71,0.71,-0.71,0.71,(199.82,   -154.23)}] (text289) at (0.0, 177.8){2016-2019};
  \node[text=c4d4d4d,anchor=south east,cm={ 0.71,0.71,-0.71,0.71,(208.4,   -154.23)}] (text290) at (0.0, 177.8){2020-2023};
  \path[draw=c333333,line cap=butt,line join=round,line width=0.38mm,miter   limit=10.0] (217.06, 26.48) -- (217.06, 27.45);
  \path[draw=c333333,line cap=butt,line join=round,line width=0.38mm,miter   limit=10.0] (225.64, 26.48) -- (225.64, 27.45);
  \path[draw=c333333,line cap=butt,line join=round,line width=0.38mm,miter   limit=10.0] (234.22, 26.48) -- (234.22, 27.45);
  \node[text=c4d4d4d,anchor=south east,cm={ 0.71,0.71,-0.71,0.71,(219.2,   -154.23)}] (text292) at (0.0, 177.8){2012-2015};
  \node[text=c4d4d4d,anchor=south east,cm={ 0.71,0.71,-0.71,0.71,(227.78,   -154.23)}] (text293) at (0.0, 177.8){2016-2019};
  \node[text=c4d4d4d,anchor=south east,cm={ 0.71,0.71,-0.71,0.71,(236.36,   -154.23)}] (text294) at (0.0, 177.8){2020-2023};
  \node[text=c4d4d4d,anchor=south east] (text295) at (14.51, 32.13){0\%};
  \node[text=c4d4d4d,anchor=south east] (text296) at (14.51, 61.1){25\%};
  \node[text=c4d4d4d,anchor=south east] (text297) at (14.51, 90.07){50\%};
  \node[text=c4d4d4d,anchor=south east] (text298) at (14.51, 119.04){75\%};
  \node[text=c4d4d4d,anchor=south east] (text299) at (14.51, 148.0){100\%};
  \path[draw=c333333,line cap=butt,line join=round,line width=0.38mm,miter   limit=10.0] (15.28, 33.24) -- (16.25, 33.24);
  \path[draw=c333333,line cap=butt,line join=round,line width=0.38mm,miter   limit=10.0] (15.28, 62.21) -- (16.25, 62.21);
  \path[draw=c333333,line cap=butt,line join=round,line width=0.38mm,miter   limit=10.0] (15.28, 91.18) -- (16.25, 91.18);
  \path[draw=c333333,line cap=butt,line join=round,line width=0.38mm,miter   limit=10.0] (15.28, 120.15) -- (16.25, 120.15);
  \path[draw=c333333,line cap=butt,line join=round,line width=0.38mm,miter   limit=10.0] (15.28, 149.11) -- (16.25, 149.11);
  \node[anchor=south,cm={ 0.0,1.0,-1.0,0.0,(4.7, -86.62)}] (text304) at (0.0,   177.8){Anteil in Prozenten (\%)};
  \path[fill=white,line cap=round,line join=round,line width=0.38mm,miter   limit=10.0] (95.59, 175.87) rectangle (160.02, 165.9);
  \path[fill=cebebeb,line cap=round,line join=round,line width=0.38mm,miter   limit=10.0] (97.53, 173.93) rectangle (103.62, 167.84);
  \path[fill=c77aadd,line cap=butt,line join=miter,line width=0.38mm,miter   limit=10.0] (97.78, 173.68) rectangle (103.37, 168.09);
  \path[fill=cebebeb,line cap=round,line join=round,line width=0.38mm,miter   limit=10.0] (112.02, 173.93) rectangle (118.12, 167.84);
  \path[fill=ceedd88,line cap=butt,line join=miter,line width=0.38mm,miter   limit=10.0] (112.28, 173.68) rectangle (117.87, 168.09);
  \path[fill=cebebeb,line cap=round,line join=round,line width=0.38mm,miter   limit=10.0] (126.63, 173.93) rectangle (132.73, 167.84);
  \path[fill=cee8866,line cap=butt,line join=miter,line width=0.38mm,miter   limit=10.0] (126.88, 173.68) rectangle (132.48, 168.09);
  \node[anchor=south west] (text312) at (105.55, 169.63){DE};
  \node[anchor=south west] (text313) at (120.05, 169.63){EN};
  \node[anchor=south west] (text314) at (134.66, 169.63){Andere};
\end{tikzpicture}}
    \caption{Sprachen der \glspl{pdd} nach Fakultät und Zeitgruppe.
    Die Höhe der Barren entsprechen dem relativen Anteil zur jeweiligen angepassten $\text{\textit{Fakultät}}\times\text{\textit{Zeitgruppe}}\times\text{\textit{Sprache}}$-Gesamtanzahl.
    Absolute Werte in Klammern angegeben.}
    \label{fig:luh-repo_sprache_x_fakultät_x_zeitgruppe}
\end{figure}

Für die Beziehung zwischen \textit{Sprache} und \textit{Allgemeine~\glspl{forschungsdaten}} ist anzumerken, dass $\SI[round-mode=places,round-precision=2]{67.5603217158177}{\percent}$ aller \textit{Stufe~3} \glspl{pdd} auf Deutsch und $\SI[round-mode=places,round-precision=2]{32.171581769437}{\percent}$ auf Englisch geschrieben wurden.
Für \textit{Stufe~1} sind diese Anteile beide jeweils $\SI[round-mode=places,round-precision=2]{49.6031746031746}{\percent}$.
Die Abhängigkeit zwischen \textit{Sprache} und \textit{Allgemeine \glspl{forschungsdaten}} ist mit $\chi^2 (\num{9}, n=\num{1252}) = \num[round-mode=places,round-precision=2]{36.3201742371864}$, $p = \num[round-mode=places,round-precision=2]{3.47809061239522E-05}<\num{0.01},\phi_C=\num[round-mode=places,round-precision=2]{0.0983356900818459}$ statistisch hochsignifikant aber sehr schwach im Effekt.

Für \textit{Externe \glspl{forschungsdaten}} ist diese Verteilung umgekehrt und stärker ausgeprägt:
Für \glspl{pdd} mit externen \glspl{forschungsdaten} die mit \textit{Stufe~1} klassifiziert wurden, wurden nur $\SI[round-mode=places,round-precision=2]{25,5555555555556}{\percent}$ auf Deutsch und $\SI[round-mode=places,round-precision=2]{73,3333333333333}{\percent}$ auf Englisch geschrieben.
Die Beziehung zwischen \textit{Sprache} und \textit{Externe \glspl{forschungsdaten}} ist mit $\chi^2 (\num{9}, n=\num{1252}) = \num[round-mode=places,round-precision=2]{39.4644044440327}$, $p = \num[round-mode=places,round-precision=2]{9.49823708731411E-06}<\num{0.01},\phi_C=\num[round-mode=places,round-precision=2]{0.102503804500984}$ auch statistisch hochsignifikant und schwach in Effektstärke.

Da \textit{Sprache} und \textit{Fakultät} auch miteinander korrelieren, wie in \cref{sec:luh-repo-results-factors} gezeigt, muss hierbei beachtet werden inwiefern der Effekt von \textit{Sprache} auf \textit{Allgemeine \glspl{forschungsdaten}} und \textit{Externe \glspl{forschungsdaten}} an \textit{Sprache} liegt und wie viel der Effekt durch ihre Korrelation mit \textit{Fakultät} bedingt ist.
Siehe hierfür die Interaktion zwischen \textit{Fakultät} und den verschiedenen \gls{forschungsdaten}-Publikationsarten in \cref{sec:luh-repo-results-faculties}

\subsection{Fakultäten und Forschungsdaten}\label{sec:luh-repo-results-faculties}
\parsum{Klassifikation Fakultäten}
Für \glspl{pdd} ist die relative und absolute Verteilung der Klassifikationsstufen für alle Dokumente sowie die Verteilung nach $\text{\textit{Fakultät}}\times\text{\textit{Klassifikationsstufe}}$ in \cref{tab:luh-repo-classification-general-all-faculty-adjusted} gegeben.
\begin{table}[!htbp]
	\caption{\gls{forschungsdaten}-Klassifizierung der Dissertationen aus der Stichprobe nach $\text{\textit{Fakultät}}\times\text{\textit{Klassifikationsstufe}}$ aufgegliedert.
    Angabe relativ zu der respektiven angepassten Gesamtanzahl für \textit{Fakultät}.
    Absolute Werte in Klammern angegeben.}
    \resizebox{\ifdim\width>\textwidth\textwidth\else\width\fi}{!}{%
	\begin{tabular}{lS[table-format=3.2]@{\,}S[table-text-alignment = left]lS[table-format=3.2]@{\,}S[table-text-alignment = left]lS[table-format=3.2]@{\,}S[table-text-alignment = left]lS[table-format=3.2]@{\,}S[table-text-alignment = left]lS[table-format=3.2]@{\,}S[table-text-alignment = left]l}
		\toprule
		& \multicolumn{3}{c}{\textbf{Stufe 1}} & \multicolumn{3}{c}{\textbf{Stufe 2}} & \multicolumn{3}{c}{\textbf{Stufe 3}} & \multicolumn{3}{c}{\textbf{Keine}}  \\
		\midrule
		\textbf{\gls{fakultät2}}  & 16,00  & \si{\percent} & (8)  & 0,00  & \si{\percent} & (0)  & 56,00  & \si{\percent} & (28) & 28,00   & \si{\percent} & (24)  \\
		\textbf{\gls{fakultät3}}  & 26,67  & \si{\percent} & (28)  & 12,38  & \si{\percent} & (13)  & 12,38  & \si{\percent} & (13)  & 48,57    & \si{\percent} & (56)\\
		\textbf{\gls{fakultät4}}  & 30,00  & \si{\percent} & (42)  & 12,86  & \si{\percent} & (18)  & 18,57  & \si{\percent} & (26)  & 38,57   & \si{\percent} & (59)\\
		\textbf{\gls{fakultät5}}  & \multicolumn{2}{c}{---} & (0)   & \multicolumn{2}{c}{---} & (0)   & \multicolumn{2}{c}{---} & (0)  & \multicolumn{2}{c}{---} & (4)\\
		\textbf{\gls{fakultät6}}  & 12,50  & \si{\percent} & (15)  & 4,17  & \si{\percent} & (5)  & 19,17  & \si{\percent} & (23)  & 64,17    & \si{\percent} & (78)\\
		\textbf{\gls{fakultät7}}  & 15,42  & \si{\percent} & (33)  & 4,67  & \si{\percent} & (10)  & 22,90  & \si{\percent} & (49)  & 57,01    & \si{\percent} & (152)\\
		\textbf{\gls{fakultät8}}  & 22,42 & \si{\percent} & (102) & 21,10  & \si{\percent} & (96) & 40,88 & \si{\percent} & (186)  & 15,60    & \si{\percent} & (76)\\
		\textbf{\gls{fakultät9}}  & 22,54  & \si{\percent} & (16)  & 0,00  & \si{\percent} & (0)  & 33,80  & \si{\percent} & (24)  & 43,66    & \si{\percent} & (84)\\
		\textbf{\gls{fakultät10}} & 8,33  & \si{\percent} & (8)  & 1,04  & \si{\percent} & (1)  & 25,00  & \si{\percent} & (24)  & 65,63    & \si{\percent} & (140)\\
		\midrule
		\textbf{Alle}            & 20,14 & \si{\percent} & (252) & 11,43 & \si{\percent} & (143) & 29,82 & \si{\percent} & (373) & 38,61  & \si{\percent} & (673)\\
		\bottomrule
	\end{tabular}
}
    \label{tab:luh-repo-classification-general-all-faculty-adjusted}
\end{table}
Dieselbe Aufteilung für alle Dissertationen ist \cref{tab:luh-repo-classification-general-all-faculty} gegeben.

Die Interaktion für \glspl{pdd} zwischen \textit{Fakultät} und \textit{Allgemeine \glspl{forschungsdaten}} ist mit $\chi^2 (\num{21}, n=\num{1252}) = \num[round-mode=places,round-precision=2]{277.018852269436}$, $p = \num[round-mode=places,round-precision=2]{1.46707274481795E-46}<\num{0.01},\phi_C=\num[round-mode=places,round-precision=2]{0.271576302422829}$ statistisch hochsignifikant mit einer fast moderaten Effektstärke.
Für \glspl{forschungsdaten} sind die respektiven respektiven Anteile pro Fakultät für \textit{Stufe~1} ($\bar{x}=\SI[round-mode=places,round-precision=2]{19.1949536178319}{\percent},s=\SI[round-mode=places,round-precision=2]{7.40146945375256}{\percent}$), \textit{Stufe~2} ($\bar{x}=\SI[round-mode=places,round-precision=2]{7.02727835832392}{\percent},s=\SI[round-mode=places,round-precision=2]{7.64365985309809}{\percent}$), \textit{Stufe~3} ($\bar{x}=\SI[round-mode=places,round-precision=2]{28.4500178056966}{\percent},s=\SI[round-mode=places,round-precision=2]{13.9773224286791}{\percent}$) und \textit{Keine} ($\bar{x}=\SI[round-mode=places,round-precision=2]{45.3277502181475}{\percent},s=\SI[round-mode=places,round-precision=2]{17.3167274633444}{\percent}$).
Es variieren zwischen den Fakultäten also insbesondere \textit{Stufe~3} und \textit{Keine \glspl{forschungsdaten}}, während \textit{Stufe~1} und \textit{Stufe~2} moderatere interfakultäre Variation erleben.
Es besteht also ein vergleichsweise kleiner Kern an \glspl{pdd}, die in allen Fakultäten---zu moderat unterschiedlich großen Anteilen---\glspl{forschungsdaten} aus \textit{Stufe~1} oder \textit{Stufe~2} publizieren, während der größte Unterschied zwischen Fakultäten dadurch bestimmt wird, wie groß die jeweilige Verteilung zwischen \textit{Keine~\glspl{forschungsdaten}} und \glspl{forschungsdaten} aus \textit{Stufe~3} ist.
Dieser Effekt kann in \cref{fig:luh-repo_fakultät_x_zeitgruppe_x_fd}, welche die relative und absolute Verteilung von \glspl{forschungsdaten} für Fakultäten über die verschiedenen Zeitgruppen darstellt, gesehen werden.
\begin{figure}[!htbp]
    \resizebox{\ifdim\width>\textwidth\textwidth\else\width\fi}{!}{\begin{tikzpicture}[y=1mm, x=1mm, yscale=\globalscale,xscale=\globalscale, every node/.append style={scale=\globalscale}, inner sep=0pt, outer sep=0pt]
  \path[fill=cebebeb,line cap=round,line join=round,line width=0.38mm,miter 
  limit=10.0] (16.25, 154.91) rectangle (43.7, 27.45);



  \path[draw=white,line cap=butt,line join=round,line width=0.19mm,miter 
  limit=10.0] (16.25, 47.73) -- (43.7, 47.73);



  \path[draw=white,line cap=butt,line join=round,line width=0.19mm,miter 
  limit=10.0] (16.25, 76.69) -- (43.7, 76.69);



  \path[draw=white,line cap=butt,line join=round,line width=0.19mm,miter 
  limit=10.0] (16.25, 105.66) -- (43.7, 105.66);



  \path[draw=white,line cap=butt,line join=round,line width=0.19mm,miter 
  limit=10.0] (16.25, 134.63) -- (43.7, 134.63);



  \path[draw=white,line cap=butt,line join=round,line width=0.38mm,miter 
  limit=10.0] (16.25, 33.24) -- (43.7, 33.24);



  \path[draw=white,line cap=butt,line join=round,line width=0.38mm,miter 
  limit=10.0] (16.25, 62.21) -- (43.7, 62.21);



  \path[draw=white,line cap=butt,line join=round,line width=0.38mm,miter 
  limit=10.0] (16.25, 91.18) -- (43.7, 91.18);



  \path[draw=white,line cap=butt,line join=round,line width=0.38mm,miter 
  limit=10.0] (16.25, 120.15) -- (43.7, 120.15);



  \path[draw=white,line cap=butt,line join=round,line width=0.38mm,miter 
  limit=10.0] (16.25, 149.11) -- (43.7, 149.11);



  \path[draw=white,line cap=butt,line join=round,line width=0.38mm,miter 
  limit=10.0] (21.4, 27.45) -- (21.4, 154.91);



  \path[draw=white,line cap=butt,line join=round,line width=0.38mm,miter 
  limit=10.0] (29.97, 27.45) -- (29.97, 154.91);



  \path[draw=white,line cap=butt,line join=round,line width=0.38mm,miter 
  limit=10.0] (38.55, 27.45) -- (38.55, 154.91);



  \path[draw=black,fill=c77aadd,draw opacity=0.85,line cap=butt,line 
  join=miter,line width=0.38mm,miter limit=10.0] (17.53, 149.11) rectangle 
  (25.26, 129.8);



  \path[fill=c99dde1,line cap=butt,line join=miter,line width=0.38mm,miter 
  limit=10.0] ;



  \path[draw=black,fill=ceedd88,draw opacity=0.85,line cap=butt,line 
  join=miter,line width=0.38mm,miter limit=10.0] (17.53, 129.8) rectangle 
  (25.26, 62.21);



  \path[draw=black,fill=cee8866,draw opacity=0.85,line cap=butt,line 
  join=miter,line width=0.38mm,miter limit=10.0] (17.53, 62.21) rectangle 
  (25.26, 33.24);



  \path[draw=black,fill=c77aadd,draw opacity=0.85,line cap=butt,line 
  join=miter,line width=0.38mm,miter limit=10.0] (26.11, 149.11) rectangle 
  (33.83, 123.36);



  \path[fill=c99dde1,line cap=butt,line join=miter,line width=0.38mm,miter 
  limit=10.0] ;



  \path[draw=black,fill=ceedd88,draw opacity=0.85,line cap=butt,line 
  join=miter,line width=0.38mm,miter limit=10.0] (26.11, 123.37) rectangle 
  (33.83, 59.0);



  \path[draw=black,fill=cee8866,draw opacity=0.85,line cap=butt,line 
  join=miter,line width=0.38mm,miter limit=10.0] (26.11, 58.99) rectangle 
  (33.83, 33.24);



  \path[draw=black,fill=c77aadd,draw opacity=0.85,line cap=butt,line 
  join=miter,line width=0.38mm,miter limit=10.0] (34.69, 149.11) rectangle 
  (42.41, 138.08);



  \path[fill=c99dde1,line cap=butt,line join=miter,line width=0.38mm,miter 
  limit=10.0] ;



  \path[draw=black,fill=ceedd88,draw opacity=0.85,line cap=butt,line 
  join=miter,line width=0.38mm,miter limit=10.0] (34.69, 138.08) rectangle 
  (42.41, 77.38);



  \path[draw=black,fill=cee8866,draw opacity=0.85,line cap=butt,line 
  join=miter,line width=0.38mm,miter limit=10.0] (34.69, 77.39) rectangle 
  (42.41, 33.24);



  \node[anchor=south] (text27) at (21.4, 140.73){17};



  \node[anchor=south] (text28) at (21.4, 135.67){(2)};



  \node[anchor=south] (text29) at (21.4, 97.28){58};



  \node[anchor=south] (text30) at (21.4, 92.22){(7)};



  \node[anchor=south] (text31) at (21.4, 49.0){25};



  \node[anchor=south] (text32) at (21.4, 43.94){(3)};



  \node[anchor=south] (text33) at (29.97, 137.51){22};



  \node[anchor=south] (text34) at (29.97, 132.45){(4)};



  \node[anchor=south] (text35) at (29.97, 92.45){56};



  \node[anchor=south] (text36) at (29.97, 87.39){(10)};



  \node[anchor=south] (text37) at (29.97, 47.39){22};



  \node[anchor=south] (text38) at (29.97, 42.33){(4)};



  \node[anchor=south] (text39) at (38.55, 144.87){10};



  \node[anchor=south] (text40) at (38.55, 139.81){(2)};



  \node[anchor=south] (text41) at (38.55, 109.0){52};



  \node[anchor=south] (text42) at (38.55, 103.95){(11)};



  \node[anchor=south] (text43) at (38.55, 56.59){38};



  \node[anchor=south] (text44) at (38.55, 51.53){(8)};



  \path[fill=cebebeb,line cap=round,line join=round,line width=0.38mm,miter 
  limit=10.0] (44.2, 154.91) rectangle (71.65, 27.45);



  \path[draw=white,line cap=butt,line join=round,line width=0.19mm,miter 
  limit=10.0] (44.2, 47.73) -- (71.65, 47.73);



  \path[draw=white,line cap=butt,line join=round,line width=0.19mm,miter 
  limit=10.0] (44.2, 76.69) -- (71.65, 76.69);



  \path[draw=white,line cap=butt,line join=round,line width=0.19mm,miter 
  limit=10.0] (44.2, 105.66) -- (71.65, 105.66);



  \path[draw=white,line cap=butt,line join=round,line width=0.19mm,miter 
  limit=10.0] (44.2, 134.63) -- (71.65, 134.63);



  \path[draw=white,line cap=butt,line join=round,line width=0.38mm,miter 
  limit=10.0] (44.2, 33.24) -- (71.65, 33.24);



  \path[draw=white,line cap=butt,line join=round,line width=0.38mm,miter 
  limit=10.0] (44.2, 62.21) -- (71.65, 62.21);



  \path[draw=white,line cap=butt,line join=round,line width=0.38mm,miter 
  limit=10.0] (44.2, 91.18) -- (71.65, 91.18);



  \path[draw=white,line cap=butt,line join=round,line width=0.38mm,miter 
  limit=10.0] (44.2, 120.15) -- (71.65, 120.15);



  \path[draw=white,line cap=butt,line join=round,line width=0.38mm,miter 
  limit=10.0] (44.2, 149.11) -- (71.65, 149.11);



  \path[draw=white,line cap=butt,line join=round,line width=0.38mm,miter 
  limit=10.0] (49.35, 27.45) -- (49.35, 154.91);



  \path[draw=white,line cap=butt,line join=round,line width=0.38mm,miter 
  limit=10.0] (57.93, 27.45) -- (57.93, 154.91);



  \path[draw=white,line cap=butt,line join=round,line width=0.38mm,miter 
  limit=10.0] (66.51, 27.45) -- (66.51, 154.91);



  \path[draw=black,fill=c77aadd,draw opacity=0.85,line cap=butt,line 
  join=miter,line width=0.38mm,miter limit=10.0] (45.49, 149.11) rectangle 
  (53.21, 136.24);



  \path[draw=black,fill=c99dde1,draw opacity=0.85,line cap=butt,line 
  join=miter,line width=0.38mm,miter limit=10.0] (45.49, 136.24) rectangle 
  (53.21, 110.49);



  \path[draw=black,fill=ceedd88,draw opacity=0.85,line cap=butt,line 
  join=miter,line width=0.38mm,miter limit=10.0] (45.49, 110.49) rectangle 
  (53.21, 84.74);



  \path[draw=black,fill=cee8866,draw opacity=0.85,line cap=butt,line 
  join=miter,line width=0.38mm,miter limit=10.0] (45.49, 84.74) rectangle 
  (53.21, 33.24);



  \path[draw=black,fill=c77aadd,draw opacity=0.85,line cap=butt,line 
  join=miter,line width=0.38mm,miter limit=10.0] (54.07, 149.11) rectangle 
  (61.79, 109.16);



  \path[draw=black,fill=c99dde1,draw opacity=0.85,line cap=butt,line 
  join=miter,line width=0.38mm,miter limit=10.0] (54.07, 109.16) rectangle 
  (61.79, 101.17);



  \path[draw=black,fill=ceedd88,draw opacity=0.85,line cap=butt,line 
  join=miter,line width=0.38mm,miter limit=10.0] (54.07, 101.17) rectangle 
  (61.79, 77.19);



  \path[draw=black,fill=cee8866,draw opacity=0.85,line cap=butt,line 
  join=miter,line width=0.38mm,miter limit=10.0] (54.07, 77.19) rectangle 
  (61.79, 33.24);



  \path[draw=black,fill=c77aadd,draw opacity=0.85,line cap=butt,line 
  join=miter,line width=0.38mm,miter limit=10.0] (62.64, 149.11) rectangle 
  (70.37, 117.15);



  \path[draw=black,fill=c99dde1,draw opacity=0.85,line cap=butt,line 
  join=miter,line width=0.38mm,miter limit=10.0] (62.64, 117.15) rectangle 
  (70.37, 103.17);



  \path[draw=black,fill=ceedd88,draw opacity=0.85,line cap=butt,line 
  join=miter,line width=0.38mm,miter limit=10.0] (62.64, 103.17) rectangle 
  (70.37, 97.17);



  \path[draw=black,fill=cee8866,draw opacity=0.85,line cap=butt,line 
  join=miter,line width=0.38mm,miter limit=10.0] (62.64, 97.17) rectangle 
  (70.37, 33.24);



  \node[anchor=south] (text68) at (49.35, 143.95){11};



  \node[anchor=south] (text69) at (49.35, 138.89){(2)};



  \node[anchor=south] (text70) at (49.35, 124.64){22};



  \node[anchor=south] (text71) at (49.35, 119.58){(4)};



  \node[anchor=south] (text72) at (49.35, 98.89){22};



  \node[anchor=south] (text73) at (49.35, 93.83){(4)};



  \node[anchor=south] (text74) at (49.35, 60.27){44};



  \node[anchor=south] (text75) at (49.35, 55.21){(8)};



  \node[anchor=south] (text76) at (57.93, 130.41){34};



  \node[anchor=south] (text77) at (57.93, 125.35){(10)};



  \node[anchor=south,shift={(0.0, -0.53)}] (text78) at (57.93, 106.44){7};



  \node[anchor=south,shift={(0.0, 0.53)}] (text79) at (57.93, 101.38){(2)};



  \node[anchor=south] (text80) at (57.93, 90.46){21};



  \node[anchor=south] (text81) at (57.93, 85.4){(6)};



  \node[anchor=south] (text82) at (57.93, 56.49){38};



  \node[anchor=south] (text83) at (57.93, 51.43){(11)};



  \node[anchor=south] (text84) at (66.51, 134.4){28};



  \node[anchor=south] (text85) at (66.51, 129.35){(16)};



  \node[anchor=south] (text86) at (66.51, 111.43){12};



  \node[anchor=south] (text87) at (66.51, 106.37){(7)};



  \node[anchor=south,shift={(0.0, -2.65)}] (text88) at (66.51, 101.44){5};



  \node[anchor=south,shift={(0.0, -2.65)}] (text89) at (66.51, 96.38){(3)};



  \node[anchor=south] (text90) at (66.51, 66.48){55};



  \node[anchor=south] (text91) at (66.51, 61.42){(32)};



  \path[fill=cebebeb,line cap=round,line join=round,line width=0.38mm,miter 
  limit=10.0] (72.15, 154.91) rectangle (99.61, 27.45);



  \path[draw=white,line cap=butt,line join=round,line width=0.19mm,miter 
  limit=10.0] (72.15, 47.73) -- (99.6, 47.73);



  \path[draw=white,line cap=butt,line join=round,line width=0.19mm,miter 
  limit=10.0] (72.15, 76.69) -- (99.6, 76.69);



  \path[draw=white,line cap=butt,line join=round,line width=0.19mm,miter 
  limit=10.0] (72.15, 105.66) -- (99.6, 105.66);



  \path[draw=white,line cap=butt,line join=round,line width=0.19mm,miter 
  limit=10.0] (72.15, 134.63) -- (99.6, 134.63);



  \path[draw=white,line cap=butt,line join=round,line width=0.38mm,miter 
  limit=10.0] (72.15, 33.24) -- (99.6, 33.24);



  \path[draw=white,line cap=butt,line join=round,line width=0.38mm,miter 
  limit=10.0] (72.15, 62.21) -- (99.6, 62.21);



  \path[draw=white,line cap=butt,line join=round,line width=0.38mm,miter 
  limit=10.0] (72.15, 91.18) -- (99.6, 91.18);



  \path[draw=white,line cap=butt,line join=round,line width=0.38mm,miter 
  limit=10.0] (72.15, 120.15) -- (99.6, 120.15);



  \path[draw=white,line cap=butt,line join=round,line width=0.38mm,miter 
  limit=10.0] (72.15, 149.11) -- (99.6, 149.11);



  \path[draw=white,line cap=butt,line join=round,line width=0.38mm,miter 
  limit=10.0] (77.3, 27.45) -- (77.3, 154.91);



  \path[draw=white,line cap=butt,line join=round,line width=0.38mm,miter 
  limit=10.0] (85.88, 27.45) -- (85.88, 154.91);



  \path[draw=white,line cap=butt,line join=round,line width=0.38mm,miter 
  limit=10.0] (94.46, 27.45) -- (94.46, 154.91);



  \path[draw=black,fill=c77aadd,draw opacity=0.85,line cap=butt,line 
  join=miter,line width=0.38mm,miter limit=10.0] (73.44, 149.11) rectangle 
  (81.16, 127.38);



  \path[draw=black,fill=c99dde1,draw opacity=0.85,line cap=butt,line 
  join=miter,line width=0.38mm,miter limit=10.0] (73.44, 127.39) rectangle 
  (81.16, 105.66);



  \path[draw=black,fill=ceedd88,draw opacity=0.85,line cap=butt,line 
  join=miter,line width=0.38mm,miter limit=10.0] (73.44, 105.66) rectangle 
  (81.16, 80.31);



  \path[draw=black,fill=cee8866,draw opacity=0.85,line cap=butt,line 
  join=miter,line width=0.38mm,miter limit=10.0] (73.44, 80.32) rectangle 
  (81.16, 33.25);



  \path[draw=black,fill=c77aadd,draw opacity=0.85,line cap=butt,line 
  join=miter,line width=0.38mm,miter limit=10.0] (82.02, 149.11) rectangle 
  (89.74, 123.68);



  \path[draw=black,fill=c99dde1,draw opacity=0.85,line cap=butt,line 
  join=miter,line width=0.38mm,miter limit=10.0] (82.02, 123.68) rectangle 
  (89.74, 101.07);



  \path[draw=black,fill=ceedd88,draw opacity=0.85,line cap=butt,line 
  join=miter,line width=0.38mm,miter limit=10.0] (82.02, 101.07) rectangle 
  (89.74, 81.29);



  \path[draw=black,fill=cee8866,draw opacity=0.85,line cap=butt,line 
  join=miter,line width=0.38mm,miter limit=10.0] (82.02, 81.29) rectangle 
  (89.74, 33.24);



  \path[draw=black,fill=c77aadd,draw opacity=0.85,line cap=butt,line 
  join=miter,line width=0.38mm,miter limit=10.0] (90.6, 149.11) rectangle 
  (98.32, 102.42);



  \path[draw=black,fill=c99dde1,draw opacity=0.85,line cap=butt,line 
  join=miter,line width=0.38mm,miter limit=10.0] (90.6, 102.42) rectangle 
  (98.32, 95.5);



  \path[draw=black,fill=ceedd88,draw opacity=0.85,line cap=butt,line 
  join=miter,line width=0.38mm,miter limit=10.0] (90.6, 95.5) rectangle (98.32, 
  74.75);



  \path[draw=black,fill=cee8866,draw opacity=0.85,line cap=butt,line 
  join=miter,line width=0.38mm,miter limit=10.0] (90.6, 74.75) rectangle (98.32,
   33.24);



  \node[anchor=south] (text115) at (77.3, 139.53){19};



  \node[anchor=south] (text116) at (77.3, 134.47){(6)};



  \node[anchor=south] (text117) at (77.3, 117.8){19};



  \node[anchor=south] (text118) at (77.3, 112.74){(6)};



  \node[anchor=south] (text119) at (77.3, 94.26){22};



  \node[anchor=south] (text120) at (77.3, 89.2){(7)};



  \node[anchor=south] (text121) at (77.3, 58.05){41};



  \node[anchor=south] (text122) at (77.3, 52.99){(13)};



  \node[anchor=south] (text123) at (85.88, 137.67){22};



  \node[anchor=south] (text124) at (85.88, 132.61){(9)};



  \node[anchor=south] (text125) at (85.88, 113.65){20};



  \node[anchor=south] (text126) at (85.88, 108.59){(8)};



  \node[anchor=south] (text127) at (85.88, 92.45){17};



  \node[anchor=south] (text128) at (85.88, 87.39){(7)};



  \node[anchor=south] (text129) at (85.88, 58.54){41};



  \node[anchor=south] (text130) at (85.88, 53.48){(17)};



  \node[anchor=south] (text131) at (94.46, 127.04){40};



  \node[anchor=south] (text132) at (94.46, 121.98){(27)};



  \node[anchor=south,shift={(0.0, -0.53)}] (text133) at (94.46, 100.23){6};



  \node[anchor=south,shift={(0.0, 0.53)}] (text134) at (94.46, 95.18){(4)};



  \node[anchor=south] (text135) at (94.46, 86.4){18};



  \node[anchor=south] (text136) at (94.46, 81.34){(12)};



  \node[anchor=south] (text137) at (94.46, 55.27){36};



  \node[anchor=south] (text138) at (94.46, 50.21){(24)};



  \path[fill=cebebeb,line cap=round,line join=round,line width=0.38mm,miter 
  limit=10.0] (100.1, 154.91) rectangle (127.56, 27.45);



  \path[draw=white,line cap=butt,line join=round,line width=0.19mm,miter 
  limit=10.0] (100.1, 47.73) -- (127.56, 47.73);



  \path[draw=white,line cap=butt,line join=round,line width=0.19mm,miter 
  limit=10.0] (100.1, 76.69) -- (127.56, 76.69);



  \path[draw=white,line cap=butt,line join=round,line width=0.19mm,miter 
  limit=10.0] (100.1, 105.66) -- (127.56, 105.66);



  \path[draw=white,line cap=butt,line join=round,line width=0.19mm,miter 
  limit=10.0] (100.1, 134.63) -- (127.56, 134.63);



  \path[draw=white,line cap=butt,line join=round,line width=0.38mm,miter 
  limit=10.0] (100.1, 33.24) -- (127.56, 33.24);



  \path[draw=white,line cap=butt,line join=round,line width=0.38mm,miter 
  limit=10.0] (100.1, 62.21) -- (127.56, 62.21);



  \path[draw=white,line cap=butt,line join=round,line width=0.38mm,miter 
  limit=10.0] (100.1, 91.18) -- (127.56, 91.18);



  \path[draw=white,line cap=butt,line join=round,line width=0.38mm,miter 
  limit=10.0] (100.1, 120.15) -- (127.56, 120.15);



  \path[draw=white,line cap=butt,line join=round,line width=0.38mm,miter 
  limit=10.0] (100.1, 149.11) -- (127.56, 149.11);



  \path[draw=white,line cap=butt,line join=round,line width=0.38mm,miter 
  limit=10.0] (105.25, 27.45) -- (105.25, 154.91);



  \path[draw=white,line cap=butt,line join=round,line width=0.38mm,miter 
  limit=10.0] (113.83, 27.45) -- (113.83, 154.91);



  \path[draw=white,line cap=butt,line join=round,line width=0.38mm,miter 
  limit=10.0] (122.41, 27.45) -- (122.41, 154.91);



  \path[draw=black,fill=c77aadd,draw opacity=0.85,line cap=butt,line 
  join=miter,line width=0.38mm,miter limit=10.0] (101.39, 149.11) rectangle 
  (109.11, 137.12);



  \path[fill=c99dde1,line cap=butt,line join=miter,line width=0.38mm,miter 
  limit=10.0] ;



  \path[draw=black,fill=ceedd88,draw opacity=0.85,line cap=butt,line 
  join=miter,line width=0.38mm,miter limit=10.0] (101.39, 137.13) rectangle 
  (109.11, 93.18);



  \path[draw=black,fill=cee8866,draw opacity=0.85,line cap=butt,line 
  join=miter,line width=0.38mm,miter limit=10.0] (101.39, 93.18) rectangle 
  (109.11, 33.24);



  \path[draw=black,fill=c77aadd,draw opacity=0.85,line cap=butt,line 
  join=miter,line width=0.38mm,miter limit=10.0] (109.97, 149.11) rectangle 
  (117.69, 138.89);



  \path[draw=black,fill=c99dde1,draw opacity=0.85,line cap=butt,line 
  join=miter,line width=0.38mm,miter limit=10.0] (109.97, 138.89) rectangle 
  (117.69, 135.48);



  \path[draw=black,fill=ceedd88,draw opacity=0.85,line cap=butt,line 
  join=miter,line width=0.38mm,miter limit=10.0] (109.97, 135.48) rectangle 
  (117.69, 101.4);



  \path[draw=black,fill=cee8866,draw opacity=0.85,line cap=butt,line 
  join=miter,line width=0.38mm,miter limit=10.0] (109.97, 101.4) rectangle 
  (117.69, 33.24);



  \path[draw=black,fill=c77aadd,draw opacity=0.85,line cap=butt,line 
  join=miter,line width=0.38mm,miter limit=10.0] (118.55, 149.11) rectangle 
  (126.27, 130.82);



  \path[draw=black,fill=c99dde1,draw opacity=0.85,line cap=butt,line 
  join=miter,line width=0.38mm,miter limit=10.0] (118.55, 130.82) rectangle 
  (126.27, 122.69);



  \path[draw=black,fill=ceedd88,draw opacity=0.85,line cap=butt,line 
  join=miter,line width=0.38mm,miter limit=10.0] (118.55, 122.69) rectangle 
  (126.27, 118.62);



  \path[draw=black,fill=cee8866,draw opacity=0.85,line cap=butt,line 
  join=miter,line width=0.38mm,miter limit=10.0] (118.55, 118.62) rectangle 
  (126.27, 33.24);



  \node[anchor=south] (text162) at (105.25, 144.4){10};



  \node[anchor=south] (text163) at (105.25, 139.34){(3)};



  \node[anchor=south] (text164) at (105.25, 116.43){38};



  \node[anchor=south] (text165) at (105.25, 111.37){(11)};



  \node[anchor=south] (text166) at (105.25, 64.48){52};



  \node[anchor=south] (text167) at (105.25, 59.43){(15)};



  \node[anchor=south,shift={(0.0, 0.53)}] (text168) at (113.83, 145.28){9};



  \node[anchor=south,shift={(0.0, 1.06)}] (text169) at (113.83, 140.22){(3)};



  \node[anchor=south,shift={(0.0, -2.65)}] (text170) at (113.83, 138.46){3};



  \node[anchor=south,shift={(0.0, -1.59)}] (text171) at (113.83, 133.4){(1)};



  \node[anchor=south] (text172) at (113.83, 119.72){29};



  \node[anchor=south] (text173) at (113.83, 114.66){(10)};



  \node[anchor=south] (text174) at (113.83, 68.6){59};



  \node[anchor=south] (text175) at (113.83, 63.54){(20)};



  \node[anchor=south] (text176) at (122.41, 141.24){16};



  \node[anchor=south] (text177) at (122.41, 136.18){(9)};



  \node[anchor=south] (text178) at (122.41, 128.03){7};



  \node[anchor=south,shift={(0.0, 1.59)}] (text179) at (122.41, 122.97){(4)};



  \node[anchor=south,shift={(0.0, -2.65)}] (text180) at (122.41, 121.93){4};



  \node[anchor=south,shift={(0.0, -1.59)}] (text181) at (122.41, 116.87){(2)};



  \node[anchor=south] (text182) at (122.41, 77.21){74};



  \node[anchor=south] (text183) at (122.41, 72.15){(42)};



  \path[fill=cebebeb,line cap=round,line join=round,line width=0.38mm,miter 
  limit=10.0] (128.06, 154.91) rectangle (155.51, 27.45);



  \path[draw=white,line cap=butt,line join=round,line width=0.19mm,miter 
  limit=10.0] (128.06, 47.73) -- (155.51, 47.73);



  \path[draw=white,line cap=butt,line join=round,line width=0.19mm,miter 
  limit=10.0] (128.06, 76.69) -- (155.51, 76.69);



  \path[draw=white,line cap=butt,line join=round,line width=0.19mm,miter 
  limit=10.0] (128.06, 105.66) -- (155.51, 105.66);



  \path[draw=white,line cap=butt,line join=round,line width=0.19mm,miter 
  limit=10.0] (128.06, 134.63) -- (155.51, 134.63);



  \path[draw=white,line cap=butt,line join=round,line width=0.38mm,miter 
  limit=10.0] (128.06, 33.24) -- (155.51, 33.24);



  \path[draw=white,line cap=butt,line join=round,line width=0.38mm,miter 
  limit=10.0] (128.06, 62.21) -- (155.51, 62.21);



  \path[draw=white,line cap=butt,line join=round,line width=0.38mm,miter 
  limit=10.0] (128.06, 91.18) -- (155.51, 91.18);



  \path[draw=white,line cap=butt,line join=round,line width=0.38mm,miter 
  limit=10.0] (128.06, 120.15) -- (155.51, 120.15);



  \path[draw=white,line cap=butt,line join=round,line width=0.38mm,miter 
  limit=10.0] (128.06, 149.11) -- (155.51, 149.11);



  \path[draw=white,line cap=butt,line join=round,line width=0.38mm,miter 
  limit=10.0] (133.21, 27.45) -- (133.21, 154.91);



  \path[draw=white,line cap=butt,line join=round,line width=0.38mm,miter 
  limit=10.0] (141.78, 27.45) -- (141.78, 154.91);



  \path[draw=white,line cap=butt,line join=round,line width=0.38mm,miter 
  limit=10.0] (150.36, 27.45) -- (150.36, 154.91);



  \path[draw=black,fill=c77aadd,draw opacity=0.85,line cap=butt,line 
  join=miter,line width=0.38mm,miter limit=10.0] (129.34, 149.11) rectangle 
  (137.06, 136.44);



  \path[draw=black,fill=c99dde1,draw opacity=0.85,line cap=butt,line 
  join=miter,line width=0.38mm,miter limit=10.0] (129.34, 136.44) rectangle 
  (137.06, 132.82);



  \path[draw=black,fill=ceedd88,draw opacity=0.85,line cap=butt,line 
  join=miter,line width=0.38mm,miter limit=10.0] (129.34, 132.82) rectangle 
  (137.06, 96.61);



  \path[draw=black,fill=cee8866,draw opacity=0.85,line cap=butt,line 
  join=miter,line width=0.38mm,miter limit=10.0] (129.34, 96.61) rectangle 
  (137.06, 33.24);



  \path[draw=black,fill=c77aadd,draw opacity=0.85,line cap=butt,line 
  join=miter,line width=0.38mm,miter limit=10.0] (137.92, 149.11) rectangle 
  (145.64, 134.88);



  \path[draw=black,fill=c99dde1,draw opacity=0.85,line cap=butt,line 
  join=miter,line width=0.38mm,miter limit=10.0] (137.92, 134.88) rectangle 
  (145.64, 126.75);



  \path[draw=black,fill=ceedd88,draw opacity=0.85,line cap=butt,line 
  join=miter,line width=0.38mm,miter limit=10.0] (137.92, 126.75) rectangle 
  (145.64, 88.13);



  \path[draw=black,fill=cee8866,draw opacity=0.85,line cap=butt,line 
  join=miter,line width=0.38mm,miter limit=10.0] (137.92, 88.13) rectangle 
  (145.64, 33.24);



  \path[draw=black,fill=c77aadd,draw opacity=0.85,line cap=butt,line 
  join=miter,line width=0.38mm,miter limit=10.0] (146.5, 149.11) rectangle 
  (154.22, 125.44);



  \path[draw=black,fill=c99dde1,draw opacity=0.85,line cap=butt,line 
  join=miter,line width=0.38mm,miter limit=10.0] (146.5, 125.44) rectangle 
  (154.22, 120.46);



  \path[draw=black,fill=ceedd88,draw opacity=0.85,line cap=butt,line 
  join=miter,line width=0.38mm,miter limit=10.0] (146.5, 120.46) rectangle 
  (154.22, 108.0);



  \path[draw=black,fill=cee8866,draw opacity=0.85,line cap=butt,line 
  join=miter,line width=0.38mm,miter limit=10.0] (146.5, 108.0) rectangle 
  (154.22, 33.24);



  \node[anchor=south] (text207) at (133.21, 144.05){11};



  \node[anchor=south] (text208) at (133.21, 138.99){(7)};



  \node[anchor=south,shift={(0.0, -2.65)}] (text209) at (133.21, 135.9){3};



  \node[anchor=south,shift={(0.0, -1.06)}] (text210) at (133.21, 130.85){(2)};



  \node[anchor=south] (text211) at (133.21, 115.99){31};



  \node[anchor=south] (text212) at (133.21, 110.93){(20)};



  \node[anchor=south] (text213) at (133.21, 66.2){55};



  \node[anchor=south] (text214) at (133.21, 61.14){(35)};



  \node[anchor=south,shift={(0.0, 0.53)}] (text215) at (141.78, 143.27){12};



  \node[anchor=south,shift={(0.0, 0.53)}] (text216) at (141.78, 138.21){(7)};



  \node[anchor=south,shift={(0.0, -0.53)}] (text217) at (141.78, 132.09){7};



  \node[anchor=south,shift={(0.0, 0.53)}] (text218) at (141.78, 127.04){(4)};



  \node[anchor=south] (text219) at (141.78, 108.72){33};



  \node[anchor=south] (text220) at (141.78, 103.66){(19)};



  \node[anchor=south] (text221) at (141.78, 61.96){47};



  \node[anchor=south] (text222) at (141.78, 56.9){(27)};



  \node[anchor=south] (text223) at (150.36, 138.55){20};



  \node[anchor=south] (text224) at (150.36, 133.49){(19)};



  \node[anchor=south,shift={(0.0, 2.12)}] (text225) at (150.36, 124.22){4};



  \node[anchor=south,shift={(0.0, 2.65)}] (text226) at (150.36, 119.16){(4)};



  \node[anchor=south,shift={(0.0, -2.12)}] (text227) at (150.36, 115.5){11};



  \node[anchor=south,shift={(0.0, -1.06)}] (text228) at (150.36, 110.44){(10)};



  \node[anchor=south] (text229) at (150.36, 71.9){65};



  \node[anchor=south] (text230) at (150.36, 66.84){(60)};



  \path[fill=cebebeb,line cap=round,line join=round,line width=0.38mm,miter 
  limit=10.0] (156.01, 154.91) rectangle (183.46, 27.45);



  \path[draw=white,line cap=butt,line join=round,line width=0.19mm,miter 
  limit=10.0] (156.01, 47.73) -- (183.46, 47.73);



  \path[draw=white,line cap=butt,line join=round,line width=0.19mm,miter 
  limit=10.0] (156.01, 76.69) -- (183.46, 76.69);



  \path[draw=white,line cap=butt,line join=round,line width=0.19mm,miter 
  limit=10.0] (156.01, 105.66) -- (183.46, 105.66);



  \path[draw=white,line cap=butt,line join=round,line width=0.19mm,miter 
  limit=10.0] (156.01, 134.63) -- (183.46, 134.63);



  \path[draw=white,line cap=butt,line join=round,line width=0.38mm,miter 
  limit=10.0] (156.01, 33.24) -- (183.46, 33.24);



  \path[draw=white,line cap=butt,line join=round,line width=0.38mm,miter 
  limit=10.0] (156.01, 62.21) -- (183.46, 62.21);



  \path[draw=white,line cap=butt,line join=round,line width=0.38mm,miter 
  limit=10.0] (156.01, 91.18) -- (183.46, 91.18);



  \path[draw=white,line cap=butt,line join=round,line width=0.38mm,miter 
  limit=10.0] (156.01, 120.15) -- (183.46, 120.15);



  \path[draw=white,line cap=butt,line join=round,line width=0.38mm,miter 
  limit=10.0] (156.01, 149.11) -- (183.46, 149.11);



  \path[draw=white,line cap=butt,line join=round,line width=0.38mm,miter 
  limit=10.0] (161.16, 27.45) -- (161.16, 154.91);



  \path[draw=white,line cap=butt,line join=round,line width=0.38mm,miter 
  limit=10.0] (169.74, 27.45) -- (169.74, 154.91);



  \path[draw=white,line cap=butt,line join=round,line width=0.38mm,miter 
  limit=10.0] (178.32, 27.45) -- (178.32, 154.91);



  \path[draw=black,fill=c77aadd,draw opacity=0.85,line cap=butt,line 
  join=miter,line width=0.38mm,miter limit=10.0] (157.3, 149.11) rectangle 
  (165.02, 127.91);



  \path[draw=black,fill=c99dde1,draw opacity=0.85,line cap=butt,line 
  join=miter,line width=0.38mm,miter limit=10.0] (157.3, 127.91) rectangle 
  (165.02, 100.65);



  \path[draw=black,fill=ceedd88,draw opacity=0.85,line cap=butt,line 
  join=miter,line width=0.38mm,miter limit=10.0] (157.3, 100.64) rectangle 
  (165.02, 49.9);



  \path[draw=black,fill=cee8866,draw opacity=0.85,line cap=butt,line 
  join=miter,line width=0.38mm,miter limit=10.0] (157.3, 49.9) rectangle 
  (165.02, 33.24);



  \path[draw=black,fill=c77aadd,draw opacity=0.85,line cap=butt,line 
  join=miter,line width=0.38mm,miter limit=10.0] (165.88, 149.11) rectangle 
  (173.6, 121.8);



  \path[draw=black,fill=c99dde1,draw opacity=0.85,line cap=butt,line 
  join=miter,line width=0.38mm,miter limit=10.0] (165.88, 121.8) rectangle 
  (173.6, 96.97);



  \path[draw=black,fill=ceedd88,draw opacity=0.85,line cap=butt,line 
  join=miter,line width=0.38mm,miter limit=10.0] (165.88, 96.97) rectangle 
  (173.6, 48.14);



  \path[draw=black,fill=cee8866,draw opacity=0.85,line cap=butt,line 
  join=miter,line width=0.38mm,miter limit=10.0] (165.88, 48.14) rectangle 
  (173.6, 33.24);



  \path[draw=black,fill=c77aadd,draw opacity=0.85,line cap=butt,line 
  join=miter,line width=0.38mm,miter limit=10.0] (174.46, 149.11) rectangle 
  (182.18, 119.79);



  \path[draw=black,fill=c99dde1,draw opacity=0.85,line cap=butt,line 
  join=miter,line width=0.38mm,miter limit=10.0] (174.46, 119.79) rectangle 
  (182.18, 98.33);



  \path[draw=black,fill=ceedd88,draw opacity=0.85,line cap=butt,line 
  join=miter,line width=0.38mm,miter limit=10.0] (174.46, 98.33) rectangle 
  (182.18, 55.41);



  \path[draw=black,fill=cee8866,draw opacity=0.85,line cap=butt,line 
  join=miter,line width=0.38mm,miter limit=10.0] (174.46, 55.41) rectangle 
  (182.18, 33.24);



  \node[anchor=south] (text254) at (161.16, 139.78){18};



  \node[anchor=south] (text255) at (161.16, 134.73){(28)};



  \node[anchor=south] (text256) at (161.16, 115.55){24};



  \node[anchor=south] (text257) at (161.16, 110.49){(36)};



  \node[anchor=south] (text258) at (161.16, 76.55){44};



  \node[anchor=south] (text259) at (161.16, 71.49){(67)};



  \node[anchor=south] (text260) at (161.16, 42.85){14};



  \node[anchor=south] (text261) at (161.16, 37.79){(22)};



  \node[anchor=south] (text262) at (169.74, 136.73){24};



  \node[anchor=south] (text263) at (169.74, 131.67){(33)};



  \node[anchor=south] (text264) at (169.74, 110.66){21};



  \node[anchor=south] (text265) at (169.74, 105.6){(30)};



  \node[anchor=south] (text266) at (169.74, 73.83){42};



  \node[anchor=south] (text267) at (169.74, 68.77){(59)};



  \node[anchor=south] (text268) at (169.74, 41.97){13};



  \node[anchor=south] (text269) at (169.74, 36.91){(18)};



  \node[anchor=south] (text270) at (178.32, 135.72){25};



  \node[anchor=south] (text271) at (178.32, 130.67){(41)};



  \node[anchor=south] (text272) at (178.32, 110.33){19};



  \node[anchor=south] (text273) at (178.32, 105.28){(30)};



  \node[anchor=south] (text274) at (178.32, 78.15){37};



  \node[anchor=south] (text275) at (178.32, 73.09){(60)};



  \node[anchor=south] (text276) at (178.32, 45.6){19};



  \node[anchor=south] (text277) at (178.32, 40.54){(31)};



  \path[fill=cebebeb,line cap=round,line join=round,line width=0.38mm,miter 
  limit=10.0] (183.96, 154.91) rectangle (211.42, 27.45);



  \path[draw=white,line cap=butt,line join=round,line width=0.19mm,miter 
  limit=10.0] (183.96, 47.73) -- (211.42, 47.73);



  \path[draw=white,line cap=butt,line join=round,line width=0.19mm,miter 
  limit=10.0] (183.96, 76.69) -- (211.42, 76.69);



  \path[draw=white,line cap=butt,line join=round,line width=0.19mm,miter 
  limit=10.0] (183.96, 105.66) -- (211.42, 105.66);



  \path[draw=white,line cap=butt,line join=round,line width=0.19mm,miter 
  limit=10.0] (183.96, 134.63) -- (211.42, 134.63);



  \path[draw=white,line cap=butt,line join=round,line width=0.38mm,miter 
  limit=10.0] (183.96, 33.24) -- (211.42, 33.24);



  \path[draw=white,line cap=butt,line join=round,line width=0.38mm,miter 
  limit=10.0] (183.96, 62.21) -- (211.42, 62.21);



  \path[draw=white,line cap=butt,line join=round,line width=0.38mm,miter 
  limit=10.0] (183.96, 91.18) -- (211.42, 91.18);



  \path[draw=white,line cap=butt,line join=round,line width=0.38mm,miter 
  limit=10.0] (183.96, 120.15) -- (211.42, 120.15);



  \path[draw=white,line cap=butt,line join=round,line width=0.38mm,miter 
  limit=10.0] (183.96, 149.11) -- (211.42, 149.11);



  \path[draw=white,line cap=butt,line join=round,line width=0.38mm,miter 
  limit=10.0] (189.11, 27.45) -- (189.11, 154.91);



  \path[draw=white,line cap=butt,line join=round,line width=0.38mm,miter 
  limit=10.0] (197.69, 27.45) -- (197.69, 154.91);



  \path[draw=white,line cap=butt,line join=round,line width=0.38mm,miter 
  limit=10.0] (206.27, 27.45) -- (206.27, 154.91);



  \path[draw=black,fill=c77aadd,draw opacity=0.85,line cap=butt,line 
  join=miter,line width=0.38mm,miter limit=10.0] (185.25, 149.11) rectangle 
  (192.97, 120.15);



  \path[draw=black,fill=ceedd88,draw opacity=0.85,line cap=butt,line 
  join=miter,line width=0.38mm,miter limit=10.0] (185.25, 120.15) rectangle 
  (192.97, 86.35);



  \path[draw=black,fill=cee8866,draw opacity=0.85,line cap=butt,line 
  join=miter,line width=0.38mm,miter limit=10.0] (185.25, 86.35) rectangle 
  (192.97, 33.24);



  \path[draw=black,fill=c77aadd,draw opacity=0.85,line cap=butt,line 
  join=miter,line width=0.38mm,miter limit=10.0] (193.83, 149.11) rectangle 
  (201.55, 131.73);



  \path[draw=black,fill=ceedd88,draw opacity=0.85,line cap=butt,line 
  join=miter,line width=0.38mm,miter limit=10.0] (193.83, 131.73) rectangle 
  (201.55, 85.39);



  \path[draw=black,fill=cee8866,draw opacity=0.85,line cap=butt,line 
  join=miter,line width=0.38mm,miter limit=10.0] (193.83, 85.39) rectangle 
  (201.55, 33.25);



  \path[draw=black,fill=c77aadd,draw opacity=0.85,line cap=butt,line 
  join=miter,line width=0.38mm,miter limit=10.0] (202.41, 149.11) rectangle 
  (210.13, 119.07);



  \path[draw=black,fill=ceedd88,draw opacity=0.85,line cap=butt,line 
  join=miter,line width=0.38mm,miter limit=10.0] (202.41, 119.07) rectangle 
  (210.13, 80.45);



  \path[draw=black,fill=cee8866,draw opacity=0.85,line cap=butt,line 
  join=miter,line width=0.38mm,miter limit=10.0] (202.41, 80.45) rectangle 
  (210.13, 33.25);



  \node[anchor=south] (text298) at (189.11, 135.9){25};



  \node[anchor=south] (text299) at (189.11, 130.85){(6)};



  \node[anchor=south] (text300) at (189.11, 104.52){29};



  \node[anchor=south] (text301) at (189.11, 99.46){(7)};



  \node[anchor=south] (text302) at (189.11, 61.07){46};



  \node[anchor=south] (text303) at (189.11, 56.01){(11)};



  \node[anchor=south] (text304) at (197.69, 141.7){15};



  \node[anchor=south] (text305) at (197.69, 136.64){(3)};



  \node[anchor=south] (text306) at (197.69, 109.83){40};



  \node[anchor=south] (text307) at (197.69, 104.78){(8)};



  \node[anchor=south] (text308) at (197.69, 60.59){45};



  \node[anchor=south] (text309) at (197.69, 55.53){(9)};



  \node[anchor=south] (text310) at (206.27, 135.37){26};



  \node[anchor=south] (text311) at (206.27, 130.31){(7)};



  \node[anchor=south] (text312) at (206.27, 101.04){33};



  \node[anchor=south] (text313) at (206.27, 95.98){(9)};



  \node[anchor=south] (text314) at (206.27, 58.12){41};



  \node[anchor=south] (text315) at (206.27, 53.06){(11)};



  \path[fill=cebebeb,line cap=round,line join=round,line width=0.38mm,miter 
  limit=10.0] (211.91, 154.91) rectangle (239.37, 27.45);



  \path[draw=white,line cap=butt,line join=round,line width=0.19mm,miter 
  limit=10.0] (211.91, 47.73) -- (239.37, 47.73);



  \path[draw=white,line cap=butt,line join=round,line width=0.19mm,miter 
  limit=10.0] (211.91, 76.69) -- (239.37, 76.69);



  \path[draw=white,line cap=butt,line join=round,line width=0.19mm,miter 
  limit=10.0] (211.91, 105.66) -- (239.37, 105.66);



  \path[draw=white,line cap=butt,line join=round,line width=0.19mm,miter 
  limit=10.0] (211.91, 134.63) -- (239.37, 134.63);



  \path[draw=white,line cap=butt,line join=round,line width=0.38mm,miter 
  limit=10.0] (211.91, 33.24) -- (239.37, 33.24);



  \path[draw=white,line cap=butt,line join=round,line width=0.38mm,miter 
  limit=10.0] (211.91, 62.21) -- (239.37, 62.21);



  \path[draw=white,line cap=butt,line join=round,line width=0.38mm,miter 
  limit=10.0] (211.91, 91.18) -- (239.37, 91.18);



  \path[draw=white,line cap=butt,line join=round,line width=0.38mm,miter 
  limit=10.0] (211.91, 120.15) -- (239.37, 120.15);



  \path[draw=white,line cap=butt,line join=round,line width=0.38mm,miter 
  limit=10.0] (211.91, 149.11) -- (239.37, 149.11);



  \path[draw=white,line cap=butt,line join=round,line width=0.38mm,miter 
  limit=10.0] (217.06, 27.45) -- (217.06, 154.91);



  \path[draw=white,line cap=butt,line join=round,line width=0.38mm,miter 
  limit=10.0] (225.64, 27.45) -- (225.64, 154.91);



  \path[draw=white,line cap=butt,line join=round,line width=0.38mm,miter 
  limit=10.0] (234.22, 27.45) -- (234.22, 154.91);



  \path[draw=black,fill=c77aadd,draw opacity=0.85,line cap=butt,line 
  join=miter,line width=0.38mm,miter limit=10.0] (213.2, 149.11) rectangle 
  (220.92, 144.97);



  \path[fill=c99dde1,line cap=butt,line join=miter,line width=0.38mm,miter 
  limit=10.0] ;



  \path[draw=black,fill=ceedd88,draw opacity=0.85,line cap=butt,line 
  join=miter,line width=0.38mm,miter limit=10.0] (213.2, 144.97) rectangle 
  (220.92, 107.73);



  \path[draw=black,fill=cee8866,draw opacity=0.85,line cap=butt,line 
  join=miter,line width=0.38mm,miter limit=10.0] (213.2, 107.73) rectangle 
  (220.92, 33.24);



  \path[draw=black,fill=c77aadd,draw opacity=0.85,line cap=butt,line 
  join=miter,line width=0.38mm,miter limit=10.0] (221.78, 149.11) rectangle 
  (229.5, 132.56);



  \path[draw=black,fill=c99dde1,draw opacity=0.85,line cap=butt,line 
  join=miter,line width=0.38mm,miter limit=10.0] (221.78, 132.56) rectangle 
  (229.5, 129.25);



  \path[draw=black,fill=ceedd88,draw opacity=0.85,line cap=butt,line 
  join=miter,line width=0.38mm,miter limit=10.0] (221.78, 129.25) rectangle 
  (229.5, 109.39);



  \path[draw=black,fill=cee8866,draw opacity=0.85,line cap=butt,line 
  join=miter,line width=0.38mm,miter limit=10.0] (221.78, 109.39) rectangle 
  (229.5, 33.24);



  \path[draw=black,fill=c77aadd,draw opacity=0.85,line cap=butt,line 
  join=miter,line width=0.38mm,miter limit=10.0] (230.36, 149.11) rectangle 
  (238.08, 142.09);



  \path[fill=c99dde1,line cap=butt,line join=miter,line width=0.38mm,miter 
  limit=10.0] ;



  \path[draw=black,fill=ceedd88,draw opacity=0.85,line cap=butt,line 
  join=miter,line width=0.38mm,miter limit=10.0] (230.36, 142.09) rectangle 
  (238.08, 110.49);



  \path[draw=black,fill=cee8866,draw opacity=0.85,line cap=butt,line 
  join=miter,line width=0.38mm,miter limit=10.0] (230.36, 110.49) rectangle 
  (238.08, 33.24);



  \node[anchor=south,shift={(0.0, 1.7)}] (text339) at (217.06, 148.32){4};



  \node[anchor=south,shift={(0.0, 2.79)}] (text340) at (217.06, 143.26){(1)};



  \node[anchor=south] (text341) at (217.06, 127.63){32};



  \node[anchor=south] (text342) at (217.06, 122.57){(9)};



  \node[anchor=south] (text343) at (217.06, 71.76){64};



  \node[anchor=south] (text344) at (217.06, 66.7){(18)};



  \node[anchor=south,shift={(0.0, 2.65)}] (text345) at (225.64, 142.11){14};



  \node[anchor=south,shift={(0.0, 2.65)}] (text346) at (225.64, 137.05){(5)};



  \node[anchor=south,shift={(0.0, -2.12)}] (text347) at (225.64, 132.18){3};



  \node[anchor=south,shift={(0.0, -1.06)}] (text348) at (225.64, 127.12){(1)};



  \node[anchor=south,shift={(0.0, -4.23)}] (text349) at (225.64, 120.59){17};



  \node[anchor=south,shift={(0.0, -4.23)}] (text350) at (225.64, 115.53){(6)};



  \node[anchor=south] (text351) at (225.64, 72.59){66};



  \node[anchor=south] (text352) at (225.64, 67.53){(23)};



  \node[anchor=south,shift={(0.0, 3.17)}] (text353) at (234.22, 146.88){6};



  \node[anchor=south,shift={(0.0, 4.23)}] (text354) at (234.22, 141.82){(2)};



  \node[anchor=south] (text355) at (234.22, 127.56){27};



  \node[anchor=south] (text356) at (234.22, 122.51){(9)};



  \node[anchor=south] (text357) at (234.22, 73.14){67};



  \node[anchor=south] (text358) at (234.22, 68.08){(22)};



  \node[text=c1a1a1a,anchor=south] (text359) at (29.97, 156.96){\gls{fakultät2}};



  \node[text=c1a1a1a,anchor=south] (text361) at (57.93, 156.96){\gls{fakultät3}};



  \node[text=c1a1a1a,anchor=south] (text363) at (85.88, 156.96){\gls{fakultät4}};



  \node[text=c1a1a1a,anchor=south] (text365) at (113.83, 156.96){\gls{fakultät6}};



  \node[text=c1a1a1a,anchor=south] (text367) at (141.78, 156.96){\gls{fakultät7}};



  \node[text=c1a1a1a,anchor=south] (text369) at (169.74, 156.96){\gls{fakultät8}};



  \node[text=c1a1a1a,anchor=south] (text371) at (197.69, 156.96){\gls{fakultät9}};



  \node[text=c1a1a1a,anchor=south] (text373) at (225.64, 156.96){\gls{fakultät10}};



  \path[draw=c333333,line cap=butt,line join=round,line width=0.38mm,miter 
  limit=10.0] (21.4, 26.48) -- (21.4, 27.45);



  \path[draw=c333333,line cap=butt,line join=round,line width=0.38mm,miter 
  limit=10.0] (29.97, 26.48) -- (29.97, 27.45);



  \path[draw=c333333,line cap=butt,line join=round,line width=0.38mm,miter 
  limit=10.0] (38.55, 26.48) -- (38.55, 27.45);



  \node[text=c4d4d4d,anchor=south east,cm={ 0.71,0.71,-0.71,0.71,(23.53, 
  -154.23)}] (text376) at (0.0, 177.8){2012-2015};



  \node[text=c4d4d4d,anchor=south east,cm={ 0.71,0.71,-0.71,0.71,(32.11, 
  -154.23)}] (text377) at (0.0, 177.8){2016-2019};



  \node[text=c4d4d4d,anchor=south east,cm={ 0.71,0.71,-0.71,0.71,(40.69, 
  -154.23)}] (text378) at (0.0, 177.8){2020-2023};



  \path[draw=c333333,line cap=butt,line join=round,line width=0.38mm,miter 
  limit=10.0] (49.35, 26.48) -- (49.35, 27.45);



  \path[draw=c333333,line cap=butt,line join=round,line width=0.38mm,miter 
  limit=10.0] (57.93, 26.48) -- (57.93, 27.45);



  \path[draw=c333333,line cap=butt,line join=round,line width=0.38mm,miter 
  limit=10.0] (66.51, 26.48) -- (66.51, 27.45);



  \node[text=c4d4d4d,anchor=south east,cm={ 0.71,0.71,-0.71,0.71,(51.48, 
  -154.23)}] (text380) at (0.0, 177.8){2012-2015};



  \node[text=c4d4d4d,anchor=south east,cm={ 0.71,0.71,-0.71,0.71,(60.06, 
  -154.23)}] (text381) at (0.0, 177.8){2016-2019};



  \node[text=c4d4d4d,anchor=south east,cm={ 0.71,0.71,-0.71,0.71,(68.64, 
  -154.23)}] (text382) at (0.0, 177.8){2020-2023};



  \path[draw=c333333,line cap=butt,line join=round,line width=0.38mm,miter 
  limit=10.0] (77.3, 26.48) -- (77.3, 27.45);



  \path[draw=c333333,line cap=butt,line join=round,line width=0.38mm,miter 
  limit=10.0] (85.88, 26.48) -- (85.88, 27.45);



  \path[draw=c333333,line cap=butt,line join=round,line width=0.38mm,miter 
  limit=10.0] (94.46, 26.48) -- (94.46, 27.45);



  \node[text=c4d4d4d,anchor=south east,cm={ 0.71,0.71,-0.71,0.71,(79.43, 
  -154.23)}] (text384) at (0.0, 177.8){2012-2015};



  \node[text=c4d4d4d,anchor=south east,cm={ 0.71,0.71,-0.71,0.71,(88.01, 
  -154.23)}] (text385) at (0.0, 177.8){2016-2019};



  \node[text=c4d4d4d,anchor=south east,cm={ 0.71,0.71,-0.71,0.71,(96.59, 
  -154.23)}] (text386) at (0.0, 177.8){2020-2023};



  \path[draw=c333333,line cap=butt,line join=round,line width=0.38mm,miter 
  limit=10.0] (105.25, 26.48) -- (105.25, 27.45);



  \path[draw=c333333,line cap=butt,line join=round,line width=0.38mm,miter 
  limit=10.0] (113.83, 26.48) -- (113.83, 27.45);



  \path[draw=c333333,line cap=butt,line join=round,line width=0.38mm,miter 
  limit=10.0] (122.41, 26.48) -- (122.41, 27.45);



  \node[text=c4d4d4d,anchor=south east,cm={ 0.71,0.71,-0.71,0.71,(107.39, 
  -154.23)}] (text388) at (0.0, 177.8){2012-2015};



  \node[text=c4d4d4d,anchor=south east,cm={ 0.71,0.71,-0.71,0.71,(115.97, 
  -154.23)}] (text389) at (0.0, 177.8){2016-2019};



  \node[text=c4d4d4d,anchor=south east,cm={ 0.71,0.71,-0.71,0.71,(124.55, 
  -154.23)}] (text390) at (0.0, 177.8){2020-2023};



  \path[draw=c333333,line cap=butt,line join=round,line width=0.38mm,miter 
  limit=10.0] (133.21, 26.48) -- (133.21, 27.45);



  \path[draw=c333333,line cap=butt,line join=round,line width=0.38mm,miter 
  limit=10.0] (141.78, 26.48) -- (141.78, 27.45);



  \path[draw=c333333,line cap=butt,line join=round,line width=0.38mm,miter 
  limit=10.0] (150.36, 26.48) -- (150.36, 27.45);



  \node[text=c4d4d4d,anchor=south east,cm={ 0.71,0.71,-0.71,0.71,(135.34, 
  -154.23)}] (text392) at (0.0, 177.8){2012-2015};



  \node[text=c4d4d4d,anchor=south east,cm={ 0.71,0.71,-0.71,0.71,(143.92, 
  -154.23)}] (text393) at (0.0, 177.8){2016-2019};



  \node[text=c4d4d4d,anchor=south east,cm={ 0.71,0.71,-0.71,0.71,(152.5, 
  -154.23)}] (text394) at (0.0, 177.8){2020-2023};



  \path[draw=c333333,line cap=butt,line join=round,line width=0.38mm,miter 
  limit=10.0] (161.16, 26.48) -- (161.16, 27.45);



  \path[draw=c333333,line cap=butt,line join=round,line width=0.38mm,miter 
  limit=10.0] (169.74, 26.48) -- (169.74, 27.45);



  \path[draw=c333333,line cap=butt,line join=round,line width=0.38mm,miter 
  limit=10.0] (178.32, 26.48) -- (178.32, 27.45);



  \node[text=c4d4d4d,anchor=south east,cm={ 0.71,0.71,-0.71,0.71,(163.29, 
  -154.23)}] (text396) at (0.0, 177.8){2012-2015};



  \node[text=c4d4d4d,anchor=south east,cm={ 0.71,0.71,-0.71,0.71,(171.87, 
  -154.23)}] (text397) at (0.0, 177.8){2016-2019};



  \node[text=c4d4d4d,anchor=south east,cm={ 0.71,0.71,-0.71,0.71,(180.45, 
  -154.23)}] (text398) at (0.0, 177.8){2020-2023};



  \path[draw=c333333,line cap=butt,line join=round,line width=0.38mm,miter 
  limit=10.0] (189.11, 26.48) -- (189.11, 27.45);



  \path[draw=c333333,line cap=butt,line join=round,line width=0.38mm,miter 
  limit=10.0] (197.69, 26.48) -- (197.69, 27.45);



  \path[draw=c333333,line cap=butt,line join=round,line width=0.38mm,miter 
  limit=10.0] (206.27, 26.48) -- (206.27, 27.45);



  \node[text=c4d4d4d,anchor=south east,cm={ 0.71,0.71,-0.71,0.71,(191.25, 
  -154.23)}] (text400) at (0.0, 177.8){2012-2015};



  \node[text=c4d4d4d,anchor=south east,cm={ 0.71,0.71,-0.71,0.71,(199.82, 
  -154.23)}] (text401) at (0.0, 177.8){2016-2019};



  \node[text=c4d4d4d,anchor=south east,cm={ 0.71,0.71,-0.71,0.71,(208.4, 
  -154.23)}] (text402) at (0.0, 177.8){2020-2023};



  \path[draw=c333333,line cap=butt,line join=round,line width=0.38mm,miter 
  limit=10.0] (217.06, 26.48) -- (217.06, 27.45);



  \path[draw=c333333,line cap=butt,line join=round,line width=0.38mm,miter 
  limit=10.0] (225.64, 26.48) -- (225.64, 27.45);



  \path[draw=c333333,line cap=butt,line join=round,line width=0.38mm,miter 
  limit=10.0] (234.22, 26.48) -- (234.22, 27.45);



  \node[text=c4d4d4d,anchor=south east,cm={ 0.71,0.71,-0.71,0.71,(219.2, 
  -154.23)}] (text404) at (0.0, 177.8){2012-2015};



  \node[text=c4d4d4d,anchor=south east,cm={ 0.71,0.71,-0.71,0.71,(227.78, 
  -154.23)}] (text405) at (0.0, 177.8){2016-2019};



  \node[text=c4d4d4d,anchor=south east,cm={ 0.71,0.71,-0.71,0.71,(236.36, 
  -154.23)}] (text406) at (0.0, 177.8){2020-2023};



  \node[text=c4d4d4d,anchor=south east] (text407) at (14.51, 32.13){0\%};



  \node[text=c4d4d4d,anchor=south east] (text408) at (14.51, 61.1){25\%};



  \node[text=c4d4d4d,anchor=south east] (text409) at (14.51, 90.07){50\%};



  \node[text=c4d4d4d,anchor=south east] (text410) at (14.51, 119.04){75\%};



  \node[text=c4d4d4d,anchor=south east] (text411) at (14.51, 148.0){100\%};



  \path[draw=c333333,line cap=butt,line join=round,line width=0.38mm,miter 
  limit=10.0] (15.28, 33.24) -- (16.25, 33.24);



  \path[draw=c333333,line cap=butt,line join=round,line width=0.38mm,miter 
  limit=10.0] (15.28, 62.21) -- (16.25, 62.21);



  \path[draw=c333333,line cap=butt,line join=round,line width=0.38mm,miter 
  limit=10.0] (15.28, 91.18) -- (16.25, 91.18);



  \path[draw=c333333,line cap=butt,line join=round,line width=0.38mm,miter 
  limit=10.0] (15.28, 120.15) -- (16.25, 120.15);



  \path[draw=c333333,line cap=butt,line join=round,line width=0.38mm,miter 
  limit=10.0] (15.28, 149.11) -- (16.25, 149.11);



  \node[anchor=south,cm={ 0.0,1.0,-1.0,0.0,(4.7, -86.62)}] (text416) at (0.0, 
  177.8){Anteil in Prozenten (\%)};



  \path[fill=cebebeb,line cap=round,line join=round,line width=0.38mm,miter 
  limit=10.0] (86.94, 173.93) rectangle (93.03, 167.84);



  \path[fill=c77aadd,line cap=butt,line join=miter,line width=0.38mm,miter 
  limit=10.0] (87.19, 173.68) rectangle (92.78, 168.09);



  \path[fill=cebebeb,line cap=round,line join=round,line width=0.38mm,miter 
  limit=10.0] (108.42, 173.93) rectangle (114.52, 167.84);



  \path[fill=c99dde1,line cap=butt,line join=miter,line width=0.38mm,miter 
  limit=10.0] (108.67, 173.68) rectangle (114.27, 168.09);



  \path[fill=cebebeb,line cap=round,line join=round,line width=0.38mm,miter 
  limit=10.0] (129.91, 173.93) rectangle (136.01, 167.84);



  \path[fill=ceedd88,line cap=butt,line join=miter,line width=0.38mm,miter 
  limit=10.0] (130.16, 173.68) rectangle (135.76, 168.09);



  \path[fill=cebebeb,line cap=round,line join=round,line width=0.38mm,miter 
  limit=10.0] (151.4, 173.93) rectangle (157.5, 167.84);



  \path[fill=cee8866,line cap=butt,line join=miter,line width=0.38mm,miter 
  limit=10.0] (151.65, 173.68) rectangle (157.24, 168.09);



  \node[anchor=south west] (text424) at (94.96, 169.63){Stufe 1};



  \node[anchor=south west] (text425) at (116.45, 169.63){Stufe 2};



  \node[anchor=south west] (text426) at (137.94, 169.63){Stufe 3};



  \node[anchor=south west] (text427) at (159.43, 169.63){Keine};




\end{tikzpicture}
}
    \caption{\gls{forschungsdaten} für \glspl{pdd} nach Fakultät, Zeitgruppe und Klassifikationsstufe.
    Die Höhe der Barren entsprechen dem relativen Anteil zur jeweiligen angepassten $\text{\textit{Fakultät}}\times\text{\textit{Zeitgruppe}}\times\text{\textit{\gls{forschungsdaten}}}$-Gesamtanzahl.
    Absolute Werte in Klammern angegeben.}
    \label{fig:luh-repo_fakultät_x_zeitgruppe_x_fd}
\end{figure}
Mit $\chi^2 (\num{21}, n=\num{1252}) = \num[round-mode=places,round-precision=2]{277.018852269436}$, $p = \num[round-mode=places,round-precision=2]{1.46707274481795E-46},\phi_C=\num[round-mode=places,round-precision=2]{0.271576302422829}$ ist die Interaktion zwischen \textit{Fakultät} und \textit{Allgemeine \glspl{forschungsdaten}} für \glspl{pdd} statistisch hochsignifikant mit einem Effekt fast-moderater Stärke.

Von den in \cref{fig:luh-repo_fakultät_x_zeitgruppe_x_fd} dargestellten fakultätsspezifischen Interaktionen zwischen \textit{Zeitgruppe} und \textit{Allgemeine \glspl{forschungsdaten}} waren \gls{fakultät6} ($\chi^2 (\num{6}, n=\num{120}) = \num[round-mode=places,round-precision=2]{19.5476964585537}$, $p = \num[round-mode=places,round-precision=2]{0.00333202926016077},\phi_C=\num[round-mode=places,round-precision=2]{0.285392248044641}$) und \gls{fakultät7} ($\chi^2 (\num{6}, n=\num{214}) = \num[round-mode=places,round-precision=2]{16.2992330069594}$, $p = \num[round-mode=places,round-precision=2]{0.0122348687338923},\phi_C=\num[round-mode=places,round-precision=2]{0.195146919293435}$) statistisch signifikant, wobei der Effekt für \gls{fakultät6} etwas stärker ist als für \gls{fakultät7}.
Für die restlichen Fakultäten war die Interaktion statistisch nicht signifkant.

\subsubsection{Integrierte Forschungsdaten}
Die relative und absolute Verteilung von integrierten \glspl{forschungsdaten} für Fakultäten über die verschiedenen Zeitgruppen hinweg ist in \cref{fig:luh-repo_fakultät_x_zeitgruppe_x_intern-fd} dargestellt.
\begin{figure}[!htbp]
    \resizebox{\ifdim\width>\textwidth\textwidth\else\width\fi}{!}{\begin{tikzpicture}[y=1mm, x=1mm, yscale=\globalscale,xscale=\globalscale, every node/.append style={scale=\globalscale}, inner sep=0pt, outer sep=0pt]
  \path[fill=cebebeb,line cap=round,line join=round,line width=0.38mm,miter 
  limit=10.0] (16.25, 154.91) rectangle (43.7, 27.45);



  \path[draw=white,line cap=butt,line join=round,line width=0.19mm,miter 
  limit=10.0] (16.25, 47.73) -- (43.7, 47.73);



  \path[draw=white,line cap=butt,line join=round,line width=0.19mm,miter 
  limit=10.0] (16.25, 76.69) -- (43.7, 76.69);



  \path[draw=white,line cap=butt,line join=round,line width=0.19mm,miter 
  limit=10.0] (16.25, 105.66) -- (43.7, 105.66);



  \path[draw=white,line cap=butt,line join=round,line width=0.19mm,miter 
  limit=10.0] (16.25, 134.63) -- (43.7, 134.63);



  \path[draw=white,line cap=butt,line join=round,line width=0.38mm,miter 
  limit=10.0] (16.25, 33.24) -- (43.7, 33.24);



  \path[draw=white,line cap=butt,line join=round,line width=0.38mm,miter 
  limit=10.0] (16.25, 62.21) -- (43.7, 62.21);



  \path[draw=white,line cap=butt,line join=round,line width=0.38mm,miter 
  limit=10.0] (16.25, 91.18) -- (43.7, 91.18);



  \path[draw=white,line cap=butt,line join=round,line width=0.38mm,miter 
  limit=10.0] (16.25, 120.15) -- (43.7, 120.15);



  \path[draw=white,line cap=butt,line join=round,line width=0.38mm,miter 
  limit=10.0] (16.25, 149.11) -- (43.7, 149.11);



  \path[draw=white,line cap=butt,line join=round,line width=0.38mm,miter 
  limit=10.0] (21.4, 27.45) -- (21.4, 154.91);



  \path[draw=white,line cap=butt,line join=round,line width=0.38mm,miter 
  limit=10.0] (29.97, 27.45) -- (29.97, 154.91);



  \path[draw=white,line cap=butt,line join=round,line width=0.38mm,miter 
  limit=10.0] (38.55, 27.45) -- (38.55, 154.91);



  \path[draw=black,fill=c77aadd,line cap=butt,line join=miter,line 
  width=0.38mm,miter limit=10.0] (17.53, 149.11) rectangle (25.26, 129.8);



  \path[fill=c99dde1,line cap=butt,line join=miter,line width=0.38mm,miter 
  limit=10.0] ;



  \path[draw=black,fill=ceedd88,line cap=butt,line join=miter,line 
  width=0.38mm,miter limit=10.0] (17.53, 129.8) rectangle (25.26, 62.21);



  \path[draw=black,fill=cee8866,line cap=butt,line join=miter,line 
  width=0.38mm,miter limit=10.0] (17.53, 62.21) rectangle (25.26, 33.24);



  \path[draw=black,fill=c77aadd,line cap=butt,line join=miter,line 
  width=0.38mm,miter limit=10.0] (26.11, 149.11) rectangle (33.83, 123.36);



  \path[fill=c99dde1,line cap=butt,line join=miter,line width=0.38mm,miter 
  limit=10.0] ;



  \path[draw=black,fill=ceedd88,line cap=butt,line join=miter,line 
  width=0.38mm,miter limit=10.0] (26.11, 123.37) rectangle (33.83, 59.0);



  \path[draw=black,fill=cee8866,line cap=butt,line join=miter,line 
  width=0.38mm,miter limit=10.0] (26.11, 58.99) rectangle (33.83, 33.24);



  \path[draw=black,fill=c77aadd,line cap=butt,line join=miter,line 
  width=0.38mm,miter limit=10.0] (34.69, 149.11) rectangle (42.41, 143.59);



  \path[fill=c99dde1,line cap=butt,line join=miter,line width=0.38mm,miter 
  limit=10.0] ;



  \path[draw=black,fill=ceedd88,line cap=butt,line join=miter,line 
  width=0.38mm,miter limit=10.0] (34.69, 143.59) rectangle (42.41, 88.42);



  \path[draw=black,fill=cee8866,line cap=butt,line join=miter,line 
  width=0.38mm,miter limit=10.0] (34.69, 88.42) rectangle (42.41, 33.24);



  \node[anchor=south] (text27) at (21.4, 140.73){17};



  \node[anchor=south] (text28) at (21.4, 135.67){(2)};



  \node[anchor=south] (text29) at (21.4, 97.28){58};



  \node[anchor=south] (text30) at (21.4, 92.22){(7)};



  \node[anchor=south] (text31) at (21.4, 49.0){25};



  \node[anchor=south] (text32) at (21.4, 43.94){(3)};



  \node[anchor=south] (text33) at (29.97, 137.51){22};



  \node[anchor=south] (text34) at (29.97, 132.45){(4)};



  \node[anchor=south] (text35) at (29.97, 92.45){56};



  \node[anchor=south] (text36) at (29.97, 87.39){(10)};



  \node[anchor=south] (text37) at (29.97, 47.39){22};



  \node[anchor=south] (text38) at (29.97, 42.33){(4)};



  \node[anchor=south,shift={(0.0, 2.65)}] (text39) at (38.55, 147.63){5};



  \node[anchor=south,shift={(0.0, 2.65)}] (text40) at (38.55, 142.57){(1)};



  \node[anchor=south] (text41) at (38.55, 117.28){48};



  \node[anchor=south] (text42) at (38.55, 112.22){(10)};



  \node[anchor=south] (text43) at (38.55, 62.11){48};



  \node[anchor=south] (text44) at (38.55, 57.05){(10)};



  \path[fill=cebebeb,line cap=round,line join=round,line width=0.38mm,miter 
  limit=10.0] (44.2, 154.91) rectangle (71.65, 27.45);



  \path[draw=white,line cap=butt,line join=round,line width=0.19mm,miter 
  limit=10.0] (44.2, 47.73) -- (71.65, 47.73);



  \path[draw=white,line cap=butt,line join=round,line width=0.19mm,miter 
  limit=10.0] (44.2, 76.69) -- (71.65, 76.69);



  \path[draw=white,line cap=butt,line join=round,line width=0.19mm,miter 
  limit=10.0] (44.2, 105.66) -- (71.65, 105.66);



  \path[draw=white,line cap=butt,line join=round,line width=0.19mm,miter 
  limit=10.0] (44.2, 134.63) -- (71.65, 134.63);



  \path[draw=white,line cap=butt,line join=round,line width=0.38mm,miter 
  limit=10.0] (44.2, 33.24) -- (71.65, 33.24);



  \path[draw=white,line cap=butt,line join=round,line width=0.38mm,miter 
  limit=10.0] (44.2, 62.21) -- (71.65, 62.21);



  \path[draw=white,line cap=butt,line join=round,line width=0.38mm,miter 
  limit=10.0] (44.2, 91.18) -- (71.65, 91.18);



  \path[draw=white,line cap=butt,line join=round,line width=0.38mm,miter 
  limit=10.0] (44.2, 120.15) -- (71.65, 120.15);



  \path[draw=white,line cap=butt,line join=round,line width=0.38mm,miter 
  limit=10.0] (44.2, 149.11) -- (71.65, 149.11);



  \path[draw=white,line cap=butt,line join=round,line width=0.38mm,miter 
  limit=10.0] (49.35, 27.45) -- (49.35, 154.91);



  \path[draw=white,line cap=butt,line join=round,line width=0.38mm,miter 
  limit=10.0] (57.93, 27.45) -- (57.93, 154.91);



  \path[draw=white,line cap=butt,line join=round,line width=0.38mm,miter 
  limit=10.0] (66.51, 27.45) -- (66.51, 154.91);



  \path[draw=black,fill=c77aadd,line cap=butt,line join=miter,line 
  width=0.38mm,miter limit=10.0] (45.49, 149.11) rectangle (53.21, 136.24);



  \path[draw=black,fill=c99dde1,line cap=butt,line join=miter,line 
  width=0.38mm,miter limit=10.0] (45.49, 136.24) rectangle (53.21, 110.49);



  \path[draw=black,fill=ceedd88,line cap=butt,line join=miter,line 
  width=0.38mm,miter limit=10.0] (45.49, 110.49) rectangle (53.21, 84.74);



  \path[draw=black,fill=cee8866,line cap=butt,line join=miter,line 
  width=0.38mm,miter limit=10.0] (45.49, 84.74) rectangle (53.21, 33.24);



  \path[draw=black,fill=c77aadd,line cap=butt,line join=miter,line 
  width=0.38mm,miter limit=10.0] (54.07, 149.11) rectangle (61.79, 121.14);



  \path[draw=black,fill=c99dde1,line cap=butt,line join=miter,line 
  width=0.38mm,miter limit=10.0] (54.07, 121.14) rectangle (61.79, 109.16);



  \path[draw=black,fill=ceedd88,line cap=butt,line join=miter,line 
  width=0.38mm,miter limit=10.0] (54.07, 109.16) rectangle (61.79, 85.18);



  \path[draw=black,fill=cee8866,line cap=butt,line join=miter,line 
  width=0.38mm,miter limit=10.0] (54.07, 85.19) rectangle (61.79, 33.24);



  \path[draw=black,fill=c77aadd,line cap=butt,line join=miter,line 
  width=0.38mm,miter limit=10.0] (62.64, 149.11) rectangle (70.37, 135.13);



  \path[draw=black,fill=c99dde1,line cap=butt,line join=miter,line 
  width=0.38mm,miter limit=10.0] (62.64, 135.13) rectangle (70.37, 111.15);



  \path[draw=black,fill=ceedd88,line cap=butt,line join=miter,line 
  width=0.38mm,miter limit=10.0] (62.64, 111.16) rectangle (70.37, 105.16);



  \path[draw=black,fill=cee8866,line cap=butt,line join=miter,line 
  width=0.38mm,miter limit=10.0] (62.64, 105.16) rectangle (70.37, 33.24);



  \node[anchor=south] (text68) at (49.35, 143.95){11};



  \node[anchor=south] (text69) at (49.35, 138.89){(2)};



  \node[anchor=south] (text70) at (49.35, 124.64){22};



  \node[anchor=south] (text71) at (49.35, 119.58){(4)};



  \node[anchor=south] (text72) at (49.35, 98.89){22};



  \node[anchor=south] (text73) at (49.35, 93.83){(4)};



  \node[anchor=south] (text74) at (49.35, 60.27){44};



  \node[anchor=south] (text75) at (49.35, 55.21){(8)};



  \node[anchor=south] (text76) at (57.93, 136.41){24};



  \node[anchor=south] (text77) at (57.93, 131.35){(7)};



  \node[anchor=south] (text78) at (57.93, 116.43){10};



  \node[anchor=south] (text79) at (57.93, 111.37){(3)};



  \node[anchor=south] (text80) at (57.93, 98.45){21};



  \node[anchor=south] (text81) at (57.93, 93.39){(6)};



  \node[anchor=south] (text82) at (57.93, 60.49){45};



  \node[anchor=south] (text83) at (57.93, 55.43){(13)};



  \node[anchor=south] (text84) at (66.51, 143.4){12};



  \node[anchor=south] (text85) at (66.51, 138.34){(7)};



  \node[anchor=south] (text86) at (66.51, 124.42){21};



  \node[anchor=south] (text87) at (66.51, 119.36){(12)};



  \node[anchor=south,shift={(0.0, -2.65)}] (text88) at (66.51, 109.44){5};



  \node[anchor=south,shift={(0.0, -2.65)}] (text89) at (66.51, 104.38){(3)};



  \node[anchor=south] (text90) at (66.51, 70.48){62};



  \node[anchor=south] (text91) at (66.51, 65.42){(36)};



  \path[fill=cebebeb,line cap=round,line join=round,line width=0.38mm,miter 
  limit=10.0] (72.15, 154.91) rectangle (99.61, 27.45);



  \path[draw=white,line cap=butt,line join=round,line width=0.19mm,miter 
  limit=10.0] (72.15, 47.73) -- (99.6, 47.73);



  \path[draw=white,line cap=butt,line join=round,line width=0.19mm,miter 
  limit=10.0] (72.15, 76.69) -- (99.6, 76.69);



  \path[draw=white,line cap=butt,line join=round,line width=0.19mm,miter 
  limit=10.0] (72.15, 105.66) -- (99.6, 105.66);



  \path[draw=white,line cap=butt,line join=round,line width=0.19mm,miter 
  limit=10.0] (72.15, 134.63) -- (99.6, 134.63);



  \path[draw=white,line cap=butt,line join=round,line width=0.38mm,miter 
  limit=10.0] (72.15, 33.24) -- (99.6, 33.24);



  \path[draw=white,line cap=butt,line join=round,line width=0.38mm,miter 
  limit=10.0] (72.15, 62.21) -- (99.6, 62.21);



  \path[draw=white,line cap=butt,line join=round,line width=0.38mm,miter 
  limit=10.0] (72.15, 91.18) -- (99.6, 91.18);



  \path[draw=white,line cap=butt,line join=round,line width=0.38mm,miter 
  limit=10.0] (72.15, 120.15) -- (99.6, 120.15);



  \path[draw=white,line cap=butt,line join=round,line width=0.38mm,miter 
  limit=10.0] (72.15, 149.11) -- (99.6, 149.11);



  \path[draw=white,line cap=butt,line join=round,line width=0.38mm,miter 
  limit=10.0] (77.3, 27.45) -- (77.3, 154.91);



  \path[draw=white,line cap=butt,line join=round,line width=0.38mm,miter 
  limit=10.0] (85.88, 27.45) -- (85.88, 154.91);



  \path[draw=white,line cap=butt,line join=round,line width=0.38mm,miter 
  limit=10.0] (94.46, 27.45) -- (94.46, 154.91);



  \path[draw=black,fill=c77aadd,line cap=butt,line join=miter,line 
  width=0.38mm,miter limit=10.0] (73.44, 149.11) rectangle (81.16, 127.38);



  \path[draw=black,fill=c99dde1,line cap=butt,line join=miter,line 
  width=0.38mm,miter limit=10.0] (73.44, 127.39) rectangle (81.16, 105.66);



  \path[draw=black,fill=ceedd88,line cap=butt,line join=miter,line 
  width=0.38mm,miter limit=10.0] (73.44, 105.66) rectangle (81.16, 80.31);



  \path[draw=black,fill=cee8866,line cap=butt,line join=miter,line 
  width=0.38mm,miter limit=10.0] (73.44, 80.32) rectangle (81.16, 33.25);



  \path[draw=black,fill=c77aadd,line cap=butt,line join=miter,line 
  width=0.38mm,miter limit=10.0] (82.02, 149.11) rectangle (89.74, 134.98);



  \path[draw=black,fill=c99dde1,line cap=butt,line join=miter,line 
  width=0.38mm,miter limit=10.0] (82.02, 134.98) rectangle (89.74, 106.72);



  \path[draw=black,fill=ceedd88,line cap=butt,line join=miter,line 
  width=0.38mm,miter limit=10.0] (82.02, 106.72) rectangle (89.74, 86.94);



  \path[draw=black,fill=cee8866,line cap=butt,line join=miter,line 
  width=0.38mm,miter limit=10.0] (82.02, 86.94) rectangle (89.74, 33.24);



  \path[draw=black,fill=c77aadd,line cap=butt,line join=miter,line 
  width=0.38mm,miter limit=10.0] (90.6, 149.11) rectangle (98.32, 140.47);



  \path[draw=black,fill=c99dde1,line cap=butt,line join=miter,line 
  width=0.38mm,miter limit=10.0] (90.6, 140.47) rectangle (98.32, 119.71);



  \path[draw=black,fill=ceedd88,line cap=butt,line join=miter,line 
  width=0.38mm,miter limit=10.0] (90.6, 119.72) rectangle (98.32, 97.23);



  \path[draw=black,fill=cee8866,line cap=butt,line join=miter,line 
  width=0.38mm,miter limit=10.0] (90.6, 97.23) rectangle (98.32, 33.24);



  \node[anchor=south] (text115) at (77.3, 139.53){19};



  \node[anchor=south] (text116) at (77.3, 134.47){(6)};



  \node[anchor=south] (text117) at (77.3, 117.8){19};



  \node[anchor=south] (text118) at (77.3, 112.74){(6)};



  \node[anchor=south] (text119) at (77.3, 94.26){22};



  \node[anchor=south] (text120) at (77.3, 89.2){(7)};



  \node[anchor=south] (text121) at (77.3, 58.05){41};



  \node[anchor=south] (text122) at (77.3, 52.99){(13)};



  \node[anchor=south] (text123) at (85.88, 143.32){12};



  \node[anchor=south] (text124) at (85.88, 138.26){(5)};



  \node[anchor=south] (text125) at (85.88, 122.13){24};



  \node[anchor=south] (text126) at (85.88, 117.07){(10)};



  \node[anchor=south] (text127) at (85.88, 98.1){17};



  \node[anchor=south] (text128) at (85.88, 93.05){(7)};



  \node[anchor=south] (text129) at (85.88, 61.37){46};



  \node[anchor=south] (text130) at (85.88, 56.31){(19)};



  \node[anchor=south,shift={(0.0, -0.53)}] (text131) at (94.46, 146.06){7};



  \node[anchor=south,shift={(0.0, 0.53)}] (text132) at (94.46, 141.01){(5)};



  \node[anchor=south] (text133) at (94.46, 131.36){18};



  \node[anchor=south] (text134) at (94.46, 126.31){(12)};



  \node[anchor=south] (text135) at (94.46, 109.75){19};



  \node[anchor=south] (text136) at (94.46, 104.69){(13)};



  \node[anchor=south] (text137) at (94.46, 66.51){55};



  \node[anchor=south] (text138) at (94.46, 61.45){(37)};



  \path[fill=cebebeb,line cap=round,line join=round,line width=0.38mm,miter 
  limit=10.0] (100.1, 154.91) rectangle (127.56, 27.45);



  \path[draw=white,line cap=butt,line join=round,line width=0.19mm,miter 
  limit=10.0] (100.1, 47.73) -- (127.56, 47.73);



  \path[draw=white,line cap=butt,line join=round,line width=0.19mm,miter 
  limit=10.0] (100.1, 76.69) -- (127.56, 76.69);



  \path[draw=white,line cap=butt,line join=round,line width=0.19mm,miter 
  limit=10.0] (100.1, 105.66) -- (127.56, 105.66);



  \path[draw=white,line cap=butt,line join=round,line width=0.19mm,miter 
  limit=10.0] (100.1, 134.63) -- (127.56, 134.63);



  \path[draw=white,line cap=butt,line join=round,line width=0.38mm,miter 
  limit=10.0] (100.1, 33.24) -- (127.56, 33.24);



  \path[draw=white,line cap=butt,line join=round,line width=0.38mm,miter 
  limit=10.0] (100.1, 62.21) -- (127.56, 62.21);



  \path[draw=white,line cap=butt,line join=round,line width=0.38mm,miter 
  limit=10.0] (100.1, 91.18) -- (127.56, 91.18);



  \path[draw=white,line cap=butt,line join=round,line width=0.38mm,miter 
  limit=10.0] (100.1, 120.15) -- (127.56, 120.15);



  \path[draw=white,line cap=butt,line join=round,line width=0.38mm,miter 
  limit=10.0] (100.1, 149.11) -- (127.56, 149.11);



  \path[draw=white,line cap=butt,line join=round,line width=0.38mm,miter 
  limit=10.0] (105.25, 27.45) -- (105.25, 154.91);



  \path[draw=white,line cap=butt,line join=round,line width=0.38mm,miter 
  limit=10.0] (113.83, 27.45) -- (113.83, 154.91);



  \path[draw=white,line cap=butt,line join=round,line width=0.38mm,miter 
  limit=10.0] (122.41, 27.45) -- (122.41, 154.91);



  \path[draw=black,fill=c77aadd,line cap=butt,line join=miter,line 
  width=0.38mm,miter limit=10.0] (101.39, 149.11) rectangle (109.11, 141.12);



  \path[fill=c99dde1,line cap=butt,line join=miter,line width=0.38mm,miter 
  limit=10.0] ;



  \path[draw=black,fill=ceedd88,line cap=butt,line join=miter,line 
  width=0.38mm,miter limit=10.0] (101.39, 141.12) rectangle (109.11, 97.17);



  \path[draw=black,fill=cee8866,line cap=butt,line join=miter,line 
  width=0.38mm,miter limit=10.0] (101.39, 97.17) rectangle (109.11, 33.24);



  \path[draw=black,fill=c77aadd,line cap=butt,line join=miter,line 
  width=0.38mm,miter limit=10.0] (109.97, 149.11) rectangle (117.69, 138.89);



  \path[draw=black,fill=c99dde1,line cap=butt,line join=miter,line 
  width=0.38mm,miter limit=10.0] (109.97, 138.89) rectangle (117.69, 135.48);



  \path[draw=black,fill=ceedd88,line cap=butt,line join=miter,line 
  width=0.38mm,miter limit=10.0] (109.97, 135.48) rectangle (117.69, 101.4);



  \path[draw=black,fill=cee8866,line cap=butt,line join=miter,line 
  width=0.38mm,miter limit=10.0] (109.97, 101.4) rectangle (117.69, 33.24);



  \path[draw=black,fill=c77aadd,line cap=butt,line join=miter,line 
  width=0.38mm,miter limit=10.0] (118.55, 149.11) rectangle (126.27, 134.88);



  \path[draw=black,fill=c99dde1,line cap=butt,line join=miter,line 
  width=0.38mm,miter limit=10.0] (118.55, 134.88) rectangle (126.27, 126.75);



  \path[draw=black,fill=ceedd88,line cap=butt,line join=miter,line 
  width=0.38mm,miter limit=10.0] (118.55, 126.75) rectangle (126.27, 120.65);



  \path[draw=black,fill=cee8866,line cap=butt,line join=miter,line 
  width=0.38mm,miter limit=10.0] (118.55, 120.65) rectangle (126.27, 33.24);



  \node[anchor=south,shift={(0.0, -0.53)}] (text162) at (105.25, 146.39){7};



  \node[anchor=south,shift={(0.0, 0.53)}] (text163) at (105.25, 141.33){(2)};



  \node[anchor=south] (text164) at (105.25, 120.42){38};



  \node[anchor=south] (text165) at (105.25, 115.36){(11)};



  \node[anchor=south] (text166) at (105.25, 66.48){55};



  \node[anchor=south] (text167) at (105.25, 61.42){(16)};



  \node[anchor=south] (text168) at (113.83, 145.28){9};



  \node[anchor=south,shift={(0.0, 1.06)}] (text169) at (113.83, 140.22){(3)};



  \node[anchor=south,shift={(0.0, -2.65)}] (text170) at (113.83, 138.46){3};



  \node[anchor=south,shift={(0.0, -2.65)}] (text171) at (113.83, 133.4){(1)};



  \node[anchor=south] (text172) at (113.83, 119.72){29};



  \node[anchor=south] (text173) at (113.83, 114.66){(10)};



  \node[anchor=south] (text174) at (113.83, 68.6){59};



  \node[anchor=south] (text175) at (113.83, 63.54){(20)};



  \node[anchor=south,shift={(0.0, 1.59)}] (text176) at (122.41, 143.27){12};



  \node[anchor=south,shift={(0.0, 1.59)}] (text177) at (122.41, 138.21){(7)};



  \node[anchor=south,shift={(0.0, -0.53)}] (text178) at (122.41, 132.09){7};



  \node[anchor=south,shift={(0.0, 1.06)}] (text179) at (122.41, 127.04){(4)};



  \node[anchor=south,shift={(0.0, -2.65)}] (text180) at (122.41, 124.98){5};



  \node[anchor=south,shift={(0.0, -2.65)}] (text181) at (122.41, 119.92){(3)};



  \node[anchor=south] (text182) at (122.41, 78.22){75};



  \node[anchor=south] (text183) at (122.41, 73.16){(43)};



  \path[fill=cebebeb,line cap=round,line join=round,line width=0.38mm,miter 
  limit=10.0] (128.06, 154.91) rectangle (155.51, 27.45);



  \path[draw=white,line cap=butt,line join=round,line width=0.19mm,miter 
  limit=10.0] (128.06, 47.73) -- (155.51, 47.73);



  \path[draw=white,line cap=butt,line join=round,line width=0.19mm,miter 
  limit=10.0] (128.06, 76.69) -- (155.51, 76.69);



  \path[draw=white,line cap=butt,line join=round,line width=0.19mm,miter 
  limit=10.0] (128.06, 105.66) -- (155.51, 105.66);



  \path[draw=white,line cap=butt,line join=round,line width=0.19mm,miter 
  limit=10.0] (128.06, 134.63) -- (155.51, 134.63);



  \path[draw=white,line cap=butt,line join=round,line width=0.38mm,miter 
  limit=10.0] (128.06, 33.24) -- (155.51, 33.24);



  \path[draw=white,line cap=butt,line join=round,line width=0.38mm,miter 
  limit=10.0] (128.06, 62.21) -- (155.51, 62.21);



  \path[draw=white,line cap=butt,line join=round,line width=0.38mm,miter 
  limit=10.0] (128.06, 91.18) -- (155.51, 91.18);



  \path[draw=white,line cap=butt,line join=round,line width=0.38mm,miter 
  limit=10.0] (128.06, 120.15) -- (155.51, 120.15);



  \path[draw=white,line cap=butt,line join=round,line width=0.38mm,miter 
  limit=10.0] (128.06, 149.11) -- (155.51, 149.11);



  \path[draw=white,line cap=butt,line join=round,line width=0.38mm,miter 
  limit=10.0] (133.21, 27.45) -- (133.21, 154.91);



  \path[draw=white,line cap=butt,line join=round,line width=0.38mm,miter 
  limit=10.0] (141.78, 27.45) -- (141.78, 154.91);



  \path[draw=white,line cap=butt,line join=round,line width=0.38mm,miter 
  limit=10.0] (150.36, 27.45) -- (150.36, 154.91);



  \path[draw=black,fill=c77aadd,line cap=butt,line join=miter,line 
  width=0.38mm,miter limit=10.0] (129.34, 149.11) rectangle (137.06, 136.44);



  \path[draw=black,fill=c99dde1,line cap=butt,line join=miter,line 
  width=0.38mm,miter limit=10.0] (129.34, 136.44) rectangle (137.06, 132.82);



  \path[draw=black,fill=ceedd88,line cap=butt,line join=miter,line 
  width=0.38mm,miter limit=10.0] (129.34, 132.82) rectangle (137.06, 96.61);



  \path[draw=black,fill=cee8866,line cap=butt,line join=miter,line 
  width=0.38mm,miter limit=10.0] (129.34, 96.61) rectangle (137.06, 33.24);



  \path[draw=black,fill=c77aadd,line cap=butt,line join=miter,line 
  width=0.38mm,miter limit=10.0] (137.92, 149.11) rectangle (145.64, 136.92);



  \path[draw=black,fill=c99dde1,line cap=butt,line join=miter,line 
  width=0.38mm,miter limit=10.0] (137.92, 136.92) rectangle (145.64, 128.79);



  \path[draw=black,fill=ceedd88,line cap=butt,line join=miter,line 
  width=0.38mm,miter limit=10.0] (137.92, 128.79) rectangle (145.64, 90.16);



  \path[draw=black,fill=cee8866,line cap=butt,line join=miter,line 
  width=0.38mm,miter limit=10.0] (137.92, 90.16) rectangle (145.64, 33.24);



  \path[draw=black,fill=c77aadd,line cap=butt,line join=miter,line 
  width=0.38mm,miter limit=10.0] (146.5, 149.11) rectangle (154.22, 139.15);



  \path[draw=black,fill=c99dde1,line cap=butt,line join=miter,line 
  width=0.38mm,miter limit=10.0] (146.5, 139.15) rectangle (154.22, 132.92);



  \path[draw=black,fill=ceedd88,line cap=butt,line join=miter,line 
  width=0.38mm,miter limit=10.0] (146.5, 132.92) rectangle (154.22, 119.21);



  \path[draw=black,fill=cee8866,line cap=butt,line join=miter,line 
  width=0.38mm,miter limit=10.0] (146.5, 119.21) rectangle (154.22, 33.24);



  \node[anchor=south] (text207) at (133.21, 144.05){11};



  \node[anchor=south] (text208) at (133.21, 138.99){(7)};



  \node[anchor=south,shift={(0.0, -2.65)}] (text209) at (133.21, 135.9){3};



  \node[anchor=south,shift={(0.0, -2.12)}] (text210) at (133.21, 130.85){(2)};



  \node[anchor=south] (text211) at (133.21, 115.99){31};



  \node[anchor=south] (text212) at (133.21, 110.93){(20)};



  \node[anchor=south] (text213) at (133.21, 66.2){55};



  \node[anchor=south] (text214) at (133.21, 61.14){(35)};



  \node[anchor=south] (text215) at (141.78, 144.29){11};



  \node[anchor=south] (text216) at (141.78, 139.23){(6)};



  \node[anchor=south,shift={(0.0, -1.06)}] (text217) at (141.78, 134.13){7};



  \node[anchor=south,shift={(0.0, 0.53)}] (text218) at (141.78, 129.07){(4)};



  \node[anchor=south] (text219) at (141.78, 110.75){33};



  \node[anchor=south] (text220) at (141.78, 105.69){(19)};



  \node[anchor=south] (text221) at (141.78, 62.98){49};



  \node[anchor=south] (text222) at (141.78, 57.92){(28)};



  \node[anchor=south] (text223) at (150.36, 145.4){9};



  \node[anchor=south] (text224) at (150.36, 140.35){(8)};



  \node[anchor=south,shift={(0.0, -2.65)}] (text225) at (150.36, 137.3){5};



  \node[anchor=south,shift={(0.0, -2.65)}] (text226) at (150.36, 132.25){(5)};



  \node[anchor=south,shift={(0.0, -6.35)}] (text227) at (150.36, 127.34){12};



  \node[anchor=south,shift={(0.0, -6.88)}] (text228) at (150.36, 122.28){(11)};



  \node[anchor=south] (text229) at (150.36, 77.5){74};



  \node[anchor=south] (text230) at (150.36, 72.44){(69)};



  \path[fill=cebebeb,line cap=round,line join=round,line width=0.38mm,miter 
  limit=10.0] (156.01, 154.91) rectangle (183.46, 27.45);



  \path[draw=white,line cap=butt,line join=round,line width=0.19mm,miter 
  limit=10.0] (156.01, 47.73) -- (183.46, 47.73);



  \path[draw=white,line cap=butt,line join=round,line width=0.19mm,miter 
  limit=10.0] (156.01, 76.69) -- (183.46, 76.69);



  \path[draw=white,line cap=butt,line join=round,line width=0.19mm,miter 
  limit=10.0] (156.01, 105.66) -- (183.46, 105.66);



  \path[draw=white,line cap=butt,line join=round,line width=0.19mm,miter 
  limit=10.0] (156.01, 134.63) -- (183.46, 134.63);



  \path[draw=white,line cap=butt,line join=round,line width=0.38mm,miter 
  limit=10.0] (156.01, 33.24) -- (183.46, 33.24);



  \path[draw=white,line cap=butt,line join=round,line width=0.38mm,miter 
  limit=10.0] (156.01, 62.21) -- (183.46, 62.21);



  \path[draw=white,line cap=butt,line join=round,line width=0.38mm,miter 
  limit=10.0] (156.01, 91.18) -- (183.46, 91.18);



  \path[draw=white,line cap=butt,line join=round,line width=0.38mm,miter 
  limit=10.0] (156.01, 120.15) -- (183.46, 120.15);



  \path[draw=white,line cap=butt,line join=round,line width=0.38mm,miter 
  limit=10.0] (156.01, 149.11) -- (183.46, 149.11);



  \path[draw=white,line cap=butt,line join=round,line width=0.38mm,miter 
  limit=10.0] (161.16, 27.45) -- (161.16, 154.91);



  \path[draw=white,line cap=butt,line join=round,line width=0.38mm,miter 
  limit=10.0] (169.74, 27.45) -- (169.74, 154.91);



  \path[draw=white,line cap=butt,line join=round,line width=0.38mm,miter 
  limit=10.0] (178.32, 27.45) -- (178.32, 154.91);



  \path[draw=black,fill=c77aadd,line cap=butt,line join=miter,line 
  width=0.38mm,miter limit=10.0] (157.3, 149.11) rectangle (165.02, 130.18);



  \path[draw=black,fill=c99dde1,line cap=butt,line join=miter,line 
  width=0.38mm,miter limit=10.0] (157.3, 130.18) rectangle (165.02, 102.92);



  \path[draw=black,fill=ceedd88,line cap=butt,line join=miter,line 
  width=0.38mm,miter limit=10.0] (157.3, 102.92) rectangle (165.02, 51.42);



  \path[draw=black,fill=cee8866,line cap=butt,line join=miter,line 
  width=0.38mm,miter limit=10.0] (157.3, 51.42) rectangle (165.02, 33.24);



  \path[draw=black,fill=c77aadd,line cap=butt,line join=miter,line 
  width=0.38mm,miter limit=10.0] (165.88, 149.11) rectangle (173.6, 130.08);



  \path[draw=black,fill=c99dde1,line cap=butt,line join=miter,line 
  width=0.38mm,miter limit=10.0] (165.88, 130.08) rectangle (173.6, 101.11);



  \path[draw=black,fill=ceedd88,line cap=butt,line join=miter,line 
  width=0.38mm,miter limit=10.0] (165.88, 101.11) rectangle (173.6, 50.62);



  \path[draw=black,fill=cee8866,line cap=butt,line join=miter,line 
  width=0.38mm,miter limit=10.0] (165.88, 50.62) rectangle (173.6, 33.24);



  \path[draw=black,fill=c77aadd,line cap=butt,line join=miter,line 
  width=0.38mm,miter limit=10.0] (174.46, 149.11) rectangle (182.18, 134.81);



  \path[draw=black,fill=c99dde1,line cap=butt,line join=miter,line 
  width=0.38mm,miter limit=10.0] (174.46, 134.81) rectangle (182.18, 109.78);



  \path[draw=black,fill=ceedd88,line cap=butt,line join=miter,line 
  width=0.38mm,miter limit=10.0] (174.46, 109.77) rectangle (182.18, 61.85);



  \path[draw=black,fill=cee8866,line cap=butt,line join=miter,line 
  width=0.38mm,miter limit=10.0] (174.46, 61.85) rectangle (182.18, 33.24);



  \node[anchor=south] (text254) at (161.16, 140.92){16};



  \node[anchor=south] (text255) at (161.16, 135.86){(25)};



  \node[anchor=south] (text256) at (161.16, 117.82){24};



  \node[anchor=south] (text257) at (161.16, 112.77){(36)};



  \node[anchor=south] (text258) at (161.16, 78.44){44};



  \node[anchor=south] (text259) at (161.16, 73.38){(68)};



  \node[anchor=south] (text260) at (161.16, 43.61){16};



  \node[anchor=south] (text261) at (161.16, 38.55){(24)};



  \node[anchor=south] (text262) at (169.74, 140.87){16};



  \node[anchor=south] (text263) at (169.74, 135.81){(23)};



  \node[anchor=south] (text264) at (169.74, 116.87){25};



  \node[anchor=south] (text265) at (169.74, 111.81){(35)};



  \node[anchor=south] (text266) at (169.74, 77.14){44};



  \node[anchor=south] (text267) at (169.74, 72.08){(61)};



  \node[anchor=south] (text268) at (169.74, 43.21){15};



  \node[anchor=south] (text269) at (169.74, 38.15){(21)};



  \node[anchor=south] (text270) at (178.32, 143.23){12};



  \node[anchor=south] (text271) at (178.32, 138.18){(20)};



  \node[anchor=south] (text272) at (178.32, 123.57){22};



  \node[anchor=south] (text273) at (178.32, 118.51){(35)};



  \node[anchor=south] (text274) at (178.32, 87.09){41};



  \node[anchor=south] (text275) at (178.32, 82.03){(67)};



  \node[anchor=south] (text276) at (178.32, 48.82){25};



  \node[anchor=south] (text277) at (178.32, 43.76){(40)};



  \path[fill=cebebeb,line cap=round,line join=round,line width=0.38mm,miter 
  limit=10.0] (183.96, 154.91) rectangle (211.42, 27.45);



  \path[draw=white,line cap=butt,line join=round,line width=0.19mm,miter 
  limit=10.0] (183.96, 47.73) -- (211.42, 47.73);



  \path[draw=white,line cap=butt,line join=round,line width=0.19mm,miter 
  limit=10.0] (183.96, 76.69) -- (211.42, 76.69);



  \path[draw=white,line cap=butt,line join=round,line width=0.19mm,miter 
  limit=10.0] (183.96, 105.66) -- (211.42, 105.66);



  \path[draw=white,line cap=butt,line join=round,line width=0.19mm,miter 
  limit=10.0] (183.96, 134.63) -- (211.42, 134.63);



  \path[draw=white,line cap=butt,line join=round,line width=0.38mm,miter 
  limit=10.0] (183.96, 33.24) -- (211.42, 33.24);



  \path[draw=white,line cap=butt,line join=round,line width=0.38mm,miter 
  limit=10.0] (183.96, 62.21) -- (211.42, 62.21);



  \path[draw=white,line cap=butt,line join=round,line width=0.38mm,miter 
  limit=10.0] (183.96, 91.18) -- (211.42, 91.18);



  \path[draw=white,line cap=butt,line join=round,line width=0.38mm,miter 
  limit=10.0] (183.96, 120.15) -- (211.42, 120.15);



  \path[draw=white,line cap=butt,line join=round,line width=0.38mm,miter 
  limit=10.0] (183.96, 149.11) -- (211.42, 149.11);



  \path[draw=white,line cap=butt,line join=round,line width=0.38mm,miter 
  limit=10.0] (189.11, 27.45) -- (189.11, 154.91);



  \path[draw=white,line cap=butt,line join=round,line width=0.38mm,miter 
  limit=10.0] (197.69, 27.45) -- (197.69, 154.91);



  \path[draw=white,line cap=butt,line join=round,line width=0.38mm,miter 
  limit=10.0] (206.27, 27.45) -- (206.27, 154.91);



  \path[draw=black,fill=c77aadd,line cap=butt,line join=miter,line 
  width=0.38mm,miter limit=10.0] (185.25, 149.11) rectangle (192.97, 124.97);



  \path[draw=black,fill=ceedd88,line cap=butt,line join=miter,line 
  width=0.38mm,miter limit=10.0] (185.25, 124.98) rectangle (192.97, 91.18);



  \path[draw=black,fill=cee8866,line cap=butt,line join=miter,line 
  width=0.38mm,miter limit=10.0] (185.25, 91.18) rectangle (192.97, 33.24);



  \path[draw=black,fill=c77aadd,line cap=butt,line join=miter,line 
  width=0.38mm,miter limit=10.0] (193.83, 149.11) rectangle (201.55, 143.32);



  \path[draw=black,fill=ceedd88,line cap=butt,line join=miter,line 
  width=0.38mm,miter limit=10.0] (193.83, 143.32) rectangle (201.55, 96.97);



  \path[draw=black,fill=cee8866,line cap=butt,line join=miter,line 
  width=0.38mm,miter limit=10.0] (193.83, 96.97) rectangle (201.55, 33.24);



  \path[draw=black,fill=c77aadd,line cap=butt,line join=miter,line 
  width=0.38mm,miter limit=10.0] (202.41, 149.11) rectangle (210.13, 123.36);



  \path[draw=black,fill=ceedd88,line cap=butt,line join=miter,line 
  width=0.38mm,miter limit=10.0] (202.41, 123.37) rectangle (210.13, 84.74);



  \path[draw=black,fill=cee8866,line cap=butt,line join=miter,line 
  width=0.38mm,miter limit=10.0] (202.41, 84.74) rectangle (210.13, 33.24);



  \node[anchor=south] (text298) at (189.11, 138.32){21};



  \node[anchor=south] (text299) at (189.11, 133.26){(5)};



  \node[anchor=south] (text300) at (189.11, 109.35){29};



  \node[anchor=south] (text301) at (189.11, 104.29){(7)};



  \node[anchor=south] (text302) at (189.11, 63.49){50};



  \node[anchor=south] (text303) at (189.11, 58.43){(12)};



  \node[anchor=south,shift={(0.0, 2.65)}] (text304) at (197.69, 147.49){5};



  \node[anchor=south,shift={(0.0, 2.65)}] (text305) at (197.69, 142.43){(1)};



  \node[anchor=south] (text306) at (197.69, 121.42){40};



  \node[anchor=south] (text307) at (197.69, 116.36){(8)};



  \node[anchor=south] (text308) at (197.69, 66.38){55};



  \node[anchor=south] (text309) at (197.69, 61.32){(11)};



  \node[anchor=south] (text310) at (206.27, 137.51){22};



  \node[anchor=south] (text311) at (206.27, 132.45){(6)};



  \node[anchor=south] (text312) at (206.27, 105.33){33};



  \node[anchor=south] (text313) at (206.27, 100.27){(9)};



  \node[anchor=south] (text314) at (206.27, 60.27){44};



  \node[anchor=south] (text315) at (206.27, 55.21){(12)};



  \path[fill=cebebeb,line cap=round,line join=round,line width=0.38mm,miter 
  limit=10.0] (211.91, 154.91) rectangle (239.37, 27.45);



  \path[draw=white,line cap=butt,line join=round,line width=0.19mm,miter 
  limit=10.0] (211.91, 47.73) -- (239.37, 47.73);



  \path[draw=white,line cap=butt,line join=round,line width=0.19mm,miter 
  limit=10.0] (211.91, 76.69) -- (239.37, 76.69);



  \path[draw=white,line cap=butt,line join=round,line width=0.19mm,miter 
  limit=10.0] (211.91, 105.66) -- (239.37, 105.66);



  \path[draw=white,line cap=butt,line join=round,line width=0.19mm,miter 
  limit=10.0] (211.91, 134.63) -- (239.37, 134.63);



  \path[draw=white,line cap=butt,line join=round,line width=0.38mm,miter 
  limit=10.0] (211.91, 33.24) -- (239.37, 33.24);



  \path[draw=white,line cap=butt,line join=round,line width=0.38mm,miter 
  limit=10.0] (211.91, 62.21) -- (239.37, 62.21);



  \path[draw=white,line cap=butt,line join=round,line width=0.38mm,miter 
  limit=10.0] (211.91, 91.18) -- (239.37, 91.18);



  \path[draw=white,line cap=butt,line join=round,line width=0.38mm,miter 
  limit=10.0] (211.91, 120.15) -- (239.37, 120.15);



  \path[draw=white,line cap=butt,line join=round,line width=0.38mm,miter 
  limit=10.0] (211.91, 149.11) -- (239.37, 149.11);



  \path[draw=white,line cap=butt,line join=round,line width=0.38mm,miter 
  limit=10.0] (217.06, 27.45) -- (217.06, 154.91);



  \path[draw=white,line cap=butt,line join=round,line width=0.38mm,miter 
  limit=10.0] (225.64, 27.45) -- (225.64, 154.91);



  \path[draw=white,line cap=butt,line join=round,line width=0.38mm,miter 
  limit=10.0] (234.22, 27.45) -- (234.22, 154.91);



  \path[draw=black,fill=c77aadd,line cap=butt,line join=miter,line 
  width=0.38mm,miter limit=10.0] (213.2, 149.11) rectangle (220.92, 144.97);



  \path[fill=c99dde1,line cap=butt,line join=miter,line width=0.38mm,miter 
  limit=10.0] ;



  \path[draw=black,fill=ceedd88,line cap=butt,line join=miter,line 
  width=0.38mm,miter limit=10.0] (213.2, 144.97) rectangle (220.92, 107.73);



  \path[draw=black,fill=cee8866,line cap=butt,line join=miter,line 
  width=0.38mm,miter limit=10.0] (213.2, 107.73) rectangle (220.92, 33.24);



  \path[draw=black,fill=c77aadd,line cap=butt,line join=miter,line 
  width=0.38mm,miter limit=10.0] (221.78, 149.11) rectangle (229.5, 132.56);



  \path[draw=black,fill=c99dde1,line cap=butt,line join=miter,line 
  width=0.38mm,miter limit=10.0] (221.78, 132.56) rectangle (229.5, 129.25);



  \path[draw=black,fill=ceedd88,line cap=butt,line join=miter,line 
  width=0.38mm,miter limit=10.0] (221.78, 129.25) rectangle (229.5, 109.39);



  \path[draw=black,fill=cee8866,line cap=butt,line join=miter,line 
  width=0.38mm,miter limit=10.0] (221.78, 109.39) rectangle (229.5, 33.24);



  \path[draw=black,fill=c77aadd,line cap=butt,line join=miter,line 
  width=0.38mm,miter limit=10.0] (230.36, 149.11) rectangle (238.08, 145.6);



  \path[fill=c99dde1,line cap=butt,line join=miter,line width=0.38mm,miter 
  limit=10.0] ;



  \path[draw=black,fill=ceedd88,line cap=butt,line join=miter,line 
  width=0.38mm,miter limit=10.0] (230.36, 145.6) rectangle (238.08, 114.0);



  \path[draw=black,fill=cee8866,line cap=butt,line join=miter,line 
  width=0.38mm,miter limit=10.0] (230.36, 114.0) rectangle (238.08, 33.24);



  \node[anchor=south,shift={(0.0, 2.65)}] (text339) at (217.06, 148.32){4};



  \node[anchor=south,shift={(0.0, 2.65)}] (text340) at (217.06, 143.26){(1)};



  \node[anchor=south] (text341) at (217.06, 127.63){32};



  \node[anchor=south] (text342) at (217.06, 122.57){(9)};



  \node[anchor=south] (text343) at (217.06, 71.76){64};



  \node[anchor=south] (text344) at (217.06, 66.7){(18)};



  \node[anchor=south] (text345) at (225.64, 142.11){14};



  \node[anchor=south] (text346) at (225.64, 137.05){(5)};



  \node[anchor=south,shift={(0.0, -2.65)}] (text347) at (225.64, 132.18){3};



  \node[anchor=south,shift={(0.0, -1.59)}] (text348) at (225.64, 127.12){(1)};



  \node[anchor=south,shift={(0.0, -3.7)}] (text349) at (225.64, 120.59){17};



  \node[anchor=south,shift={(0.0, -3.7)}] (text350) at (225.64, 115.53){(6)};



  \node[anchor=south] (text351) at (225.64, 72.59){66};



  \node[anchor=south] (text352) at (225.64, 67.53){(23)};



  \node[anchor=south,shift={(0.0, 2.65)}] (text353) at (234.22, 148.63){3};



  \node[anchor=south,shift={(0.0, 2.65)}] (text354) at (234.22, 143.57){(1)};



  \node[anchor=south] (text355) at (234.22, 131.08){27};



  \node[anchor=south] (text356) at (234.22, 126.02){(9)};



  \node[anchor=south] (text357) at (234.22, 74.9){70};



  \node[anchor=south] (text358) at (234.22, 69.84){(23)};



  \node[text=c1a1a1a,anchor=south] (text359) at (29.97, 156.96){\gls{fakultät2}};



  \node[text=c1a1a1a,anchor=south] (text361) at (57.93, 156.96){\gls{fakultät3}};



  \node[text=c1a1a1a,anchor=south] (text363) at (85.88, 156.96){\gls{fakultät4}};



  \node[text=c1a1a1a,anchor=south] (text365) at (113.83, 156.96){\gls{fakultät6}};



  \node[text=c1a1a1a,anchor=south] (text367) at (141.78, 156.96){\gls{fakultät7}};



  \node[text=c1a1a1a,anchor=south] (text369) at (169.74, 156.96){\gls{fakultät8}};



  \node[text=c1a1a1a,anchor=south] (text371) at (197.69, 156.96){\gls{fakultät9}};



  \node[text=c1a1a1a,anchor=south] (text373) at (225.64, 156.96){\gls{fakultät10}};



  \path[draw=c333333,line cap=butt,line join=round,line width=0.38mm,miter 
  limit=10.0] (21.4, 26.48) -- (21.4, 27.45);



  \path[draw=c333333,line cap=butt,line join=round,line width=0.38mm,miter 
  limit=10.0] (29.97, 26.48) -- (29.97, 27.45);



  \path[draw=c333333,line cap=butt,line join=round,line width=0.38mm,miter 
  limit=10.0] (38.55, 26.48) -- (38.55, 27.45);



  \node[text=c4d4d4d,anchor=south east,cm={ 0.71,0.71,-0.71,0.71,(23.53, 
  -154.23)}] (text376) at (0.0, 177.8){2012-2015};



  \node[text=c4d4d4d,anchor=south east,cm={ 0.71,0.71,-0.71,0.71,(32.11, 
  -154.23)}] (text377) at (0.0, 177.8){2016-2019};



  \node[text=c4d4d4d,anchor=south east,cm={ 0.71,0.71,-0.71,0.71,(40.69, 
  -154.23)}] (text378) at (0.0, 177.8){2020-2023};



  \path[draw=c333333,line cap=butt,line join=round,line width=0.38mm,miter 
  limit=10.0] (49.35, 26.48) -- (49.35, 27.45);



  \path[draw=c333333,line cap=butt,line join=round,line width=0.38mm,miter 
  limit=10.0] (57.93, 26.48) -- (57.93, 27.45);



  \path[draw=c333333,line cap=butt,line join=round,line width=0.38mm,miter 
  limit=10.0] (66.51, 26.48) -- (66.51, 27.45);



  \node[text=c4d4d4d,anchor=south east,cm={ 0.71,0.71,-0.71,0.71,(51.48, 
  -154.23)}] (text380) at (0.0, 177.8){2012-2015};



  \node[text=c4d4d4d,anchor=south east,cm={ 0.71,0.71,-0.71,0.71,(60.06, 
  -154.23)}] (text381) at (0.0, 177.8){2016-2019};



  \node[text=c4d4d4d,anchor=south east,cm={ 0.71,0.71,-0.71,0.71,(68.64, 
  -154.23)}] (text382) at (0.0, 177.8){2020-2023};



  \path[draw=c333333,line cap=butt,line join=round,line width=0.38mm,miter 
  limit=10.0] (77.3, 26.48) -- (77.3, 27.45);



  \path[draw=c333333,line cap=butt,line join=round,line width=0.38mm,miter 
  limit=10.0] (85.88, 26.48) -- (85.88, 27.45);



  \path[draw=c333333,line cap=butt,line join=round,line width=0.38mm,miter 
  limit=10.0] (94.46, 26.48) -- (94.46, 27.45);



  \node[text=c4d4d4d,anchor=south east,cm={ 0.71,0.71,-0.71,0.71,(79.43, 
  -154.23)}] (text384) at (0.0, 177.8){2012-2015};



  \node[text=c4d4d4d,anchor=south east,cm={ 0.71,0.71,-0.71,0.71,(88.01, 
  -154.23)}] (text385) at (0.0, 177.8){2016-2019};



  \node[text=c4d4d4d,anchor=south east,cm={ 0.71,0.71,-0.71,0.71,(96.59, 
  -154.23)}] (text386) at (0.0, 177.8){2020-2023};



  \path[draw=c333333,line cap=butt,line join=round,line width=0.38mm,miter 
  limit=10.0] (105.25, 26.48) -- (105.25, 27.45);



  \path[draw=c333333,line cap=butt,line join=round,line width=0.38mm,miter 
  limit=10.0] (113.83, 26.48) -- (113.83, 27.45);



  \path[draw=c333333,line cap=butt,line join=round,line width=0.38mm,miter 
  limit=10.0] (122.41, 26.48) -- (122.41, 27.45);



  \node[text=c4d4d4d,anchor=south east,cm={ 0.71,0.71,-0.71,0.71,(107.39, 
  -154.23)}] (text388) at (0.0, 177.8){2012-2015};



  \node[text=c4d4d4d,anchor=south east,cm={ 0.71,0.71,-0.71,0.71,(115.97, 
  -154.23)}] (text389) at (0.0, 177.8){2016-2019};



  \node[text=c4d4d4d,anchor=south east,cm={ 0.71,0.71,-0.71,0.71,(124.55, 
  -154.23)}] (text390) at (0.0, 177.8){2020-2023};



  \path[draw=c333333,line cap=butt,line join=round,line width=0.38mm,miter 
  limit=10.0] (133.21, 26.48) -- (133.21, 27.45);



  \path[draw=c333333,line cap=butt,line join=round,line width=0.38mm,miter 
  limit=10.0] (141.78, 26.48) -- (141.78, 27.45);



  \path[draw=c333333,line cap=butt,line join=round,line width=0.38mm,miter 
  limit=10.0] (150.36, 26.48) -- (150.36, 27.45);



  \node[text=c4d4d4d,anchor=south east,cm={ 0.71,0.71,-0.71,0.71,(135.34, 
  -154.23)}] (text392) at (0.0, 177.8){2012-2015};



  \node[text=c4d4d4d,anchor=south east,cm={ 0.71,0.71,-0.71,0.71,(143.92, 
  -154.23)}] (text393) at (0.0, 177.8){2016-2019};



  \node[text=c4d4d4d,anchor=south east,cm={ 0.71,0.71,-0.71,0.71,(152.5, 
  -154.23)}] (text394) at (0.0, 177.8){2020-2023};



  \path[draw=c333333,line cap=butt,line join=round,line width=0.38mm,miter 
  limit=10.0] (161.16, 26.48) -- (161.16, 27.45);



  \path[draw=c333333,line cap=butt,line join=round,line width=0.38mm,miter 
  limit=10.0] (169.74, 26.48) -- (169.74, 27.45);



  \path[draw=c333333,line cap=butt,line join=round,line width=0.38mm,miter 
  limit=10.0] (178.32, 26.48) -- (178.32, 27.45);



  \node[text=c4d4d4d,anchor=south east,cm={ 0.71,0.71,-0.71,0.71,(163.29, 
  -154.23)}] (text396) at (0.0, 177.8){2012-2015};



  \node[text=c4d4d4d,anchor=south east,cm={ 0.71,0.71,-0.71,0.71,(171.87, 
  -154.23)}] (text397) at (0.0, 177.8){2016-2019};



  \node[text=c4d4d4d,anchor=south east,cm={ 0.71,0.71,-0.71,0.71,(180.45, 
  -154.23)}] (text398) at (0.0, 177.8){2020-2023};



  \path[draw=c333333,line cap=butt,line join=round,line width=0.38mm,miter 
  limit=10.0] (189.11, 26.48) -- (189.11, 27.45);



  \path[draw=c333333,line cap=butt,line join=round,line width=0.38mm,miter 
  limit=10.0] (197.69, 26.48) -- (197.69, 27.45);



  \path[draw=c333333,line cap=butt,line join=round,line width=0.38mm,miter 
  limit=10.0] (206.27, 26.48) -- (206.27, 27.45);



  \node[text=c4d4d4d,anchor=south east,cm={ 0.71,0.71,-0.71,0.71,(191.25, 
  -154.23)}] (text400) at (0.0, 177.8){2012-2015};



  \node[text=c4d4d4d,anchor=south east,cm={ 0.71,0.71,-0.71,0.71,(199.82, 
  -154.23)}] (text401) at (0.0, 177.8){2016-2019};



  \node[text=c4d4d4d,anchor=south east,cm={ 0.71,0.71,-0.71,0.71,(208.4, 
  -154.23)}] (text402) at (0.0, 177.8){2020-2023};



  \path[draw=c333333,line cap=butt,line join=round,line width=0.38mm,miter 
  limit=10.0] (217.06, 26.48) -- (217.06, 27.45);



  \path[draw=c333333,line cap=butt,line join=round,line width=0.38mm,miter 
  limit=10.0] (225.64, 26.48) -- (225.64, 27.45);



  \path[draw=c333333,line cap=butt,line join=round,line width=0.38mm,miter 
  limit=10.0] (234.22, 26.48) -- (234.22, 27.45);



  \node[text=c4d4d4d,anchor=south east,cm={ 0.71,0.71,-0.71,0.71,(219.2, 
  -154.23)}] (text404) at (0.0, 177.8){2012-2015};



  \node[text=c4d4d4d,anchor=south east,cm={ 0.71,0.71,-0.71,0.71,(227.78, 
  -154.23)}] (text405) at (0.0, 177.8){2016-2019};



  \node[text=c4d4d4d,anchor=south east,cm={ 0.71,0.71,-0.71,0.71,(236.36, 
  -154.23)}] (text406) at (0.0, 177.8){2020-2023};



  \node[text=c4d4d4d,anchor=south east] (text407) at (14.51, 32.13){0\%};



  \node[text=c4d4d4d,anchor=south east] (text408) at (14.51, 61.1){25\%};



  \node[text=c4d4d4d,anchor=south east] (text409) at (14.51, 90.07){50\%};



  \node[text=c4d4d4d,anchor=south east] (text410) at (14.51, 119.04){75\%};



  \node[text=c4d4d4d,anchor=south east] (text411) at (14.51, 148.0){100\%};



  \path[draw=c333333,line cap=butt,line join=round,line width=0.38mm,miter 
  limit=10.0] (15.28, 33.24) -- (16.25, 33.24);



  \path[draw=c333333,line cap=butt,line join=round,line width=0.38mm,miter 
  limit=10.0] (15.28, 62.21) -- (16.25, 62.21);



  \path[draw=c333333,line cap=butt,line join=round,line width=0.38mm,miter 
  limit=10.0] (15.28, 91.18) -- (16.25, 91.18);



  \path[draw=c333333,line cap=butt,line join=round,line width=0.38mm,miter 
  limit=10.0] (15.28, 120.15) -- (16.25, 120.15);



  \path[draw=c333333,line cap=butt,line join=round,line width=0.38mm,miter 
  limit=10.0] (15.28, 149.11) -- (16.25, 149.11);



  \node[anchor=south,cm={ 0.0,1.0,-1.0,0.0,(4.7, -86.62)}] (text416) at (0.0, 
  177.8){Anteil in Prozenten (\%)};



  \path[fill=cebebeb,line cap=round,line join=round,line width=0.38mm,miter 
  limit=10.0] (86.94, 173.93) rectangle (93.03, 167.84);



  \path[fill=c77aadd,line cap=butt,line join=miter,line width=0.38mm,miter 
  limit=10.0] (87.19, 173.68) rectangle (92.78, 168.09);



  \path[fill=cebebeb,line cap=round,line join=round,line width=0.38mm,miter 
  limit=10.0] (108.42, 173.93) rectangle (114.52, 167.84);



  \path[fill=c99dde1,line cap=butt,line join=miter,line width=0.38mm,miter 
  limit=10.0] (108.67, 173.68) rectangle (114.27, 168.09);



  \path[fill=cebebeb,line cap=round,line join=round,line width=0.38mm,miter 
  limit=10.0] (129.91, 173.93) rectangle (136.01, 167.84);



  \path[fill=ceedd88,line cap=butt,line join=miter,line width=0.38mm,miter 
  limit=10.0] (130.16, 173.68) rectangle (135.76, 168.09);



  \path[fill=cebebeb,line cap=round,line join=round,line width=0.38mm,miter 
  limit=10.0] (151.4, 173.93) rectangle (157.5, 167.84);



  \path[fill=cee8866,line cap=butt,line join=miter,line width=0.38mm,miter 
  limit=10.0] (151.65, 173.68) rectangle (157.24, 168.09);



  \node[anchor=south west] (text424) at (94.96, 169.63){Stufe 1};



  \node[anchor=south west] (text425) at (116.45, 169.63){Stufe 2};



  \node[anchor=south west] (text426) at (137.94, 169.63){Stufe 3};



  \node[anchor=south west] (text427) at (159.43, 169.63){Keine};




\end{tikzpicture}
}
    \caption{Integrierte \gls{forschungsdaten} für \glspl{pdd} nach Fakultät, Zeitgruppe und Klassifikationsstufe.
    Die Höhe der Barren entsprechen dem relativen Anteil zur jeweiligen angepassten $\text{\textit{Fakultät}}\times\text{\textit{Zeitgruppe}}\times\text{\textit{Integrierte \gls{forschungsdaten}}}$-Gesamtanzahl.
    Absolute Werte in Klammern angegeben.}
    \label{fig:luh-repo_fakultät_x_zeitgruppe_x_intern-fd}
\end{figure}
Mit $\chi^2 (\num{21}, n=\num{1252}) = \num[round-mode=places,round-precision=2]{263.002145678598}$, $p = \num[round-mode=places,round-precision=2]{9.94476930299567E-44},\phi_C=\num[round-mode=places,round-precision=2]{0.264616459279196}$ ist die Interaktion zwischen \textit{Fakultät} und \textit{Integrierten \glspl{forschungsdaten}} für \glspl{pdd} statistisch hochsignifikant mit einem Effekt fast-moderater Stärke.

Von den in \cref{fig:luh-repo_fakultät_x_zeitgruppe_x_intern-fd} dargestellten fakultätsspezifischen Interaktionen zwischen \textit{Zeitgruppe} und \textit{Allgemeine \glspl{forschungsdaten}} waren auch hier nur \gls{fakultät6} ($\chi^2 (\num{6}, n=\num{120}) = \num[round-mode=places,round-precision=2]{16.9762583339752}$, $p = \num[round-mode=places,round-precision=2]{0.00937089563575401},\phi_C=\num[round-mode=places,round-precision=2]{0.265959413679788}$) und \gls{fakultät7} ($\chi^2 (\num{6}, n=\num{214}) = \num[round-mode=places,round-precision=2]{14.9081855229165}$, $p = \num[round-mode=places,round-precision=2]{0.0209829722338535},\phi_C=\num[round-mode=places,round-precision=2]{0.186633890721093}$) statistisch signifikant, wobei der Effekt für \gls{fakultät6} etwas stärker ist als für \gls{fakultät7}.
Hier ist feststellbar, dass die Interaktionen sowohl weniger signifikant wie auch schwächer sind, als die respektiven Interaktionen zwischen \textit{Zeitgruppe} und \textit{Integrierte \gls{forschungsdaten}}.
Für die restlichen Fakultäten war die Interaktion statistisch nicht signifkant.

\subsubsection{Begleitende Forschungsdaten}
Die relative und absolute Verteilung von begleitenden \glspl{forschungsdaten} für Fakultäten über die verschiedenen Zeitgruppen hinweg ist in \cref{fig:luh-repo_fakultät_x_zeitgruppe_x_begleit-fd} dargestellt.
\begin{figure}[!htbp]
    \resizebox{\ifdim\width>\textwidth\textwidth\else\width\fi}{!}{\begin{tikzpicture}[y=1mm, x=1mm, yscale=\globalscale,xscale=\globalscale, every node/.append style={scale=\globalscale}, inner sep=0pt, outer sep=0pt]
  \path[fill=cebebeb,line cap=round,line join=round,line width=0.38mm,miter 
  limit=10.0] (16.25, 154.91) rectangle (43.7, 27.45);



  \path[draw=white,line cap=butt,line join=round,line width=0.19mm,miter 
  limit=10.0] (16.25, 47.73) -- (43.7, 47.73);



  \path[draw=white,line cap=butt,line join=round,line width=0.19mm,miter 
  limit=10.0] (16.25, 76.69) -- (43.7, 76.69);



  \path[draw=white,line cap=butt,line join=round,line width=0.19mm,miter 
  limit=10.0] (16.25, 105.66) -- (43.7, 105.66);



  \path[draw=white,line cap=butt,line join=round,line width=0.19mm,miter 
  limit=10.0] (16.25, 134.63) -- (43.7, 134.63);



  \path[draw=white,line cap=butt,line join=round,line width=0.38mm,miter 
  limit=10.0] (16.25, 33.24) -- (43.7, 33.24);



  \path[draw=white,line cap=butt,line join=round,line width=0.38mm,miter 
  limit=10.0] (16.25, 62.21) -- (43.7, 62.21);



  \path[draw=white,line cap=butt,line join=round,line width=0.38mm,miter 
  limit=10.0] (16.25, 91.18) -- (43.7, 91.18);



  \path[draw=white,line cap=butt,line join=round,line width=0.38mm,miter 
  limit=10.0] (16.25, 120.15) -- (43.7, 120.15);



  \path[draw=white,line cap=butt,line join=round,line width=0.38mm,miter 
  limit=10.0] (16.25, 149.11) -- (43.7, 149.11);



  \path[draw=white,line cap=butt,line join=round,line width=0.38mm,miter 
  limit=10.0] (21.4, 27.45) -- (21.4, 154.91);



  \path[draw=white,line cap=butt,line join=round,line width=0.38mm,miter 
  limit=10.0] (29.97, 27.45) -- (29.97, 154.91);



  \path[draw=white,line cap=butt,line join=round,line width=0.38mm,miter 
  limit=10.0] (38.55, 27.45) -- (38.55, 154.91);



  \path[fill=c77aadd,line cap=butt,line join=miter,line width=0.38mm,miter 
  limit=10.0] ;



  \path[fill=c99dde1,line cap=butt,line join=miter,line width=0.38mm,miter 
  limit=10.0] ;



  \path[fill=ceedd88,line cap=butt,line join=miter,line width=0.38mm,miter 
  limit=10.0] ;



  \path[draw=black,fill=cee8866,line cap=butt,line join=miter,line 
  width=0.38mm,miter limit=10.0] (17.53, 149.11) rectangle (25.26, 33.24);



  \path[fill=c77aadd,line cap=butt,line join=miter,line width=0.38mm,miter 
  limit=10.0] ;



  \path[fill=c99dde1,line cap=butt,line join=miter,line width=0.38mm,miter 
  limit=10.0] ;



  \path[fill=ceedd88,line cap=butt,line join=miter,line width=0.38mm,miter 
  limit=10.0] ;



  \path[draw=black,fill=cee8866,line cap=butt,line join=miter,line 
  width=0.38mm,miter limit=10.0] (26.11, 149.11) rectangle (33.83, 33.24);



  \path[draw=black,fill=c77aadd,line cap=butt,line join=miter,line 
  width=0.38mm,miter limit=10.0] (34.69, 149.11) rectangle (42.41, 143.59);



  \path[fill=c99dde1,line cap=butt,line join=miter,line width=0.38mm,miter 
  limit=10.0] ;



  \path[draw=black,fill=ceedd88,line cap=butt,line join=miter,line 
  width=0.38mm,miter limit=10.0] (34.69, 143.59) rectangle (42.41, 138.08);



  \path[draw=black,fill=cee8866,line cap=butt,line join=miter,line 
  width=0.38mm,miter limit=10.0] (34.69, 138.08) rectangle (42.41, 33.24);



  \node[anchor=south] (text27) at (21.4, 92.45){100};



  \node[anchor=south] (text28) at (21.4, 87.39){(12)};



  \node[anchor=south] (text29) at (29.97, 92.45){100};



  \node[anchor=south] (text30) at (29.97, 87.39){(18)};



  \node[anchor=south,shift={(0.0, 2.65)}] (text31) at (38.55, 147.63){5};



  \node[anchor=south,shift={(0.0, 2.65)}] (text32) at (38.55, 142.57){(1)};



  \node[anchor=south,shift={(0.0, -2.65)}] (text33) at (38.55, 142.11){5};



  \node[anchor=south,shift={(0.0, -2.65)}] (text34) at (38.55, 137.05){(1)};



  \node[anchor=south] (text35) at (38.55, 86.94){90};



  \node[anchor=south] (text36) at (38.55, 81.88){(19)};



  \path[fill=cebebeb,line cap=round,line join=round,line width=0.38mm,miter 
  limit=10.0] (44.2, 154.91) rectangle (71.65, 27.45);



  \path[draw=white,line cap=butt,line join=round,line width=0.19mm,miter 
  limit=10.0] (44.2, 47.73) -- (71.65, 47.73);



  \path[draw=white,line cap=butt,line join=round,line width=0.19mm,miter 
  limit=10.0] (44.2, 76.69) -- (71.65, 76.69);



  \path[draw=white,line cap=butt,line join=round,line width=0.19mm,miter 
  limit=10.0] (44.2, 105.66) -- (71.65, 105.66);



  \path[draw=white,line cap=butt,line join=round,line width=0.19mm,miter 
  limit=10.0] (44.2, 134.63) -- (71.65, 134.63);



  \path[draw=white,line cap=butt,line join=round,line width=0.38mm,miter 
  limit=10.0] (44.2, 33.24) -- (71.65, 33.24);



  \path[draw=white,line cap=butt,line join=round,line width=0.38mm,miter 
  limit=10.0] (44.2, 62.21) -- (71.65, 62.21);



  \path[draw=white,line cap=butt,line join=round,line width=0.38mm,miter 
  limit=10.0] (44.2, 91.18) -- (71.65, 91.18);



  \path[draw=white,line cap=butt,line join=round,line width=0.38mm,miter 
  limit=10.0] (44.2, 120.15) -- (71.65, 120.15);



  \path[draw=white,line cap=butt,line join=round,line width=0.38mm,miter 
  limit=10.0] (44.2, 149.11) -- (71.65, 149.11);



  \path[draw=white,line cap=butt,line join=round,line width=0.38mm,miter 
  limit=10.0] (49.35, 27.45) -- (49.35, 154.91);



  \path[draw=white,line cap=butt,line join=round,line width=0.38mm,miter 
  limit=10.0] (57.93, 27.45) -- (57.93, 154.91);



  \path[draw=white,line cap=butt,line join=round,line width=0.38mm,miter 
  limit=10.0] (66.51, 27.45) -- (66.51, 154.91);



  \path[fill=c77aadd,line cap=butt,line join=miter,line width=0.38mm,miter 
  limit=10.0] ;



  \path[draw=black,fill=cee8866,line cap=butt,line join=miter,line 
  width=0.38mm,miter limit=10.0] (45.49, 149.11) rectangle (53.21, 33.24);



  \path[fill=c77aadd,line cap=butt,line join=miter,line width=0.38mm,miter 
  limit=10.0] ;



  \path[draw=black,fill=cee8866,line cap=butt,line join=miter,line 
  width=0.38mm,miter limit=10.0] (54.07, 149.11) rectangle (61.79, 33.24);



  \path[draw=black,fill=c77aadd,line cap=butt,line join=miter,line 
  width=0.38mm,miter limit=10.0] (62.64, 149.11) rectangle (70.37, 147.12);



  \path[draw=black,fill=cee8866,line cap=butt,line join=miter,line 
  width=0.38mm,miter limit=10.0] (62.64, 147.12) rectangle (70.37, 33.24);



  \node[anchor=south] (text54) at (49.35, 92.45){100};



  \node[anchor=south] (text55) at (49.35, 87.39){(18)};



  \node[anchor=south] (text56) at (57.93, 92.45){100};



  \node[anchor=south] (text57) at (57.93, 87.39){(29)};



  \node[anchor=south,shift={(0.0, 2.65)}] (text58) at (66.51, 149.39){2};



  \node[anchor=south,shift={(0.0, 2.65)}] (text59) at (66.51, 144.33){(1)};



  \node[anchor=south] (text60) at (66.51, 91.45){98};



  \node[anchor=south] (text61) at (66.51, 86.4){(57)};



  \path[fill=cebebeb,line cap=round,line join=round,line width=0.38mm,miter 
  limit=10.0] (72.15, 154.91) rectangle (99.61, 27.45);



  \path[draw=white,line cap=butt,line join=round,line width=0.19mm,miter 
  limit=10.0] (72.15, 47.73) -- (99.6, 47.73);



  \path[draw=white,line cap=butt,line join=round,line width=0.19mm,miter 
  limit=10.0] (72.15, 76.69) -- (99.6, 76.69);



  \path[draw=white,line cap=butt,line join=round,line width=0.19mm,miter 
  limit=10.0] (72.15, 105.66) -- (99.6, 105.66);



  \path[draw=white,line cap=butt,line join=round,line width=0.19mm,miter 
  limit=10.0] (72.15, 134.63) -- (99.6, 134.63);



  \path[draw=white,line cap=butt,line join=round,line width=0.38mm,miter 
  limit=10.0] (72.15, 33.24) -- (99.6, 33.24);



  \path[draw=white,line cap=butt,line join=round,line width=0.38mm,miter 
  limit=10.0] (72.15, 62.21) -- (99.6, 62.21);



  \path[draw=white,line cap=butt,line join=round,line width=0.38mm,miter 
  limit=10.0] (72.15, 91.18) -- (99.6, 91.18);



  \path[draw=white,line cap=butt,line join=round,line width=0.38mm,miter 
  limit=10.0] (72.15, 120.15) -- (99.6, 120.15);



  \path[draw=white,line cap=butt,line join=round,line width=0.38mm,miter 
  limit=10.0] (72.15, 149.11) -- (99.6, 149.11);



  \path[draw=white,line cap=butt,line join=round,line width=0.38mm,miter 
  limit=10.0] (77.3, 27.45) -- (77.3, 154.91);



  \path[draw=white,line cap=butt,line join=round,line width=0.38mm,miter 
  limit=10.0] (85.88, 27.45) -- (85.88, 154.91);



  \path[draw=white,line cap=butt,line join=round,line width=0.38mm,miter 
  limit=10.0] (94.46, 27.45) -- (94.46, 154.91);



  \path[draw=black,fill=c77aadd,line cap=butt,line join=miter,line 
  width=0.38mm,miter limit=10.0] (73.44, 149.11) rectangle (81.16, 145.49);



  \path[draw=black,fill=cee8866,line cap=butt,line join=miter,line 
  width=0.38mm,miter limit=10.0] (73.44, 145.49) rectangle (81.16, 33.24);



  \path[fill=c77aadd,line cap=butt,line join=miter,line width=0.38mm,miter 
  limit=10.0] ;



  \path[draw=black,fill=cee8866,line cap=butt,line join=miter,line 
  width=0.38mm,miter limit=10.0] (82.02, 149.11) rectangle (89.74, 33.24);



  \path[fill=c77aadd,line cap=butt,line join=miter,line width=0.38mm,miter 
  limit=10.0] ;



  \path[draw=black,fill=cee8866,line cap=butt,line join=miter,line 
  width=0.38mm,miter limit=10.0] (90.6, 149.11) rectangle (98.32, 33.24);



  \node[anchor=south,shift={(0.0, 3.17)}] (text79) at (77.3, 148.58){3};



  \node[anchor=south,shift={(0.0, 3.17)}] (text80) at (77.3, 143.52){(1)};



  \node[anchor=south] (text81) at (77.3, 90.64){97};



  \node[anchor=south] (text82) at (77.3, 85.58){(31)};



  \node[anchor=south] (text83) at (85.88, 92.45){100};



  \node[anchor=south] (text84) at (85.88, 87.39){(41)};



  \node[anchor=south] (text85) at (94.46, 92.45){100};



  \node[anchor=south] (text86) at (94.46, 87.39){(67)};



  \path[fill=cebebeb,line cap=round,line join=round,line width=0.38mm,miter 
  limit=10.0] (100.1, 154.91) rectangle (127.56, 27.45);



  \path[draw=white,line cap=butt,line join=round,line width=0.19mm,miter 
  limit=10.0] (100.1, 47.73) -- (127.56, 47.73);



  \path[draw=white,line cap=butt,line join=round,line width=0.19mm,miter 
  limit=10.0] (100.1, 76.69) -- (127.56, 76.69);



  \path[draw=white,line cap=butt,line join=round,line width=0.19mm,miter 
  limit=10.0] (100.1, 105.66) -- (127.56, 105.66);



  \path[draw=white,line cap=butt,line join=round,line width=0.19mm,miter 
  limit=10.0] (100.1, 134.63) -- (127.56, 134.63);



  \path[draw=white,line cap=butt,line join=round,line width=0.38mm,miter 
  limit=10.0] (100.1, 33.24) -- (127.56, 33.24);



  \path[draw=white,line cap=butt,line join=round,line width=0.38mm,miter 
  limit=10.0] (100.1, 62.21) -- (127.56, 62.21);



  \path[draw=white,line cap=butt,line join=round,line width=0.38mm,miter 
  limit=10.0] (100.1, 91.18) -- (127.56, 91.18);



  \path[draw=white,line cap=butt,line join=round,line width=0.38mm,miter 
  limit=10.0] (100.1, 120.15) -- (127.56, 120.15);



  \path[draw=white,line cap=butt,line join=round,line width=0.38mm,miter 
  limit=10.0] (100.1, 149.11) -- (127.56, 149.11);



  \path[draw=white,line cap=butt,line join=round,line width=0.38mm,miter 
  limit=10.0] (105.25, 27.45) -- (105.25, 154.91);



  \path[draw=white,line cap=butt,line join=round,line width=0.38mm,miter 
  limit=10.0] (113.83, 27.45) -- (113.83, 154.91);



  \path[draw=white,line cap=butt,line join=round,line width=0.38mm,miter 
  limit=10.0] (122.41, 27.45) -- (122.41, 154.91);



  \path[draw=black,fill=c77aadd,line cap=butt,line join=miter,line 
  width=0.38mm,miter limit=10.0] (101.39, 149.11) rectangle (109.11, 145.12);



  \path[draw=black,fill=cee8866,line cap=butt,line join=miter,line 
  width=0.38mm,miter limit=10.0] (101.39, 145.12) rectangle (109.11, 33.24);



  \path[fill=c77aadd,line cap=butt,line join=miter,line width=0.38mm,miter 
  limit=10.0] ;



  \path[draw=black,fill=cee8866,line cap=butt,line join=miter,line 
  width=0.38mm,miter limit=10.0] (109.97, 149.11) rectangle (117.69, 33.24);



  \path[fill=c77aadd,line cap=butt,line join=miter,line width=0.38mm,miter 
  limit=10.0] ;



  \path[draw=black,fill=cee8866,line cap=butt,line join=miter,line 
  width=0.38mm,miter limit=10.0] (118.55, 149.11) rectangle (126.27, 33.24);



  \node[anchor=south,shift={(0.0, 2.65)}] (text104) at (105.25, 148.39){3};



  \node[anchor=south,shift={(0.0, 2.65)}] (text105) at (105.25, 143.33){(1)};



  \node[anchor=south] (text106) at (105.25, 90.46){97};



  \node[anchor=south] (text107) at (105.25, 85.4){(28)};



  \node[anchor=south] (text108) at (113.83, 92.45){100};



  \node[anchor=south] (text109) at (113.83, 87.39){(34)};



  \node[anchor=south] (text110) at (122.41, 92.45){100};



  \node[anchor=south] (text111) at (122.41, 87.39){(57)};



  \path[fill=cebebeb,line cap=round,line join=round,line width=0.38mm,miter 
  limit=10.0] (128.06, 154.91) rectangle (155.51, 27.45);



  \path[draw=white,line cap=butt,line join=round,line width=0.19mm,miter 
  limit=10.0] (128.06, 47.73) -- (155.51, 47.73);



  \path[draw=white,line cap=butt,line join=round,line width=0.19mm,miter 
  limit=10.0] (128.06, 76.69) -- (155.51, 76.69);



  \path[draw=white,line cap=butt,line join=round,line width=0.19mm,miter 
  limit=10.0] (128.06, 105.66) -- (155.51, 105.66);



  \path[draw=white,line cap=butt,line join=round,line width=0.19mm,miter 
  limit=10.0] (128.06, 134.63) -- (155.51, 134.63);



  \path[draw=white,line cap=butt,line join=round,line width=0.38mm,miter 
  limit=10.0] (128.06, 33.24) -- (155.51, 33.24);



  \path[draw=white,line cap=butt,line join=round,line width=0.38mm,miter 
  limit=10.0] (128.06, 62.21) -- (155.51, 62.21);



  \path[draw=white,line cap=butt,line join=round,line width=0.38mm,miter 
  limit=10.0] (128.06, 91.18) -- (155.51, 91.18);



  \path[draw=white,line cap=butt,line join=round,line width=0.38mm,miter 
  limit=10.0] (128.06, 120.15) -- (155.51, 120.15);



  \path[draw=white,line cap=butt,line join=round,line width=0.38mm,miter 
  limit=10.0] (128.06, 149.11) -- (155.51, 149.11);



  \path[draw=white,line cap=butt,line join=round,line width=0.38mm,miter 
  limit=10.0] (133.21, 27.45) -- (133.21, 154.91);



  \path[draw=white,line cap=butt,line join=round,line width=0.38mm,miter 
  limit=10.0] (141.78, 27.45) -- (141.78, 154.91);



  \path[draw=white,line cap=butt,line join=round,line width=0.38mm,miter 
  limit=10.0] (150.36, 27.45) -- (150.36, 154.91);



  \path[fill=c77aadd,line cap=butt,line join=miter,line width=0.38mm,miter 
  limit=10.0] ;



  \path[draw=black,fill=cee8866,line cap=butt,line join=miter,line 
  width=0.38mm,miter limit=10.0] (129.34, 149.11) rectangle (137.06, 33.24);



  \path[draw=black,fill=c77aadd,line cap=butt,line join=miter,line 
  width=0.38mm,miter limit=10.0] (137.92, 149.11) rectangle (145.64, 147.08);



  \path[draw=black,fill=cee8866,line cap=butt,line join=miter,line 
  width=0.38mm,miter limit=10.0] (137.92, 147.08) rectangle (145.64, 33.24);



  \path[fill=c77aadd,line cap=butt,line join=miter,line width=0.38mm,miter 
  limit=10.0] ;



  \path[draw=black,fill=cee8866,line cap=butt,line join=miter,line 
  width=0.38mm,miter limit=10.0] (146.5, 149.11) rectangle (154.22, 33.24);



  \node[anchor=south] (text129) at (133.21, 92.45){100};



  \node[anchor=south] (text130) at (133.21, 87.39){(64)};



  \node[anchor=south] (text131) at (141.78, 149.37){2};



  \node[anchor=south] (text132) at (141.78, 144.31){(1)};



  \node[anchor=south] (text133) at (141.78, 91.44){98};



  \node[anchor=south] (text134) at (141.78, 86.38){(56)};



  \node[anchor=south] (text135) at (150.36, 92.45){100};



  \node[anchor=south] (text136) at (150.36, 87.39){(93)};



  \path[fill=cebebeb,line cap=round,line join=round,line width=0.38mm,miter 
  limit=10.0] (156.01, 154.91) rectangle (183.46, 27.45);



  \path[draw=white,line cap=butt,line join=round,line width=0.19mm,miter 
  limit=10.0] (156.01, 47.73) -- (183.46, 47.73);



  \path[draw=white,line cap=butt,line join=round,line width=0.19mm,miter 
  limit=10.0] (156.01, 76.69) -- (183.46, 76.69);



  \path[draw=white,line cap=butt,line join=round,line width=0.19mm,miter 
  limit=10.0] (156.01, 105.66) -- (183.46, 105.66);



  \path[draw=white,line cap=butt,line join=round,line width=0.19mm,miter 
  limit=10.0] (156.01, 134.63) -- (183.46, 134.63);



  \path[draw=white,line cap=butt,line join=round,line width=0.38mm,miter 
  limit=10.0] (156.01, 33.24) -- (183.46, 33.24);



  \path[draw=white,line cap=butt,line join=round,line width=0.38mm,miter 
  limit=10.0] (156.01, 62.21) -- (183.46, 62.21);



  \path[draw=white,line cap=butt,line join=round,line width=0.38mm,miter 
  limit=10.0] (156.01, 91.18) -- (183.46, 91.18);



  \path[draw=white,line cap=butt,line join=round,line width=0.38mm,miter 
  limit=10.0] (156.01, 120.15) -- (183.46, 120.15);



  \path[draw=white,line cap=butt,line join=round,line width=0.38mm,miter 
  limit=10.0] (156.01, 149.11) -- (183.46, 149.11);



  \path[draw=white,line cap=butt,line join=round,line width=0.38mm,miter 
  limit=10.0] (161.16, 27.45) -- (161.16, 154.91);



  \path[draw=white,line cap=butt,line join=round,line width=0.38mm,miter 
  limit=10.0] (169.74, 27.45) -- (169.74, 154.91);



  \path[draw=white,line cap=butt,line join=round,line width=0.38mm,miter 
  limit=10.0] (178.32, 27.45) -- (178.32, 154.91);



  \path[draw=black,fill=c77aadd,line cap=butt,line join=miter,line 
  width=0.38mm,miter limit=10.0] (157.3, 149.11) rectangle (165.02, 147.6);



  \path[fill=ceedd88,line cap=butt,line join=miter,line width=0.38mm,miter 
  limit=10.0] ;



  \path[draw=black,fill=cee8866,line cap=butt,line join=miter,line 
  width=0.38mm,miter limit=10.0] (157.3, 147.6) rectangle (165.02, 33.24);



  \path[draw=black,fill=c77aadd,line cap=butt,line join=miter,line 
  width=0.38mm,miter limit=10.0] (165.88, 149.11) rectangle (173.6, 144.15);



  \path[draw=black,fill=ceedd88,line cap=butt,line join=miter,line 
  width=0.38mm,miter limit=10.0] (165.88, 144.15) rectangle (173.6, 143.32);



  \path[draw=black,fill=cee8866,line cap=butt,line join=miter,line 
  width=0.38mm,miter limit=10.0] (165.88, 143.32) rectangle (173.6, 33.24);



  \path[draw=black,fill=c77aadd,line cap=butt,line join=miter,line 
  width=0.38mm,miter limit=10.0] (174.46, 149.11) rectangle (182.18, 146.25);



  \path[draw=black,fill=ceedd88,line cap=butt,line join=miter,line 
  width=0.38mm,miter limit=10.0] (174.46, 146.25) rectangle (182.18, 144.82);



  \path[draw=black,fill=cee8866,line cap=butt,line join=miter,line 
  width=0.38mm,miter limit=10.0] (174.46, 144.82) rectangle (182.18, 33.24);



  \node[anchor=south] (text157) at (161.16, 149.63){1};



  \node[anchor=south] (text158) at (161.16, 144.57){(2)};



  \node[anchor=south] (text159) at (161.16, 91.69){99};



  \node[anchor=south] (text160) at (161.16, 86.64){(151)};



  \node[anchor=south,shift={(0.0, 2.65)}] (text161) at (169.74, 147.91){4};



  \node[anchor=south,shift={(0.0, 2.65)}] (text162) at (169.74, 142.85){(6)};



  \node[anchor=south,shift={(0.0, -4.76)}] (text163) at (169.74, 145.01){1};



  \node[anchor=south,shift={(0.0, -4.76)}] (text164) at (169.74, 139.95){(1)};



  \node[anchor=south] (text165) at (169.74, 89.56){95};



  \node[anchor=south] (text166) at (169.74, 84.5){(133)};



  \node[anchor=south,shift={(0.0, 2.65)}] (text167) at (178.32, 148.96){2};



  \node[anchor=south,shift={(0.0, 2.65)}] (text168) at (178.32, 143.9){(4)};



  \node[anchor=south,shift={(0.0, -5.29)}] (text169) at (178.32, 146.81){1};



  \node[anchor=south,shift={(0.0, -5.29)}] (text170) at (178.32, 141.75){(2)};



  \node[anchor=south] (text171) at (178.32, 90.31){96};



  \node[anchor=south] (text172) at (178.32, 85.25){(156)};



  \path[fill=cebebeb,line cap=round,line join=round,line width=0.38mm,miter 
  limit=10.0] (183.96, 154.91) rectangle (211.42, 27.45);



  \path[draw=white,line cap=butt,line join=round,line width=0.19mm,miter 
  limit=10.0] (183.96, 47.73) -- (211.42, 47.73);



  \path[draw=white,line cap=butt,line join=round,line width=0.19mm,miter 
  limit=10.0] (183.96, 76.69) -- (211.42, 76.69);



  \path[draw=white,line cap=butt,line join=round,line width=0.19mm,miter 
  limit=10.0] (183.96, 105.66) -- (211.42, 105.66);



  \path[draw=white,line cap=butt,line join=round,line width=0.19mm,miter 
  limit=10.0] (183.96, 134.63) -- (211.42, 134.63);



  \path[draw=white,line cap=butt,line join=round,line width=0.38mm,miter 
  limit=10.0] (183.96, 33.24) -- (211.42, 33.24);



  \path[draw=white,line cap=butt,line join=round,line width=0.38mm,miter 
  limit=10.0] (183.96, 62.21) -- (211.42, 62.21);



  \path[draw=white,line cap=butt,line join=round,line width=0.38mm,miter 
  limit=10.0] (183.96, 91.18) -- (211.42, 91.18);



  \path[draw=white,line cap=butt,line join=round,line width=0.38mm,miter 
  limit=10.0] (183.96, 120.15) -- (211.42, 120.15);



  \path[draw=white,line cap=butt,line join=round,line width=0.38mm,miter 
  limit=10.0] (183.96, 149.11) -- (211.42, 149.11);



  \path[draw=white,line cap=butt,line join=round,line width=0.38mm,miter 
  limit=10.0] (189.11, 27.45) -- (189.11, 154.91);



  \path[draw=white,line cap=butt,line join=round,line width=0.38mm,miter 
  limit=10.0] (197.69, 27.45) -- (197.69, 154.91);



  \path[draw=white,line cap=butt,line join=round,line width=0.38mm,miter 
  limit=10.0] (206.27, 27.45) -- (206.27, 154.91);



  \path[draw=black,fill=c77aadd,line cap=butt,line join=miter,line 
  width=0.38mm,miter limit=10.0] (185.25, 149.11) rectangle (192.97, 144.28);



  \path[draw=black,fill=ceedd88,line cap=butt,line join=miter,line 
  width=0.38mm,miter limit=10.0] (185.25, 144.29) rectangle (192.97, 134.63);



  \path[draw=black,fill=cee8866,line cap=butt,line join=miter,line 
  width=0.38mm,miter limit=10.0] (185.25, 134.63) rectangle (192.97, 33.24);



  \path[draw=black,fill=c77aadd,line cap=butt,line join=miter,line 
  width=0.38mm,miter limit=10.0] (193.83, 149.11) rectangle (201.55, 137.52);



  \path[fill=ceedd88,line cap=butt,line join=miter,line width=0.38mm,miter 
  limit=10.0] ;



  \path[draw=black,fill=cee8866,line cap=butt,line join=miter,line 
  width=0.38mm,miter limit=10.0] (193.83, 137.53) rectangle (201.55, 33.24);



  \path[draw=black,fill=c77aadd,line cap=butt,line join=miter,line 
  width=0.38mm,miter limit=10.0] (202.41, 149.11) rectangle (210.13, 144.82);



  \path[fill=ceedd88,line cap=butt,line join=miter,line width=0.38mm,miter 
  limit=10.0] ;



  \path[draw=black,fill=cee8866,line cap=butt,line join=miter,line 
  width=0.38mm,miter limit=10.0] (202.41, 144.82) rectangle (210.13, 33.24);



  \node[anchor=south,shift={(0.0, 2.65)}] (text193) at (189.11, 147.98){4};



  \node[anchor=south,shift={(0.0, 2.65)}] (text194) at (189.11, 142.92){(1)};



  \node[anchor=south] (text195) at (189.11, 140.73){8};



  \node[anchor=south] (text196) at (189.11, 135.67){(2)};



  \node[anchor=south] (text197) at (189.11, 85.21){88};



  \node[anchor=south] (text198) at (189.11, 80.15){(21)};



  \node[anchor=south] (text199) at (197.69, 144.6){10};



  \node[anchor=south] (text200) at (197.69, 139.54){(2)};



  \node[anchor=south] (text201) at (197.69, 86.66){90};



  \node[anchor=south] (text202) at (197.69, 81.6){(18)};



  \node[anchor=south,shift={(0.0, 2.65)}] (text203) at (206.27, 148.24){4};



  \node[anchor=south,shift={(0.0, 2.65)}] (text204) at (206.27, 143.19){(1)};



  \node[anchor=south] (text205) at (206.27, 90.31){96};



  \node[anchor=south] (text206) at (206.27, 85.25){(26)};



  \path[fill=cebebeb,line cap=round,line join=round,line width=0.38mm,miter 
  limit=10.0] (211.91, 154.91) rectangle (239.37, 27.45);



  \path[draw=white,line cap=butt,line join=round,line width=0.19mm,miter 
  limit=10.0] (211.91, 47.73) -- (239.37, 47.73);



  \path[draw=white,line cap=butt,line join=round,line width=0.19mm,miter 
  limit=10.0] (211.91, 76.69) -- (239.37, 76.69);



  \path[draw=white,line cap=butt,line join=round,line width=0.19mm,miter 
  limit=10.0] (211.91, 105.66) -- (239.37, 105.66);



  \path[draw=white,line cap=butt,line join=round,line width=0.19mm,miter 
  limit=10.0] (211.91, 134.63) -- (239.37, 134.63);



  \path[draw=white,line cap=butt,line join=round,line width=0.38mm,miter 
  limit=10.0] (211.91, 33.24) -- (239.37, 33.24);



  \path[draw=white,line cap=butt,line join=round,line width=0.38mm,miter 
  limit=10.0] (211.91, 62.21) -- (239.37, 62.21);



  \path[draw=white,line cap=butt,line join=round,line width=0.38mm,miter 
  limit=10.0] (211.91, 91.18) -- (239.37, 91.18);



  \path[draw=white,line cap=butt,line join=round,line width=0.38mm,miter 
  limit=10.0] (211.91, 120.15) -- (239.37, 120.15);



  \path[draw=white,line cap=butt,line join=round,line width=0.38mm,miter 
  limit=10.0] (211.91, 149.11) -- (239.37, 149.11);



  \path[draw=white,line cap=butt,line join=round,line width=0.38mm,miter 
  limit=10.0] (217.06, 27.45) -- (217.06, 154.91);



  \path[draw=white,line cap=butt,line join=round,line width=0.38mm,miter 
  limit=10.0] (225.64, 27.45) -- (225.64, 154.91);



  \path[draw=white,line cap=butt,line join=round,line width=0.38mm,miter 
  limit=10.0] (234.22, 27.45) -- (234.22, 154.91);



  \path[fill=c77aadd,line cap=butt,line join=miter,line width=0.38mm,miter 
  limit=10.0] ;



  \path[draw=black,fill=cee8866,line cap=butt,line join=miter,line 
  width=0.38mm,miter limit=10.0] (213.2, 149.11) rectangle (220.92, 33.24);



  \path[draw=black,fill=c77aadd,line cap=butt,line join=miter,line 
  width=0.38mm,miter limit=10.0] (221.78, 149.11) rectangle (229.5, 145.8);



  \path[draw=black,fill=cee8866,line cap=butt,line join=miter,line 
  width=0.38mm,miter limit=10.0] (221.78, 145.8) rectangle (229.5, 33.24);



  \path[fill=c77aadd,line cap=butt,line join=miter,line width=0.38mm,miter 
  limit=10.0] ;



  \path[draw=black,fill=cee8866,line cap=butt,line join=miter,line 
  width=0.38mm,miter limit=10.0] (230.36, 149.11) rectangle (238.08, 33.24);



  \node[anchor=south] (text224) at (217.06, 92.45){100};



  \node[anchor=south] (text225) at (217.06, 87.39){(28)};



  \node[anchor=south,shift={(0.0, 2.65)}] (text226) at (225.64, 148.73){3};



  \node[anchor=south,shift={(0.0, 2.65)}] (text227) at (225.64, 143.68){(1)};



  \node[anchor=south] (text228) at (225.64, 90.8){97};



  \node[anchor=south] (text229) at (225.64, 85.74){(34)};



  \node[anchor=south] (text230) at (234.22, 92.45){100};



  \node[anchor=south] (text231) at (234.22, 87.39){(33)};



  \node[text=c1a1a1a,anchor=south] (text232) at (29.97, 156.96){\gls{fakultät2}};



  \node[text=c1a1a1a,anchor=south] (text234) at (57.93, 156.96){\gls{fakultät3}};



  \node[text=c1a1a1a,anchor=south] (text236) at (85.88, 156.96){\gls{fakultät4}};



  \node[text=c1a1a1a,anchor=south] (text238) at (113.83, 156.96){\gls{fakultät6}};



  \node[text=c1a1a1a,anchor=south] (text240) at (141.78, 156.96){\gls{fakultät7}};



  \node[text=c1a1a1a,anchor=south] (text242) at (169.74, 156.96){\gls{fakultät8}};



  \node[text=c1a1a1a,anchor=south] (text244) at (197.69, 156.96){\gls{fakultät9}};



  \node[text=c1a1a1a,anchor=south] (text246) at (225.64, 156.96){\gls{fakultät10}};



  \path[draw=c333333,line cap=butt,line join=round,line width=0.38mm,miter 
  limit=10.0] (21.4, 26.48) -- (21.4, 27.45);



  \path[draw=c333333,line cap=butt,line join=round,line width=0.38mm,miter 
  limit=10.0] (29.97, 26.48) -- (29.97, 27.45);



  \path[draw=c333333,line cap=butt,line join=round,line width=0.38mm,miter 
  limit=10.0] (38.55, 26.48) -- (38.55, 27.45);



  \node[text=c4d4d4d,anchor=south east,cm={ 0.71,0.71,-0.71,0.71,(23.53, 
  -154.23)}] (text249) at (0.0, 177.8){2012-2015};



  \node[text=c4d4d4d,anchor=south east,cm={ 0.71,0.71,-0.71,0.71,(32.11, 
  -154.23)}] (text250) at (0.0, 177.8){2016-2019};



  \node[text=c4d4d4d,anchor=south east,cm={ 0.71,0.71,-0.71,0.71,(40.69, 
  -154.23)}] (text251) at (0.0, 177.8){2020-2023};



  \path[draw=c333333,line cap=butt,line join=round,line width=0.38mm,miter 
  limit=10.0] (49.35, 26.48) -- (49.35, 27.45);



  \path[draw=c333333,line cap=butt,line join=round,line width=0.38mm,miter 
  limit=10.0] (57.93, 26.48) -- (57.93, 27.45);



  \path[draw=c333333,line cap=butt,line join=round,line width=0.38mm,miter 
  limit=10.0] (66.51, 26.48) -- (66.51, 27.45);



  \node[text=c4d4d4d,anchor=south east,cm={ 0.71,0.71,-0.71,0.71,(51.48, 
  -154.23)}] (text253) at (0.0, 177.8){2012-2015};



  \node[text=c4d4d4d,anchor=south east,cm={ 0.71,0.71,-0.71,0.71,(60.06, 
  -154.23)}] (text254) at (0.0, 177.8){2016-2019};



  \node[text=c4d4d4d,anchor=south east,cm={ 0.71,0.71,-0.71,0.71,(68.64, 
  -154.23)}] (text255) at (0.0, 177.8){2020-2023};



  \path[draw=c333333,line cap=butt,line join=round,line width=0.38mm,miter 
  limit=10.0] (77.3, 26.48) -- (77.3, 27.45);



  \path[draw=c333333,line cap=butt,line join=round,line width=0.38mm,miter 
  limit=10.0] (85.88, 26.48) -- (85.88, 27.45);



  \path[draw=c333333,line cap=butt,line join=round,line width=0.38mm,miter 
  limit=10.0] (94.46, 26.48) -- (94.46, 27.45);



  \node[text=c4d4d4d,anchor=south east,cm={ 0.71,0.71,-0.71,0.71,(79.43, 
  -154.23)}] (text257) at (0.0, 177.8){2012-2015};



  \node[text=c4d4d4d,anchor=south east,cm={ 0.71,0.71,-0.71,0.71,(88.01, 
  -154.23)}] (text258) at (0.0, 177.8){2016-2019};



  \node[text=c4d4d4d,anchor=south east,cm={ 0.71,0.71,-0.71,0.71,(96.59, 
  -154.23)}] (text259) at (0.0, 177.8){2020-2023};



  \path[draw=c333333,line cap=butt,line join=round,line width=0.38mm,miter 
  limit=10.0] (105.25, 26.48) -- (105.25, 27.45);



  \path[draw=c333333,line cap=butt,line join=round,line width=0.38mm,miter 
  limit=10.0] (113.83, 26.48) -- (113.83, 27.45);



  \path[draw=c333333,line cap=butt,line join=round,line width=0.38mm,miter 
  limit=10.0] (122.41, 26.48) -- (122.41, 27.45);



  \node[text=c4d4d4d,anchor=south east,cm={ 0.71,0.71,-0.71,0.71,(107.39, 
  -154.23)}] (text261) at (0.0, 177.8){2012-2015};



  \node[text=c4d4d4d,anchor=south east,cm={ 0.71,0.71,-0.71,0.71,(115.97, 
  -154.23)}] (text262) at (0.0, 177.8){2016-2019};



  \node[text=c4d4d4d,anchor=south east,cm={ 0.71,0.71,-0.71,0.71,(124.55, 
  -154.23)}] (text263) at (0.0, 177.8){2020-2023};



  \path[draw=c333333,line cap=butt,line join=round,line width=0.38mm,miter 
  limit=10.0] (133.21, 26.48) -- (133.21, 27.45);



  \path[draw=c333333,line cap=butt,line join=round,line width=0.38mm,miter 
  limit=10.0] (141.78, 26.48) -- (141.78, 27.45);



  \path[draw=c333333,line cap=butt,line join=round,line width=0.38mm,miter 
  limit=10.0] (150.36, 26.48) -- (150.36, 27.45);



  \node[text=c4d4d4d,anchor=south east,cm={ 0.71,0.71,-0.71,0.71,(135.34, 
  -154.23)}] (text265) at (0.0, 177.8){2012-2015};



  \node[text=c4d4d4d,anchor=south east,cm={ 0.71,0.71,-0.71,0.71,(143.92, 
  -154.23)}] (text266) at (0.0, 177.8){2016-2019};



  \node[text=c4d4d4d,anchor=south east,cm={ 0.71,0.71,-0.71,0.71,(152.5, 
  -154.23)}] (text267) at (0.0, 177.8){2020-2023};



  \path[draw=c333333,line cap=butt,line join=round,line width=0.38mm,miter 
  limit=10.0] (161.16, 26.48) -- (161.16, 27.45);



  \path[draw=c333333,line cap=butt,line join=round,line width=0.38mm,miter 
  limit=10.0] (169.74, 26.48) -- (169.74, 27.45);



  \path[draw=c333333,line cap=butt,line join=round,line width=0.38mm,miter 
  limit=10.0] (178.32, 26.48) -- (178.32, 27.45);



  \node[text=c4d4d4d,anchor=south east,cm={ 0.71,0.71,-0.71,0.71,(163.29, 
  -154.23)}] (text269) at (0.0, 177.8){2012-2015};



  \node[text=c4d4d4d,anchor=south east,cm={ 0.71,0.71,-0.71,0.71,(171.87, 
  -154.23)}] (text270) at (0.0, 177.8){2016-2019};



  \node[text=c4d4d4d,anchor=south east,cm={ 0.71,0.71,-0.71,0.71,(180.45, 
  -154.23)}] (text271) at (0.0, 177.8){2020-2023};



  \path[draw=c333333,line cap=butt,line join=round,line width=0.38mm,miter 
  limit=10.0] (189.11, 26.48) -- (189.11, 27.45);



  \path[draw=c333333,line cap=butt,line join=round,line width=0.38mm,miter 
  limit=10.0] (197.69, 26.48) -- (197.69, 27.45);



  \path[draw=c333333,line cap=butt,line join=round,line width=0.38mm,miter 
  limit=10.0] (206.27, 26.48) -- (206.27, 27.45);



  \node[text=c4d4d4d,anchor=south east,cm={ 0.71,0.71,-0.71,0.71,(191.25, 
  -154.23)}] (text273) at (0.0, 177.8){2012-2015};



  \node[text=c4d4d4d,anchor=south east,cm={ 0.71,0.71,-0.71,0.71,(199.82, 
  -154.23)}] (text274) at (0.0, 177.8){2016-2019};



  \node[text=c4d4d4d,anchor=south east,cm={ 0.71,0.71,-0.71,0.71,(208.4, 
  -154.23)}] (text275) at (0.0, 177.8){2020-2023};



  \path[draw=c333333,line cap=butt,line join=round,line width=0.38mm,miter 
  limit=10.0] (217.06, 26.48) -- (217.06, 27.45);



  \path[draw=c333333,line cap=butt,line join=round,line width=0.38mm,miter 
  limit=10.0] (225.64, 26.48) -- (225.64, 27.45);



  \path[draw=c333333,line cap=butt,line join=round,line width=0.38mm,miter 
  limit=10.0] (234.22, 26.48) -- (234.22, 27.45);



  \node[text=c4d4d4d,anchor=south east,cm={ 0.71,0.71,-0.71,0.71,(219.2, 
  -154.23)}] (text277) at (0.0, 177.8){2012-2015};



  \node[text=c4d4d4d,anchor=south east,cm={ 0.71,0.71,-0.71,0.71,(227.78, 
  -154.23)}] (text278) at (0.0, 177.8){2016-2019};



  \node[text=c4d4d4d,anchor=south east,cm={ 0.71,0.71,-0.71,0.71,(236.36, 
  -154.23)}] (text279) at (0.0, 177.8){2020-2023};



  \node[text=c4d4d4d,anchor=south east] (text280) at (14.51, 32.13){0\%};



  \node[text=c4d4d4d,anchor=south east] (text281) at (14.51, 61.1){25\%};



  \node[text=c4d4d4d,anchor=south east] (text282) at (14.51, 90.07){50\%};



  \node[text=c4d4d4d,anchor=south east] (text283) at (14.51, 119.04){75\%};



  \node[text=c4d4d4d,anchor=south east] (text284) at (14.51, 148.0){100\%};



  \path[draw=c333333,line cap=butt,line join=round,line width=0.38mm,miter 
  limit=10.0] (15.28, 33.24) -- (16.25, 33.24);



  \path[draw=c333333,line cap=butt,line join=round,line width=0.38mm,miter 
  limit=10.0] (15.28, 62.21) -- (16.25, 62.21);



  \path[draw=c333333,line cap=butt,line join=round,line width=0.38mm,miter 
  limit=10.0] (15.28, 91.18) -- (16.25, 91.18);



  \path[draw=c333333,line cap=butt,line join=round,line width=0.38mm,miter 
  limit=10.0] (15.28, 120.15) -- (16.25, 120.15);



  \path[draw=c333333,line cap=butt,line join=round,line width=0.38mm,miter 
  limit=10.0] (15.28, 149.11) -- (16.25, 149.11);



  \node[anchor=south,cm={ 0.0,1.0,-1.0,0.0,(4.7, -86.62)}] (text289) at (0.0, 
  177.8){Anteil in Prozenten (\%)};



  \path[fill=cebebeb,line cap=round,line join=round,line width=0.38mm,miter 
  limit=10.0] (86.94, 173.93) rectangle (93.03, 167.84);



  \path[fill=c77aadd,line cap=butt,line join=miter,line width=0.38mm,miter 
  limit=10.0] (87.19, 173.68) rectangle (92.78, 168.09);



  \path[fill=cebebeb,line cap=round,line join=round,line width=0.38mm,miter 
  limit=10.0] (108.42, 173.93) rectangle (114.52, 167.84);



  \path[fill=c99dde1,line cap=butt,line join=miter,line width=0.38mm,miter 
  limit=10.0] (108.67, 173.68) rectangle (114.27, 168.09);



  \path[fill=cebebeb,line cap=round,line join=round,line width=0.38mm,miter 
  limit=10.0] (129.91, 173.93) rectangle (136.01, 167.84);



  \path[fill=ceedd88,line cap=butt,line join=miter,line width=0.38mm,miter 
  limit=10.0] (130.16, 173.68) rectangle (135.76, 168.09);



  \path[fill=cebebeb,line cap=round,line join=round,line width=0.38mm,miter 
  limit=10.0] (151.4, 173.93) rectangle (157.5, 167.84);



  \path[fill=cee8866,line cap=butt,line join=miter,line width=0.38mm,miter 
  limit=10.0] (151.65, 173.68) rectangle (157.24, 168.09);



  \node[anchor=south west] (text297) at (94.96, 169.63){Stufe 1};



  \node[anchor=south west] (text298) at (116.45, 169.63){Stufe 2};



  \node[anchor=south west] (text299) at (137.94, 169.63){Stufe 3};



  \node[anchor=south west] (text300) at (159.43, 169.63){Keine};




\end{tikzpicture}
}
    \caption{Externe \gls{forschungsdaten} für \glspl{pdd} nach Fakultät, Zeitgruppe und Klassifikationsstufe.
    Die Höhe der Barren entsprechen dem relativen Anteil zur jeweiligen angepassten $\text{\textit{Fakultät}}\times\text{\textit{Zeitgruppe}}\times\text{\textit{Begleitende \gls{forschungsdaten}}}$-Gesamtanzahl.
    Absolute Werte in Klammern angegeben.}
    \label{fig:luh-repo_fakultät_x_zeitgruppe_x_begleit-fd}
\end{figure}
Mit $\chi^2 (\num{14}, n=\num{1252}) = \num[round-mode=places,round-precision=2]{26.7127108745513}$, $p = \num[round-mode=places,round-precision=2]{0.0209769860572188},\phi_C=\num[round-mode=places,round-precision=2]{0.103286085824667}$ ist die Interaktion zwischen \textit{Fakultät} und \textit{Begleitende \glspl{forschungsdaten}} für \glspl{pdd} statistisch signifikant mit einer schwachen Effektstärke.

Von den in \cref{fig:luh-repo_fakultät_x_zeitgruppe_x_begleit-fd} dargestellten fakultätsspezifischen Interaktionen zwischen \textit{Zeitgruppe} und \textit{Begleitende \glspl{forschungsdaten}} waren keine Interaktionen statistisch signifikant.

\subsubsection{Externe Forschungsdaten}
Die relative und absolute Verteilung von externen \glspl{forschungsdaten} für Fakultäten über die verschiedenen Zeitgruppen hinweg ist in \cref{fig:luh-repo_fakultät_x_zeitgruppe_x_externe-fd} dargestellt.
\begin{figure}[!htbp]
    \resizebox{\ifdim\width>\textwidth\textwidth\else\width\fi}{!}{\begin{tikzpicture}[y=1mm, x=1mm, yscale=\globalscale,xscale=\globalscale, every node/.append style={scale=\globalscale}, inner sep=0pt, outer sep=0pt]
  \path[fill=cebebeb,line cap=round,line join=round,line width=0.38mm,miter 
  limit=10.0] (16.25, 154.91) rectangle (43.7, 27.45);



  \path[draw=white,line cap=butt,line join=round,line width=0.19mm,miter 
  limit=10.0] (16.25, 47.73) -- (43.7, 47.73);



  \path[draw=white,line cap=butt,line join=round,line width=0.19mm,miter 
  limit=10.0] (16.25, 76.69) -- (43.7, 76.69);



  \path[draw=white,line cap=butt,line join=round,line width=0.19mm,miter 
  limit=10.0] (16.25, 105.66) -- (43.7, 105.66);



  \path[draw=white,line cap=butt,line join=round,line width=0.19mm,miter 
  limit=10.0] (16.25, 134.63) -- (43.7, 134.63);



  \path[draw=white,line cap=butt,line join=round,line width=0.38mm,miter 
  limit=10.0] (16.25, 33.24) -- (43.7, 33.24);



  \path[draw=white,line cap=butt,line join=round,line width=0.38mm,miter 
  limit=10.0] (16.25, 62.21) -- (43.7, 62.21);



  \path[draw=white,line cap=butt,line join=round,line width=0.38mm,miter 
  limit=10.0] (16.25, 91.18) -- (43.7, 91.18);



  \path[draw=white,line cap=butt,line join=round,line width=0.38mm,miter 
  limit=10.0] (16.25, 120.15) -- (43.7, 120.15);



  \path[draw=white,line cap=butt,line join=round,line width=0.38mm,miter 
  limit=10.0] (16.25, 149.11) -- (43.7, 149.11);



  \path[draw=white,line cap=butt,line join=round,line width=0.38mm,miter 
  limit=10.0] (21.4, 27.45) -- (21.4, 154.91);



  \path[draw=white,line cap=butt,line join=round,line width=0.38mm,miter 
  limit=10.0] (29.97, 27.45) -- (29.97, 154.91);



  \path[draw=white,line cap=butt,line join=round,line width=0.38mm,miter 
  limit=10.0] (38.55, 27.45) -- (38.55, 154.91);



  \path[fill=c77aadd,line cap=butt,line join=miter,line width=0.38mm,miter 
  limit=10.0] ;



  \path[fill=c99dde1,line cap=butt,line join=miter,line width=0.38mm,miter 
  limit=10.0] ;



  \path[fill=ceedd88,line cap=butt,line join=miter,line width=0.38mm,miter 
  limit=10.0] ;



  \path[draw=black,fill=cee8866,line cap=butt,line join=miter,line 
  width=0.38mm,miter limit=10.0] (17.53, 149.11) rectangle (25.26, 33.24);



  \path[fill=c77aadd,line cap=butt,line join=miter,line width=0.38mm,miter 
  limit=10.0] ;



  \path[fill=c99dde1,line cap=butt,line join=miter,line width=0.38mm,miter 
  limit=10.0] ;



  \path[fill=ceedd88,line cap=butt,line join=miter,line width=0.38mm,miter 
  limit=10.0] ;



  \path[draw=black,fill=cee8866,line cap=butt,line join=miter,line 
  width=0.38mm,miter limit=10.0] (26.11, 149.11) rectangle (33.83, 33.24);



  \path[fill=c77aadd,line cap=butt,line join=miter,line width=0.38mm,miter 
  limit=10.0] ;



  \path[fill=c99dde1,line cap=butt,line join=miter,line width=0.38mm,miter 
  limit=10.0] ;



  \path[fill=ceedd88,line cap=butt,line join=miter,line width=0.38mm,miter 
  limit=10.0] ;



  \path[draw=black,fill=cee8866,line cap=butt,line join=miter,line 
  width=0.38mm,miter limit=10.0] (34.69, 149.11) rectangle (42.41, 33.24);



  \node[anchor=south] (text27) at (21.4, 92.45){100};



  \node[anchor=south] (text28) at (21.4, 87.39){(12)};



  \node[anchor=south] (text29) at (29.97, 92.45){100};



  \node[anchor=south] (text30) at (29.97, 87.39){(18)};



  \node[anchor=south] (text31) at (38.55, 92.45){100};



  \node[anchor=south] (text32) at (38.55, 87.39){(21)};



  \path[fill=cebebeb,line cap=round,line join=round,line width=0.38mm,miter 
  limit=10.0] (44.2, 154.91) rectangle (71.65, 27.45);



  \path[draw=white,line cap=butt,line join=round,line width=0.19mm,miter 
  limit=10.0] (44.2, 47.73) -- (71.65, 47.73);



  \path[draw=white,line cap=butt,line join=round,line width=0.19mm,miter 
  limit=10.0] (44.2, 76.69) -- (71.65, 76.69);



  \path[draw=white,line cap=butt,line join=round,line width=0.19mm,miter 
  limit=10.0] (44.2, 105.66) -- (71.65, 105.66);



  \path[draw=white,line cap=butt,line join=round,line width=0.19mm,miter 
  limit=10.0] (44.2, 134.63) -- (71.65, 134.63);



  \path[draw=white,line cap=butt,line join=round,line width=0.38mm,miter 
  limit=10.0] (44.2, 33.24) -- (71.65, 33.24);



  \path[draw=white,line cap=butt,line join=round,line width=0.38mm,miter 
  limit=10.0] (44.2, 62.21) -- (71.65, 62.21);



  \path[draw=white,line cap=butt,line join=round,line width=0.38mm,miter 
  limit=10.0] (44.2, 91.18) -- (71.65, 91.18);



  \path[draw=white,line cap=butt,line join=round,line width=0.38mm,miter 
  limit=10.0] (44.2, 120.15) -- (71.65, 120.15);



  \path[draw=white,line cap=butt,line join=round,line width=0.38mm,miter 
  limit=10.0] (44.2, 149.11) -- (71.65, 149.11);



  \path[draw=white,line cap=butt,line join=round,line width=0.38mm,miter 
  limit=10.0] (49.35, 27.45) -- (49.35, 154.91);



  \path[draw=white,line cap=butt,line join=round,line width=0.38mm,miter 
  limit=10.0] (57.93, 27.45) -- (57.93, 154.91);



  \path[draw=white,line cap=butt,line join=round,line width=0.38mm,miter 
  limit=10.0] (66.51, 27.45) -- (66.51, 154.91);



  \path[fill=c77aadd,line cap=butt,line join=miter,line width=0.38mm,miter 
  limit=10.0] ;



  \path[draw=black,fill=cee8866,line cap=butt,line join=miter,line 
  width=0.38mm,miter limit=10.0] (45.49, 149.11) rectangle (53.21, 33.24);



  \path[draw=black,fill=c77aadd,line cap=butt,line join=miter,line 
  width=0.38mm,miter limit=10.0] (54.07, 149.11) rectangle (61.79, 137.12);



  \path[draw=black,fill=cee8866,line cap=butt,line join=miter,line 
  width=0.38mm,miter limit=10.0] (54.07, 137.13) rectangle (61.79, 33.24);



  \path[draw=black,fill=c77aadd,line cap=butt,line join=miter,line 
  width=0.38mm,miter limit=10.0] (62.64, 149.11) rectangle (70.37, 131.13);



  \path[draw=black,fill=cee8866,line cap=butt,line join=miter,line 
  width=0.38mm,miter limit=10.0] (62.64, 131.13) rectangle (70.37, 33.24);



  \node[anchor=south] (text50) at (49.35, 92.45){100};



  \node[anchor=south] (text51) at (49.35, 87.39){(18)};



  \node[anchor=south] (text52) at (57.93, 144.4){10};



  \node[anchor=south] (text53) at (57.93, 139.34){(3)};



  \node[anchor=south] (text54) at (57.93, 86.46){90};



  \node[anchor=south] (text55) at (57.93, 81.4){(26)};



  \node[anchor=south] (text56) at (66.51, 141.4){16};



  \node[anchor=south] (text57) at (66.51, 136.34){(9)};



  \node[anchor=south] (text58) at (66.51, 83.46){84};



  \node[anchor=south] (text59) at (66.51, 78.4){(49)};



  \path[fill=cebebeb,line cap=round,line join=round,line width=0.38mm,miter 
  limit=10.0] (72.15, 154.91) rectangle (99.61, 27.45);



  \path[draw=white,line cap=butt,line join=round,line width=0.19mm,miter 
  limit=10.0] (72.15, 47.73) -- (99.6, 47.73);



  \path[draw=white,line cap=butt,line join=round,line width=0.19mm,miter 
  limit=10.0] (72.15, 76.69) -- (99.6, 76.69);



  \path[draw=white,line cap=butt,line join=round,line width=0.19mm,miter 
  limit=10.0] (72.15, 105.66) -- (99.6, 105.66);



  \path[draw=white,line cap=butt,line join=round,line width=0.19mm,miter 
  limit=10.0] (72.15, 134.63) -- (99.6, 134.63);



  \path[draw=white,line cap=butt,line join=round,line width=0.38mm,miter 
  limit=10.0] (72.15, 33.24) -- (99.6, 33.24);



  \path[draw=white,line cap=butt,line join=round,line width=0.38mm,miter 
  limit=10.0] (72.15, 62.21) -- (99.6, 62.21);



  \path[draw=white,line cap=butt,line join=round,line width=0.38mm,miter 
  limit=10.0] (72.15, 91.18) -- (99.6, 91.18);



  \path[draw=white,line cap=butt,line join=round,line width=0.38mm,miter 
  limit=10.0] (72.15, 120.15) -- (99.6, 120.15);



  \path[draw=white,line cap=butt,line join=round,line width=0.38mm,miter 
  limit=10.0] (72.15, 149.11) -- (99.6, 149.11);



  \path[draw=white,line cap=butt,line join=round,line width=0.38mm,miter 
  limit=10.0] (77.3, 27.45) -- (77.3, 154.91);



  \path[draw=white,line cap=butt,line join=round,line width=0.38mm,miter 
  limit=10.0] (85.88, 27.45) -- (85.88, 154.91);



  \path[draw=white,line cap=butt,line join=round,line width=0.38mm,miter 
  limit=10.0] (94.46, 27.45) -- (94.46, 154.91);



  \path[draw=black,fill=c77aadd,line cap=butt,line join=miter,line 
  width=0.38mm,miter limit=10.0] (73.44, 149.11) rectangle (81.16, 145.49);



  \path[draw=black,fill=cee8866,line cap=butt,line join=miter,line 
  width=0.38mm,miter limit=10.0] (73.44, 145.49) rectangle (81.16, 33.24);



  \path[draw=black,fill=c77aadd,line cap=butt,line join=miter,line 
  width=0.38mm,miter limit=10.0] (82.02, 149.11) rectangle (89.74, 134.98);



  \path[draw=black,fill=cee8866,line cap=butt,line join=miter,line 
  width=0.38mm,miter limit=10.0] (82.02, 134.98) rectangle (89.74, 33.24);



  \path[draw=black,fill=c77aadd,line cap=butt,line join=miter,line 
  width=0.38mm,miter limit=10.0] (90.6, 149.11) rectangle (98.32, 109.34);



  \path[draw=black,fill=cee8866,line cap=butt,line join=miter,line 
  width=0.38mm,miter limit=10.0] (90.6, 109.34) rectangle (98.32, 33.24);



  \node[anchor=south,shift={(0.0, 2.81)}] (text77) at (77.3, 148.58){3};



  \node[anchor=south,shift={(0.0, 2.81)}] (text78) at (77.3, 143.52){(1)};



  \node[anchor=south] (text79) at (77.3, 90.64){97};



  \node[anchor=south] (text80) at (77.3, 85.58){(31)};



  \node[anchor=south] (text81) at (85.88, 143.32){12};



  \node[anchor=south] (text82) at (85.88, 138.26){(5)};



  \node[anchor=south] (text83) at (85.88, 85.39){88};



  \node[anchor=south] (text84) at (85.88, 80.33){(36)};



  \node[anchor=south] (text85) at (94.46, 130.5){34};



  \node[anchor=south] (text86) at (94.46, 125.44){(23)};



  \node[anchor=south] (text87) at (94.46, 72.56){66};



  \node[anchor=south] (text88) at (94.46, 67.5){(44)};



  \path[fill=cebebeb,line cap=round,line join=round,line width=0.38mm,miter 
  limit=10.0] (100.1, 154.91) rectangle (127.56, 27.45);



  \path[draw=white,line cap=butt,line join=round,line width=0.19mm,miter 
  limit=10.0] (100.1, 47.73) -- (127.56, 47.73);



  \path[draw=white,line cap=butt,line join=round,line width=0.19mm,miter 
  limit=10.0] (100.1, 76.69) -- (127.56, 76.69);



  \path[draw=white,line cap=butt,line join=round,line width=0.19mm,miter 
  limit=10.0] (100.1, 105.66) -- (127.56, 105.66);



  \path[draw=white,line cap=butt,line join=round,line width=0.19mm,miter 
  limit=10.0] (100.1, 134.63) -- (127.56, 134.63);



  \path[draw=white,line cap=butt,line join=round,line width=0.38mm,miter 
  limit=10.0] (100.1, 33.24) -- (127.56, 33.24);



  \path[draw=white,line cap=butt,line join=round,line width=0.38mm,miter 
  limit=10.0] (100.1, 62.21) -- (127.56, 62.21);



  \path[draw=white,line cap=butt,line join=round,line width=0.38mm,miter 
  limit=10.0] (100.1, 91.18) -- (127.56, 91.18);



  \path[draw=white,line cap=butt,line join=round,line width=0.38mm,miter 
  limit=10.0] (100.1, 120.15) -- (127.56, 120.15);



  \path[draw=white,line cap=butt,line join=round,line width=0.38mm,miter 
  limit=10.0] (100.1, 149.11) -- (127.56, 149.11);



  \path[draw=white,line cap=butt,line join=round,line width=0.38mm,miter 
  limit=10.0] (105.25, 27.45) -- (105.25, 154.91);



  \path[draw=white,line cap=butt,line join=round,line width=0.38mm,miter 
  limit=10.0] (113.83, 27.45) -- (113.83, 154.91);



  \path[draw=white,line cap=butt,line join=round,line width=0.38mm,miter 
  limit=10.0] (122.41, 27.45) -- (122.41, 154.91);



  \path[fill=c77aadd,line cap=butt,line join=miter,line width=0.38mm,miter 
  limit=10.0] ;



  \path[draw=black,fill=cee8866,line cap=butt,line join=miter,line 
  width=0.38mm,miter limit=10.0] (101.39, 149.11) rectangle (109.11, 33.24);



  \path[fill=c77aadd,line cap=butt,line join=miter,line width=0.38mm,miter 
  limit=10.0] ;



  \path[draw=black,fill=cee8866,line cap=butt,line join=miter,line 
  width=0.38mm,miter limit=10.0] (109.97, 149.11) rectangle (117.69, 33.24);



  \path[draw=black,fill=c77aadd,line cap=butt,line join=miter,line 
  width=0.38mm,miter limit=10.0] (118.55, 149.11) rectangle (126.27, 143.01);



  \path[draw=black,fill=cee8866,line cap=butt,line join=miter,line 
  width=0.38mm,miter limit=10.0] (118.55, 143.02) rectangle (126.27, 33.24);



  \node[anchor=south] (text106) at (105.25, 92.45){100};



  \node[anchor=south] (text107) at (105.25, 87.39){(29)};



  \node[anchor=south] (text108) at (113.83, 92.45){100};



  \node[anchor=south] (text109) at (113.83, 87.39){(34)};



  \node[anchor=south,shift={(0.0, 3.17)}] (text110) at (122.41, 147.34){5};



  \node[anchor=south,shift={(0.0, 3.17)}] (text111) at (122.41, 142.28){(3)};



  \node[anchor=south] (text112) at (122.41, 89.4){95};



  \node[anchor=south] (text113) at (122.41, 84.35){(54)};



  \path[fill=cebebeb,line cap=round,line join=round,line width=0.38mm,miter 
  limit=10.0] (128.06, 154.91) rectangle (155.51, 27.45);



  \path[draw=white,line cap=butt,line join=round,line width=0.19mm,miter 
  limit=10.0] (128.06, 47.73) -- (155.51, 47.73);



  \path[draw=white,line cap=butt,line join=round,line width=0.19mm,miter 
  limit=10.0] (128.06, 76.69) -- (155.51, 76.69);



  \path[draw=white,line cap=butt,line join=round,line width=0.19mm,miter 
  limit=10.0] (128.06, 105.66) -- (155.51, 105.66);



  \path[draw=white,line cap=butt,line join=round,line width=0.19mm,miter 
  limit=10.0] (128.06, 134.63) -- (155.51, 134.63);



  \path[draw=white,line cap=butt,line join=round,line width=0.38mm,miter 
  limit=10.0] (128.06, 33.24) -- (155.51, 33.24);



  \path[draw=white,line cap=butt,line join=round,line width=0.38mm,miter 
  limit=10.0] (128.06, 62.21) -- (155.51, 62.21);



  \path[draw=white,line cap=butt,line join=round,line width=0.38mm,miter 
  limit=10.0] (128.06, 91.18) -- (155.51, 91.18);



  \path[draw=white,line cap=butt,line join=round,line width=0.38mm,miter 
  limit=10.0] (128.06, 120.15) -- (155.51, 120.15);



  \path[draw=white,line cap=butt,line join=round,line width=0.38mm,miter 
  limit=10.0] (128.06, 149.11) -- (155.51, 149.11);



  \path[draw=white,line cap=butt,line join=round,line width=0.38mm,miter 
  limit=10.0] (133.21, 27.45) -- (133.21, 154.91);



  \path[draw=white,line cap=butt,line join=round,line width=0.38mm,miter 
  limit=10.0] (141.78, 27.45) -- (141.78, 154.91);



  \path[draw=white,line cap=butt,line join=round,line width=0.38mm,miter 
  limit=10.0] (150.36, 27.45) -- (150.36, 154.91);



  \path[fill=c77aadd,line cap=butt,line join=miter,line width=0.38mm,miter 
  limit=10.0] ;



  \path[draw=black,fill=cee8866,line cap=butt,line join=miter,line 
  width=0.38mm,miter limit=10.0] (129.34, 149.11) rectangle (137.06, 33.24);



  \path[fill=c77aadd,line cap=butt,line join=miter,line width=0.38mm,miter 
  limit=10.0] ;



  \path[draw=black,fill=cee8866,line cap=butt,line join=miter,line 
  width=0.38mm,miter limit=10.0] (137.92, 149.11) rectangle (145.64, 33.24);



  \path[draw=black,fill=c77aadd,line cap=butt,line join=miter,line 
  width=0.38mm,miter limit=10.0] (146.5, 149.11) rectangle (154.22, 135.41);



  \path[draw=black,fill=cee8866,line cap=butt,line join=miter,line 
  width=0.38mm,miter limit=10.0] (146.5, 135.41) rectangle (154.22, 33.25);



  \node[anchor=south] (text131) at (133.21, 92.45){100};



  \node[anchor=south] (text132) at (133.21, 87.39){(64)};



  \node[anchor=south] (text133) at (141.78, 92.45){100};



  \node[anchor=south] (text134) at (141.78, 87.39){(57)};



  \node[anchor=south] (text135) at (150.36, 143.53){12};



  \node[anchor=south] (text136) at (150.36, 138.48){(11)};



  \node[anchor=south] (text137) at (150.36, 85.6){88};



  \node[anchor=south] (text138) at (150.36, 80.54){(82)};



  \path[fill=cebebeb,line cap=round,line join=round,line width=0.38mm,miter 
  limit=10.0] (156.01, 154.91) rectangle (183.46, 27.45);



  \path[draw=white,line cap=butt,line join=round,line width=0.19mm,miter 
  limit=10.0] (156.01, 47.73) -- (183.46, 47.73);



  \path[draw=white,line cap=butt,line join=round,line width=0.19mm,miter 
  limit=10.0] (156.01, 76.69) -- (183.46, 76.69);



  \path[draw=white,line cap=butt,line join=round,line width=0.19mm,miter 
  limit=10.0] (156.01, 105.66) -- (183.46, 105.66);



  \path[draw=white,line cap=butt,line join=round,line width=0.19mm,miter 
  limit=10.0] (156.01, 134.63) -- (183.46, 134.63);



  \path[draw=white,line cap=butt,line join=round,line width=0.38mm,miter 
  limit=10.0] (156.01, 33.24) -- (183.46, 33.24);



  \path[draw=white,line cap=butt,line join=round,line width=0.38mm,miter 
  limit=10.0] (156.01, 62.21) -- (183.46, 62.21);



  \path[draw=white,line cap=butt,line join=round,line width=0.38mm,miter 
  limit=10.0] (156.01, 91.18) -- (183.46, 91.18);



  \path[draw=white,line cap=butt,line join=round,line width=0.38mm,miter 
  limit=10.0] (156.01, 120.15) -- (183.46, 120.15);



  \path[draw=white,line cap=butt,line join=round,line width=0.38mm,miter 
  limit=10.0] (156.01, 149.11) -- (183.46, 149.11);



  \path[draw=white,line cap=butt,line join=round,line width=0.38mm,miter 
  limit=10.0] (161.16, 27.45) -- (161.16, 154.91);



  \path[draw=white,line cap=butt,line join=round,line width=0.38mm,miter 
  limit=10.0] (169.74, 27.45) -- (169.74, 154.91);



  \path[draw=white,line cap=butt,line join=round,line width=0.38mm,miter 
  limit=10.0] (178.32, 27.45) -- (178.32, 154.91);



  \path[draw=black,fill=c77aadd,line cap=butt,line join=miter,line 
  width=0.38mm,miter limit=10.0] (157.3, 149.11) rectangle (165.02, 146.84);



  \path[fill=c99dde1,line cap=butt,line join=miter,line width=0.38mm,miter 
  limit=10.0] ;



  \path[fill=ceedd88,line cap=butt,line join=miter,line width=0.38mm,miter 
  limit=10.0] ;



  \path[draw=black,fill=cee8866,line cap=butt,line join=miter,line 
  width=0.38mm,miter limit=10.0] (157.3, 146.84) rectangle (165.02, 33.24);



  \path[draw=black,fill=c77aadd,line cap=butt,line join=miter,line 
  width=0.38mm,miter limit=10.0] (165.88, 149.11) rectangle (173.6, 145.8);



  \path[fill=c99dde1,line cap=butt,line join=miter,line width=0.38mm,miter 
  limit=10.0] ;



  \path[draw=black,fill=ceedd88,line cap=butt,line join=miter,line 
  width=0.38mm,miter limit=10.0] (165.88, 145.8) rectangle (173.6, 144.97);



  \path[draw=black,fill=cee8866,line cap=butt,line join=miter,line 
  width=0.38mm,miter limit=10.0] (165.88, 144.97) rectangle (173.6, 33.24);



  \path[draw=black,fill=c77aadd,line cap=butt,line join=miter,line 
  width=0.38mm,miter limit=10.0] (174.46, 149.11) rectangle (182.18, 129.8);



  \path[draw=black,fill=c99dde1,line cap=butt,line join=miter,line 
  width=0.38mm,miter limit=10.0] (174.46, 129.8) rectangle (182.18, 129.08);



  \path[draw=black,fill=ceedd88,line cap=butt,line join=miter,line 
  width=0.38mm,miter limit=10.0] (174.46, 129.09) rectangle (182.18, 128.37);



  \path[draw=black,fill=cee8866,line cap=butt,line join=miter,line 
  width=0.38mm,miter limit=10.0] (174.46, 128.37) rectangle (182.18, 33.24);



  \node[anchor=south,shift={(0.0, 2.65)}] (text162) at (161.16, 149.25){2};



  \node[anchor=south,shift={(0.0, 2.65)}] (text163) at (161.16, 144.19){(3)};



  \node[anchor=south] (text164) at (161.16, 91.32){98};



  \node[anchor=south] (text165) at (161.16, 86.26){(150)};



  \node[anchor=south,shift={(0.0, 3.17)}] (text166) at (169.74, 148.73){3};



  \node[anchor=south,shift={(0.0, 3.17)}] (text167) at (169.74, 143.68){(4)};



  \node[anchor=south,shift={(0.0, -5.29)}] (text168) at (169.74, 146.66){1};



  \node[anchor=south,shift={(0.0, -5.29)}] (text169) at (169.74, 141.61){(1)};



  \node[anchor=south] (text170) at (169.74, 90.39){96};



  \node[anchor=south] (text171) at (169.74, 85.33){(135)};



  \node[anchor=south] (text172) at (178.32, 140.73){17};



  \node[anchor=south] (text173) at (178.32, 135.67){(27)};



  \node[anchor=south,shift={(0.0, -9.0)}] (text174) at (178.32, 130.72){1};



  \node[anchor=south,shift={(0.0, -7.41)}] (text175) at (178.32, 125.66){(1)};



  \node[anchor=south] (text176) at (178.32, 130.0){1};



  \node[anchor=south,shift={(0.0, 1.59)}] (text177) at (178.32, 124.94){(1)};



  \node[anchor=south] (text178) at (178.32, 82.08){82};



  \node[anchor=south] (text179) at (178.32, 77.02){(133)};



  \path[fill=cebebeb,line cap=round,line join=round,line width=0.38mm,miter 
  limit=10.0] (183.96, 154.91) rectangle (211.42, 27.45);



  \path[draw=white,line cap=butt,line join=round,line width=0.19mm,miter 
  limit=10.0] (183.96, 47.73) -- (211.42, 47.73);



  \path[draw=white,line cap=butt,line join=round,line width=0.19mm,miter 
  limit=10.0] (183.96, 76.69) -- (211.42, 76.69);



  \path[draw=white,line cap=butt,line join=round,line width=0.19mm,miter 
  limit=10.0] (183.96, 105.66) -- (211.42, 105.66);



  \path[draw=white,line cap=butt,line join=round,line width=0.19mm,miter 
  limit=10.0] (183.96, 134.63) -- (211.42, 134.63);



  \path[draw=white,line cap=butt,line join=round,line width=0.38mm,miter 
  limit=10.0] (183.96, 33.24) -- (211.42, 33.24);



  \path[draw=white,line cap=butt,line join=round,line width=0.38mm,miter 
  limit=10.0] (183.96, 62.21) -- (211.42, 62.21);



  \path[draw=white,line cap=butt,line join=round,line width=0.38mm,miter 
  limit=10.0] (183.96, 91.18) -- (211.42, 91.18);



  \path[draw=white,line cap=butt,line join=round,line width=0.38mm,miter 
  limit=10.0] (183.96, 120.15) -- (211.42, 120.15);



  \path[draw=white,line cap=butt,line join=round,line width=0.38mm,miter 
  limit=10.0] (183.96, 149.11) -- (211.42, 149.11);



  \path[draw=white,line cap=butt,line join=round,line width=0.38mm,miter 
  limit=10.0] (189.11, 27.45) -- (189.11, 154.91);



  \path[draw=white,line cap=butt,line join=round,line width=0.38mm,miter 
  limit=10.0] (197.69, 27.45) -- (197.69, 154.91);



  \path[draw=white,line cap=butt,line join=round,line width=0.38mm,miter 
  limit=10.0] (206.27, 27.45) -- (206.27, 154.91);



  \path[draw=black,fill=cee8866,line cap=butt,line join=miter,line 
  width=0.38mm,miter limit=10.0] (185.25, 149.11) rectangle (192.97, 33.24);



  \path[draw=black,fill=cee8866,line cap=butt,line join=miter,line 
  width=0.38mm,miter limit=10.0] (193.83, 149.11) rectangle (201.55, 33.24);



  \path[draw=black,fill=cee8866,line cap=butt,line join=miter,line 
  width=0.38mm,miter limit=10.0] (202.41, 149.11) rectangle (210.13, 33.24);



  \node[anchor=south] (text194) at (189.11, 92.45){100};



  \node[anchor=south] (text195) at (189.11, 87.39){(24)};



  \node[anchor=south] (text196) at (197.69, 92.45){100};



  \node[anchor=south] (text197) at (197.69, 87.39){(20)};



  \node[anchor=south] (text198) at (206.27, 92.45){100};



  \node[anchor=south] (text199) at (206.27, 87.39){(27)};



  \path[fill=cebebeb,line cap=round,line join=round,line width=0.38mm,miter 
  limit=10.0] (211.91, 154.91) rectangle (239.37, 27.45);



  \path[draw=white,line cap=butt,line join=round,line width=0.19mm,miter 
  limit=10.0] (211.91, 47.73) -- (239.37, 47.73);



  \path[draw=white,line cap=butt,line join=round,line width=0.19mm,miter 
  limit=10.0] (211.91, 76.69) -- (239.37, 76.69);



  \path[draw=white,line cap=butt,line join=round,line width=0.19mm,miter 
  limit=10.0] (211.91, 105.66) -- (239.37, 105.66);



  \path[draw=white,line cap=butt,line join=round,line width=0.19mm,miter 
  limit=10.0] (211.91, 134.63) -- (239.37, 134.63);



  \path[draw=white,line cap=butt,line join=round,line width=0.38mm,miter 
  limit=10.0] (211.91, 33.24) -- (239.37, 33.24);



  \path[draw=white,line cap=butt,line join=round,line width=0.38mm,miter 
  limit=10.0] (211.91, 62.21) -- (239.37, 62.21);



  \path[draw=white,line cap=butt,line join=round,line width=0.38mm,miter 
  limit=10.0] (211.91, 91.18) -- (239.37, 91.18);



  \path[draw=white,line cap=butt,line join=round,line width=0.38mm,miter 
  limit=10.0] (211.91, 120.15) -- (239.37, 120.15);



  \path[draw=white,line cap=butt,line join=round,line width=0.38mm,miter 
  limit=10.0] (211.91, 149.11) -- (239.37, 149.11);



  \path[draw=white,line cap=butt,line join=round,line width=0.38mm,miter 
  limit=10.0] (217.06, 27.45) -- (217.06, 154.91);



  \path[draw=white,line cap=butt,line join=round,line width=0.38mm,miter 
  limit=10.0] (225.64, 27.45) -- (225.64, 154.91);



  \path[draw=white,line cap=butt,line join=round,line width=0.38mm,miter 
  limit=10.0] (234.22, 27.45) -- (234.22, 154.91);



  \path[fill=c77aadd,line cap=butt,line join=miter,line width=0.38mm,miter 
  limit=10.0] ;



  \path[draw=black,fill=cee8866,line cap=butt,line join=miter,line 
  width=0.38mm,miter limit=10.0] (213.2, 149.11) rectangle (220.92, 33.24);



  \path[fill=c77aadd,line cap=butt,line join=miter,line width=0.38mm,miter 
  limit=10.0] ;



  \path[draw=black,fill=cee8866,line cap=butt,line join=miter,line 
  width=0.38mm,miter limit=10.0] (221.78, 149.11) rectangle (229.5, 33.24);



  \path[draw=black,fill=c77aadd,line cap=butt,line join=miter,line 
  width=0.38mm,miter limit=10.0] (230.36, 149.11) rectangle (238.08, 145.6);



  \path[draw=black,fill=cee8866,line cap=butt,line join=miter,line 
  width=0.38mm,miter limit=10.0] (230.36, 145.6) rectangle (238.08, 33.24);



  \node[anchor=south] (text217) at (217.06, 92.45){100};



  \node[anchor=south] (text218) at (217.06, 87.39){(28)};



  \node[anchor=south] (text219) at (225.64, 92.45){100};



  \node[anchor=south] (text220) at (225.64, 87.39){(35)};



  \node[anchor=south,shift={(0.0, 2.65)}] (text221) at (234.22, 148.63){3};



  \node[anchor=south,shift={(0.0, 2.65)}] (text222) at (234.22, 143.57){(1)};



  \node[anchor=south] (text223) at (234.22, 90.7){97};



  \node[anchor=south] (text224) at (234.22, 85.64){(32)};



  \node[text=c1a1a1a,anchor=south] (text225) at (29.97, 156.96){\gls{fakultät2}};



  \node[text=c1a1a1a,anchor=south] (text227) at (57.93, 156.96){\gls{fakultät3}};



  \node[text=c1a1a1a,anchor=south] (text229) at (85.88, 156.96){\gls{fakultät4}};



  \node[text=c1a1a1a,anchor=south] (text231) at (113.83, 156.96){\gls{fakultät6}};



  \node[text=c1a1a1a,anchor=south] (text233) at (141.78, 156.96){\gls{fakultät7}};



  \node[text=c1a1a1a,anchor=south] (text235) at (169.74, 156.96){\gls{fakultät8}};



  \node[text=c1a1a1a,anchor=south] (text237) at (197.69, 156.96){\gls{fakultät9}};



  \node[text=c1a1a1a,anchor=south] (text239) at (225.64, 156.96){\gls{fakultät10}};



  \path[draw=c333333,line cap=butt,line join=round,line width=0.38mm,miter 
  limit=10.0] (21.4, 26.48) -- (21.4, 27.45);



  \path[draw=c333333,line cap=butt,line join=round,line width=0.38mm,miter 
  limit=10.0] (29.97, 26.48) -- (29.97, 27.45);



  \path[draw=c333333,line cap=butt,line join=round,line width=0.38mm,miter 
  limit=10.0] (38.55, 26.48) -- (38.55, 27.45);



  \node[text=c4d4d4d,anchor=south east,cm={ 0.71,0.71,-0.71,0.71,(23.53, 
  -154.23)}] (text242) at (0.0, 177.8){2012-2015};



  \node[text=c4d4d4d,anchor=south east,cm={ 0.71,0.71,-0.71,0.71,(32.11, 
  -154.23)}] (text243) at (0.0, 177.8){2016-2019};



  \node[text=c4d4d4d,anchor=south east,cm={ 0.71,0.71,-0.71,0.71,(40.69, 
  -154.23)}] (text244) at (0.0, 177.8){2020-2023};



  \path[draw=c333333,line cap=butt,line join=round,line width=0.38mm,miter 
  limit=10.0] (49.35, 26.48) -- (49.35, 27.45);



  \path[draw=c333333,line cap=butt,line join=round,line width=0.38mm,miter 
  limit=10.0] (57.93, 26.48) -- (57.93, 27.45);



  \path[draw=c333333,line cap=butt,line join=round,line width=0.38mm,miter 
  limit=10.0] (66.51, 26.48) -- (66.51, 27.45);



  \node[text=c4d4d4d,anchor=south east,cm={ 0.71,0.71,-0.71,0.71,(51.48, 
  -154.23)}] (text246) at (0.0, 177.8){2012-2015};



  \node[text=c4d4d4d,anchor=south east,cm={ 0.71,0.71,-0.71,0.71,(60.06, 
  -154.23)}] (text247) at (0.0, 177.8){2016-2019};



  \node[text=c4d4d4d,anchor=south east,cm={ 0.71,0.71,-0.71,0.71,(68.64, 
  -154.23)}] (text248) at (0.0, 177.8){2020-2023};



  \path[draw=c333333,line cap=butt,line join=round,line width=0.38mm,miter 
  limit=10.0] (77.3, 26.48) -- (77.3, 27.45);



  \path[draw=c333333,line cap=butt,line join=round,line width=0.38mm,miter 
  limit=10.0] (85.88, 26.48) -- (85.88, 27.45);



  \path[draw=c333333,line cap=butt,line join=round,line width=0.38mm,miter 
  limit=10.0] (94.46, 26.48) -- (94.46, 27.45);



  \node[text=c4d4d4d,anchor=south east,cm={ 0.71,0.71,-0.71,0.71,(79.43, 
  -154.23)}] (text250) at (0.0, 177.8){2012-2015};



  \node[text=c4d4d4d,anchor=south east,cm={ 0.71,0.71,-0.71,0.71,(88.01, 
  -154.23)}] (text251) at (0.0, 177.8){2016-2019};



  \node[text=c4d4d4d,anchor=south east,cm={ 0.71,0.71,-0.71,0.71,(96.59, 
  -154.23)}] (text252) at (0.0, 177.8){2020-2023};



  \path[draw=c333333,line cap=butt,line join=round,line width=0.38mm,miter 
  limit=10.0] (105.25, 26.48) -- (105.25, 27.45);



  \path[draw=c333333,line cap=butt,line join=round,line width=0.38mm,miter 
  limit=10.0] (113.83, 26.48) -- (113.83, 27.45);



  \path[draw=c333333,line cap=butt,line join=round,line width=0.38mm,miter 
  limit=10.0] (122.41, 26.48) -- (122.41, 27.45);



  \node[text=c4d4d4d,anchor=south east,cm={ 0.71,0.71,-0.71,0.71,(107.39, 
  -154.23)}] (text254) at (0.0, 177.8){2012-2015};



  \node[text=c4d4d4d,anchor=south east,cm={ 0.71,0.71,-0.71,0.71,(115.97, 
  -154.23)}] (text255) at (0.0, 177.8){2016-2019};



  \node[text=c4d4d4d,anchor=south east,cm={ 0.71,0.71,-0.71,0.71,(124.55, 
  -154.23)}] (text256) at (0.0, 177.8){2020-2023};



  \path[draw=c333333,line cap=butt,line join=round,line width=0.38mm,miter 
  limit=10.0] (133.21, 26.48) -- (133.21, 27.45);



  \path[draw=c333333,line cap=butt,line join=round,line width=0.38mm,miter 
  limit=10.0] (141.78, 26.48) -- (141.78, 27.45);



  \path[draw=c333333,line cap=butt,line join=round,line width=0.38mm,miter 
  limit=10.0] (150.36, 26.48) -- (150.36, 27.45);



  \node[text=c4d4d4d,anchor=south east,cm={ 0.71,0.71,-0.71,0.71,(135.34, 
  -154.23)}] (text258) at (0.0, 177.8){2012-2015};



  \node[text=c4d4d4d,anchor=south east,cm={ 0.71,0.71,-0.71,0.71,(143.92, 
  -154.23)}] (text259) at (0.0, 177.8){2016-2019};



  \node[text=c4d4d4d,anchor=south east,cm={ 0.71,0.71,-0.71,0.71,(152.5, 
  -154.23)}] (text260) at (0.0, 177.8){2020-2023};



  \path[draw=c333333,line cap=butt,line join=round,line width=0.38mm,miter 
  limit=10.0] (161.16, 26.48) -- (161.16, 27.45);



  \path[draw=c333333,line cap=butt,line join=round,line width=0.38mm,miter 
  limit=10.0] (169.74, 26.48) -- (169.74, 27.45);



  \path[draw=c333333,line cap=butt,line join=round,line width=0.38mm,miter 
  limit=10.0] (178.32, 26.48) -- (178.32, 27.45);



  \node[text=c4d4d4d,anchor=south east,cm={ 0.71,0.71,-0.71,0.71,(163.29, 
  -154.23)}] (text262) at (0.0, 177.8){2012-2015};



  \node[text=c4d4d4d,anchor=south east,cm={ 0.71,0.71,-0.71,0.71,(171.87, 
  -154.23)}] (text263) at (0.0, 177.8){2016-2019};



  \node[text=c4d4d4d,anchor=south east,cm={ 0.71,0.71,-0.71,0.71,(180.45, 
  -154.23)}] (text264) at (0.0, 177.8){2020-2023};



  \path[draw=c333333,line cap=butt,line join=round,line width=0.38mm,miter 
  limit=10.0] (189.11, 26.48) -- (189.11, 27.45);



  \path[draw=c333333,line cap=butt,line join=round,line width=0.38mm,miter 
  limit=10.0] (197.69, 26.48) -- (197.69, 27.45);



  \path[draw=c333333,line cap=butt,line join=round,line width=0.38mm,miter 
  limit=10.0] (206.27, 26.48) -- (206.27, 27.45);



  \node[text=c4d4d4d,anchor=south east,cm={ 0.71,0.71,-0.71,0.71,(191.25, 
  -154.23)}] (text266) at (0.0, 177.8){2012-2015};



  \node[text=c4d4d4d,anchor=south east,cm={ 0.71,0.71,-0.71,0.71,(199.82, 
  -154.23)}] (text267) at (0.0, 177.8){2016-2019};



  \node[text=c4d4d4d,anchor=south east,cm={ 0.71,0.71,-0.71,0.71,(208.4, 
  -154.23)}] (text268) at (0.0, 177.8){2020-2023};



  \path[draw=c333333,line cap=butt,line join=round,line width=0.38mm,miter 
  limit=10.0] (217.06, 26.48) -- (217.06, 27.45);



  \path[draw=c333333,line cap=butt,line join=round,line width=0.38mm,miter 
  limit=10.0] (225.64, 26.48) -- (225.64, 27.45);



  \path[draw=c333333,line cap=butt,line join=round,line width=0.38mm,miter 
  limit=10.0] (234.22, 26.48) -- (234.22, 27.45);



  \node[text=c4d4d4d,anchor=south east,cm={ 0.71,0.71,-0.71,0.71,(219.2, 
  -154.23)}] (text270) at (0.0, 177.8){2012-2015};



  \node[text=c4d4d4d,anchor=south east,cm={ 0.71,0.71,-0.71,0.71,(227.78, 
  -154.23)}] (text271) at (0.0, 177.8){2016-2019};



  \node[text=c4d4d4d,anchor=south east,cm={ 0.71,0.71,-0.71,0.71,(236.36, 
  -154.23)}] (text272) at (0.0, 177.8){2020-2023};



  \node[text=c4d4d4d,anchor=south east] (text273) at (14.51, 32.13){0\%};



  \node[text=c4d4d4d,anchor=south east] (text274) at (14.51, 61.1){25\%};



  \node[text=c4d4d4d,anchor=south east] (text275) at (14.51, 90.07){50\%};



  \node[text=c4d4d4d,anchor=south east] (text276) at (14.51, 119.04){75\%};



  \node[text=c4d4d4d,anchor=south east] (text277) at (14.51, 148.0){100\%};



  \path[draw=c333333,line cap=butt,line join=round,line width=0.38mm,miter 
  limit=10.0] (15.28, 33.24) -- (16.25, 33.24);



  \path[draw=c333333,line cap=butt,line join=round,line width=0.38mm,miter 
  limit=10.0] (15.28, 62.21) -- (16.25, 62.21);



  \path[draw=c333333,line cap=butt,line join=round,line width=0.38mm,miter 
  limit=10.0] (15.28, 91.18) -- (16.25, 91.18);



  \path[draw=c333333,line cap=butt,line join=round,line width=0.38mm,miter 
  limit=10.0] (15.28, 120.15) -- (16.25, 120.15);



  \path[draw=c333333,line cap=butt,line join=round,line width=0.38mm,miter 
  limit=10.0] (15.28, 149.11) -- (16.25, 149.11);



  \node[anchor=south,cm={ 0.0,1.0,-1.0,0.0,(4.7, -86.62)}] (text282) at (0.0, 
  177.8){Anteil in Prozenten (\%)};



  \path[fill=cebebeb,line cap=round,line join=round,line width=0.38mm,miter 
  limit=10.0] (86.94, 173.93) rectangle (93.03, 167.84);



  \path[fill=c77aadd,line cap=butt,line join=miter,line width=0.38mm,miter 
  limit=10.0] (87.19, 173.68) rectangle (92.78, 168.09);



  \path[fill=cebebeb,line cap=round,line join=round,line width=0.38mm,miter 
  limit=10.0] (108.42, 173.93) rectangle (114.52, 167.84);



  \path[fill=c99dde1,line cap=butt,line join=miter,line width=0.38mm,miter 
  limit=10.0] (108.67, 173.68) rectangle (114.27, 168.09);



  \path[fill=cebebeb,line cap=round,line join=round,line width=0.38mm,miter 
  limit=10.0] (129.91, 173.93) rectangle (136.01, 167.84);



  \path[fill=ceedd88,line cap=butt,line join=miter,line width=0.38mm,miter 
  limit=10.0] (130.16, 173.68) rectangle (135.76, 168.09);



  \path[fill=cebebeb,line cap=round,line join=round,line width=0.38mm,miter 
  limit=10.0] (151.4, 173.93) rectangle (157.5, 167.84);



  \path[fill=cee8866,line cap=butt,line join=miter,line width=0.38mm,miter 
  limit=10.0] (151.65, 173.68) rectangle (157.24, 168.09);



  \node[anchor=south west] (text290) at (94.96, 169.63){Stufe 1};



  \node[anchor=south west] (text291) at (116.45, 169.63){Stufe 2};



  \node[anchor=south west] (text292) at (137.94, 169.63){Stufe 3};



  \node[anchor=south west] (text293) at (159.43, 169.63){Keine};




\end{tikzpicture}
}
    \caption{Externe \gls{forschungsdaten} für \glspl{pdd} nach Fakultät, Zeitgruppe und Klassifikationsstufe.
    Die Höhe der Barren entsprechen dem relativen Anteil zur jeweiligen angepassten $\text{\textit{Fakultät}}\times\text{\textit{Zeitgruppe}}\times\text{\textit{Externe \gls{forschungsdaten}}}$-Gesamtanzahl.
    Absolute Werte in Klammern angegeben.}
    \label{fig:luh-repo_fakultät_x_zeitgruppe_x_externe-fd}
\end{figure}
Mit $\chi^2 (\num{21}, n=\num{1252}) = \num[round-mode=places,round-precision=2]{66.7577903096558}$, $p = \num[round-mode=places,round-precision=2]{1.1517458402742E-06},\phi_C=\num[round-mode=places,round-precision=2]{0.133317814235608}$ ist die Interaktion zwischen \textit{Fakultät} und \textit{Begleitende \glspl{forschungsdaten}} für \glspl{pdd} statistisch hochsignifikant mit einer schwachen Effektstärke.

Von den in \cref{fig:luh-repo_fakultät_x_zeitgruppe_x_externe-fd} dargestellten fakultätsspezifischen Interaktionen zwischen \textit{Zeitgruppe} und \textit{Allgemeine \glspl{forschungsdaten}} waren auch hier nur \gls{fakultät4} ($\chi^2 (\num{2}, n=\num{140}) = \num[round-mode=places,round-precision=2]{15.4010432799519}$, $p = \num[round-mode=places,round-precision=4]{0.000452591031724297},\phi_C=\num[round-mode=places,round-precision=2]{0.331673713157459}$) und \gls{fakultät7} ($\chi^2 (\num{2}, n=\num{214}) = \num[round-mode=places,round-precision=2]{15.0873457280576}$, $p = \num[round-mode=places,round-precision=3]{0.000529449450462421},\phi_C=\num[round-mode=places,round-precision=2]{0.265521403189406}$) statistisch signifikant, wobei der Effekt für \gls{fakultät6} etwas stärker ist als für \gls{fakultät7}.

\subsection{Externe Publikationen und Metadaten}\label{sec:luh-repo-results-external-metadata}
\parsum{Externe Publikation}
Die Untersuchung von Metadatenangaben für externe Publikationen hat ergeben, dass keine der externen Publikationsformen in ihren Metadaten festhält, dass die Daten einer Dissertation aus dem \gls{luh-repo} zugehörig sind.
Dies ist zumindest im Teil auch dadurch bedingt, dass die Nutzung dedizierter \gls{forschungsdaten}-Repositorien bei den externen Publikationsmöglichkeiten in der Stichprobe in der Minderheit lag:
Viele Daten wurden extern in dedizierten Datenbanken (z.B.~Gensequenzen), auf GitHub~/~GitLab publiziert oder entsprachen bei kumulativen Dissertationen begleitenden Dateien, die in einem Journal zusammen mit der PDF-Datei hochgeladen wurden.
Insgesamt wurden fast keine Plattformen genutzt, die von sich aus eine Kodifizierung entsprechender Metadaten ermöglichen.
Zusätzlich wurden zum Zeitpunkt dieser Arbeit noch keine Informationen zu externen \gls{forschungsdaten}-Publikationen in den Metadaten von \gls{luh-repo} festgehalten.
Durch die Kombination dieser Faktoren, konnten keine empirischen Daten zur Nutzung unterschiedlicher Metadatenstandards auf Basis des \gls{luh-repo} und der sich darin befindenden Dissertationen gesammelt werden.

\section{Diskussion}\label{sec:luh-repo-discussion}
Die Ergebnisse in \cref{sec:luh-repo-results-factors} zeigten, dass alle evaluierten Faktoren zu unterschiedlich starken---aber insgesamt dennoch schwachen---Graden voneinander abhängig sind.
Dies verkompliziert die Analyse der jeweiligen Faktoren in Bezug auf deren Wirkung auf \glspl{forschungsdaten} und deren unterschiedlichen Publikationsformen.

\subsection{Sprache}
Ein gutes Beispiel hierfür ist der Faktor \textit{Sprache}.
Die Annahme hinter diesem Faktor war, dass die Nutzung einer internationalen Sprache (i.e.~\textit{Englisch}) eventuell einen Einfluss auf die Beachtung und Einhaltung internationaler \gls{fdm}-Standards haben könnte.
Auf ersten Blick scheint sich dies auch zu bewahrheiten, wie an zwei Beispielen gezeigt werden kann: 
Einerseits wurden ca.~zwei Drittel aller \glspl{pdd} mit externen \glspl{forschungsdaten} auf Englisch verfasst, obwohl die Mehrheit aller \glspl{pdd} auf Deutsch verfasst wurden, wie in \cref{sec:luh-repo-results-language} aufgezeigt wurde.
Andererseits sinkt die Nutzungsrate von \textit{Deutsch} und steigt für \textit{Englisch} an, je höher die Klassifikationsstufe einer \gls{pdd} ist---bis sie schließlich auf \textit{Level~1} Parität erreichen.
Beachtet man jedoch die Verteilung von \textit{Sprache} in \cref{fig:luh-repo_sprache_x_fakultät_x_zeitgruppe} über \textit{Fakultät}, so zeigt sich, dass die Nutzung von \textit{Englisch} für die meisten Fakultäten über die Jahre zugenommen hat.
In dem selben Zeitrahmen stieg auch die Nutzung von externen \glspl{forschungsdaten} an, wie sich in der Verteilung von \textit{Externe \glspl{forschungsdaten}} über die verschiedenen Fakultäten und Zeitgruppen hinweg in \cref{fig:luh-repo_fakultät_x_zeitgruppe_x_externe-fd} gezeigt wurde.

Dazu kommt, dass in einigen Fakultäten, die vergleichsweise viel \glspl{pdd} der dritten Stufe generieren, wie z.B.~in \gls{fakultät2}, der Anteil an englischsprachigen \glspl{pdd} sehr gering ist und auch nicht über die Jahre zugenommen hat, was die relative Verteilung von \textit{Stufe~3} unter den zwei Sprachen maßgeblich beeinflusst.

Insgesamt ergibt sich für \textit{Sprache} das Bild, dass der Faktor nur indirekt mit denen der \glspl{forschungsdaten} korreliert.
Es erscheint maßgeblicher, wie \textit{Sprache} durch \textit{Fakulät} geprägt wird und wie Fakultäten unterschiedliches Publikationsverhalten besitzen (wie in \cref{sec:luh-repo-results-faculties} gezeigt wurde).

\subsection{Externe Forschungsdaten}
Besonders interessant ist das Publikationsverhalten zu externen \glspl{forschungsdaten}---auch da diese die eigentlich ideale Publikationsform darstellen.
Betrachtet man \cref{fig:luh-repo_fakultät_x_zeitgruppe_x_externe-fd}, so ist deutlich erkennbar, dass nur drei der neun Fakultäten für den Hauptteil aller extern publizierten \glspl{forschungsdaten} verantwortlich sind und in mehr als einer Zeitgruppe externe \glspl{forschungsdaten} produzierten: \gls{fakultät3}, \gls{fakultät4} und \gls{fakultät8}.
Von diesen ist nutzt insbesondere \gls{fakultät4} externe \glspl{forschungsdaten} zu einem besonders hohen Anteil.
Jedoch nutzen, auch von diesen Fakultäten, nur die wenigsten dedizierte \gls{forschungsdaten}-Repositorien.
Stattdessen sind die meisten externen Publikationen Code auf GitHub~/~GitLab (für \gls{fakultät4}), in Datenbanken eingereichte Gensequenzen (für \gls{fakultät8}) oder Beireichungen zu Journal-Artikeln für kumulative Dissertationen.
Das bedeutet, dass der Anteil an extern publizierten \glspl{forschungsdaten} zwar insgesamt zunimmt, die Nutzung von entsprechenden Fachrepositorien bei Dissertationen jedoch weiterhin verschwindend gering bleibt.
Tatsächlich war der Anteil so gering, dass es keine externe \gls{forschungsdaten} Veröffentlichung gab, in deren Metadaten spezifiziert wurde, dass diese dem Werk einer Dissertation zugehörig sei, was es unmöglich machte, empirisch zu untersuchen, welche Metadateneinträge hierfür am häufigsten genutzt werden.
Insgesamt besteht dennoch ein Trend dazu, besonders hochwertige Daten eher extern zu publizieren, statt diese in der Dissertation integriert zu veröffentlichen.
Dies lässt sich daran erkennen, dass der Gesamtanteil an allgemeinen und externen \textit{Stufe~1} \glspl{pdd} über die Zeitgruppen stetig zugenommen hat, gleichzeitig aber für integrierte \glspl{forschungsdaten} abgenommen hat, wie in \cref{tab:luh-repo-classification-general-publication-adjusted} gezeigt wurde.

\subsection{Zeitliche Entwicklung von Forschungsdaten und Fakultäten}
Abgesehen davon, dass extern publizierte \glspl{forschungsdaten} über die Jahre zugenommen haben, lassen sich einige andere allgemeine Trends für \glspl{pdd} im \gls{luh-repo} erkennen.
Ein auf ersten Blick kontraintuitiver Trend ist, dass der Anteil an allgemeinen \glspl{pdd} ohne publizierte \glspl{forschungsdaten} für die letzte Jahresgruppe 2020--2023 stark zugenommen hat.
Betrachtet man jedoch die allgemeine Nutzungsrate des \gls{luh-repo}s nach Fakultäten und Zeitgruppen in \cref{tab:luh-repo-zahlenspiegel-summary}, lässt sich feststellen, dass sich die Nutzungsrate für die letzte Zeitgruppe besonders stark verändert hat:
Insbesondere in der vorherigen Zeitgruppe schwach vertretene Fakultäten wie \gls{fakultät3}, \gls{fakultät4} und \gls{fakultät9} haben ihren Anteil an \gls{luh-repo}-Dissertationen stark erhöht.
Auch \gls{fakultät6}, welche besonders notorisch zu sein scheint, wenige \glspl{forschungsdaten} zu veröffentlichen, hat seinen Anteil um ca.~$\SI{65}{\percent}$ gegenüber der vorherigen Zeitgruppe erhöht.
Während diese Kompositionsveränderung des \gls{luh-repo}s den abnehmenden \gls{forschungsdaten} Trend zumindest zum Teil erklären, zeigt \cref{fig:luh-repo_fakultät_x_zeitgruppe_x_fd} dennoch auf, dass dieser Trend auch einige Fakultäten intern betrifft:
Nämlich \gls{fakultät2}, \gls{fakultät3}, \gls{fakultät5} und \gls{fakultät6} (und in deutlich geringerem Umfang auch \gls{fakultät8} und \gls{fakultät10}).
Was genau diese Abnahme verursacht ist nicht bekannt und Bedarf weiterer Beobachtung und Erforschung.

Was die Option betrifft, \glspl{forschungsdaten} begleitend zu der Dissertation im \gls{luh-repo} zu veröffentlichen, so lässt sich hierzu sagen, dass es eine sehr wenig genutzte Möglichkeit ist.
Wie in \cref{fig:luh-repo_fakultät_x_zeitgruppe_x_begleit-fd} gezeigt wurde, gibt es allerdings einen kleinen aber konsistenten Kern an Nutzern, welcher statistisch signifikant insbesondere \gls{fakultät8} und \gls{fakultät9} zugeordnet werden kann.
Es erscheint unwahrscheinlich, dass die Nutzungsrate dieser Publikationsart sich in über die nächsten Jahre ändern wird, da zumindest bisher Zeit kein statistisch signifikanter Faktor war (siehe \cref{sec:luh-repo-results-time}).

\subsection{Handlungsempfehlungen}
Die Resultate in \cref{sec:luh-repo-results} zeigen mehrere Möglichkeiten auf, was die \gls{tib}~/~die \gls{luh} tun oder beachten können, um ihre Services weiter auszubauen.
Diese Möglichkeiten teilen sich in zwei Hauptbereiche: Nutzung des \gls{luh-repo}s und verbessertes \gls{fdm} durch die Promovierenden.

\subsubsection{Nutzung des \gls{luh-repo}s}
Wie in \cref{sec:luh-repo-results-zahlenspiegel} dargestellt wird das \gls{luh-repo} von den verschiedenen Fakultäten der \gls{luh} unterschiedlich stark für die Publikation von Dissertationen genutzt.
Während Promovierende aus sieben der neun Fakultäten inzwischen mindestens annäherend die Hälfte aller ihrer neuen Dissertationen auf \gls{luh-repo} publizieren, nämlich die Intermediär- und Intensivnutzer des Repositoriums, gibt es hierbei zwei bedeutende Ausreißer:
Die Geringnutzer des \gls{luh-repo}, \gls{fakultät5} und \gls{fakultät6}, welche weniger als ein Viertel ihrer neuen Dissertationen im \gls{luh-repo} publizieren.
Dieser Befund stimmt mit früherer Forschung überein, da diese zeigte, dass sowohl die Ingenieurswissenschaften (und verwandte Unterdisziplinen) wie auch die Rechtswissenschaftne besonders schwach im Open Access Bereich vertreten sind \autocite{Archambault2014,Piwowar2018,Severin2020-Jura,Hamann2019-OA,Fischer2022-Jura}.
Dennoch bedeuten diese geringen Werte, dass hier aktiver Handlungsbedarf seitens der \gls{luh} und der \gls{tib} besteht.
So sollte vermehrt Aufklärungs-, Beratungs- und Werbearbeit zielgerichtet auf \gls{fakultät5} und \gls{fakultät6} gerichtet werden.

So kann hier insbesondere im Bereich der Rechtswissenschaft erfolgreiche Aufklärungsarbeit stattfinden, da Rechtswissenschaftler sich häufig nicht der Möglichkeit von Open Access bewusst sind---so sind z.B. die wenigsten mit ihrem Zweitveröffentlichungsrecht vertraut \autocite[91]{Eisentraut}.
Gleichzeitig ist in den Rechtswissenschaft die Verbreitung von Dissertationen in Deutschland \enquote{äußerst mangelhaft} \autocite[50]{Steinhauer2019-OA}, wobei ein vermehrter Wandel auf Open Access Abhilfe schaffen könnte.

Allgemein sollte bei diesen beiden Fakultäten Kontakt aufgebaut werden und jeweils gemeinsam besprochen werden, welche Bedenken vor einer höheren Nutzung des \gls{luh-repo}s bestehen, die diversen Vorteile von Open Access erörtert \autocite{Bautista-Puig2020} und auch aufgezeigt werden, dass z.B.~ein bisheriger Mangel an High Impact Open Access Publikationen kein Indikator dafür ist, dass ein entsprechender Markt nicht sehr schnell entwickelt werden kann \autocite{Björk2012}.

\subsubsection{Forschungsdaten}
\cref{sec:luh-repo-results} hat insgesamt gezeigt, dass nur ein Drittel aller Promovierenden von \glspl{pdd} aus dem \gls{luh-repo} \glspl{forschungsdaten} aus \textit{Stufe~1} und \textit{Stufe~2} publiziert hat, während ein weiteres Drittel überhaupt keine \glspl{forschungsdaten} publiziert hat.
Unter Ausschluss integrierter \glspl{forschungsdaten} stieg hierbei der Anteil jener, die keine \glspl{forschungsdaten} publizieren, sogar auf über \SI{90}{\percent}.

Entsprechend gibt es hier zwei separate aber verwandte Bereiche, bei der Promovierende an der \gls{luh} eventuell zusätzliche Unterstützung erfahren oder durch vermerte Werbearbeit auf existierende Angebote der \gls{luh}~/~\gls{tib} aufmerksam gemaht werden sollten:
die Identifikation von eigenen \glspl{pd}, die zur Verfügung gestellt werden sollten, und die Möglichkeit, wie dies durch externe Repositorien nach den \gls{fair}-Prinzipien stattfinden kann.

Ersteres kann insofern ein wichtiger Faktor sein, da die \gls{fdm}-Richtlinie der \gls{luh} eigentlich vorsieht, dass die \glspl{forschungsdaten} nach Möglichkeit publiziert werden sollten.
Dies scheint aber nicht in vollem Umfang durch die Promovierenden umgesetzt zu werden.
Eine Möglichkeit für diesen Umstand könnte sein, dass Promovierende einfach unterschätzen, welche \glspl{pd} für die Veröffentlichung relevant sein könnten.

Eine Möglichkeit, wie für das bestehende Beratungs- und Informationsangebot zu \gls{fdm} unter den Studierenden und Promovierenden Werbearbeit gemacht und gleichzeitig erweitert werden könnte, ist die Nutzung eines \gls{ki} Modells, welches darauf trainiert wurde, zu erkennen, welche Datensätze in einer Arbeit implizit genutzt werden und daher eventuell auch explizit veröffentlicht werden sollten.
Eine Möglichkeit hierzu wäre das \textit{DataSeer}-Projekt \autocite{dataseer}, welches entwickelt wurde, um Forschenden und Betreibern von wissenschaftlichen Journals dabei zu helfen, zu identifizieren, welche Datensätze einem Artikel zugrunde liegen und welche idealerweise geteilt werden sollten.
Da es sich bei \textit{DataSeer} um ein quelloffenes Projekt handelt, wäre hier eine Möglichkeit, dass die \gls{tib} oder die \gls{luh} eine eigene Instanz hostet, welche \gls{luh}-Angehörigen zur Verfügung steht.
Hier kann dann durch Plakate und andere Medien eine \enquote{Check Your Data}-Kampagne gestartet werden, welche den \gls{ki}-Aspekt des Projektes in den Vordergrund stellt, um die Neugier der Studierenden und Promovierenden dafür zu nutzen, ihre wissenschaftliche Praxis oder zumindest ihren wissenschaftliche \gls{forschungsdaten}-Achtsamkeit zu verbessern.

Was die Möglichkeit betrifft, \glspl{forschungsdaten} in ein externes Repositorium hochzuladen, so könnte ein mangelndes Wissen um welche Repositorien existieren und wie diese genutzt werden können ein Faktor sein, warum so wenige \glspl{pdd} im \gls{luh-repo} externe \glspl{forschungsdaten} in dedizierte \gls{forschungsdaten}-Repositorien hochgeladen haben.
Hier sollte vermehrt Werbearbeit für das institutionelle \gls{forschungsdaten}-Repositorium und für die allgemeine \gls{fdm}-Informationswebseite der \gls{luh} gemacht werden.
Eine Option hierfür wäre, den bereits existierenden autodidaktischen \gls{fdm}-Einführungskurs der \gls{luh} für Studierende und Promovierende verpflichtend zu machen, sodass diese einen Nachweis hierzu erbringen müssen.
Für Studierende müsste dies in den Semestern vor der Abschlussarbeit und für Promovierende in den ersten Semestern ihrer Forschungstätigkeit der Fall sein, um die gewünschte Wirkung zu maximieren.
Auch könnte, unter Absprache mit den Fakultäten und der Universitätsleitung, eine verpflichtende Erstellung eines \gls{dmp} einerseits die wissenschaftliche Praxis im Allgemeinen verbessern und andererseits die Thematik des \gls{fdm} den Studierenden bzw. Promovierenden eine höhere Relevanz vermitteln.


Hierbei Bedarf es einer tatkräftigen und überzeugten Unterstützung durch die Universitätsleitung der \gls{luh}, da die Erfahrung von im \gls{fdm} tätigen Bibliothekaren gezeigt hat, dass dies eine unbedingte Notwendigkeit ist \autocite{Dellmann2022}.

\glsresetall
        \chapter{Schlussfolgerungen}\label{ch:schlussfolgerungen}

    %
    %% Hinterer Teil
    {\backmatter%GESCHWEIFTE KLAMMERN NICHT LÖSCHEN! Beeinflusst sonst den Appendix!
        \emergencystretch=1em\printbibliography %Bibliografie mit besserem Umbruch
    }%GESCHWEIFTE KLAMMERN NICHT LÖSCHEN! Beeinflusst sonst den Appendix!
    %
    %% Appendix
    \appendix
    \part*{Appendix}
    \chapter{Richtlinienklassifikation}\label{appendix:richtlinienklassifikation}
\begin{table}[!htbp]
	\caption{Klassifikation der allgemeingültigen verwaltungsrechtlichen Dokumente in relativer Angabe nach Bundesland. Absolute Werte in Klammern angegeben.}
    \resizebox{\ifdim\width>\textwidth\textwidth\else\width\fi}{!}{%
        \begin{tabular}{lS[table-format=3.2]@{\,}S[table-text-alignment = left]lS[table-format=3.2]@{\,}S[table-text-alignment = left]lS[table-format=3.2]@{\,}S[table-text-alignment = left]lS[table-format=3.2]@{\,}S[table-text-alignment = left]l}
            \toprule
            & \multicolumn{3}{c}{\textbf{Keine Verfügbar}} & \multicolumn{3}{c}{\textbf{\gls{gwp}-Richtlinien}} & \multicolumn{3}{c}{\textbf{Andere Richtlinien}} & \multicolumn{3}{c}{\textbf{\gls{forschungsdaten}-Richtlinien}}    \\
            \midrule
            \textbf{DE-BW}          & 23,53 & \si{\percent}  & (4)  & 41,18 & \si{\percent} & (7)  & 0,00 & \si{\percent} & (0)  & 35,29  & \si{\percent} & (6) \\
            \textbf{DE-BY}          & 16,67  & \si{\percent} & (2)  & 50,00 & \si{\percent} & (6)  & 0,00 & \si{\percent} & (0)  & 33,33  & \si{\percent} & (4)  \\
            \textbf{DE-BE}          & 20,00 & \si{\percent}  & (1)  & 20,00 & \si{\percent} & (1)  & 0,00 & \si{\percent} & (0)  & 60,00  & \si{\percent} & (3)  \\
            \textbf{DE-BB}          & 0,00 & \si{\percent}   & (0)  & 25,00 & \si{\percent} & (1)  & 0,00 & \si{\percent} & (0)  & 75,00  & \si{\percent} & (3)  \\
            \textbf{DE-HB}          & 12,50 & \si{\percent}  & (0)  & 50,00 & \si{\percent} & (1)  & 0,00 & \si{\percent} & (0)  & 50,00  & \si{\percent} & (1)  \\
            \textbf{DE-HH}          & 10,00 & \si{\percent}  & (1)  & 75,00 & \si{\percent} & (6)  & 0,00 & \si{\percent} & (0)  & 12,50  & \si{\percent} & (1)  \\
            \textbf{DE-HE}          & 50,00 & \si{\percent}  & (1)  & 40,00 & \si{\percent} & (4)  & 0,00 & \si{\percent} & (0)  & 50,00  & \si{\percent} & (5)  \\
            \textbf{DE-MV}          & 9,09 & \si{\percent}   & (1)  & 50,00 & \si{\percent} & (1)  & 0,00 & \si{\percent} & (0)  & 0,00   & \si{\percent} & (0)  \\
            \textbf{DE-NI}          & 9,09 & \si{\percent}   & (1)  & 0,00  & \si{\percent} & (0)  & 0,00 & \si{\percent} & (0)  & 90,91  & \si{\percent} & (10)  \\
            \textbf{DE-NW}          & 0,00 & \si{\percent}   & (2)  & 40,91 & \si{\percent} & (9)  & 0,00 & \si{\percent} & (0)  & 50,00  & \si{\percent} & (11)  \\
            \textbf{DE-RP}          & 0,00 & \si{\percent}   & (0)  & 80,00 & \si{\percent} & (4)  & 0,00 & \si{\percent} & (0)  & 20,00  & \si{\percent} & (1)  \\
            \textbf{DE-SL}          & 0,00 & \si{\percent}   & (0)  & 50,00 & \si{\percent} & (1)  & 0,00 & \si{\percent} & (0)  & 50,00  & \si{\percent} & (1)  \\
            \textbf{DE-SN}          & 0,00 & \si{\percent}   & (0)  & 60,00 & \si{\percent} & (3)  & 0,00 & \si{\percent} & (0)  & 40,00  & \si{\percent} & (2)  \\
            \textbf{DE-ST}          & 0,00 & \si{\percent}   & (0)  & 66,67 & \si{\percent} & (4)  & 0,00 & \si{\percent} & (0)  & 33,33  & \si{\percent} & (2)  \\
            \textbf{DE-SH}          & 50,00 & \si{\percent}  & (1)  & 50,00 & \si{\percent} & (1)  & 0,00 & \si{\percent} & (0)  & 0,00   & \si{\percent} & (0)  \\
            \textbf{DE-TH}          & 0,00 & \si{\percent}   & (0)  & 50,00 & \si{\percent} & (1)  & 0,00 & \si{\percent} & (0)  & 50,00  & \si{\percent} & (1)  \\
            \midrule
            \textbf{Alle}           & 12,17 & \si{\percent}  & (14) & 43,48 & \si{\percent} & (50) & 0,00 & \si{\percent} & (0) & 44,35 & \si{\percent} & (51) \\
            \bottomrule
        \end{tabular}
    }
	\label{tab:stichprobe-klassifikation-allgemein-bundesland}
\end{table}

\begin{table}[!htbp]
	\caption{Klassifikation der allgemeingültigen verwaltungsrechtlichen Dokumente in relativer Angabe nach Bundesland. Absolute Werte in Klammern angegeben. Angabe der Bundesländer nach ISO 3166-2:2020.~\autocite{ISO3166}}
    \resizebox{\ifdim\width>\textwidth\textwidth\else\width\fi}{!}{%
        \begin{tabular}{l S[table-format=3.2]@{\,}S[table-text-alignment = left]l S[table-format=3.2]@{\,}S[table-text-alignment = left]l S[table-format=3.2]@{\,}S[table-text-alignment = left]l S[table-format=3.2]@{\,}S[table-text-alignment = left]l S[table-format=3.2]@{\,}S[table-text-alignment = left]l S[table-format=3.2]@{\,}S[table-text-alignment = left]l}
            \toprule
            & \multicolumn{6}{c}{\textbf{Kein(e)}} & \multicolumn{6}{c}{\textbf{\gls{gwp}-Richtlinie}} & \multicolumn{6}{c}{\textbf{\gls{forschungsdaten}-Richtlinie}}\\
            \cmidrule(lr){2-7}\cmidrule(lr){8-13}\cmidrule(lr){14-19}
            & \multicolumn{3}{c}{\textbf{Zugang}} & \multicolumn{3}{c}{\textbf{\gls{forschungsdaten}-Richtlinie}} & \multicolumn{3}{c}{\textbf{Empfehlung}} & \multicolumn{3}{c}{\textbf{Verpflichtung}} & \multicolumn{3}{c}{\textbf{Empfehlung}} & \multicolumn{3}{c}{\textbf{Verpflichtung}}\\
            \midrule
            \textbf{DE-BW}   & 0,00  & \si{\percent} & (0)  & 5,88  & \si{\percent} & (1)  & 0,00  & \si{\percent} & (0) & 82,35  & \si{\percent} & (14) & 0,00 & \si{\percent} & (0) & 11,76 & \si{\percent} & (2) \\
            \textbf{DE-BY}   & 0,00  & \si{\percent} & (0)  & 50,00 & \si{\percent} & (6)  & 0,00  & \si{\percent} & (0) & 50,00  & \si{\percent} & (6)  & 0,00 & \si{\percent} & (0) & 0,00  & \si{\percent} & (0)  \\
            \textbf{DE-BE}   & 20,00 & \si{\percent} & (1)  & 20,00 & \si{\percent} & (1)  & 20,00 & \si{\percent} & (1) & 0,00   & \si{\percent} & (0)  & 0,00 & \si{\percent} & (0) & 40,00 & \si{\percent} & (2)  \\
            \textbf{DE-BB}   & 0,00  & \si{\percent} & (0)  & 25,00 & \si{\percent} & (1)  & 0,00  & \si{\percent} & (0) & 50,00  & \si{\percent} & (2)  & 0,00 & \si{\percent} & (0) & 25,00 & \si{\percent} & (1)  \\
            \textbf{DE-HB}   & 0,00  & \si{\percent} & (0)  & 50,00 & \si{\percent} & (1)  & 0,00  & \si{\percent} & (0) & 50,00  & \si{\percent} & (1)  & 0,00 & \si{\percent} & (0) & 0,00  & \si{\percent} & (0)  \\
            \textbf{DE-HH}   & 0,00  & \si{\percent} & (0)  & 25,00 & \si{\percent} & (2)  & 20,00 & \si{\percent} & (2) & 50,00  & \si{\percent} & (4)  & 0,00 & \si{\percent} & (0) & 0,00  & \si{\percent} & (0)  \\
            \textbf{DE-HE}   & 10,00 & \si{\percent} & (1)  & 20,00 & \si{\percent} & (2)  & 0,00  & \si{\percent} & (0) & 20,00  & \si{\percent} & (2)  & 10,00& \si{\percent} & (1) & 40,00 & \si{\percent} & (4)  \\
            \textbf{DE-MV}   & 0,00  & \si{\percent} & (0)  & 0,00  & \si{\percent} & (0)  & 0,00  & \si{\percent} & (0) & 100,00 & \si{\percent} & (2)  & 0,00 & \si{\percent} & (0) & 0,00  & \si{\percent} & (0)  \\
            \textbf{DE-NI}   & 9,09  & \si{\percent} & (1)  & 9,09  & \si{\percent} & (1)  & 0,00  & \si{\percent} & (0) & 81,82  & \si{\percent} & (9)  & 0,00 & \si{\percent} & (0) & 0,00  & \si{\percent} & (0)  \\
            \textbf{DE-NW}   & 0,00  & \si{\percent} & (0)  & 40,91 & \si{\percent} & (9)  & 0,00  & \si{\percent} & (0) & 50,00  & \si{\percent} & (11) & 0,00 & \si{\percent} & (0) & 9,09  & \si{\percent} & (2)  \\
            \textbf{DE-RP}   & 0,00  & \si{\percent} & (0)  & 20,00 & \si{\percent} & (1)  & 0,00  & \si{\percent} & (0) & 40,00  & \si{\percent} & (2)  & 0,00 & \si{\percent} & (0) & 40,00 & \si{\percent} & (2)  \\
            \textbf{DE-SL}   & 0,00  & \si{\percent} & (0)  & 50,00 & \si{\percent} & (1)  & 50,00 & \si{\percent} & (1) & 0,00   & \si{\percent} & (0)  & 0,00 & \si{\percent} & (0) & 0,00  & \si{\percent} & (0)  \\
            \textbf{DE-SN}   & 0,00  & \si{\percent} & (0)  & 20,00 & \si{\percent} & (1)  & 0,00  & \si{\percent} & (0) & 60,00  & \si{\percent} & (3)  & 0,00 & \si{\percent} & (0) & 20,00 & \si{\percent} & (1)  \\
            \textbf{DE-ST}   & 0,00  & \si{\percent} & (0)  & 16,67 & \si{\percent} & (1)  & 0,00  & \si{\percent} & (0) & 50,00  & \si{\percent} & (3)  & 0,00 & \si{\percent} & (0) & 33,33 & \si{\percent} & (2)  \\
            \textbf{DE-SH}   & 0,00  & \si{\percent} & (0)  & 50,00 & \si{\percent} & (1)  & 0,00  & \si{\percent} & (0) & 50,00  & \si{\percent} & (1)  & 0,00 & \si{\percent} & (0) & 0,00  & \si{\percent} & (0)  \\
            \textbf{DE-TH}   & 50,00 & \si{\percent} & (1)  & 0,00  & \si{\percent} & (0)  & 0,00  & \si{\percent} & (0) & 50,00  & \si{\percent} & (1)  & 0,00 & \si{\percent} & (0) & 0,00  & \si{\percent} & (0)  \\
            \midrule
            \textbf{Alle}    & 3,48 & \si{\percent}  & (4)  & 25,22 & \si{\percent} & (29)  & 3,48 & \si{\percent} & (4) & 53,04  & \si{\percent} & (61) & 0,87 & \si{\percent} & (1) & 13,91  & \si{\percent} & (16) \\
            \bottomrule
        \end{tabular}
    }
	\label{tab:stichprobe-klassifikation-spezifisch-bundesland}
\end{table}
    \chapter{LUH-Repo Skripte}
\section{Tabellen}
\begin{table}[!htbp]
	\caption{\gls{forschungsdaten}-Klassifizierung der Dissertationen aus der Stichprobe nach $\text{\textit{Fakultät}}\times\text{\textit{Klassifikationsstufe}}$ aufgegliedert.
    Angabe relativ zu der respektiven Gesamtanzahl für \textit{Fakultät}.
    Absolute Werte in Klammern angegeben.}
    \resizebox{\ifdim\width>\textwidth\textwidth\else\width\fi}{!}{%
	\begin{tabular}{lS[table-format=3.2]@{\,}S[table-text-alignment = left]lS[table-format=3.2]@{\,}S[table-text-alignment = left]lS[table-format=3.2]@{\,}S[table-text-alignment = left]lS[table-format=3.2]@{\,}S[table-text-alignment = left]lS[table-format=3.2]@{\,}S[table-text-alignment = left]l}
		\toprule
		& \multicolumn{3}{c}{\textbf{Stufe 1}} & \multicolumn{3}{c}{\textbf{Stufe 2}} & \multicolumn{3}{c}{\textbf{Stufe 3}} & \multicolumn{3}{c}{\textbf{Keine}}  \\
		\midrule
		\textbf{\gls{fakultät2}}  & 13,33  & \si{\percent} & (8)  & 0,00  & \si{\percent} & (0)  & 46,67  & \si{\percent} & (28) & 40,00   & \si{\percent} & (24)  \\
		\textbf{\gls{fakultät3}}  & 25,45  & \si{\percent} & (28)  & 11,82  & \si{\percent} & (13)  & 11,82  & \si{\percent} & (13)  & 50,91    & \si{\percent} & (56)\\
		\textbf{\gls{fakultät4}}  & 28,97  & \si{\percent} & (42)  & 12,41  & \si{\percent} & (18)  & 17,93  & \si{\percent} & (26)  & 40,69   & \si{\percent} & (59)\\
		\textbf{\gls{fakultät5}}  & 0,00   & \si{\percent} & (0)   & 0,00 & \si{\percent} & (0)   & 0,00 & \si{\percent} & (0)  & 100,00 & \si{\percent} & (4)\\
		\textbf{\gls{fakultät6}}  & 12,40  & \si{\percent} & (15)  & 4,13  & \si{\percent} & (5)  & 19,01  & \si{\percent} & (23)  & 64,46    & \si{\percent} & (78)\\
		\textbf{\gls{fakultät7}}  & 13,52  & \si{\percent} & (33)  & 4,10  & \si{\percent} & (10)  & 20,08  & \si{\percent} & (49)  & 62,30    & \si{\percent} & (152)\\
		\textbf{\gls{fakultät8}}  & 22,17 & \si{\percent} & (102) & 20,87  & \si{\percent} & (96) & 40,43 & \si{\percent} & (186)  & 16,52    & \si{\percent} & (76)\\
		\textbf{\gls{fakultät9}}  & 12,90  & \si{\percent} & (16)  & 0,00  & \si{\percent} & (0)  & 19,35  & \si{\percent} & (24)  & 67,74    & \si{\percent} & (84)\\
		\textbf{\gls{fakultät10}} & 4,62  & \si{\percent} & (8)  & 0,58  & \si{\percent} & (1)  & 13,87  & \si{\percent} & (24)  & 80,92    & \si{\percent} & (140)\\
		\midrule
		\textbf{Alle}            & 17,49 & \si{\percent} & (252) & 9,92 & \si{\percent} & (143) & 25,88 & \si{\percent} & (373) & 46,70 & \si{\percent} & (673)\\
		\bottomrule
	\end{tabular}
}
    \label{tab:luh-repo-classification-general-all-faculty}
\end{table}

\begin{table}[!htbp]
	\caption{\gls{forschungsdaten}-Klassifizierung der Dissertationen aus der Stichprobe nach $\text{\textit{Publikationsart}}\times\text{\textit{Klassifikationsstufe}}\times\text{\textit{Jahresgruppe}}$ aufgegliedert.
    Angaben relativ zu der Gesamtanzahl der Jahresgruppe.
    Absolute Werte in Klammern angegeben.}
    \resizebox{\ifdim\width>\textwidth\textwidth\else\width\fi}{!}{%
	\begin{tabular}{clS[table-format=3.2]@{\,}S[table-text-alignment = left]lS[table-format=3.2]@{\,}S[table-text-alignment = left]lS[table-format=3.2]@{\,}S[table-text-alignment = left]lS[table-format=3.2]@{\,}S[table-text-alignment = left]lS[table-format=3.2]@{\,}S[table-text-alignment = left]l}
		\toprule
		& & \multicolumn{3}{c}{\textbf{2012-2015}} & \multicolumn{3}{c}{\textbf{2016-2019}} & \multicolumn{3}{c}{\textbf{2020-2023}} & \multicolumn{3}{c}{\textbf{Alle}}  \\
		\midrule
		\parbox[t]{2mm}{\multirow{4}{*}{\rotatebox[origin=c]{90}{\textbf{Intern}}}}  & \textbf{Stufe 1} & 11,99 & \si{\percent} & (50)  & 12,36  & \si{\percent} & (54)  & 9,37  & \si{\percent} & (55)  & 11,03            & \si{\percent} & (159)\\
		                                                                             & \textbf{Stufe 2} & 11,51 & \si{\percent} & (48)  & 12,36  & \si{\percent} & (54)  & 11,58 & \si{\percent} & (68)  & 11,80            & \si{\percent} & (170)\\
		                                                                             & \textbf{Stufe 3} & 31,89 & \si{\percent} & (133) & 29,06  & \si{\percent} & (127) & 21,29 & \si{\percent} & (125) & 26,72            & \si{\percent} & (385)\\
		                                                                             & \textbf{Keine}   & 44,60 & \si{\percent} & (186) & 46,22  & \si{\percent} & (202) & 57,75 & \si{\percent} & (339) & 50,45            & \si{\percent} & (727)\\
        \midrule
		\parbox[t]{2mm}{\multirow{4}{*}{\rotatebox[origin=c]{90}{\textbf{Beilage}}}} & \textbf{Stufe 1} & 1,20  & \si{\percent} & (5)   & 2,29   & \si{\percent} & (10)  & 1,19  & \si{\percent} & (7)   & 1,53            & \si{\percent} & (22)\\
		                                                                             & \textbf{Stufe 2} & 0,00  & \si{\percent} & (0)   & 0,00   & \si{\percent} & (0)   & 0,00  & \si{\percent} & (0)   & 0,00            & \si{\percent} & (0)\\
		                                                                             & \textbf{Stufe 3} & 0,48  & \si{\percent} & (2)   & 0,23   & \si{\percent} & (1)   & 0,51  & \si{\percent} & (3)   & 0,42            & \si{\percent} & (6)\\
                                                                                     & \textbf{Keine}   & 98,32 & \si{\percent} & (410) & 97,48  & \si{\percent} & (426) & 98,30 & \si{\percent} & (577) & 98,06            & \si{\percent} & (1413)\\
        \midrule
		\parbox[t]{2mm}{\multirow{4}{*}{\rotatebox[origin=c]{90}{\textbf{Extern}}}}  & \textbf{Stufe 1} & 0,96  & \si{\percent} & (4)   & 2,75   & \si{\percent} & (12)  & 12,61 & \si{\percent} & (74)  & 6,25            & \si{\percent} & (90)\\
		                                                                             & \textbf{Stufe 2} & 0,00  & \si{\percent} & (0)   & 0,00   & \si{\percent} & (0)   & 0,17  & \si{\percent} & (1)   & 0,07            & \si{\percent} & (1)\\
		                                                                             & \textbf{Stufe 3} & 0,00  & \si{\percent} & (0)   & 0,23   & \si{\percent} & (1)   & 0,17  & \si{\percent} & (1)   & 0,14            & \si{\percent} & (2)\\
                                                                                     & \textbf{Keine}   & 99,04 & \si{\percent} & (413) & 97,03  & \si{\percent} & (424) & 87,05  & \si{\percent} & (511) & 93,55            & \si{\percent} & (1348)\\
        \midrule
        \parbox[t]{2mm}{\multirow{4}{*}{\rotatebox[origin=c]{90}{\textbf{Alle}}}}    & \textbf{Stufe 1} & 13,19 & \si{\percent} & (55)  & 16,93  & \si{\percent} & (74)  & 20,95 & \si{\percent} & (123) & 17,49            & \si{\percent} & (252)\\
                                                                                     & \textbf{Stufe 2} & 11,51 & \si{\percent} & (48)  & 10,53  & \si{\percent} & (46)  & 8,35  & \si{\percent} & (49)  & 9,92            & \si{\percent} & (143)\\
                                                                                     & \textbf{Stufe 3} & 31,65 & \si{\percent} & (132) & 28,60  & \si{\percent} & (125) & 19,76 & \si{\percent} & (116) & 25,88            & \si{\percent} & (373)\\
                                                                                     & \textbf{Keine}   & 43,65 & \si{\percent} & (182) & 43,94  & \si{\percent} & (192) & 50,94 & \si{\percent} & (299) & 46,70            & \si{\percent} & (673)\\
		\bottomrule
	\end{tabular}
}
    \label{tab:luh-repo-classification-general-publication}
\end{table}
\section{Grafiken}
\begin{figure}[!htbp]
    \resizebox{\ifdim\width>\textwidth\textwidth\else\width\fi}{!}{\begin{tikzpicture}[y=1cm, x=1cm, yscale=\globalscale,xscale=\globalscale, every node/.append style={scale=\globalscale}, inner sep=0pt, outer sep=0pt]
  \path[fill=white,line cap=round,line join=round,miter limit=10.0] ;
  \path[draw=white,fill=white,line cap=round,line join=round,line width=0.04cm,miter limit=10.0] (0.0, 17.78) rectangle (24.13, 0.0);
  \path[fill=cebebeb,line cap=round,line join=round,line width=0.04cm,miter limit=10.0] (1.62, 15.49) rectangle (4.06, 2.74);
  \path[draw=white,line cap=butt,line join=round,line width=0.02cm,miter limit=10.0] (1.62, 4.77) -- (4.06, 4.77);
  \path[draw=white,line cap=butt,line join=round,line width=0.02cm,miter limit=10.0] (1.62, 7.67) -- (4.06, 7.67);
  \path[draw=white,line cap=butt,line join=round,line width=0.02cm,miter limit=10.0] (1.62, 10.57) -- (4.06, 10.57);
  \path[draw=white,line cap=butt,line join=round,line width=0.02cm,miter limit=10.0] (1.62, 13.46) -- (4.06, 13.46);
  \path[draw=white,line cap=butt,line join=round,line width=0.04cm,miter limit=10.0] (1.62, 3.32) -- (4.06, 3.32);
  \path[draw=white,line cap=butt,line join=round,line width=0.04cm,miter limit=10.0] (1.62, 6.22) -- (4.06, 6.22);
  \path[draw=white,line cap=butt,line join=round,line width=0.04cm,miter limit=10.0] (1.62, 9.12) -- (4.06, 9.12);
  \path[draw=white,line cap=butt,line join=round,line width=0.04cm,miter limit=10.0] (1.62, 12.01) -- (4.06, 12.01);
  \path[draw=white,line cap=butt,line join=round,line width=0.04cm,miter limit=10.0] (1.62, 14.91) -- (4.06, 14.91);
  \path[draw=white,line cap=butt,line join=round,line width=0.04cm,miter limit=10.0] (2.08, 2.74) -- (2.08, 15.49);
  \path[draw=white,line cap=butt,line join=round,line width=0.04cm,miter limit=10.0] (2.84, 2.74) -- (2.84, 15.49);
  \path[draw=white,line cap=butt,line join=round,line width=0.04cm,miter limit=10.0] (3.6, 2.74) -- (3.6, 15.49);
  \path[fill=c77aadd,line cap=butt,line join=miter,line width=0.04cm,miter limit=10.0] (1.74, 14.91) rectangle (2.42, 13.13);
  \path[fill=c99dde1,line cap=butt,line join=miter,line width=0.04cm,miter limit=10.0] ;
  \path[fill=ceedd88,line cap=butt,line join=miter,line width=0.04cm,miter limit=10.0] (1.74, 13.13) rectangle (2.42, 6.89);
  \path[fill=cee8866,line cap=butt,line join=miter,line width=0.04cm,miter limit=10.0] (1.74, 6.89) rectangle (2.42, 3.32);
  \path[fill=c77aadd,line cap=butt,line join=miter,line width=0.04cm,miter limit=10.0] (2.5, 14.91) rectangle (3.18, 12.8);
  \path[fill=c99dde1,line cap=butt,line join=miter,line width=0.04cm,miter limit=10.0] ;
  \path[fill=ceedd88,line cap=butt,line join=miter,line width=0.04cm,miter limit=10.0] (2.5, 12.8) rectangle (3.18, 7.54);
  \path[fill=cee8866,line cap=butt,line join=miter,line width=0.04cm,miter limit=10.0] (2.5, 7.54) rectangle (3.18, 3.32);
  \path[fill=c77aadd,line cap=butt,line join=miter,line width=0.04cm,miter limit=10.0] (3.26, 14.91) rectangle (3.95, 13.98);
  \path[fill=c99dde1,line cap=butt,line join=miter,line width=0.04cm,miter limit=10.0] ;
  \path[fill=ceedd88,line cap=butt,line join=miter,line width=0.04cm,miter limit=10.0] (3.26, 13.98) rectangle (3.95, 8.89);
  \path[fill=cee8866,line cap=butt,line join=miter,line width=0.04cm,miter limit=10.0] (3.26, 8.89) rectangle (3.95, 3.32);
  \node[anchor=south] (text27) at (2.08, 14.15){15};
  \node[anchor=south] (text28) at (2.08, 13.64){(2)};
  \node[anchor=south] (text29) at (2.08, 10.14){54};
  \node[anchor=south] (text30) at (2.08, 9.63){(7)};
  \node[anchor=south] (text31) at (2.08, 5.23){31};
  \node[anchor=south] (text32) at (2.08, 4.73){(4)};
  \node[anchor=south] (text33) at (2.84, 13.99){18};
  \node[anchor=south] (text34) at (2.84, 13.48){(4)};
  \node[anchor=south] (text35) at (2.84, 10.3){45};
  \node[anchor=south] (text36) at (2.84, 9.79){(10)};
  \node[anchor=south] (text37) at (2.84, 5.56){36};
  \node[anchor=south] (text38) at (2.84, 5.05){(8)};
  \node[anchor=south] (text39) at (3.6, 14.58){8};
  \node[anchor=south] (text40) at (3.6, 14.07){(2)};
  \node[anchor=south] (text41) at (3.6, 11.56){44};
  \node[anchor=south] (text42) at (3.6, 11.06){(11)};
  \node[anchor=south] (text43) at (3.6, 6.23){48};
  \node[anchor=south] (text44) at (3.6, 5.73){(12)};
  \path[fill=cebebeb,line cap=round,line join=round,line width=0.04cm,miter limit=10.0] (4.11, 15.49) rectangle (6.54, 2.74);
  \path[draw=white,line cap=butt,line join=round,line width=0.02cm,miter limit=10.0] (4.11, 4.77) -- (6.54, 4.77);
  \path[draw=white,line cap=butt,line join=round,line width=0.02cm,miter limit=10.0] (4.11, 7.67) -- (6.54, 7.67);
  \path[draw=white,line cap=butt,line join=round,line width=0.02cm,miter limit=10.0] (4.11, 10.57) -- (6.54, 10.57);
  \path[draw=white,line cap=butt,line join=round,line width=0.02cm,miter limit=10.0] (4.11, 13.46) -- (6.54, 13.46);
  \path[draw=white,line cap=butt,line join=round,line width=0.04cm,miter limit=10.0] (4.11, 3.32) -- (6.54, 3.32);
  \path[draw=white,line cap=butt,line join=round,line width=0.04cm,miter limit=10.0] (4.11, 6.22) -- (6.54, 6.22);
  \path[draw=white,line cap=butt,line join=round,line width=0.04cm,miter limit=10.0] (4.11, 9.12) -- (6.54, 9.12);
  \path[draw=white,line cap=butt,line join=round,line width=0.04cm,miter limit=10.0] (4.11, 12.01) -- (6.54, 12.01);
  \path[draw=white,line cap=butt,line join=round,line width=0.04cm,miter limit=10.0] (4.11, 14.91) -- (6.54, 14.91);
  \path[draw=white,line cap=butt,line join=round,line width=0.04cm,miter limit=10.0] (4.57, 2.74) -- (4.57, 15.49);
  \path[draw=white,line cap=butt,line join=round,line width=0.04cm,miter limit=10.0] (5.33, 2.74) -- (5.33, 15.49);
  \path[draw=white,line cap=butt,line join=round,line width=0.04cm,miter limit=10.0] (6.09, 2.74) -- (6.09, 15.49);
  \path[fill=c77aadd,line cap=butt,line join=miter,line width=0.04cm,miter limit=10.0] (4.22, 14.91) rectangle (4.91, 13.69);
  \path[fill=c99dde1,line cap=butt,line join=miter,line width=0.04cm,miter limit=10.0] (4.22, 13.69) rectangle (4.91, 11.25);
  \path[fill=ceedd88,line cap=butt,line join=miter,line width=0.04cm,miter limit=10.0] (4.22, 11.25) rectangle (4.91, 8.81);
  \path[fill=cee8866,line cap=butt,line join=miter,line width=0.04cm,miter limit=10.0] (4.22, 8.81) rectangle (4.91, 3.32);
  \path[fill=c77aadd,line cap=butt,line join=miter,line width=0.04cm,miter limit=10.0] (4.98, 14.91) rectangle (5.67, 11.29);
  \path[fill=c99dde1,line cap=butt,line join=miter,line width=0.04cm,miter limit=10.0] (4.98, 11.29) rectangle (5.67, 10.57);
  \path[fill=ceedd88,line cap=butt,line join=miter,line width=0.04cm,miter limit=10.0] (4.98, 10.57) rectangle (5.67, 8.39);
  \path[fill=cee8866,line cap=butt,line join=miter,line width=0.04cm,miter limit=10.0] (4.98, 8.39) rectangle (5.67, 3.32);
  \path[fill=c77aadd,line cap=butt,line join=miter,line width=0.04cm,miter limit=10.0] (5.75, 14.91) rectangle (6.43, 11.77);
  \path[fill=c99dde1,line cap=butt,line join=miter,line width=0.04cm,miter limit=10.0] (5.75, 11.77) rectangle (6.43, 10.39);
  \path[fill=ceedd88,line cap=butt,line join=miter,line width=0.04cm,miter limit=10.0] (5.75, 10.39) rectangle (6.43, 9.81);
  \path[fill=cee8866,line cap=butt,line join=miter,line width=0.04cm,miter limit=10.0] (5.75, 9.81) rectangle (6.43, 3.32);
  \node[anchor=south] (text68) at (4.57, 14.43){11};
  \node[anchor=south] (text69) at (4.57, 13.92){(2)};
  \node[anchor=south] (text70) at (4.57, 12.6){21};
  \node[anchor=south] (text71) at (4.57, 12.09){(4)};
  \node[anchor=south] (text72) at (4.57, 10.16){21};
  \node[anchor=south] (text73) at (4.57, 9.65){(4)};
  \node[anchor=south] (text74) at (4.57, 6.2){47};
  \node[anchor=south] (text75) at (4.57, 5.69){(9)};
  \node[anchor=south] (text76) at (5.33, 13.23){31};
  \node[anchor=south] (text77) at (5.33, 12.72){(10)};
  \node[anchor=south] (text78) at (5.33, 11.06){6};
  \node[anchor=south] (text79) at (5.33, 10.55){(2)};
  \node[anchor=south] (text80) at (5.33, 9.61){19};
  \node[anchor=south] (text81) at (5.33, 9.1){(6)};
  \node[anchor=south] (text82) at (5.33, 5.99){44};
  \node[anchor=south] (text83) at (5.33, 5.48){(14)};
  \node[anchor=south] (text84) at (6.09, 13.47){27};
  \node[anchor=south] (text85) at (6.09, 12.96){(16)};
  \node[anchor=south] (text86) at (6.09, 11.21){12};
  \node[anchor=south] (text87) at (6.09, 10.7){(7)};
  \node[anchor=south,shift={(0.0, -0.16)}] (text88) at (6.09, 10.23){5};
  \node[anchor=south,shift={(0.0, -0.05)}] (text89) at (6.09, 9.72){(3)};
  \node[anchor=south] (text90) at (6.09, 6.69){56};
  \node[anchor=south] (text91) at (6.09, 6.19){(33)};
  \path[fill=cebebeb,line cap=round,line join=round,line width=0.04cm,miter limit=10.0] (6.59, 15.49) rectangle (9.03, 2.74);
  \path[draw=white,line cap=butt,line join=round,line width=0.02cm,miter limit=10.0] (6.59, 4.77) -- (9.03, 4.77);
  \path[draw=white,line cap=butt,line join=round,line width=0.02cm,miter limit=10.0] (6.59, 7.67) -- (9.03, 7.67);
  \path[draw=white,line cap=butt,line join=round,line width=0.02cm,miter limit=10.0] (6.59, 10.57) -- (9.03, 10.57);
  \path[draw=white,line cap=butt,line join=round,line width=0.02cm,miter limit=10.0] (6.59, 13.46) -- (9.03, 13.46);
  \path[draw=white,line cap=butt,line join=round,line width=0.04cm,miter limit=10.0] (6.59, 3.32) -- (9.03, 3.32);
  \path[draw=white,line cap=butt,line join=round,line width=0.04cm,miter limit=10.0] (6.59, 6.22) -- (9.03, 6.22);
  \path[draw=white,line cap=butt,line join=round,line width=0.04cm,miter limit=10.0] (6.59, 9.12) -- (9.03, 9.12);
  \path[draw=white,line cap=butt,line join=round,line width=0.04cm,miter limit=10.0] (6.59, 12.01) -- (9.03, 12.01);
  \path[draw=white,line cap=butt,line join=round,line width=0.04cm,miter limit=10.0] (6.59, 14.91) -- (9.03, 14.91);
  \path[draw=white,line cap=butt,line join=round,line width=0.04cm,miter limit=10.0] (7.05, 2.74) -- (7.05, 15.49);
  \path[draw=white,line cap=butt,line join=round,line width=0.04cm,miter limit=10.0] (7.81, 2.74) -- (7.81, 15.49);
  \path[draw=white,line cap=butt,line join=round,line width=0.04cm,miter limit=10.0] (8.57, 2.74) -- (8.57, 15.49);
  \path[fill=c77aadd,line cap=butt,line join=miter,line width=0.04cm,miter limit=10.0] (6.71, 14.91) rectangle (7.39, 12.8);
  \path[fill=c99dde1,line cap=butt,line join=miter,line width=0.04cm,miter limit=10.0] (6.71, 12.8) rectangle (7.39, 10.7);
  \path[fill=ceedd88,line cap=butt,line join=miter,line width=0.04cm,miter limit=10.0] (6.71, 10.7) rectangle (7.39, 8.24);
  \path[fill=cee8866,line cap=butt,line join=miter,line width=0.04cm,miter limit=10.0] (6.71, 8.24) rectangle (7.39, 3.32);
  \path[fill=c77aadd,line cap=butt,line join=miter,line width=0.04cm,miter limit=10.0] (7.47, 14.91) rectangle (8.15, 12.59);
  \path[fill=c99dde1,line cap=butt,line join=miter,line width=0.04cm,miter limit=10.0] (7.47, 12.59) rectangle (8.15, 10.53);
  \path[fill=ceedd88,line cap=butt,line join=miter,line width=0.04cm,miter limit=10.0] (7.47, 10.53) rectangle (8.15, 8.73);
  \path[fill=cee8866,line cap=butt,line join=miter,line width=0.04cm,miter limit=10.0] (7.47, 8.73) rectangle (8.15, 3.32);
  \path[fill=c77aadd,line cap=butt,line join=miter,line width=0.04cm,miter limit=10.0] (8.23, 14.91) rectangle (8.91, 10.24);
  \path[fill=c99dde1,line cap=butt,line join=miter,line width=0.04cm,miter limit=10.0] (8.23, 10.24) rectangle (8.91, 9.55);
  \path[fill=ceedd88,line cap=butt,line join=miter,line width=0.04cm,miter limit=10.0] (8.23, 9.55) rectangle (8.91, 7.47);
  \path[fill=cee8866,line cap=butt,line join=miter,line width=0.04cm,miter limit=10.0] (8.23, 7.48) rectangle (8.91, 3.32);
  \node[anchor=south] (text115) at (7.05, 13.99){18};
  \node[anchor=south] (text116) at (7.05, 13.48){(6)};
  \node[anchor=south] (text117) at (7.05, 11.88){18};
  \node[anchor=south] (text118) at (7.05, 11.37){(6)};
  \node[anchor=south] (text119) at (7.05, 9.6){21};
  \node[anchor=south] (text120) at (7.05, 9.09){(7)};
  \node[anchor=south] (text121) at (7.05, 5.91){42};
  \node[anchor=south] (text122) at (7.05, 5.4){(14)};
  \node[anchor=south] (text123) at (7.81, 13.88){20};
  \node[anchor=south] (text124) at (7.81, 13.37){(9)};
  \node[anchor=south] (text125) at (7.81, 11.69){18};
  \node[anchor=south] (text126) at (7.81, 11.19){(8)};
  \node[anchor=south] (text127) at (7.81, 9.76){16};
  \node[anchor=south] (text128) at (7.81, 9.25){(7)};
  \node[anchor=south] (text129) at (7.81, 6.16){47};
  \node[anchor=south] (text130) at (7.81, 5.65){(21)};
  \node[anchor=south] (text131) at (8.57, 12.7){40};
  \node[anchor=south] (text132) at (8.57, 12.2){(27)};
  \node[anchor=south,shift={(0.0, -0.11)}] (text133) at (8.57, 10.02){6};
  \node[anchor=south] (text134) at (8.57, 9.52){(4)};
  \node[anchor=south] (text135) at (8.57, 8.64){18};
  \node[anchor=south] (text136) at (8.57, 8.13){(12)};
  \node[anchor=south] (text137) at (8.57, 5.53){36};
  \node[anchor=south] (text138) at (8.57, 5.02){(24)};
  \path[fill=cebebeb,line cap=round,line join=round,line width=0.04cm,miter limit=10.0] (9.08, 15.49) rectangle (11.51, 2.74);
  \path[draw=white,line cap=butt,line join=round,line width=0.02cm,miter limit=10.0] (9.08, 4.77) -- (11.51, 4.77);
  \path[draw=white,line cap=butt,line join=round,line width=0.02cm,miter limit=10.0] (9.08, 7.67) -- (11.51, 7.67);
  \path[draw=white,line cap=butt,line join=round,line width=0.02cm,miter limit=10.0] (9.08, 10.57) -- (11.51, 10.57);
  \path[draw=white,line cap=butt,line join=round,line width=0.02cm,miter limit=10.0] (9.08, 13.46) -- (11.51, 13.46);
  \path[draw=white,line cap=butt,line join=round,line width=0.04cm,miter limit=10.0] (9.08, 3.32) -- (11.51, 3.32);
  \path[draw=white,line cap=butt,line join=round,line width=0.04cm,miter limit=10.0] (9.08, 6.22) -- (11.51, 6.22);
  \path[draw=white,line cap=butt,line join=round,line width=0.04cm,miter limit=10.0] (9.08, 9.12) -- (11.51, 9.12);
  \path[draw=white,line cap=butt,line join=round,line width=0.04cm,miter limit=10.0] (9.08, 12.01) -- (11.51, 12.01);
  \path[draw=white,line cap=butt,line join=round,line width=0.04cm,miter limit=10.0] (9.08, 14.91) -- (11.51, 14.91);
  \path[draw=white,line cap=butt,line join=round,line width=0.04cm,miter limit=10.0] (9.54, 2.74) -- (9.54, 15.49);
  \path[draw=white,line cap=butt,line join=round,line width=0.04cm,miter limit=10.0] (10.3, 2.74) -- (10.3, 15.49);
  \path[draw=white,line cap=butt,line join=round,line width=0.04cm,miter limit=10.0] (11.06, 2.74) -- (11.06, 15.49);
  \path[fill=c77aadd,line cap=butt,line join=miter,line width=0.04cm,miter limit=10.0] ;
  \path[fill=c99dde1,line cap=butt,line join=miter,line width=0.04cm,miter limit=10.0] ;
  \path[fill=ceedd88,line cap=butt,line join=miter,line width=0.04cm,miter limit=10.0] ;
  \path[fill=cee8866,line cap=butt,line join=miter,line width=0.04cm,miter limit=10.0] (9.95, 14.91) rectangle (10.64, 3.32);
  \path[fill=c77aadd,line cap=butt,line join=miter,line width=0.04cm,miter limit=10.0] ;
  \path[fill=c99dde1,line cap=butt,line join=miter,line width=0.04cm,miter limit=10.0] ;
  \path[fill=ceedd88,line cap=butt,line join=miter,line width=0.04cm,miter limit=10.0] ;
  \path[fill=cee8866,line cap=butt,line join=miter,line width=0.04cm,miter limit=10.0] (10.71, 14.91) rectangle (11.4, 3.32);
  \node[anchor=south] (text158) at (10.3, 9.25){100};
  \node[anchor=south] (text159) at (10.3, 8.74){(3)};
  \node[anchor=south] (text160) at (11.06, 9.25){100};
  \node[anchor=south] (text161) at (11.06, 8.74){(1)};
  \path[fill=cebebeb,line cap=round,line join=round,line width=0.04cm,miter limit=10.0] (11.56, 15.49) rectangle (14.0, 2.74);
  \path[draw=white,line cap=butt,line join=round,line width=0.02cm,miter limit=10.0] (11.56, 4.77) -- (14.0, 4.77);
  \path[draw=white,line cap=butt,line join=round,line width=0.02cm,miter limit=10.0] (11.56, 7.67) -- (14.0, 7.67);
  \path[draw=white,line cap=butt,line join=round,line width=0.02cm,miter limit=10.0] (11.56, 10.57) -- (14.0, 10.57);
  \path[draw=white,line cap=butt,line join=round,line width=0.02cm,miter limit=10.0] (11.56, 13.46) -- (14.0, 13.46);
  \path[draw=white,line cap=butt,line join=round,line width=0.04cm,miter limit=10.0] (11.56, 3.32) -- (14.0, 3.32);
  \path[draw=white,line cap=butt,line join=round,line width=0.04cm,miter limit=10.0] (11.56, 6.22) -- (14.0, 6.22);
  \path[draw=white,line cap=butt,line join=round,line width=0.04cm,miter limit=10.0] (11.56, 9.12) -- (14.0, 9.12);
  \path[draw=white,line cap=butt,line join=round,line width=0.04cm,miter limit=10.0] (11.56, 12.01) -- (14.0, 12.01);
  \path[draw=white,line cap=butt,line join=round,line width=0.04cm,miter limit=10.0] (11.56, 14.91) -- (14.0, 14.91);
  \path[draw=white,line cap=butt,line join=round,line width=0.04cm,miter limit=10.0] (12.02, 2.74) -- (12.02, 15.49);
  \path[draw=white,line cap=butt,line join=round,line width=0.04cm,miter limit=10.0] (12.78, 2.74) -- (12.78, 15.49);
  \path[draw=white,line cap=butt,line join=round,line width=0.04cm,miter limit=10.0] (13.54, 2.74) -- (13.54, 15.49);
  \path[fill=c77aadd,line cap=butt,line join=miter,line width=0.04cm,miter limit=10.0] (11.68, 14.91) rectangle (12.36, 13.75);
  \path[fill=c99dde1,line cap=butt,line join=miter,line width=0.04cm,miter limit=10.0] ;
  \path[fill=ceedd88,line cap=butt,line join=miter,line width=0.04cm,miter limit=10.0] (11.68, 13.75) rectangle (12.36, 9.5);
  \path[fill=cee8866,line cap=butt,line join=miter,line width=0.04cm,miter limit=10.0] (11.68, 9.5) rectangle (12.36, 3.32);
  \path[fill=c77aadd,line cap=butt,line join=miter,line width=0.04cm,miter limit=10.0] (12.44, 14.91) rectangle (13.12, 13.89);
  \path[fill=c99dde1,line cap=butt,line join=miter,line width=0.04cm,miter limit=10.0] (12.44, 13.89) rectangle (13.12, 13.55);
  \path[fill=ceedd88,line cap=butt,line join=miter,line width=0.04cm,miter limit=10.0] (12.44, 13.55) rectangle (13.12, 10.14);
  \path[fill=cee8866,line cap=butt,line join=miter,line width=0.04cm,miter limit=10.0] (12.44, 10.14) rectangle (13.12, 3.32);
  \path[fill=c77aadd,line cap=butt,line join=miter,line width=0.04cm,miter limit=10.0] (13.2, 14.91) rectangle (13.88, 13.08);
  \path[fill=c99dde1,line cap=butt,line join=miter,line width=0.04cm,miter limit=10.0] (13.2, 13.08) rectangle (13.88, 12.27);
  \path[fill=ceedd88,line cap=butt,line join=miter,line width=0.04cm,miter limit=10.0] (13.2, 12.27) rectangle (13.88, 11.86);
  \path[fill=cee8866,line cap=butt,line join=miter,line width=0.04cm,miter limit=10.0] (13.2, 11.86) rectangle (13.88, 3.32);
  \node[anchor=south] (text185) at (12.02, 14.46){10};
  \node[anchor=south] (text186) at (12.02, 13.95){(3)};
  \node[anchor=south] (text187) at (12.02, 11.76){37};
  \node[anchor=south] (text188) at (12.02, 11.25){(11)};
  \node[anchor=south] (text189) at (12.02, 6.54){53};
  \node[anchor=south] (text190) at (12.02, 6.04){(16)};
  \node[anchor=south] (text191) at (12.78, 14.53){9};
  \node[anchor=south,shift={(0.0, 0.11)}] (text192) at (12.78, 14.02){(3)};
  \node[anchor=south,shift={(0.0, -0.16)}] (text193) at (12.78, 13.85){3};
  \node[anchor=south,shift={(0.0, -0.05)}] (text194) at (12.78, 13.34){(1)};
  \node[anchor=south] (text195) at (12.78, 11.97){29};
  \node[anchor=south] (text196) at (12.78, 11.47){(10)};
  \node[anchor=south] (text197) at (12.78, 6.86){59};
  \node[anchor=south] (text198) at (12.78, 6.35){(20)};
  \node[anchor=south] (text199) at (13.54, 14.12){16};
  \node[anchor=south] (text200) at (13.54, 13.62){(9)};
  \node[anchor=south] (text201) at (13.54, 12.8){7};
  \node[anchor=south,shift={(0.0, 0.16)}] (text202) at (13.54, 12.3){(4)};
  \node[anchor=south,shift={(0.0, -0.11)}] (text203) at (13.54, 12.19){4};
  \node[anchor=south] (text204) at (13.54, 11.69){(2)};
  \node[anchor=south] (text205) at (13.54, 7.72){74};
  \node[anchor=south] (text206) at (13.54, 7.21){(42)};
  \path[fill=cebebeb,line cap=round,line join=round,line width=0.04cm,miter limit=10.0] (14.05, 15.49) rectangle (16.48, 2.74);
  \path[draw=white,line cap=butt,line join=round,line width=0.02cm,miter limit=10.0] (14.05, 4.77) -- (16.48, 4.77);
  \path[draw=white,line cap=butt,line join=round,line width=0.02cm,miter limit=10.0] (14.05, 7.67) -- (16.48, 7.67);
  \path[draw=white,line cap=butt,line join=round,line width=0.02cm,miter limit=10.0] (14.05, 10.57) -- (16.48, 10.57);
  \path[draw=white,line cap=butt,line join=round,line width=0.02cm,miter limit=10.0] (14.05, 13.46) -- (16.48, 13.46);
  \path[draw=white,line cap=butt,line join=round,line width=0.04cm,miter limit=10.0] (14.05, 3.32) -- (16.48, 3.32);
  \path[draw=white,line cap=butt,line join=round,line width=0.04cm,miter limit=10.0] (14.05, 6.22) -- (16.48, 6.22);
  \path[draw=white,line cap=butt,line join=round,line width=0.04cm,miter limit=10.0] (14.05, 9.12) -- (16.48, 9.12);
  \path[draw=white,line cap=butt,line join=round,line width=0.04cm,miter limit=10.0] (14.05, 12.01) -- (16.48, 12.01);
  \path[draw=white,line cap=butt,line join=round,line width=0.04cm,miter limit=10.0] (14.05, 14.91) -- (16.48, 14.91);
  \path[draw=white,line cap=butt,line join=round,line width=0.04cm,miter limit=10.0] (14.5, 2.74) -- (14.5, 15.49);
  \path[draw=white,line cap=butt,line join=round,line width=0.04cm,miter limit=10.0] (15.27, 2.74) -- (15.27, 15.49);
  \path[draw=white,line cap=butt,line join=round,line width=0.04cm,miter limit=10.0] (16.03, 2.74) -- (16.03, 15.49);
  \path[fill=c77aadd,line cap=butt,line join=miter,line width=0.04cm,miter limit=10.0] (14.16, 14.91) rectangle (14.85, 13.91);
  \path[fill=c99dde1,line cap=butt,line join=miter,line width=0.04cm,miter limit=10.0] (14.16, 13.91) rectangle (14.85, 13.62);
  \path[fill=ceedd88,line cap=butt,line join=miter,line width=0.04cm,miter limit=10.0] (14.16, 13.62) rectangle (14.85, 10.76);
  \path[fill=cee8866,line cap=butt,line join=miter,line width=0.04cm,miter limit=10.0] (14.16, 10.76) rectangle (14.85, 3.32);
  \path[fill=c77aadd,line cap=butt,line join=miter,line width=0.04cm,miter limit=10.0] (14.92, 14.91) rectangle (15.61, 13.75);
  \path[fill=c99dde1,line cap=butt,line join=miter,line width=0.04cm,miter limit=10.0] (14.92, 13.75) rectangle (15.61, 13.09);
  \path[fill=ceedd88,line cap=butt,line join=miter,line width=0.04cm,miter limit=10.0] (14.92, 13.09) rectangle (15.61, 9.95);
  \path[fill=cee8866,line cap=butt,line join=miter,line width=0.04cm,miter limit=10.0] (14.92, 9.95) rectangle (15.61, 3.32);
  \path[fill=c77aadd,line cap=butt,line join=miter,line width=0.04cm,miter limit=10.0] (15.68, 14.91) rectangle (16.37, 12.54);
  \path[fill=c99dde1,line cap=butt,line join=miter,line width=0.04cm,miter limit=10.0] (15.68, 12.54) rectangle (16.37, 12.05);
  \path[fill=ceedd88,line cap=butt,line join=miter,line width=0.04cm,miter limit=10.0] (15.68, 12.05) rectangle (16.37, 10.8);
  \path[fill=cee8866,line cap=butt,line join=miter,line width=0.04cm,miter limit=10.0] (15.68, 10.8) rectangle (16.37, 3.32);
  \node[anchor=south] (text230) at (14.5, 14.54){9};
  \node[anchor=south,shift={(0.0, 0.11)}] (text231) at (14.5, 14.03){(7)};
  \node[anchor=south,shift={(0.0, -0.26)}] (text232) at (14.5, 13.89){2};
  \node[anchor=south,shift={(0.0, -0.11)}] (text233) at (14.5, 13.39){(2)};
  \node[anchor=south] (text234) at (14.5, 12.32){25};
  \node[anchor=south] (text235) at (14.5, 11.81){(20)};
  \node[anchor=south] (text236) at (14.5, 7.17){64};
  \node[anchor=south] (text237) at (14.5, 6.67){(52)};
  \node[anchor=south] (text238) at (15.27, 14.46){10};
  \node[anchor=south,shift={(0.0, 0.11)}] (text239) at (15.27, 13.95){(7)};
  \node[anchor=south,shift={(0.0, -0.11)}] (text240) at (15.27, 13.55){6};
  \node[anchor=south,shift={(0.0, -0.05)}] (text241) at (15.27, 13.04){(4)};
  \node[anchor=south] (text242) at (15.27, 11.65){27};
  \node[anchor=south] (text243) at (15.27, 11.14){(19)};
  \node[anchor=south] (text244) at (15.27, 6.76){57};
  \node[anchor=south] (text245) at (15.27, 6.26){(40)};
  \node[anchor=south] (text246) at (16.03, 13.86){20};
  \node[anchor=south] (text247) at (16.03, 13.35){(19)};
  \node[anchor=south,shift={(0.0, -0.11)}] (text248) at (16.03, 12.42){4};
  \node[anchor=south,shift={(0.0, 0.05)}] (text249) at (16.03, 11.92){(4)};
  \node[anchor=south,shift={(0.0, -0.11)}] (text250) at (16.03, 11.55){11};
  \node[anchor=south,shift={(0.0, -0.11)}] (text251) at (16.03, 11.04){(10)};
  \node[anchor=south] (text252) at (16.03, 7.19){65};
  \node[anchor=south] (text253) at (16.03, 6.68){(60)};
  \path[fill=cebebeb,line cap=round,line join=round,line width=0.04cm,miter limit=10.0] (16.53, 15.49) rectangle (18.97, 2.74);
  \path[draw=white,line cap=butt,line join=round,line width=0.02cm,miter limit=10.0] (16.53, 4.77) -- (18.97, 4.77);
  \path[draw=white,line cap=butt,line join=round,line width=0.02cm,miter limit=10.0] (16.53, 7.67) -- (18.97, 7.67);
  \path[draw=white,line cap=butt,line join=round,line width=0.02cm,miter limit=10.0] (16.53, 10.57) -- (18.97, 10.57);
  \path[draw=white,line cap=butt,line join=round,line width=0.02cm,miter limit=10.0] (16.53, 13.46) -- (18.97, 13.46);
  \path[draw=white,line cap=butt,line join=round,line width=0.04cm,miter limit=10.0] (16.53, 3.32) -- (18.97, 3.32);
  \path[draw=white,line cap=butt,line join=round,line width=0.04cm,miter limit=10.0] (16.53, 6.22) -- (18.97, 6.22);
  \path[draw=white,line cap=butt,line join=round,line width=0.04cm,miter limit=10.0] (16.53, 9.12) -- (18.97, 9.12);
  \path[draw=white,line cap=butt,line join=round,line width=0.04cm,miter limit=10.0] (16.53, 12.01) -- (18.97, 12.01);
  \path[draw=white,line cap=butt,line join=round,line width=0.04cm,miter limit=10.0] (16.53, 14.91) -- (18.97, 14.91);
  \path[draw=white,line cap=butt,line join=round,line width=0.04cm,miter limit=10.0] (16.99, 2.74) -- (16.99, 15.49);
  \path[draw=white,line cap=butt,line join=round,line width=0.04cm,miter limit=10.0] (17.75, 2.74) -- (17.75, 15.49);
  \path[draw=white,line cap=butt,line join=round,line width=0.04cm,miter limit=10.0] (18.51, 2.74) -- (18.51, 15.49);
  \path[fill=c77aadd,line cap=butt,line join=miter,line width=0.04cm,miter limit=10.0] (16.65, 14.91) rectangle (17.33, 12.79);
  \path[fill=c99dde1,line cap=butt,line join=miter,line width=0.04cm,miter limit=10.0] (16.65, 12.79) rectangle (17.33, 10.06);
  \path[fill=ceedd88,line cap=butt,line join=miter,line width=0.04cm,miter limit=10.0] (16.65, 10.06) rectangle (17.33, 4.99);
  \path[fill=cee8866,line cap=butt,line join=miter,line width=0.04cm,miter limit=10.0] (16.65, 4.99) rectangle (17.33, 3.32);
  \path[fill=c77aadd,line cap=butt,line join=miter,line width=0.04cm,miter limit=10.0] (17.41, 14.91) rectangle (18.09, 12.22);
  \path[fill=c99dde1,line cap=butt,line join=miter,line width=0.04cm,miter limit=10.0] (17.41, 12.22) rectangle (18.09, 9.77);
  \path[fill=ceedd88,line cap=butt,line join=miter,line width=0.04cm,miter limit=10.0] (17.41, 9.77) rectangle (18.09, 4.96);
  \path[fill=cee8866,line cap=butt,line join=miter,line width=0.04cm,miter limit=10.0] (17.41, 4.96) rectangle (18.09, 3.32);
  \path[fill=c77aadd,line cap=butt,line join=miter,line width=0.04cm,miter limit=10.0] (18.17, 14.91) rectangle (18.85, 12.03);
  \path[fill=c99dde1,line cap=butt,line join=miter,line width=0.04cm,miter limit=10.0] (18.17, 12.03) rectangle (18.85, 9.93);
  \path[fill=ceedd88,line cap=butt,line join=miter,line width=0.04cm,miter limit=10.0] (18.17, 9.93) rectangle (18.85, 5.71);
  \path[fill=cee8866,line cap=butt,line join=miter,line width=0.04cm,miter limit=10.0] (18.17, 5.71) rectangle (18.85, 3.32);
  \node[anchor=south] (text277) at (16.99, 13.98){18};
  \node[anchor=south] (text278) at (16.99, 13.47){(28)};
  \node[anchor=south] (text279) at (16.99, 11.56){24};
  \node[anchor=south] (text280) at (16.99, 11.05){(36)};
  \node[anchor=south] (text281) at (16.99, 7.65){44};
  \node[anchor=south] (text282) at (16.99, 7.15){(67)};
  \node[anchor=south] (text283) at (16.99, 4.28){14};
  \node[anchor=south] (text284) at (16.99, 3.78){(22)};
  \node[anchor=south] (text285) at (17.75, 13.69){23};
  \node[anchor=south] (text286) at (17.75, 13.19){(33)};
  \node[anchor=south] (text287) at (17.75, 11.12){21};
  \node[anchor=south] (text288) at (17.75, 10.62){(30)};
  \node[anchor=south] (text289) at (17.75, 7.49){42};
  \node[anchor=south] (text290) at (17.75, 6.98){(59)};
  \node[anchor=south] (text291) at (17.75, 4.27){14};
  \node[anchor=south] (text292) at (17.75, 3.76){(20)};
  \node[anchor=south] (text293) at (18.51, 13.6){25};
  \node[anchor=south] (text294) at (18.51, 13.09){(41)};
  \node[anchor=south] (text295) at (18.51, 11.11){18};
  \node[anchor=south] (text296) at (18.51, 10.6){(30)};
  \node[anchor=south] (text297) at (18.51, 7.95){36};
  \node[anchor=south] (text298) at (18.51, 7.44){(60)};
  \node[anchor=south] (text299) at (18.51, 4.65){21};
  \node[anchor=south] (text300) at (18.51, 4.14){(34)};
  \path[fill=cebebeb,line cap=round,line join=round,line width=0.04cm,miter limit=10.0] (19.02, 15.49) rectangle (21.45, 2.74);
  \path[draw=white,line cap=butt,line join=round,line width=0.02cm,miter limit=10.0] (19.02, 4.77) -- (21.45, 4.77);
  \path[draw=white,line cap=butt,line join=round,line width=0.02cm,miter limit=10.0] (19.02, 7.67) -- (21.45, 7.67);
  \path[draw=white,line cap=butt,line join=round,line width=0.02cm,miter limit=10.0] (19.02, 10.57) -- (21.45, 10.57);
  \path[draw=white,line cap=butt,line join=round,line width=0.02cm,miter limit=10.0] (19.02, 13.46) -- (21.45, 13.46);
  \path[draw=white,line cap=butt,line join=round,line width=0.04cm,miter limit=10.0] (19.02, 3.32) -- (21.45, 3.32);
  \path[draw=white,line cap=butt,line join=round,line width=0.04cm,miter limit=10.0] (19.02, 6.22) -- (21.45, 6.22);
  \path[draw=white,line cap=butt,line join=round,line width=0.04cm,miter limit=10.0] (19.02, 9.12) -- (21.45, 9.12);
  \path[draw=white,line cap=butt,line join=round,line width=0.04cm,miter limit=10.0] (19.02, 12.01) -- (21.45, 12.01);
  \path[draw=white,line cap=butt,line join=round,line width=0.04cm,miter limit=10.0] (19.02, 14.91) -- (21.45, 14.91);
  \path[draw=white,line cap=butt,line join=round,line width=0.04cm,miter limit=10.0] (19.47, 2.74) -- (19.47, 15.49);
  \path[draw=white,line cap=butt,line join=round,line width=0.04cm,miter limit=10.0] (20.23, 2.74) -- (20.23, 15.49);
  \path[draw=white,line cap=butt,line join=round,line width=0.04cm,miter limit=10.0] (21.0, 2.74) -- (21.0, 15.49);
  \path[fill=c77aadd,line cap=butt,line join=miter,line width=0.04cm,miter limit=10.0] (19.13, 14.91) rectangle (19.82, 12.92);
  \path[fill=c99dde1,line cap=butt,line join=miter,line width=0.04cm,miter limit=10.0] ;
  \path[fill=ceedd88,line cap=butt,line join=miter,line width=0.04cm,miter limit=10.0] (19.13, 12.93) rectangle (19.82, 10.61);
  \path[fill=cee8866,line cap=butt,line join=miter,line width=0.04cm,miter limit=10.0] (19.13, 10.61) rectangle (19.82, 3.32);
  \path[fill=c77aadd,line cap=butt,line join=miter,line width=0.04cm,miter limit=10.0] (19.89, 14.91) rectangle (20.58, 13.89);
  \path[fill=c99dde1,line cap=butt,line join=miter,line width=0.04cm,miter limit=10.0] ;
  \path[fill=ceedd88,line cap=butt,line join=miter,line width=0.04cm,miter limit=10.0] (19.89, 13.89) rectangle (20.58, 11.16);
  \path[fill=cee8866,line cap=butt,line join=miter,line width=0.04cm,miter limit=10.0] (19.89, 11.16) rectangle (20.58, 3.32);
  \path[fill=c77aadd,line cap=butt,line join=miter,line width=0.04cm,miter limit=10.0] (20.65, 14.91) rectangle (21.34, 13.44);
  \path[fill=c99dde1,line cap=butt,line join=miter,line width=0.04cm,miter limit=10.0] ;
  \path[fill=ceedd88,line cap=butt,line join=miter,line width=0.04cm,miter limit=10.0] (20.65, 13.44) rectangle (21.34, 11.54);
  \path[fill=cee8866,line cap=butt,line join=miter,line width=0.04cm,miter limit=10.0] (20.65, 11.54) rectangle (21.34, 3.32);
  \node[anchor=south] (text324) at (19.47, 14.05){17};
  \node[anchor=south] (text325) at (19.47, 13.54){(6)};
  \node[anchor=south] (text326) at (19.47, 11.89){20};
  \node[anchor=south] (text327) at (19.47, 11.39){(7)};
  \node[anchor=south] (text328) at (19.47, 7.09){63};
  \node[anchor=south] (text329) at (19.47, 6.59){(22)};
  \node[anchor=south] (text330) at (20.23, 14.53){9};
  \node[anchor=south] (text331) at (20.23, 14.02){(3)};
  \node[anchor=south] (text332) at (20.23, 12.65){24};
  \node[anchor=south] (text333) at (20.23, 12.15){(8)};
  \node[anchor=south] (text334) at (20.23, 7.37){68};
  \node[anchor=south] (text335) at (20.23, 6.87){(23)};
  \node[anchor=south] (text336) at (21.0, 14.3){13};
  \node[anchor=south] (text337) at (21.0, 13.8){(7)};
  \node[anchor=south] (text338) at (21.0, 12.62){16};
  \node[anchor=south] (text339) at (21.0, 12.11){(9)};
  \node[anchor=south] (text340) at (21.0, 7.56){71};
  \node[anchor=south] (text341) at (21.0, 7.05){(39)};
  \path[fill=cebebeb,line cap=round,line join=round,line width=0.04cm,miter limit=10.0] (21.5, 15.49) rectangle (23.94, 2.74);
  \path[draw=white,line cap=butt,line join=round,line width=0.02cm,miter limit=10.0] (21.5, 4.77) -- (23.94, 4.77);
  \path[draw=white,line cap=butt,line join=round,line width=0.02cm,miter limit=10.0] (21.5, 7.67) -- (23.94, 7.67);
  \path[draw=white,line cap=butt,line join=round,line width=0.02cm,miter limit=10.0] (21.5, 10.57) -- (23.94, 10.57);
  \path[draw=white,line cap=butt,line join=round,line width=0.02cm,miter limit=10.0] (21.5, 13.46) -- (23.94, 13.46);
  \path[draw=white,line cap=butt,line join=round,line width=0.04cm,miter limit=10.0] (21.5, 3.32) -- (23.94, 3.32);
  \path[draw=white,line cap=butt,line join=round,line width=0.04cm,miter limit=10.0] (21.5, 6.22) -- (23.94, 6.22);
  \path[draw=white,line cap=butt,line join=round,line width=0.04cm,miter limit=10.0] (21.5, 9.12) -- (23.94, 9.12);
  \path[draw=white,line cap=butt,line join=round,line width=0.04cm,miter limit=10.0] (21.5, 12.01) -- (23.94, 12.01);
  \path[draw=white,line cap=butt,line join=round,line width=0.04cm,miter limit=10.0] (21.5, 14.91) -- (23.94, 14.91);
  \path[draw=white,line cap=butt,line join=round,line width=0.04cm,miter limit=10.0] (21.96, 2.74) -- (21.96, 15.49);
  \path[draw=white,line cap=butt,line join=round,line width=0.04cm,miter limit=10.0] (22.72, 2.74) -- (22.72, 15.49);
  \path[draw=white,line cap=butt,line join=round,line width=0.04cm,miter limit=10.0] (23.48, 2.74) -- (23.48, 15.49);
  \path[fill=c77aadd,line cap=butt,line join=miter,line width=0.04cm,miter limit=10.0] (21.62, 14.91) rectangle (22.3, 14.69);
  \path[fill=c99dde1,line cap=butt,line join=miter,line width=0.04cm,miter limit=10.0] ;
  \path[fill=ceedd88,line cap=butt,line join=miter,line width=0.04cm,miter limit=10.0] (21.62, 14.69) rectangle (22.3, 12.73);
  \path[fill=cee8866,line cap=butt,line join=miter,line width=0.04cm,miter limit=10.0] (21.62, 12.73) rectangle (22.3, 3.32);
  \path[fill=c77aadd,line cap=butt,line join=miter,line width=0.04cm,miter limit=10.0] (22.38, 14.91) rectangle (23.06, 13.86);
  \path[fill=c99dde1,line cap=butt,line join=miter,line width=0.04cm,miter limit=10.0] (22.38, 13.86) rectangle (23.06, 13.65);
  \path[fill=ceedd88,line cap=butt,line join=miter,line width=0.04cm,miter limit=10.0] (22.38, 13.65) rectangle (23.06, 12.38);
  \path[fill=cee8866,line cap=butt,line join=miter,line width=0.04cm,miter limit=10.0] (22.38, 12.38) rectangle (23.06, 3.32);
  \path[fill=c77aadd,line cap=butt,line join=miter,line width=0.04cm,miter limit=10.0] (23.14, 14.91) rectangle (23.82, 14.55);
  \path[fill=c99dde1,line cap=butt,line join=miter,line width=0.04cm,miter limit=10.0] ;
  \path[fill=ceedd88,line cap=butt,line join=miter,line width=0.04cm,miter limit=10.0] (23.14, 14.55) rectangle (23.82, 12.95);
  \path[fill=cee8866,line cap=butt,line join=miter,line width=0.04cm,miter limit=10.0] (23.14, 12.95) rectangle (23.82, 3.32);
  \node[anchor=south] (text365) at (21.96, 14.93){2};
  \node[anchor=south] (text366) at (21.96, 14.42){(1)};
  \node[anchor=south] (text367) at (21.96, 13.84){17};
  \node[anchor=south] (text368) at (21.96, 13.33){(9)};
  \node[anchor=south] (text369) at (21.96, 8.15){81};
  \node[anchor=south] (text370) at (21.96, 7.65){(43)};
  \node[anchor=south,shift={(0.0, 0.42)}] (text371) at (22.72, 14.51){9};
  \node[anchor=south,shift={(0.0, 0.42)}] (text372) at (22.72, 14.01){(5)};
  \node[anchor=south,shift={(0.0, -0.05)}] (text373) at (22.72, 13.88){2};
  \node[anchor=south,shift={(0.0, 0.11)}] (text374) at (22.72, 13.37){(1)};
  \node[anchor=south,shift={(0.0, -0.32)}] (text375) at (22.72, 13.14){11};
  \node[anchor=south,shift={(0.0, -0.26)}] (text376) at (22.72, 12.64){(6)};
  \node[anchor=south] (text377) at (22.72, 7.98){78};
  \node[anchor=south] (text378) at (22.72, 7.48){(43)};
  \node[anchor=south,shift={(0.0, 0.07)}] (text379) at (23.48, 14.86){3};
  \node[anchor=south,shift={(0.0, 0.07)}] (text380) at (23.48, 14.35){(2)};
  \node[anchor=south] (text381) at (23.48, 13.88){14};
  \node[anchor=south] (text382) at (23.48, 13.37){(9)};
  \node[anchor=south] (text383) at (23.48, 8.26){83};
  \node[anchor=south] (text384) at (23.48, 7.76){(54)};
  \node[text=c1a1a1a,anchor=south] (text385) at (2.84, 15.7){\gls{fakultät2}};
  \node[text=c1a1a1a,anchor=south] (text387) at (5.33, 15.7){\gls{fakultät3}};
  \node[text=c1a1a1a,anchor=south] (text389) at (7.81, 15.7){\gls{fakultät4}};
  \node[text=c1a1a1a,anchor=south] (text391) at (10.3, 15.7){\gls{fakultät5}};
  \node[text=c1a1a1a,anchor=south] (text393) at (12.78, 15.7){\gls{fakultät6}};
  \node[text=c1a1a1a,anchor=south] (text395) at (15.27, 15.7){\gls{fakultät7}};
  \node[text=c1a1a1a,anchor=south] (text397) at (17.75, 15.7){\gls{fakultät8}};
  \node[text=c1a1a1a,anchor=south] (text399) at (20.23, 15.7){\gls{fakultät9}};
  \node[text=c1a1a1a,anchor=south] (text401) at (22.72, 15.7){\gls{fakultät10}};
  \path[draw=c333333,line cap=butt,line join=round,line width=0.04cm,miter limit=10.0] (2.08, 2.65) -- (2.08, 2.74);
  \path[draw=c333333,line cap=butt,line join=round,line width=0.04cm,miter limit=10.0] (2.84, 2.65) -- (2.84, 2.74);
  \path[draw=c333333,line cap=butt,line join=round,line width=0.04cm,miter limit=10.0] (3.6, 2.65) -- (3.6, 2.74);
  \node[text=c4d4d4d,anchor=south east,cm={ 0.71,0.71,-0.71,0.71,(2.29, -15.42)}] (text404) at (0.0, 17.78){2012--2015};
  \node[text=c4d4d4d,anchor=south east,cm={ 0.71,0.71,-0.71,0.71,(3.06, -15.42)}] (text405) at (0.0, 17.78){2016--2019};
  \node[text=c4d4d4d,anchor=south east,cm={ 0.71,0.71,-0.71,0.71,(3.82, -15.42)}] (text406) at (0.0, 17.78){2020--2023};
  \path[draw=c333333,line cap=butt,line join=round,line width=0.04cm,miter limit=10.0] (4.57, 2.65) -- (4.57, 2.74);
  \path[draw=c333333,line cap=butt,line join=round,line width=0.04cm,miter limit=10.0] (5.33, 2.65) -- (5.33, 2.74);
  \path[draw=c333333,line cap=butt,line join=round,line width=0.04cm,miter limit=10.0] (6.09, 2.65) -- (6.09, 2.74);
  \node[text=c4d4d4d,anchor=south east,cm={ 0.71,0.71,-0.71,0.71,(4.78, -15.42)}] (text408) at (0.0, 17.78){2012--2015};
  \node[text=c4d4d4d,anchor=south east,cm={ 0.71,0.71,-0.71,0.71,(5.54, -15.42)}] (text409) at (0.0, 17.78){2016--2019};
  \node[text=c4d4d4d,anchor=south east,cm={ 0.71,0.71,-0.71,0.71,(6.3, -15.42)}] (text410) at (0.0, 17.78){2020--2023};
  \path[draw=c333333,line cap=butt,line join=round,line width=0.04cm,miter limit=10.0] (7.05, 2.65) -- (7.05, 2.74);
  \path[draw=c333333,line cap=butt,line join=round,line width=0.04cm,miter limit=10.0] (7.81, 2.65) -- (7.81, 2.74);
  \path[draw=c333333,line cap=butt,line join=round,line width=0.04cm,miter limit=10.0] (8.57, 2.65) -- (8.57, 2.74);
  \node[text=c4d4d4d,anchor=south east,cm={ 0.71,0.71,-0.71,0.71,(7.26, -15.42)}] (text412) at (0.0, 17.78){2012--2015};
  \node[text=c4d4d4d,anchor=south east,cm={ 0.71,0.71,-0.71,0.71,(8.02, -15.42)}] (text413) at (0.0, 17.78){2016--2019};
  \node[text=c4d4d4d,anchor=south east,cm={ 0.71,0.71,-0.71,0.71,(8.79, -15.42)}] (text414) at (0.0, 17.78){2020--2023};
  \path[draw=c333333,line cap=butt,line join=round,line width=0.04cm,miter limit=10.0] (9.54, 2.65) -- (9.54, 2.74);
  \path[draw=c333333,line cap=butt,line join=round,line width=0.04cm,miter limit=10.0] (10.3, 2.65) -- (10.3, 2.74);
  \path[draw=c333333,line cap=butt,line join=round,line width=0.04cm,miter limit=10.0] (11.06, 2.65) -- (11.06, 2.74);
  \node[text=c4d4d4d,anchor=south east,cm={ 0.71,0.71,-0.71,0.71,(9.75, -15.42)}] (text416) at (0.0, 17.78){2012--2015};
  \node[text=c4d4d4d,anchor=south east,cm={ 0.71,0.71,-0.71,0.71,(10.51, -15.42)}] (text417) at (0.0, 17.78){2016--2019};
  \node[text=c4d4d4d,anchor=south east,cm={ 0.71,0.71,-0.71,0.71,(11.27, -15.42)}] (text418) at (0.0, 17.78){2020--2023};
  \path[draw=c333333,line cap=butt,line join=round,line width=0.04cm,miter limit=10.0] (12.02, 2.65) -- (12.02, 2.74);
  \path[draw=c333333,line cap=butt,line join=round,line width=0.04cm,miter limit=10.0] (12.78, 2.65) -- (12.78, 2.74);
  \path[draw=c333333,line cap=butt,line join=round,line width=0.04cm,miter limit=10.0] (13.54, 2.65) -- (13.54, 2.74);
  \node[text=c4d4d4d,anchor=south east,cm={ 0.71,0.71,-0.71,0.71,(12.23, -15.42)}] (text420) at (0.0, 17.78){2012--2015};
  \node[text=c4d4d4d,anchor=south east,cm={ 0.71,0.71,-0.71,0.71,(12.99, -15.42)}] (text421) at (0.0, 17.78){2016--2019};
  \node[text=c4d4d4d,anchor=south east,cm={ 0.71,0.71,-0.71,0.71,(13.76, -15.42)}] (text422) at (0.0, 17.78){2020--2023};
  \path[draw=c333333,line cap=butt,line join=round,line width=0.04cm,miter limit=10.0] (14.5, 2.65) -- (14.5, 2.74);
  \path[draw=c333333,line cap=butt,line join=round,line width=0.04cm,miter limit=10.0] (15.27, 2.65) -- (15.27, 2.74);
  \path[draw=c333333,line cap=butt,line join=round,line width=0.04cm,miter limit=10.0] (16.03, 2.65) -- (16.03, 2.74);
  \node[text=c4d4d4d,anchor=south east,cm={ 0.71,0.71,-0.71,0.71,(14.72, -15.42)}] (text424) at (0.0, 17.78){2012--2015};
  \node[text=c4d4d4d,anchor=south east,cm={ 0.71,0.71,-0.71,0.71,(15.48, -15.42)}] (text425) at (0.0, 17.78){2016--2019};
  \node[text=c4d4d4d,anchor=south east,cm={ 0.71,0.71,-0.71,0.71,(16.24, -15.42)}] (text426) at (0.0, 17.78){2020--2023};
  \path[draw=c333333,line cap=butt,line join=round,line width=0.04cm,miter limit=10.0] (16.99, 2.65) -- (16.99, 2.74);
  \path[draw=c333333,line cap=butt,line join=round,line width=0.04cm,miter limit=10.0] (17.75, 2.65) -- (17.75, 2.74);
  \path[draw=c333333,line cap=butt,line join=round,line width=0.04cm,miter limit=10.0] (18.51, 2.65) -- (18.51, 2.74);
  \node[text=c4d4d4d,anchor=south east,cm={ 0.71,0.71,-0.71,0.71,(17.2, -15.42)}] (text428) at (0.0, 17.78){2012--2015};
  \node[text=c4d4d4d,anchor=south east,cm={ 0.71,0.71,-0.71,0.71,(17.96, -15.42)}] (text429) at (0.0, 17.78){2016--2019};
  \node[text=c4d4d4d,anchor=south east,cm={ 0.71,0.71,-0.71,0.71,(18.72, -15.42)}] (text430) at (0.0, 17.78){2020--2023};
  \path[draw=c333333,line cap=butt,line join=round,line width=0.04cm,miter limit=10.0] (19.47, 2.65) -- (19.47, 2.74);
  \path[draw=c333333,line cap=butt,line join=round,line width=0.04cm,miter limit=10.0] (20.23, 2.65) -- (20.23, 2.74);
  \path[draw=c333333,line cap=butt,line join=round,line width=0.04cm,miter limit=10.0] (21.0, 2.65) -- (21.0, 2.74);
  \node[text=c4d4d4d,anchor=south east,cm={ 0.71,0.71,-0.71,0.71,(19.69, -15.42)}] (text432) at (0.0, 17.78){2012--2015};
  \node[text=c4d4d4d,anchor=south east,cm={ 0.71,0.71,-0.71,0.71,(20.45, -15.42)}] (text433) at (0.0, 17.78){2016--2019};
  \node[text=c4d4d4d,anchor=south east,cm={ 0.71,0.71,-0.71,0.71,(21.21, -15.42)}] (text434) at (0.0, 17.78){2020--2023};
  \path[draw=c333333,line cap=butt,line join=round,line width=0.04cm,miter limit=10.0] (21.96, 2.65) -- (21.96, 2.74);
  \path[draw=c333333,line cap=butt,line join=round,line width=0.04cm,miter limit=10.0] (22.72, 2.65) -- (22.72, 2.74);
  \path[draw=c333333,line cap=butt,line join=round,line width=0.04cm,miter limit=10.0] (23.48, 2.65) -- (23.48, 2.74);
  \node[text=c4d4d4d,anchor=south east,cm={ 0.71,0.71,-0.71,0.71,(22.17, -15.42)}] (text436) at (0.0, 17.78){2012--2015};
  \node[text=c4d4d4d,anchor=south east,cm={ 0.71,0.71,-0.71,0.71,(22.93, -15.42)}] (text437) at (0.0, 17.78){2016--2019};
  \node[text=c4d4d4d,anchor=south east,cm={ 0.71,0.71,-0.71,0.71,(23.69, -15.42)}] (text438) at (0.0, 17.78){2020--2023};
  \node[text=c4d4d4d,anchor=south east] (text439) at (1.45, 3.21){0\%};
  \node[text=c4d4d4d,anchor=south east] (text440) at (1.45, 6.11){25\%};
  \node[text=c4d4d4d,anchor=south east] (text441) at (1.45, 9.01){50\%};
  \node[text=c4d4d4d,anchor=south east] (text442) at (1.45, 11.9){75\%};
  \node[text=c4d4d4d,anchor=south east] (text443) at (1.45, 14.8){100\%};
  \path[draw=c333333,line cap=butt,line join=round,line width=0.04cm,miter limit=10.0] (1.53, 3.32) -- (1.62, 3.32);
  \path[draw=c333333,line cap=butt,line join=round,line width=0.04cm,miter limit=10.0] (1.53, 6.22) -- (1.62, 6.22);
  \path[draw=c333333,line cap=butt,line join=round,line width=0.04cm,miter limit=10.0] (1.53, 9.12) -- (1.62, 9.12);
  \path[draw=c333333,line cap=butt,line join=round,line width=0.04cm,miter limit=10.0] (1.53, 12.01) -- (1.62, 12.01);
  \path[draw=c333333,line cap=butt,line join=round,line width=0.04cm,miter limit=10.0] (1.53, 14.91) -- (1.62, 14.91);
  \node[anchor=south,cm={ 0.0,1.0,-1.0,0.0,(0.47, -8.66)}] (text448) at (0.0, 17.78){Anteil in Prozent (\%)};
  \path[fill=white,line cap=round,line join=round,line width=0.04cm,miter limit=10.0] (8.52, 17.59) rectangle (17.04, 16.59);
  \path[fill=cebebeb,line cap=round,line join=round,line width=0.04cm,miter limit=10.0] (8.72, 17.39) rectangle (9.33, 16.78);
  \path[fill=c77aadd,line cap=butt,line join=miter,line width=0.04cm,miter limit=10.0] (8.74, 17.37) rectangle (9.3, 16.81);
  \path[fill=cebebeb,line cap=round,line join=round,line width=0.04cm,miter limit=10.0] (10.86, 17.39) rectangle (11.47, 16.78);
  \path[fill=c99dde1,line cap=butt,line join=miter,line width=0.04cm,miter limit=10.0] (10.88, 17.37) rectangle (11.44, 16.81);
  \path[fill=cebebeb,line cap=round,line join=round,line width=0.04cm,miter limit=10.0] (13.0, 17.39) rectangle (13.61, 16.78);
  \path[fill=ceedd88,line cap=butt,line join=miter,line width=0.04cm,miter limit=10.0] (13.03, 17.37) rectangle (13.59, 16.81);
  \path[fill=cebebeb,line cap=round,line join=round,line width=0.04cm,miter limit=10.0] (15.14, 17.39) rectangle (15.75, 16.78);
  \path[fill=cee8866,line cap=butt,line join=miter,line width=0.04cm,miter limit=10.0] (15.17, 17.37) rectangle (15.73, 16.81);
  \node[anchor=south west] (text456) at (9.52, 16.96){Stufe 1};
  \node[anchor=south west] (text457) at (11.66, 16.96){Stufe 2};
  \node[anchor=south west] (text458) at (13.8, 16.96){Stufe 3};
  \node[anchor=south west] (text459) at (15.95, 16.96){Keine};
\end{tikzpicture}}
    \caption{\gls{forschungsdaten}-Klassifikation der Dissertationen aus der Stichprobe nach Fakultät, Zeitgruppe und Klassifikationsstufe.
    Die Höhe der Barren entsprechen dem relativen Anteil zur jeweiligen $\text{\textit{Fakultät}}\times\text{\textit{Zeitgruppe}}$-Gesamtanzahl.
    Absolute Werte in Klammern angegeben.}
    \label{fig:luh-repo-faculty-yeargroup-classification}
    \end{figure}

    \begin{figure}[!htbp]
        \centering%
        \resizebox{.33\textwidth}{!}{
\begin{tikzpicture}[y=1mm, x=1mm, yscale=\globalscale,xscale=\globalscale, every node/.append style={scale=\globalscale}, inner sep=0pt, outer sep=0pt]
  \path[fill=white,line cap=round,line join=round,miter limit=10.0] ;



  \path[draw=white,fill=white,line cap=round,line join=round,line 
  width=0.38mm,miter limit=10.0] (0.0, 80.0) rectangle (80.0, 0.0);



  \path[fill=cebebeb,line cap=round,line join=round,line width=0.38mm,miter 
  limit=10.0] (12.07, 72.09) rectangle (78.07, 16.31);



  \path[draw=white,line cap=butt,line join=round,line width=0.19mm,miter 
  limit=10.0] (12.07, 27.72) -- (78.07, 27.72);



  \path[draw=white,line cap=butt,line join=round,line width=0.19mm,miter 
  limit=10.0] (12.07, 40.4) -- (78.07, 40.4);



  \path[draw=white,line cap=butt,line join=round,line width=0.19mm,miter 
  limit=10.0] (12.07, 53.08) -- (78.07, 53.08);



  \path[draw=white,line cap=butt,line join=round,line width=0.19mm,miter 
  limit=10.0] (12.07, 65.75) -- (78.07, 65.75);



  \path[draw=white,line cap=butt,line join=round,line width=0.38mm,miter 
  limit=10.0] (12.07, 21.38) -- (78.07, 21.38);



  \path[draw=white,line cap=butt,line join=round,line width=0.38mm,miter 
  limit=10.0] (12.07, 34.06) -- (78.07, 34.06);



  \path[draw=white,line cap=butt,line join=round,line width=0.38mm,miter 
  limit=10.0] (12.07, 46.74) -- (78.07, 46.74);



  \path[draw=white,line cap=butt,line join=round,line width=0.38mm,miter 
  limit=10.0] (12.07, 59.41) -- (78.07, 59.41);



  \path[draw=white,line cap=butt,line join=round,line width=0.38mm,miter 
  limit=10.0] (12.07, 72.09) -- (78.07, 72.09);



  \path[draw=white,line cap=butt,line join=round,line width=0.38mm,miter 
  limit=10.0] (24.44, 16.31) -- (24.44, 72.09);



  \path[draw=white,line cap=butt,line join=round,line width=0.38mm,miter 
  limit=10.0] (45.07, 16.31) -- (45.07, 72.09);



  \path[draw=white,line cap=butt,line join=round,line width=0.38mm,miter 
  limit=10.0] (65.69, 16.31) -- (65.69, 72.09);



  \path[draw=c333333,line cap=butt,line join=round,line width=0.38mm,miter 
  limit=10.0] (24.44, 59.41) -- (24.44, 69.56);



  \path[draw=c333333,line cap=butt,line join=round,line width=0.38mm,miter 
  limit=10.0] (24.44, 39.13) -- (24.44, 28.99);



  \path[draw=c333333,fill=white,line cap=butt,line join=miter,line 
  width=0.38mm,miter limit=10.0] (16.71, 59.41) -- (16.71, 39.13) -- (32.18, 
  39.13) -- (32.18, 59.41) -- (16.71, 59.41) -- (16.71, 59.41)-- cycle;



  \path[draw=c333333,line cap=butt,line join=miter,line width=0.75mm,miter 
  limit=10.0] (16.71, 49.27) -- (32.18, 49.27);



  \path[draw=c333333,line cap=butt,line join=round,line width=0.38mm,miter 
  limit=10.0] (45.07, 23.91) -- (45.07, 28.99);



  \path[draw=c333333,line cap=butt,line join=round,line width=0.38mm,miter 
  limit=10.0] ;



  \path[draw=c333333,fill=white,line cap=butt,line join=miter,line 
  width=0.38mm,miter limit=10.0] (37.33, 23.91) -- (37.33, 18.85) -- (52.8, 
  18.85) -- (52.8, 23.91) -- (37.33, 23.91) -- (37.33, 23.91)-- cycle;



  \path[draw=c333333,line cap=butt,line join=miter,line width=0.75mm,miter 
  limit=10.0] (37.33, 18.85) -- (52.8, 18.85);



  \path[draw=c333333,line cap=butt,line join=round,line width=0.38mm,miter 
  limit=10.0] (65.69, 61.95) -- (65.69, 64.49);



  \path[draw=c333333,line cap=butt,line join=round,line width=0.38mm,miter 
  limit=10.0] (65.69, 51.81) -- (65.69, 44.2);



  \path[draw=c333333,fill=white,line cap=butt,line join=miter,line 
  width=0.38mm,miter limit=10.0] (57.96, 61.95) -- (57.96, 51.81) -- (73.43, 
  51.81) -- (73.43, 61.95) -- (57.96, 61.95) -- (57.96, 61.95)-- cycle;



  \path[draw=c333333,line cap=butt,line join=miter,line width=0.75mm,miter 
  limit=10.0] (57.96, 59.41) -- (73.43, 59.41);



  \node[text=c4d4d4d,anchor=south east] (text24) at (10.33, 19.87){2.5};



  \node[text=c4d4d4d,anchor=south east] (text25) at (10.33, 32.55){5.0};



  \node[text=c4d4d4d,anchor=south east] (text26) at (10.33, 45.23){7.5};



  \node[text=c4d4d4d,anchor=south east] (text27) at (10.33, 57.9){10.0};



  \node[text=c4d4d4d,anchor=south east] (text28) at (10.33, 70.58){12.5};



  \path[draw=c333333,line cap=butt,line join=round,line width=0.38mm,miter 
  limit=10.0] (11.1, 21.38) -- (12.07, 21.38);



  \path[draw=c333333,line cap=butt,line join=round,line width=0.38mm,miter 
  limit=10.0] (11.1, 34.06) -- (12.07, 34.06);



  \path[draw=c333333,line cap=butt,line join=round,line width=0.38mm,miter 
  limit=10.0] (11.1, 46.74) -- (12.07, 46.74);



  \path[draw=c333333,line cap=butt,line join=round,line width=0.38mm,miter 
  limit=10.0] (11.1, 59.41) -- (12.07, 59.41);



  \path[draw=c333333,line cap=butt,line join=round,line width=0.38mm,miter 
  limit=10.0] (11.1, 72.09) -- (12.07, 72.09);



  \path[draw=c333333,line cap=butt,line join=round,line width=0.38mm,miter 
  limit=10.0] (24.44, 15.34) -- (24.44, 16.31);



  \path[draw=c333333,line cap=butt,line join=round,line width=0.38mm,miter 
  limit=10.0] (45.07, 15.34) -- (45.07, 16.31);



  \path[draw=c333333,line cap=butt,line join=round,line width=0.38mm,miter 
  limit=10.0] (65.69, 15.34) -- (65.69, 16.31);



  \node[text=c4d4d4d,anchor=south east,cm={ 0.71,0.71,-0.71,0.71,(26.58, 
  -67.57)}] (text35) at (0.0, 80.0){Keine};



  \node[text=c4d4d4d,anchor=south east,cm={ 0.71,0.71,-0.71,0.71,(47.21, 
  -67.57)}] (text36) at (0.0, 80.0){Stufe 1};



  \node[text=c4d4d4d,anchor=south east,cm={ 0.71,0.71,-0.71,0.71,(67.83, 
  -67.57)}] (text37) at (0.0, 80.0){Stufe 3};



  \node[anchor=south west] (text38) at (12.07, 75.04){Zeitgruppe $\times$ \gls{forschungsdaten}};

  \node[anchor=south west] (text38) at (0, 76){\Large\textbf{(A)}};


\end{tikzpicture}
}%
        \resizebox{.33\textwidth}{!}{
\begin{tikzpicture}[y=1mm, x=1mm, yscale=\globalscale,xscale=\globalscale, every node/.append style={scale=\globalscale}, inner sep=0pt, outer sep=0pt]
  \path[fill=white,line cap=round,line join=round,miter limit=10.0] ;



  \path[draw=white,fill=white,line cap=round,line join=round,line 
  width=0.38mm,miter limit=10.0] (0.0, 80.0) rectangle (80.0, 0.0);



  \path[fill=cebebeb,line cap=round,line join=round,line width=0.38mm,miter 
  limit=10.0] (9.65, 72.09) rectangle (78.07, 16.31);



  \path[draw=white,line cap=butt,line join=round,line width=0.19mm,miter 
  limit=10.0] (9.65, 16.31) -- (78.07, 16.31);



  \path[draw=white,line cap=butt,line join=round,line width=0.19mm,miter 
  limit=10.0] (9.65, 31.52) -- (78.07, 31.52);



  \path[draw=white,line cap=butt,line join=round,line width=0.19mm,miter 
  limit=10.0] (9.65, 46.74) -- (78.07, 46.74);



  \path[draw=white,line cap=butt,line join=round,line width=0.19mm,miter 
  limit=10.0] (9.65, 61.95) -- (78.07, 61.95);



  \path[draw=white,line cap=butt,line join=round,line width=0.38mm,miter 
  limit=10.0] (9.65, 23.91) -- (78.07, 23.91);



  \path[draw=white,line cap=butt,line join=round,line width=0.38mm,miter 
  limit=10.0] (9.65, 39.13) -- (78.07, 39.13);



  \path[draw=white,line cap=butt,line join=round,line width=0.38mm,miter 
  limit=10.0] (9.65, 54.35) -- (78.07, 54.35);



  \path[draw=white,line cap=butt,line join=round,line width=0.38mm,miter 
  limit=10.0] (9.65, 69.56) -- (78.07, 69.56);



  \path[draw=white,line cap=butt,line join=round,line width=0.38mm,miter 
  limit=10.0] (22.48, 16.31) -- (22.48, 72.09);



  \path[draw=white,line cap=butt,line join=round,line width=0.38mm,miter 
  limit=10.0] (43.86, 16.31) -- (43.86, 72.09);



  \path[draw=white,line cap=butt,line join=round,line width=0.38mm,miter 
  limit=10.0] (65.24, 16.31) -- (65.24, 72.09);



  \path[draw=c333333,line cap=butt,line join=round,line width=0.38mm,miter 
  limit=10.0] (22.48, 56.88) -- (22.48, 69.56);



  \path[draw=c333333,line cap=butt,line join=round,line width=0.38mm,miter 
  limit=10.0] (22.48, 31.52) -- (22.48, 18.85);



  \path[draw=c333333,fill=white,line cap=butt,line join=miter,line 
  width=0.38mm,miter limit=10.0] (14.46, 56.88) -- (14.46, 31.52) -- (30.49, 
  31.52) -- (30.49, 56.88) -- (14.46, 56.88) -- (14.46, 56.88)-- cycle;



  \path[draw=c333333,line cap=butt,line join=miter,line width=0.75mm,miter 
  limit=10.0] (14.46, 44.2) -- (30.49, 44.2);



  \path[draw=c333333,line cap=butt,line join=round,line width=0.38mm,miter 
  limit=10.0] (43.86, 50.54) -- (43.86, 69.56);



  \path[draw=c333333,line cap=butt,line join=round,line width=0.38mm,miter 
  limit=10.0] ;



  \path[draw=c333333,fill=white,line cap=butt,line join=miter,line 
  width=0.38mm,miter limit=10.0] (35.84, 50.54) -- (35.84, 31.52) -- (51.88, 
  31.52) -- (51.88, 50.54) -- (35.84, 50.54) -- (35.84, 50.54)-- cycle;



  \path[draw=c333333,line cap=butt,line join=miter,line width=0.75mm,miter 
  limit=10.0] (35.84, 31.52) -- (51.88, 31.52);



  \path[draw=c333333,line cap=butt,line join=round,line width=0.38mm,miter 
  limit=10.0] (65.24, 50.54) -- (65.24, 53.26);



  \path[draw=c333333,line cap=butt,line join=round,line width=0.38mm,miter 
  limit=10.0] (65.24, 39.67) -- (65.24, 31.52);



  \path[draw=c333333,fill=white,line cap=butt,line join=miter,line 
  width=0.38mm,miter limit=10.0] (57.22, 50.54) -- (57.22, 39.67) -- (73.26, 
  39.67) -- (73.26, 50.54) -- (57.22, 50.54) -- (57.22, 50.54)-- cycle;



  \path[draw=c333333,line cap=butt,line join=miter,line width=0.75mm,miter 
  limit=10.0] (57.22, 47.83) -- (73.26, 47.83);



  \node[text=c4d4d4d,anchor=south east] (text23) at (7.91, 22.4){0.2};



  \node[text=c4d4d4d,anchor=south east] (text24) at (7.91, 37.62){0.3};



  \node[text=c4d4d4d,anchor=south east] (text25) at (7.91, 52.83){0.4};



  \node[text=c4d4d4d,anchor=south east] (text26) at (7.91, 68.05){0.5};



  \path[draw=c333333,line cap=butt,line join=round,line width=0.38mm,miter 
  limit=10.0] (8.68, 23.91) -- (9.65, 23.91);



  \path[draw=c333333,line cap=butt,line join=round,line width=0.38mm,miter 
  limit=10.0] (8.68, 39.13) -- (9.65, 39.13);



  \path[draw=c333333,line cap=butt,line join=round,line width=0.38mm,miter 
  limit=10.0] (8.68, 54.35) -- (9.65, 54.35);



  \path[draw=c333333,line cap=butt,line join=round,line width=0.38mm,miter 
  limit=10.0] (8.68, 69.56) -- (9.65, 69.56);



  \path[draw=c333333,line cap=butt,line join=round,line width=0.38mm,miter 
  limit=10.0] (22.48, 15.34) -- (22.48, 16.31);



  \path[draw=c333333,line cap=butt,line join=round,line width=0.38mm,miter 
  limit=10.0] (43.86, 15.34) -- (43.86, 16.31);



  \path[draw=c333333,line cap=butt,line join=round,line width=0.38mm,miter 
  limit=10.0] (65.24, 15.34) -- (65.24, 16.31);



  \node[text=c4d4d4d,anchor=south east,cm={ 0.71,0.71,-0.71,0.71,(24.61, 
  -67.57)}] (text32) at (0.0, 80.0){Keine};



  \node[text=c4d4d4d,anchor=south east,cm={ 0.71,0.71,-0.71,0.71,(46.0, 
  -67.57)}] (text33) at (0.0, 80.0){Stufe 1};



  \node[text=c4d4d4d,anchor=south east,cm={ 0.71,0.71,-0.71,0.71,(67.38, 
  -67.57)}] (text34) at (0.0, 80.0){Stufe 3};



  \node[anchor=south west] (text35) at (9.65, 75.04){Zeitgruppe $\times$ \gls{forschungsdaten} (\% zu \gls{forschungsdaten})};

  \node[anchor=south west] (text38) at (0, 76){\Large\textbf{(B)}};



\end{tikzpicture}
}%
        \resizebox{.33\textwidth}{!}{
\begin{tikzpicture}[y=1mm, x=1mm, yscale=\globalscale,xscale=\globalscale, every node/.append style={scale=\globalscale}, inner sep=0pt, outer sep=0pt]
  \path[fill=white,line cap=round,line join=round,miter limit=10.0] ;



  \path[draw=white,fill=white,line cap=round,line join=round,line 
  width=0.38mm,miter limit=10.0] (0.0, 80.0) rectangle (80.0, 0.0);



  \path[fill=cebebeb,line cap=round,line join=round,line width=0.38mm,miter 
  limit=10.0] (9.65, 72.09) rectangle (78.07, 16.31);



  \path[draw=white,line cap=butt,line join=round,line width=0.19mm,miter 
  limit=10.0] (9.65, 26.59) -- (78.07, 26.59);



  \path[draw=white,line cap=butt,line join=round,line width=0.19mm,miter 
  limit=10.0] (9.65, 37.65) -- (78.07, 37.65);



  \path[draw=white,line cap=butt,line join=round,line width=0.19mm,miter 
  limit=10.0] (9.65, 48.71) -- (78.07, 48.71);



  \path[draw=white,line cap=butt,line join=round,line width=0.19mm,miter 
  limit=10.0] (9.65, 59.77) -- (78.07, 59.77);



  \path[draw=white,line cap=butt,line join=round,line width=0.19mm,miter 
  limit=10.0] (9.65, 70.83) -- (78.07, 70.83);



  \path[draw=white,line cap=butt,line join=round,line width=0.38mm,miter 
  limit=10.0] (9.65, 21.06) -- (78.07, 21.06);



  \path[draw=white,line cap=butt,line join=round,line width=0.38mm,miter 
  limit=10.0] (9.65, 32.12) -- (78.07, 32.12);



  \path[draw=white,line cap=butt,line join=round,line width=0.38mm,miter 
  limit=10.0] (9.65, 43.18) -- (78.07, 43.18);



  \path[draw=white,line cap=butt,line join=round,line width=0.38mm,miter 
  limit=10.0] (9.65, 54.24) -- (78.07, 54.24);



  \path[draw=white,line cap=butt,line join=round,line width=0.38mm,miter 
  limit=10.0] (9.65, 65.31) -- (78.07, 65.31);



  \path[draw=white,line cap=butt,line join=round,line width=0.38mm,miter 
  limit=10.0] (22.48, 16.31) -- (22.48, 72.09);



  \path[draw=white,line cap=butt,line join=round,line width=0.38mm,miter 
  limit=10.0] (43.86, 16.31) -- (43.86, 72.09);



  \path[draw=white,line cap=butt,line join=round,line width=0.38mm,miter 
  limit=10.0] (65.24, 16.31) -- (65.24, 72.09);



  \path[draw=c333333,line cap=butt,line join=round,line width=0.38mm,miter 
  limit=10.0] (22.48, 56.66) -- (22.48, 63.09);



  \path[draw=c333333,line cap=butt,line join=round,line width=0.38mm,miter 
  limit=10.0] (22.48, 47.13) -- (22.48, 44.03);



  \path[draw=c333333,fill=white,line cap=butt,line join=miter,line 
  width=0.38mm,miter limit=10.0] (14.46, 56.66) -- (14.46, 47.13) -- (30.49, 
  47.13) -- (30.49, 56.66) -- (14.46, 56.66) -- (14.46, 56.66)-- cycle;



  \path[draw=c333333,line cap=butt,line join=miter,line width=0.75mm,miter 
  limit=10.0] (14.46, 50.22) -- (30.49, 50.22);



  \path[draw=c333333,line cap=butt,line join=round,line width=0.38mm,miter 
  limit=10.0] (43.86, 28.56) -- (43.86, 30.11);



  \path[draw=c333333,line cap=butt,line join=round,line width=0.38mm,miter 
  limit=10.0] (43.86, 22.93) -- (43.86, 18.85);



  \path[draw=c333333,fill=white,line cap=butt,line join=miter,line 
  width=0.38mm,miter limit=10.0] (35.84, 28.56) -- (35.84, 22.93) -- (51.88, 
  22.93) -- (51.88, 28.56) -- (35.84, 28.56) -- (35.84, 28.56)-- cycle;



  \path[draw=c333333,line cap=butt,line join=miter,line width=0.75mm,miter 
  limit=10.0] (35.84, 27.01) -- (51.88, 27.01);



  \path[draw=c333333,line cap=butt,line join=round,line width=0.38mm,miter 
  limit=10.0] (65.24, 64.92) -- (65.24, 69.56);



  \path[draw=c333333,line cap=butt,line join=round,line width=0.38mm,miter 
  limit=10.0] (65.24, 59.47) -- (65.24, 58.67);



  \path[draw=c333333,fill=white,line cap=butt,line join=miter,line 
  width=0.38mm,miter limit=10.0] (57.22, 64.92) -- (57.22, 59.47) -- (73.26, 
  59.47) -- (73.26, 64.92) -- (57.22, 64.92) -- (57.22, 64.92)-- cycle;



  \path[draw=c333333,line cap=butt,line join=miter,line width=0.75mm,miter 
  limit=10.0] (57.22, 60.28) -- (73.26, 60.28);



  \node[text=c4d4d4d,anchor=south east] (text25) at (7.91, 19.55){0.1};



  \node[text=c4d4d4d,anchor=south east] (text26) at (7.91, 30.61){0.2};



  \node[text=c4d4d4d,anchor=south east] (text27) at (7.91, 41.67){0.3};



  \node[text=c4d4d4d,anchor=south east] (text28) at (7.91, 52.73){0.4};



  \node[text=c4d4d4d,anchor=south east] (text29) at (7.91, 63.79){0.5};



  \path[draw=c333333,line cap=butt,line join=round,line width=0.38mm,miter 
  limit=10.0] (8.68, 21.06) -- (9.65, 21.06);



  \path[draw=c333333,line cap=butt,line join=round,line width=0.38mm,miter 
  limit=10.0] (8.68, 32.12) -- (9.65, 32.12);



  \path[draw=c333333,line cap=butt,line join=round,line width=0.38mm,miter 
  limit=10.0] (8.68, 43.18) -- (9.65, 43.18);



  \path[draw=c333333,line cap=butt,line join=round,line width=0.38mm,miter 
  limit=10.0] (8.68, 54.24) -- (9.65, 54.24);



  \path[draw=c333333,line cap=butt,line join=round,line width=0.38mm,miter 
  limit=10.0] (8.68, 65.31) -- (9.65, 65.31);



  \path[draw=c333333,line cap=butt,line join=round,line width=0.38mm,miter 
  limit=10.0] (22.48, 15.34) -- (22.48, 16.31);



  \path[draw=c333333,line cap=butt,line join=round,line width=0.38mm,miter 
  limit=10.0] (43.86, 15.34) -- (43.86, 16.31);



  \path[draw=c333333,line cap=butt,line join=round,line width=0.38mm,miter 
  limit=10.0] (65.24, 15.34) -- (65.24, 16.31);



  \node[text=c4d4d4d,anchor=south east,cm={ 0.71,0.71,-0.71,0.71,(24.61, 
  -67.57)}] (text36) at (0.0, 80.0){Keine};



  \node[text=c4d4d4d,anchor=south east,cm={ 0.71,0.71,-0.71,0.71,(46.0, 
  -67.57)}] (text37) at (0.0, 80.0){Stufe 1};



  \node[text=c4d4d4d,anchor=south east,cm={ 0.71,0.71,-0.71,0.71,(67.38, 
  -67.57)}] (text38) at (0.0, 80.0){Stufe 3};



  \node[anchor=south west] (text39) at (9.65, 75.04){Zeitgruppe $\times$ \gls{forschungsdaten} (\% zu Zeitgruppe)};

  \node[anchor=south west] (text38) at (0, 76){\Large\textbf{(C)}};



\end{tikzpicture}
}%
        \caption{Klassifikation-Kastengrafiken für $\text{\textit{Zeitgruppe}}\times\text{\textit{\gls{forschungsdaten}}}$ für \gls{fakultät2}. \textbf{(A)}~Absolute Streuung. \textbf{(B)}~Streuung relativ zum Stufengesamtwert. \textbf{(C)}~Streuung relativ zum \textit{Zeitgruppe}-Gesamtwert.}
        \label{fig:faculty_a_sampled_evaluated_factors-only_Zeitgruppe_x_FD_absolute_boxplot}
    \end{figure}


\section{Skripte}
\lstinputlisting[language=Python,label={lst:python-luh-repo-stratification},caption={FIXME!}]{content/code/python-luh-repo-stratification.py}
\lstinputlisting[language=Python,label={lst:simple-dspace5-downloader},caption={FIXME!}]{content/code/simple-dspace-downloader.py}
\lstinputlisting[language=Bash,label={lst:luh-repo-document-search},caption={FIXME!}]{content/code/bash-luh-repo-search.sh}
\lstinputlisting[language=Python,label={lst:bash-luh-repo-csv-combiner},caption={FIXME!}]{content/code/luh-repo-csv-combiner.py}

    %% Selbstständigkeitserklärung
    % Wichtig: Ersetze die PDF-Datei mit einer ausgefüllten Variante vor der Abgabe!
    % Wichtig: Die hier beigefügte Datei entspricht der Vorlage des Instituts für Bibliotheks- und Informationswissenschaft. Nutze die Vorlage des entsprechend gültigen Fachbereichs!
    \cleardoubleoddstandardpage
%% Ersetze die folgende PDF-Datei mit einer eigens ausgefüllten Variante
\includepdf[pages=-,pagecommand={\thispagestyle{empty}\begin{tikzpicture}[remember picture, overlay]
    \node[anchor=west] at (0.4, -0.85) {Krassnig};
    \node[anchor=west] at (8.3, -0.85) {David};
    \node[anchor=west] at (0.4, -2.125) {632351};
    \path [fill=black] (-0.1,-6.15) rectangle (0.04,-6.3);
    \node[draw, align=left, draw=none, anchor=north west, execute at begin node=\setlength{\baselineskip}{1.35\baselineskip}] at (-1.45,-9.775) {Publikationspraktiken für Forschungsdaten in Hochschulschriften: Eine Untersuchung der\\ Veröffentlichungsformate und -methoden};
    \node[anchor=west] at (0.4, -21.15) {12.06.2024};
    \node[inner sep=0pt, anchor=west] at (8.85,-21.15)
    {\includegraphics[width=.25\textwidth]{matter/backmatter/signature.png}};
  \end{tikzpicture}}]{matter/backmatter/selbststaendigkeitserklaerung.pdf}
\end{document}
