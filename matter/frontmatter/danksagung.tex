\chapter{Danksagung}
Als erstes möchte ich meinen zwei Gutachtern, Dr.~Sarah Dellmann und Prof.~Dr.~Robert Jäschke, für ihre Unterstützung sowie für ihre Bereitwilligkeit, dieses viel zu lange Dokument zu lesen und zu begutachten, danken.

Darüber hinaus möchte ich mich bei allen Mitarbeitenden der Technischen Informationsbibliothek bedanken, die sich bereit erklärt haben, mir in beratender Funktion zur Seite zu stehen und es mir auch ermöglicht haben, im Rahmen eines Praktikums administrativen Zugriff auf das institutionelle Repositorium der Leibniz Universität Hannover zu erhalten.
Hier ist ein besonder Dank an alle Mitglieder des Projekts \textit{FoHop!} und an die Mitarbeitenden aus dem Bereich \textit{Publikationsdienste und Open Access} gerichtet.
Mein Dank an die Mitarbeitenden der Technischen Informationsbibliothek inkludiert dabei, alphabetisch nach Nachnamen sortiert, Anika Altenberg, Olga Engelhardt, Israel Holger, Petra Mensing, Linna Lu, Anna-Karina Renziehausen, Corinna Schneider, Cäcilia Schröer und Anne Quinkenstein.

Auch bedanken möchte ich mich bei allen Mitarbeitenden der Universitätsbibliothek Johann Christian Senckenberg, die mich im Rahmen meines Bibliotheksreferendariats begleitet und betreut haben.
Hier besonders zu erwähnen sind Bernhard Wirth, welcher es uns Referendaren durch seine sorgfältige und interaktive Planung des Referendariats erlaubte, alle Facetten des bibliothekarischen Alltags zu erleben, Agnes Brauer, welche mich stets offen und enthusiastisch in meinem Schwerpunkt, den \textit{Digital Humanities}, betreut hat, Daniela Poth, welche als Direktorin der Bibliothek stets ein offenes Ohr für uns Referendare hatte und uns tiefgehende Einblicke in die Leitung einer wissenschaftlichen Bibliothek gewährte, und Dr.~Christoph Marutschke, welcher mir meinen ersten tiefergehenden und praktischen Einblick in das Thema Forschungsdatenmanagement ermöglichte.
Darüber hinaus gebührt mein Dank hier auch Michelle Kamolz, Jakob Frohmann und unzähligen weiteren Kolleginnen und Kollegen, die sich hier leider nicht alle aufzählen lassen.

Auch möchte ich mich an meine hessischen Mitreferendaren für ihre Kollegialität, gegenseitige Unterstützung und Freundschaft bedanken.
Dieses besondere Dank geht an Alicia Schwammborn, welche es zwei Jahre mit mir an derselben Bibliothek ausgehalten hat, Maureen Bössow, Marvin Gusen und Henrike Weyer.

Mein Dank geht auch an das Institut für Bibliotheks- und Informationswissenschaft, an all seinen Mitarbeitenden, an alle meine Kommilitoninnen und Kommilitonen der Matrikel~28 sowie an alle Dozentinnen und Dozenten, die mir ein unglaublich abwechslungsreiches und spannendes Studium ermöglichten.

Auch möchte ich zutiefst bei meiner Frau, Sarah C. Krassnig, bedanken.
Ich kann mir nur schwer eine Welt vorstellen, in der diese Abschlussarbeit in ihrer jetzigen Form hätte kreiert werden können, wenn sie mich nicht allzeit unterstützen würde.\enlargethispage{\baselineskip}

Mein ewigwährender Dank geht auch an die beste Familie der Welt, die ich mir hätte wünschen können: Karl-Heinz, Roswitha, Stefan, Anna, Gerlinde, Anton und Therese Krassnig, Egon und Käthe Stein sowie Mary, Leyla und Melissa Sagin.