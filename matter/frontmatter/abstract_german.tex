\chapter{Zusammenfassung}

Diese Abschlussarbeit befasst sich mit den Forschungsdaten in Dissertationen und dem dazugehörigen Forschungsdatenmanagement an deutschen Universitäten.
Die Arbeit lässt sich thematisch in zwei Teile gliedern.

Im ersten Teil werden allgemeingültige und promotionsspezifische Richtlinien einer Stichprobe aller deutschen promotionsberechtigten Institutionen auf Anweisungen zu Forschungsdaten untersucht.
Dieser Bereich wird in \cref{ch:richtlinien} behandelt.

Im zweiten Teil wird eine repräsentative, mehrschichtige Stichprobe aller Dissertationen aus dem institutionellen Repositorium der Leibniz Universität Hannover, die zwischen 2012 und 2023 erschienen sind, in Bezug auf Forschungsdaten untersucht.
Hierbei steht besonders im Fokus, wie diese Forschungsdaten publiziert worden sind: als integrierter Bestandteil der Dissertation, als begleitende Dateien oder als separate Veröffentlichung auf einer externen Plattform.
Dieser Bereich wird in \cref{ch:luh-repo} behandelt.

Die Untersuchung der allgemeingültigen Richtlinien der promotionsberechtigten Institutionen ergab, dass etwa \SI{45}{\percent} aller Institutionen in Deutschland eine allgemeine Forschungsdaten-Richtlinie besitzen (davon sind etwa \SI{60}{\percent} Universitäten).
Weitere \SI{45}{\percent} besitzen zumindest eine Richtlinie zu den Regeln der Guten Wissenschaftlichen Praxis, während die restlichen \SI{10}{\percent} keine allgemeine Richtlinie besitzen, die Forschungsdaten oder Forschungsdatenmanagement erwähnt (davon sind etwa \SI{71}{\percent} künstlerische Hochschulen).
Es wurde festgestellt, dass die Trägerschaft sowie die Institutionsart einen statistisch signifikanten Einfluss auf die höchste Klassifikationsstufe der allgemeinen Dokumente haben.
Entsprechende mögliche Erklärungen zu den einzelnen signifikanten Relationen werden in \cref{sec:policy-discussion-general} gegeben.

Die Untersuchung der promotionsspezifischen Richtlinien der promotionsberechtigten Institutionen ergab, dass nur etwa \SI{14}{\percent} aller Institutionen in Deutschland mindestens eine Promotions- oder Prüfungsordnung besitzen, die den Umgang mit Forschungsdaten verpflichtend definiert.
Von diesen Institutionen besitzen \SI{75}{\percent} auch eine allgemeine Forschungsdaten-Richtlinie.
Mehr als die Hälfte aller Institutionen haben jedoch zumindest eine Promotions- oder Prüfungsordnung, die verpflichtende Regularien zu den Regeln der Guten Wissenschaftlichen Praxis enthält. 
Ein weiteres Viertel der Institutionen besitzen keine Promotions- oder Prüfungsordnung, die Forschungsdaten oder Verpflichtungen zu den Regeln der Guten Wissenschaftlichen Praxis enthält.
Es wurde festgestellt, dass das Bundesland, die Institutionsart einen statistisch signifikanten Einfluss auf die höchste Klassifikationsstufe der promotionsspezifischen Dokumente haben.
Zusätzlich wurde festgestellt, dass die höchste Klassifikationsstufe der allgemeinen und promotionsspezifischen Richtlinien einer Institution signifiankt voneinander abängen.
Entsprechende mögliche Erklärungen zu den einzelnen signifikanten Relationen werden in \cref{sec:policy-discussion-specific} gegeben.

Die Untersuchung der Dissertationen aus dem institutionellen Repositorium der Leibniz Universität Hannover ergab, dass etwa \SI{87}{\percent} aller Dissertationen aus der Stichprobe zumindest in Teilen auf originellen Primärdaten basieren.
Von diesen Dissertationen haben etwa \SI{61}{\percent} zumindest einen Teil ihrer Forschungsdaten publiziert.
Dieser Anteil sinkt auf etwa \SI{32}{\percent}, wenn komprimierte Fotografien, Spektraldiagramme, Gensequenzen, unausgefüllte Fragebögen, Leitfäden und Montagezeichnungen wegen mangelnder Qualität, Relevanz oder schwer zu evaluierender Originalität ausgeschlossen werden.
Die häufigste Publikationsart waren in die Dissertationsdatei integrierte Forschungsdaten: Diese machten etwa \SI{57}{\percent} aller Dissertationen mit originellen Primärdaten aus.
Nur etwa \SI{7}{\percent} aller Dissertationen mit Primärdaten veröffentlichten Forschungsdaten auf einer externen Plattform und nur etwa \SI{2}{\percent} veröffentlichten Forschungsdaten als begleitende Dateien im institutionellen Repositorium der Leibniz Universität Hannover.
Es wurde festgestellt, dass es statistisch signifikante Abhängigkeiten zwischen der Art der Forschungsdatenpublikation und der jeweiligen Fakultät, dem jeweiligen Zeitraum und der genutzten Sprache gibt.
Es zeigte sich, dass der Anteil an extern publizierten Forschungsdaten über einen Zeitraum von zwölf Jahren, hauptsächlich in den letzten vier Jahren, stark angestiegen ist, sich jedoch zu großen Teilen auf nur drei Fakultäten beschränkt.
Ähnlich zu diesem Wachstum in externen Forschungsdaten wechselte auch die dominante Sprache dieser Fakultäten von Deutsch auf Englisch.
Da eine statistisch signifikante Abhängigkeit zwischen Sprache und Fakultät sowie Zeitgruppe nachgewiesen werden konnte, wurde die Sprache als eigenständiger Faktor unter Vorbehalt ausgeschlossen.
Entsprechende mögliche Erklärungen zu den einzelnen signifikanten Relationen werden in \cref{sec:luh-repo-discussion} gegeben.

Insgesamt konnte festgestellt werden, dass nur ein sehr kleiner Anteil an Forschungsdaten so veröffentlicht wurde, dass sie den FAIR-Prinzipien entsprechen.
Um diese Situation zu verbessern, wurden in \cref{sec:luh-repo-discussion} einige mögliche Handlungsempfehlungen für die LUH und die TIB ausgesprochen.