\chapter{Abstract}

This thesis deals with research data in dissertations and the associated research data management at German universities.
The thesis is thematically divided into two parts.

In the first part, general and dissertation-specific guidelines of a sample of all German institutions entitled to confer doctoral degrees are examined for instructions on research data.
This section is covered in Chapter~\ref{ch:richtlinien}.

In the second part, a representative, multi-layered sample of all dissertations from the institutional repository of the Leibniz University Hannover, published between 2012 and 2023, is examined with regard to research data.
The focus is particularly on how this research data has been published: as an integrated part of the dissertation, as accompanying files, or as a separate publication on an external platform.
This section is covered in Chapter~\ref{ch:luh-repo}.

The investigation of the general guidelines of the institutions entitled to confer doctoral degrees revealed that about \SI{45}{\percent} of all institutions in Germany have a general research data policy (of these, about \SI{60}{\percent} are universities).
A further \SI{45}{\percent} have at least a guideline on the rules of good scientific practice, while the remaining \SI{10}{\percent} have no general policy mentioning research data or research data management (of these, about \SI{71}{\percent} are art colleges).
It was found that the type of institution and the type of sponsorship have a statistically significant influence on the highest classification level of the general documents.
Corresponding possible explanations for the individual significant relationships are given in Section~\ref{sec:policy-discussion-general}.

The investigation of the dissertation-specific guidelines of the institutions entitled to confer doctoral degrees revealed that only about \SI{14}{\percent} of all institutions in Germany have at least one doctoral or examination regulation that obligatorily defines the handling of research data.
Of these institutions, \SI{75}{\percent} also have a general research data policy.
More than half of all institutions have at least one doctoral or examination regulation containing mandatory regulations on the rules of good scientific practice, while about a quarter of all institutions have no doctoral or examination regulation containing research data or the rules of good scientific practice.
It was found that the federal state, the type of institution, and the highest classification level of the general documents have a statistically significant influence on the highest classification level of the dissertation-specific documents.
Corresponding possible explanations for the individual significant relationships are given in Section~\ref{sec:policy-discussion-specific}.

The investigation of dissertations from the institutional repository of the Leibniz University Hannover revealed that about \SI{87}{\percent} of all dissertations in the sample are based at least in part on original primary data.
Of these dissertations, about \SI{61}{\percent} have published at least some of their research data.
This proportion decreases to about \SI{32}{\percent} when compressed photographs, spectral diagrams, gene sequences, unfilled questionnaires, guidelines, and assembly drawings are excluded due to lack of quality, relevance, or difficult-to-evaluate originality.
The most common type of publication was research data integrated into the dissertation file, which accounted for about \SI{57}{\percent} of all dissertations with original primary data.
Only about \SI{7}{\percent} of all dissertations with primary data published research data on an external platform, and only about \SI{2}{\percent} published research data as accompanying files in the institutional repository of the Leibniz University Hannover.
It was found that there are statistically significant dependencies between the type of research data publication and the respective faculty, the respective period, and the language used.
It was shown that the proportion of externally published research data increased significantly over a period of twelve years, mainly in the last four years, but was largely confined to only three faculties.
Similar to this growth in external research data, the dominant language in these faculties changed from German to English.
Since a statistically significant dependency between language and faculty as well as time group could be demonstrated, language was excluded as an independent factor with reservation.
Corresponding possible explanations for the individual significant relationships are given in Section~\ref{sec:luh-repo-discussion}.

Overall, it was found that only a very small proportion of research data was published in accordance with the FAIR principles.
To improve this situation, several possible recommendations for action for LUH and TIB are given in Section~\ref{sec:luh-repo-discussion}.